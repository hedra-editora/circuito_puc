\begin{itemize}
\item \textbf{1967, meio século depois} \lipsum[1]
  
\item \textbf{Felipe Scovino} é professor do Departamento de História e Teoria da Arte e do Programa de Pós"-Graduação em Artes Visuais da \versal{UFRJ}. Foi professor visitante na Universidad de Chile em 2014. É organizador dos livros \emph{Arquivo Contemporâneo} (7Letras, 2009), \emph{Cildo Meireles} (Azougue Editorial, 2009) e \emph{Carlos Zilio} (Museu de Arte Contemporânea de Niterói, 2010). É coautor de \emph{Coletivos} (Circuito, 2010). Escreveu ensaios sobre arte contemporânea para \emph{Arte \& Ensaios}, \emph{Artforum}, \emph{Art Review}, \emph{Flash Art}, \emph{Third Text}, dentre outros periódicos. Foi curador das exposições \emph{Estes Nortes} (Centro de Arte Hélio Oiticica, Rio de Janeiro, 2012), \emph{Lygia Clark: uma retrospectiva} (Itaú Cultural, São Paulo, 2012, co"-curador Paulo Sergio Duarte), Abraham Palatnik: a reinvenção da pintura (\versal{CCBB}, Brasília, 2013; \versal{MON}, Curitiba, 2014; \versal{MAM}, São Paulo, 2014; Fundação Iberê Camargo, Porto Alegre, 2015; \versal{CCBB}, Rio de Janeiro, 2017), \emph{Franz Weissmann: a forma do vazio} (Itaú Cultural, São Paulo, 2019), dentre outras. Recebeu a Bolsa de Estímulo à Produção Crítica (Minc/Funarte) em 2008.

\item \textbf{Frederico Coelho} \lipsum[1]

\item \textbf{Pedro Duarte} é professor Doutor de Filosofia na \versal{PUC}"-Rio. Foi Professor Visitante nas universidades Brown (\versal{EUA}) e Södertörns (Suécia). É autor dos livros \emph{Estio do tempo: romantismo e estética moderna} (Zahar), \emph{A palavra modernista: vanguarda e manifesto} (Casa da Palavra) e \emph{Tropicália} (Cobogó). Tradutor do livro \emph{Liberdade para ser livre}, de Hannah Arendt (Bazar do Tempo). Co"-autor, roteirista e curador da série de \versal{TV} ``Alegorias do Brasil'', junto com o diretor Murilo Salles (Canal Curta!).


\item \textbf{Sérgio Bruno Martins} \lipsum[1]

\end{itemize}

