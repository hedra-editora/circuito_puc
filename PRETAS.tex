\begin{itemize}
\item \textbf{1967, meio século depois} é um livro que sonda --- na confluência das artes plásticas, literatura, cinema, \versal{TV}, teatro e política --- o ano de 1967 e seus principais desdobramentos no pensamento e na cultura brasileira. Diversos ensaístas e intelectuais se encontram aqui para analisar do tropicalismo e os Festivais de Música às revoluções perpetradas por Zé Celso no Teatro Oficina, passando pelo Cinema Novo de Glauber Rocha e as provocantes obras de Hélio Oiticica. Às vésperas do período mais ditatorial e repressivo da história política brasileira, as artes tentaram abrir novos caminhos de liberdade e insurgência de um Brasil plurívoco e múltiplo, nos quais ora os autores procuram suas linhas de permanências e rupturas, tentando traçar suas possiblidades cinquenta anos depois.

\item \textbf{Felipe Scovino} é professor do Departamento de História e Teoria da Arte e do Programa de Pós"-Graduação em Artes Visuais da \versal{UFRJ}. Foi professor visitante na Universidad de Chile em 2014. É organizador dos livros \emph{Arquivo Contemporâneo} (7Letras, 2009), \emph{Cildo Meireles} (Azougue Editorial, 2009) e \emph{Carlos Zilio} (Museu de Arte Contemporânea de Niterói, 2010). É coautor de \emph{Coletivos} (Circuito, 2010). Escreveu ensaios sobre arte contemporânea para \emph{Arte \& Ensaios}, \emph{Artforum}, \emph{Art Review}, \emph{Flash Art}, \emph{Third Text}, dentre outros periódicos. Foi curador das exposições \emph{Estes Nortes} (Centro de Arte Hélio Oiticica, Rio de Janeiro, 2012), \emph{Lygia Clark: uma retrospectiva} (Itaú Cultural, São Paulo, 2012, co"-curador Paulo Sergio Duarte), Abraham Palatnik: a reinvenção da pintura (\versal{CCBB}, Brasília, 2013; \versal{MON}, Curitiba, 2014; \versal{MAM}, São Paulo, 2014; Fundação Iberê Camargo, Porto Alegre, 2015; \versal{CCBB}, Rio de Janeiro, 2017), \emph{Franz Weissmann: a forma do vazio} (Itaú Cultural, São Paulo, 2019), dentre outras. Recebeu a Bolsa de Estímulo à Produção Crítica (Minc/Funarte) em 2008.

\item \textbf{Frederico Coelho} é Doutor em Literatura Brasileira pela \versal{PUC}-Rio (2004"-2008) com Bolsa"-Sanduíche da Capes por um ano na New York University (2006). Entre 2001 e 2009 foi pesquisador do Núcleo de Estudos Musicais (\versal{NUM}) da Universidade Cândido Mendes e pesquisador do \versal{NELIM} (Núcleo de Estudos de Literatura e Música) da \versal{PUC}-Rio entre 2009 e 2012. Entre os livros e artigos que escreveu explorou a história cultural brasileira, a música popular, o Modernismo brasileiro, a obra de Hélio Oiticica e a cultura marginal. Em 2009 tornou"-se curador"-assistente de artes visuais do Museu de Arte Moderna do Rio de Janeiro (\versal{MAM-RJ}), onde ficou até julho de 2011 trabalhando com Luiz Camillo Osório. Desde agosto de 2014, é Professor Assistente dos cursos de Literatura e Artes Cênicas e da Pós"-Graduação em Literatura, Cultura e Contemporaneidade (\versal{PPGLCC}) do Departamento de Letras da \versal{PUC}-Rio.

\item \textbf{Pedro Duarte} é professor Doutor de Filosofia na \versal{PUC}"-Rio. Foi Professor Visitante nas universidades Brown (\versal{EUA}) e Södertörns (Suécia). É autor dos livros \emph{Estio do tempo: romantismo e estética moderna} (Zahar), \emph{A palavra modernista: vanguarda e manifesto} (Casa da Palavra) e \emph{Tropicália} (Cobogó). Tradutor do livro \emph{Liberdade para ser livre}, de Hannah Arendt (Bazar do Tempo). Co"-autor, roteirista e curador da série de \versal{TV} ``Alegorias do Brasil'', junto com o diretor Murilo Salles (Canal Curta!).


\item \textbf{Sérgio Bruno Martins} é crítico de arte, professor do departamento de História da \versal{PUC}-Rio e autor do livro \emph{Constructing an Avant-Garde: Art in Brazil, 1949} (\versal{MIT} Press, 2013). Colabora regularmente com a revista \emph{Artforum} e publicou artigos em periódicos como \emph{October}, \emph{Novos Estudos}, \emph{\versal{ARTM}argins} e \emph{Third Text}. Para esta última, organizou o dossiê ``Bursting on the Scene: Looking Back at Brazilian Art'', de 2012. É autor também de ensaios em diversos catálagos de exposições, como ``Cildo Meireles'' (Reina Sofia and Serralves, 2013), ``Alexander Calder: Performing Sculpture'' (Tate Modern, 2015), ``Hélio Oiticica: to Organize Delirium'' (Carnegie, Art Institute of Chicago and Whitney, 2016), ``Lygia Clark: uma retrospectiva'' (Itaú Cultural, 2014), ``Lygia Pape: a Multitude of Forms'' (The Metropolitan Museum of Art, 2017), ``Anna Maria Maiolino'' (\versal{M}o\versal{CA}, Los Angeles, 2017) e ``Wanda Pimentel'' (\versal{MASP}, 2017). Seu atual projeto pesquisa, financiado pela bolsa Jovem Cientista do Nosso Estado (\versal{FAPERJ}), toma a gênese do livro de artista \emph{Trama}, de Antonio Dias, como base para examinar sua trajetória em meio ao circuito de arte conceitual europeu das décadas de 1960 e 1970.

\end{itemize}

