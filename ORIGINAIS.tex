\addcontentsline{toc}{chapter}{A mercadoria enternecida e os índios, \emph{por João Camillo Penna}}
\chapter*{A mercadoria enternecida e os índios \emph{Limites~e~deslimites~do~tropicalismo~hoje}\footnote{Este texto é uma parte de um ensaio maior,
  em progresso, que por sua vez é uma versão bastante modificada de
  conferências que dei entre 2017 e 2018. Gostaria de agradecer aos
  organizadores de cada evento: Fred Coelho e Felipe Scovino (\versal{PUC"-RJ},
  novembro de 2017); Alexandre Nodari (Universidade Federal do Paraná,
  fevereiro de 2018); Virginia Figueiredo (\versal{UFMG}, abril de 2018).}}

\begin{flushright}
\emph{João Camillo Penna}
\end{flushright}

No final da introducão de \emph{Verdade tropical}, Caetano fala das
``entranhas imundas (e, no entanto, saneadoras) da internacionalizante
indústria do entretenimento''.\footnote{Caetano Veloso. \emph{Verdade
  tropical}. São Paulo: Companhia das Letras, 2017, 3ª edição, p. 53.}
Como entender essa figura dúplice da mercadoria, ao mesmo tempo
saneadora e imunda, em sua obsessiva busca do prazer? Há um aspecto
libertador no tropicalismo do gozo da mercadoria, como um antídoto ao
interdito da ascese nacionalista"-marxista, um aval permissivo de
desfrute e mergulho na cultura de massas. Esta cultura encontra hoje o
seu limite, quando se limita justamente à ``mansão das liberdades
modernas'', com a transformação do ser humano em agente geológico, isso
que Paul J. Crutzen e Eugene F. Stoermer denominaram
antropoceno,\footnote{Dipesh Chakrabarty. ``O clima da história:
  quatro teses''. Trad. Denise Bottmann et~al. \emph{Sopro}, 92,
  07/2013, p. 11.} e outros denominaram capitaloceno.\footnote{Donna
  Haraway. ``Anthropocene, Capitalocene, Plantationocene, Chthulucene:
  Making Kin''. \emph{Environmental Humanities}, vol. 6, 2015.} Uma
frase na apresentação da segunda edição de \emph{Verdade tropical}
parece indicar esse limite: ``Um tema, no entanto, abala todo o meu
esboço de visão de mundo: a questão ambiental''.\footnote{Caetano
  Veloso. \emph{Verdade tropical}, op.~cit., p. 28.} Mas antes de chegar
aí, é preciso entender esse ciclo, no qual se insere o programa
tropicalista, para poder depois enxergar os seus limites, tal qual
vistos a partir de 2017-2019. Tomemos uma canção prototípica em tantos
sentidos, como ``Alegria, alegria'', apresentada no Festival da Record
de 1967, um verdadeiro hino à liberdade com a sua permissão de gozo,
para em seguida desenhar o seu contorno e acabamento.

A dupla citação de Chacrinha e de Wilson Simonal no título, o pastiche
da marchinha ternária ``A banda'', de Chico Buarque, ganhadora do mesmo
Festival no ano anterior, que faz de ``Alegria, alegria'' em muitos
sentidos uma anti"-banda, a intenção original de que a canção fosse
executada com acompanhamento da banda de Roberto Carlos (o \versal{RC}7) --- todos
são procedimentos em que se identifica o decalque pop do objeto de
consumo, ao modo de uma Brillo Box warholiana da canção.\footnote{Ibidem,
  p. 184-185, p. 191.} Caetano percebe nela um método composicional
característico, de ``justaposição de acordes maiores em relações
insólitas'', oposto termo a termo ao método bossanovístico, de ``peças
redondas em que as vozes internas dos acordes alterados se mov{[}em{]}
com natural fluência'', que tinha por modelo ``Bom dia'' de Gilberto
Gil, canção que concorrera no mesmo Festival da Canção, na belíssima voz
de Nana Caymmi, que por sua vez traduzia algo da audição da música dos
Beatles.\footnote{Ibidem, p. 187.} Em ``Bom dia'', apresentava"-se de fato
o procedimento da segmentação silábica de acordes maiores, que vai
aparecer em certos trechos de ``Alegria, alegria'', como ``de
presidente'' (no verso ``em caras de presidente''), ou ``beijos de
amor''\ldots{} A sequência de cortes em segmentos súbitos, sem evolução,
trazia para o esquema harmônico o modelo citacional da ``colcha de
retalhos de frases musicais da tradição sentimental brasileira'',
formalizado na primeira canção tropicalista, ``Paisagem útil'', ela
também de 1967.\footnote{Ibidem, p. 140.} O que explicita, sem dúvida,
uma diferença palpável para com o fraseado redondo do melodismo
bossanovístico, ao mesmo tempo que uma tradução do estilo composicional
dos Beatles.

``Bom dia'', de fato, processa algo de ``Good day sunshine'' do álbum
\emph{Revolver} (1966) dos Beatles. Ali se encontra o achado da
sequência de acordes maiores ritmando as palavras em tempo forte, porém
com significativo deslocamento do tema da saudação do novo dia. Enquanto
a canção dos Beatles, como de praxe nesse momento, trabalha com a
imitação eufórica do amor, aonde o nascer do dia se conjuga à
felicidade de estar apaixonado (``estou apaixonado e é um dia de sol''),
cristalizando a utopia amorosa que contém o cerne do projeto
político"-cancional dos Beatles, em ``Bom dia'', o dia vem interromper o
encontro noturno dos amantes, chamando para o trabalho na ``usina do
dono do teu cansaço''. O despertar de madrugada, anterior ao sol,
registra os preparativos do trabalhador para o dia de trabalho, que
rouba o seu dia. A canção sonoriza algo do apito da fábrica de tecidos
de Noel, e se aproxima do canto de trabalho. Enquanto o dia dos Beatles
é o dia prazeroso do ócio amoroso, o de Gil formaliza a consciência
social da expropriação do trabalho, como negócio e mais"-valia do patrão.
A denúncia musical da exploração do trabalho tematizada pela canção
parece se contrapor ao esquema apropriado pela escuta de Gil, que traz
para a estrutura harmônica o modelo musical do pastiche e da repetição
pop. Esta se coaduna melhor com a arquitetura de ``Alegria, alegria'',
movida pela fusão de dois procedimentos opostos: a distância crítica da
paródia que Caetano explica ser a ``condição de liberdade'' e a fruição
alegre pelas coisas.\footnote{Ibidem, p. 184.}

O que é nomeado ali na lista de objetos tão ao gosto das canções mais
representativas do período é uma autorização para ``se \emph{atirar},
sem destino, ao puro destino das mercadorias'', para citar uma bela
análise de Tales Ab'Sáber.\footnote{Tales Ab'Sáber. \emph{Ensaio,
  fragmento. 205 apontamentos de um ano}. São Paulo: Editora 34, 2014,
  p. 38.}

%afastar um pouco da mancha?
\begin{quote}
O sol nas bancas de revista\\
Me enche de alegria e preguiça\\
Quem lê tanta notícia\\
Eu vou\footnote{Disponível em: \textless{}\emph{https://www.letras.mus.br/caetano-veloso/43867/}\textgreater{}.}
\end{quote}

A repartição do sol equívoco, ao mesmo tempo natural e tipográfico (a
referência ao jornal \emph{O sol}); ``sol de quase dezembro'', o astro
``tão bonito'', signo dos trópicos, e um sobrevoo sobre as notícias e
fotos de jornal, repousa sobre uma duplicidade de fundo entre artifício
e natureza, \emph{Sol} e sol, e contém a cifra de uma tropologia pop,
que está no cerne da proposição tropicalista. O pop relata a
transformação eufórica da natureza em produto, e do ser em produção,
colhendo uma lição de Marx.\footnote{Gérard Granel. ``Incipit Marx.
  L'ontologie marxiste de 1844 et la question de la `coupure'". In:
  \emph{Traditionis traditio}. Paris: Gallimard, 1972.} O antropoceno,
na outra ponta da linha, colhe o resultado dessa euforia, como disforia
em torno da crise climática, e limite ao gozo ilimitado da liberdade.

Diante do mundo transformado em objeto: crimes, espaçonaves, guerrilhas,
Cardinales bonitas, caras de presidentes, beijos de amor, dentes,
pernas, bandeira, bomba e Brigitte Bardot, o cancionista afirma o
destino de ir. ``Caminhando contra o vento'' retoma o motivo do vento de
``Blowin' in the wind'' (1962) de Bob Dylan, a arquetípica \emph{protest
song}, hino do movimento pelos direitos civis e da campanha contra a
guerra do Vietnã, invertendo o sentido da ação humana. A resposta a
todas as perguntas em série que protestam na canção de Dylan a respeito
do atraso do tempo em reverter as iniquidades humanas (quanto tempo será
preciso?), repousa na ação enigmática, assignificante e gratuita, cifra
da poesia da canção: ``a resposta, meu amigo, é soprar no vento'' (``The
answer, my friend, is blowin' in the wind''). Já em ``Alegria, alegria''
ela é revertida em contrariedade e deslocamento espacial, em caminhar
``\emph{contra} o vento'', o que frisa a resistência de uma
contra"-cultura cuja condição de existência é ser contra. É algo do que
Oiticica captara ao lançar o lema: ``Da adversidade vivemos'',\footnote{Hélio
  Oiticica. ``Esquema geral da Nova Objetividade''. In: \emph{Aspiro ao
  grande labirinto}. Rio de Janeiro: Rocco, 1986, p.~168.} que condensa o
estado de espírito da cultura durante a ditadura. ``Nada nos bolsos ou
nas mãos'', citação do final de \emph{As palavras} de Jean"-Paul Sartre,
condensa o manifesto geracional da ``esquerda existencial'',\footnote{Eduardo
  Viveiros de Castro, em ``Rosa e Clarice, a fera e o fora'', fala da
  divisão do campo cultural dos anos 1960 no Brasil entre duas
  esquerdas, a nacional"-popular e a existencial. \emph{Revista Letras}
  {[}no prelo{]}.} ao identificar no \emph{nada} de posses (nos bolsos,
nas mãos), de status e identidade (lenço, documento), a proposição sobre
a liberdade e a negatividade sartriana.\footnote{Jean"-Paul Sartre.
  \emph{L'être et le néant}. Paris: Gallimard, 1943.} Guilherme Wisnik
assinala a maneira com que essa liberdade do eu se afirma na canção como
independência do canto sobre uma ``cama orquestral de cordas
sintetizadas, soando como um órgão de igreja'', em linhas melódicas
ascendentes e descendentes, que modulam um passeio pelo
espaço.\footnote{Guilherme Wisnik. \emph{Folha explica Caetano}. São
  Paulo: Publifolha, 2005, p. 18-19.} O ``seguir vivendo'' contém afinal
a afirmação vital do gozo das coisas, como diz Tales Ab'Sáber, cuja
hipótese vou seguindo aqui.

A equivocidade entre natureza e artifício é, portanto, um elemento
central da leitura tropicalista do modelo de produção capitalista.
Essencial, por exemplo, à tese da ``linha evolutiva'', o próprio samba é
lido como pop, convertido em forma reproduzida audível em disco e no
rádio,\footnote{``Em suma: o samba tem sido um gênero pop para consumo
  de populações urbanas desde sua consolidação estilística no Rio de
  Janeiro, para a qual o teatro, o rádio e
  o disco contribuíram decisivamente.'' Caetano Veloso. \emph{Verdade tropical}, op.~cit., p.~70.} o samba no pé impresso em vinil, dando
lugar a uma experiência não menos autêntica. Essa equivocidade comparece
na primeira canção tropicalista, ``Paisagem útil'', paródia de ``Inútil
paisagem'', a bossa quase estática de Tom Jobim e Aloysio de Oliveira,
que imita ironicamente a dicção dos cantores do gogó como Francisco
Alves e Orlando Silva. Ali está a ``lua oval do Esso'', a referência
também pop ao outdoor hoje desaparecido da logomarca da multinacional do
varejo de gasolina, igualmente desaparecida, que pairava sobre o céu do
Castelo, no centro do Rio. Assim, como o ``frio palmeiral de cimento''
do Aterro do Flamengo de Lota Macedo Soares e das esculturas
paisagísticas vegetais de Burle Marx, elas também articialmente naturais,
recém inauguradas.

Tudo é produto, tudo é produção e citação, nessa ontologia do produto,
tudo deve ser transformado pela visão ácida tropicalista em
\emph{personagem}, segundo um ``método de dramatização'' que ressalta
sempre o aspecto segundo das coisas. É preciso que esses produtos
citados em listas se transformem na própria canção, que passa a
encarnar, sua pessoa e seu intérprete, o processo produtivo no contexto
colonial"-ditatorial. Basta lembrar o processamento pelo qual passa a
``Carolina'' de Chico Buarque de Hollanda, uma das canções prediletas do
general Costa e Silva na interpretação de Agnaldo Rayol, transformada em
personagem pelos tropicalistas, e gravada por Caetano em 1969.\footnote{O
  caráter exemplar da dramatização de ``Carolina'' é reconstituído em:
  Caetano Veloso. ``Diferentemente dos americanos do norte''. In: \emph{O
  mundo não é chato}. São Paulo: Companhia das Letras, 2005, p.~49.}
Foi igualmente o retrato de capa da revista norte"-americana
\emph{Time}, de abril de 1967, estampando Costa e Silva com uma
bananeira verde e amarela ao fundo, juntamente com a matéria da revista
``cheia de fotos para americano ver'', de exaltações das ``riquezas''
brasileiras, que ``eletrocutaram'' Zé Celso, fornecendo a cifra do
deboche que alimentará a visão da dramaturgia de \emph{O rei da
vela}.\footnote{Zé Celso Martinez Corrêa. \emph{Primeiro Ato. Cadernos,
  depoimentos, entrevistas (1958-1974)}. Ana Helena Carmargo (org.). São
  Paulo: Editora 34, 1998, p. 127.} Repetir, portanto, o próprio
general, a imagem clichê do gosto musical, o senso estético do ditador,
como enunciação característica. Caetano refere"-se sempre ao insight de
Zé Celso, que descrevia assim sem dúvida o próprio procedimento do
Oficina, sobre o ``caráter masoquista da estética tropicalista com sua
reprodução paródica do olhar estrangeiro sobre Brasil e sua eleição de
tudo o que nos parecesse a princípio insuportável''.\footnote{Caetano
  Veloso. ``Diferentemente dos americanos do norte'', op.~cit., p. 51.}
Algo que o mesmo Zé Celso descreverá de outro modo como: ``localizar em
nós mesmos o inimigo, para amá"-lo e, através do choque do momento do
gozo, destruí"-lo''. Próximo da definição de Glauber: ``tropicalismo é
aceitação, ascenção do subdesenvolvimento''.\footnote{Glauber Rocha.
  ``Tropicalismo, antropologia, mito, ideograma''. In: \emph{Revolução do
  Cinema Novo}. São Paulo: Cosac Naify, 2004, p.~151.}

A resposta pop, retomada pelo tropicalismo, consiste em repetir o
processo de produção capitalista de mercadorias, de uma certa maneira
intensificando o circuito produtivo, ao produzir simulacros de produtos,
decalques irônicos, moedas falsificadas introduzidas na livre circulação
de produtos. A repetição do pop brasileiro, que denominamos tropicalismo,
consiste no aproveitamento do resíduo da mercadoria industrial, ou
natural processada, i.~e., convertida em mercadoria industrial, colonial
ou nacional, dando a ela uma outra vida estética. De fato, Caetano
define ``Alegria, alegria'', e por consequência o tropicalismo, como um
``começar a mexer no lixo'', relacionando"-a a outras canções da época,
como ``Superbacana'', Geléia geral'', ou ``Baby'', todas elas contendo
produtos como ``Coca"-Cola'', ou palavras, como ``lanchonete'',
importadas e monstruosas, que lhe davam náusea, e o humilhavam ao
denotarem o domínio produtivo e cultural norte"-americano.\footnote{Caetano
  Veloso. ``Diferentemente dos americanos do norte'', op.~cit., p. 51.}

O que Caetano identifica aqui, com um olhar presciente, através desse
resíduo industrial colonial, matéria"-prima de suas canções, é algo muito
próximo do \emph{dumping} contemporâneo, que faz, de países como o
Brasil, o depositário do lixo ambiental dos países desenvolvidos. A
operação tropicalista consistirá precisamente em ``redimir'' esse
\emph{dumping} de produtos importados que proliferam com a abundância
conhecida. Há algo disso no procedimento que Gil aprendera com os
Beatles, de ``transformar alquimicamente lixo comercial em criação
inspiradora e livre, reforçando assim a autonomia dos criadores --- e dos
consumidores''.\footnote{Caetano Veloso. \emph{Verdade tropical},
  op.~cit., p. 187.} Pensando em termos da economia ecológica, i.~e.,
aquela expande o ciclo da produção de forma a incorporar os restos do
processo de produção dentro do ciclo produtivo, transformando lixo em
recurso, a operação se aproxima da reciclagem, ou do mais atual,
\emph{upcicling}, que descobre novas utilidades para os produtos gastos
e já utilizados.

Cabe então explicitar uma diferença significativa entre o pop e o
tropicalismo. Digamos que o pop replica a aparência da mercadoria,
produzindo, por assim dizer, uma aparência e duas essências distintas.
Como lembra Arthur Danto, a pergunta da arte pop poderia ser esta:
``dados dois objetos de aparência exatamente igual, como é possível que
um deles seja uma obra de arte e outro apenas um objeto
comum?''\footnote{Arthur Danto. \emph{Andy Warhol}. Trad. Ver Pereira.
  São Paulo: Cosac Naify, 2013, ebook.} Já no tropicalismo, não se trata
de produzir uma repetição literal, mas de redimi"-la, por uma alquimia
que resgata os harmônicos profundos da mercadoria, e a faz, por assim
dizer, sentir de novo.

O exemplo clássico dessa operação é ``Coração materno'', de
\emph{Tropicália ou Panis et circencis}, talvez a personagem mais
acabada de todas as tantas que formam a galeria dramática tropicalista.
A canção de 1937, retirada do repertório do cantor proto"-brega Vicente
Celestino, espécime do mau gosto, chama a atenção pelo tema grotesco,
entre o \emph{fait divers} e a lenda: um ingênuo e violento campônio
mata a mãe e extrai dela o seu coração, atendendo a um pedido perverso e
debochado de sua amada, a \emph{femme fatale} indiferente.\footnote{Cf.
  Diogo Araújo. ``A canção `Coração materno' no tropicalismo''. Disponível em:
  \textless{}\emph{https://bit.ly/2CjCqYL}\textgreater{}.}
A canção serviu de base para um filme estrelado também por Vicente
Celestino. O humor negro que salta aos olhos desde o original, sem dúvida
aproveitado pela repetição tropicalista, é a literalização do motivo
romântico do coração, central orgânica dos sentimentos, convertido em
dado literal, e cruel corte cirúrgico, extirpado no entanto não de si
próprio, como quereria uma fantasia autorreflexiva romântica, mas
edipianamente da mãe. Transformada pelo arranjo de Rogério Duprat e a
interpretação de Caetano, tornou"-se uma peça pop de primeira grandeza.
Uma comparação das duas versões é iluminadora. A interpretação de
Vicente Celestino decanta uma tradição operística presente na canção
popular brasileira, inscrita tanto na dívida para com o bel canto na
execução de Celestino, quanto no arranjo orquestral, que cita o
acompanhamento das árias de óperas de Giuseppe Verdi, com a sua
pontuação em pizzicato, sugerindo os passos do personagem no palco, e
pequenos desenhos em contratempo. O vibrato e a impostação exageradas da
voz dão destaque à dramaticidade da cena, centro grandiloquente do
melodrama, submetendo com mão de ferro o acompanhamento, cujos violinos
tensionados na região aguda imitam histericamente o canto.

Na execução de \emph{Panis et circenses}, por outro lado, o arranjo de
Duprat é colocado em primeiro plano, trazendo o fundo à superfície,
transformando o acompanhamento em um personagem musical, que contracena
com a voz despojada, praticamente sem vibrato de Caetano, que ressalta o
desenho emotivo da melodia. A contrário do que se diz, o esfriamento da
execução de Caetano, se comparada ao patético de Celestino, dá a ler o
sentimental, mas depurado em sobriedade tanto mais derramada porque
contida. A economia tradicional do canto operístico e da canção popular
em seu início, com seu regime próprio de trabalho, o protagonismo
suspenso da melodia apoiada sobre o pano de fundo da orquestra, é
substituída pela economia cinematográfica, com um arranjo instrumental
que tem um quê de trilha sonora, em sua imitação dos motivos narrativos.
A canção começa com três estrondos percussivos, sinal do início da
representação teatral ou de tiros de canhão, como o anúncio de mau
augúrio. As cordas em tons escuros explorando os graves preparam o
cenário de um drama, que pouco a pouco se fixa em um motivo pendular,
sinal da entrada da voz de Caetano. Os desenhos em contratempo se
desenvolvem criando autonomia com relação ao canto, os comentários da
letra se multiplicam: o harpejo pontua dramaticamente o verbo ``rasgar''
em ``rasga"-lhe o peito, o demônio'', seguido de escalas rápidas de
cordas, que indiciam o clímax. O baixo transmite o pulso, e as cordas em
motivos repetidos e percussivos que lembram a \emph{Sagração da
primavera} de Stravinsky. Na execução original, o acompanhamento é
redundante, imita o bel canto de Celestino, enquanto na de Duprat, o
acompanhamento divide democraticamente a cena com a voz de Caetano,
imitando a narrativa, como uma espécie de ``peça característica'',
procedimento de que o cinema se apropriou. O resultado é uma
intensificação afetiva, que aprofunda os relevos dramáticos da canção,
fazendo verdadeiramente o kitsch, o clichê apagado do resíduo cultural,
sentir, operando assim, de fato, a ``redenção'' dos clichês imundos,
oriundos das indústria de entretenimento de massas.

Entendamos essa dimensão afetiva e afirmativa da repetição tropicalista
inteiramente estranha à repetição pop. É o que pode ser depreendido de
um comentário de Gilberto Gil, no documentário de Andrucha Waddington,
\emph{Outros (doces) bárbaros} de 2004. Gil cita uma frase de Andy
Warhol: ``se ser pop é gostar das coisas, ele, Gil, é pop no sentido de
gostar de gostar das coisas''. A frase se presta a maus entendidos. Ela
vem sendo interpretada equivocadamente, me parece, como uma afirmação da
falta de seletividade e negação do juízo de gosto, quando se trata de
algo de muito diferente. Ao contrário, o que Gil defende não é a
permissão de gostar de tudo, um aval para o gosto qualquer, ou seja,
para a falta de gosto. Mas o prazer (gostar) de gostar, o que sublinha a
tonalidade afetiva da afirmação do gostar. O interessante, no entanto, é
que a frase de Warhol tem um sentido muito diferente do atribuído por
Gil, e é essa diferença que mostra a especificidade do pop tropicalista.
A frase está em uma entrevista de Warhol de novembro de 1963 a G.~R.
Swanson, intitulada, precisamente, ``O que é arte pop?''. Quando
perguntado sobre o sentido do pop, Warhol lembra uma frase de Brecht
segundo a qual todo mundo deveria pensar igual. E continua: ``Todo mundo
é parecido e age parecido, e estamos ficando cada vez mais assim. Acho
que todo mundo deveria ser uma máquina''. Quando então perguntado pelo
entrevistador: ``É isso que se trata na arte pop?'', Warhol aquiesce:
``Sim, é gostar de coisas'' (\emph{It's liking things}). ``E gostar de
coisas é ser uma máquina?'', insiste o entrevistador. ``Sim, porque você
faz a mesma coisa a cada vez. Você faz muitas vezes a mesma coisa''
(\emph{over and over again}).\footnote{Andy Warhol. \emph{I'll be your
  Mirror. The Selected Andy Warhol Interviews}. Kenneth Goldsmith (ed.).
  Nova York: Carroll \& Graf Publishers, 2004, p. 16.}

Gil enfatiza o aspecto afetuoso do ``gostar'', repetido reflexivamente
em um ``gostar de gostar'', enquanto Warhol sublinha o lado maquínico de
gostar de ``coisas'', de em suma \emph{se transformar em coisa} e, desta
forma, erradicar o afeto e o sujeito. Todo o contrário da negatividade
tropicalista atravessada de lado a lado pela afirmação do afeto.

Produto amenizado pelo afeto, pelo gostar de gostar, produto decalcado,
mas ainda assim, produto. A proposição tropicalista retém do pop a
hipótese de uma transgressão estética pela via da aceleração paródica da
produção. Caetano reflete sobre isso ao lançar à queima"-roupa a
pergunta, na apresentação da nova edição de \emph{Verdade tropical}:
``Serei um aceleracionista?''\footnote{Caetano Veloso. \emph{Verdade
  tropical}, op.~cit., p.~31.} Nas primeiras linhas da coletânea de
textos aceleracionistas de Mackay e Avanessian, eles definem
aceleracionismo da seguinte maneira:

\begin{quote}
Aceleracionismo {[}\ldots{}{]} é a insistência em que a única resposta
política radical ao capitalismo não é protestar, interromper ou
criticá"-lo, nem esperar a sua falência nas mãos de suas próprias
contradições, mas acelerar as suas tendências de desenraizamento,
alienação, decodificação e abstração. O termo foi introduzido em teoria
política para designar um certo alinhamento niilista do pensamento
filosófico com os excessos da cultura (ou anticultura) capitalista,
encarnado em escritos que buscavam uma imanência com esse processo de
alienação.\footnote{Robin Mackay \& Armen Avanessian (eds.).
  \emph{\#Accelerate\# The Accelerationist Reader}. Falmouth, Reino
  Unido: Urbanomic, 2014, p. 4.}
\end{quote}

As referências são as mais variadas, desde o ``Fragmento das máquinas''
de Marx, a alguns trechos do \emph{Anti"-Édipo} de Deleuze e Guattari, a
certas passagens de Baudrillard e Lyotard dos anos 1970. Cabe aqui,
portanto, a pergunta sobre o parentesco do programa tropicalista com a
afirmação marxista da lógica da circulação e multiplicação capitalistas
da mercadoria, como modo de sua reversão e regeneração vitais. Ponto
instigante, além disso: existe um aceleracionismo liberal e um
aceleracionismo de esquerda, um que se regozija na aceleração produtiva
para aumentar a produção, outro para destruí"-la. Mais uma equivocidade a
se inscrever aqui entre tantas.

Tales Ab'Sáber faz uma observação aguda sobre o fato de que a liberdade
do gozo da mercadoria proposto por ``Alegria, alegria'' cria ``uma
ponte estética real para o mercado'', que a esquerda teria levado trinta
anos para entender. Foi apenas com governo Lula, escreve Tales, que ela
finalmente se reconciliou com o gesto de pura fruição
mercadológica.\footnote{Tales Ab'Sáber. \emph{Ensaio,
  fragmento. 205 apontamentos de um ano}, op.~cit., p.~40.} A sugestão é
que existe um parentesco entre o programa tropicalista de repetição
enternecida e afirmativa do corpo, mercadorias e objetos, e o projeto
trabalhista da extensão do consumo às massas pobres, numa visão paródica
do projeto industrial petista, ao que acrescentaríamos, em sua devida e
abissal cegueira ambiental. Esse mesmo projeto de conciliação
generalizada entre ricos e pobres pela via do trabalho e do consumo
industriais, cujo fim e fracasso terminais assistimos com a prisão de
Lula, em 7 de abril de 2018, consumado com a eleição, em outubro de
2019, de Jair Messias Bolsonaro. Ora, o limite evidente desse sonho
conciliatório, o seu ponto cego e impensável pela lógica do capitalismo
petista é a catástrofe climática, e a consubstancial existência dos
povos da floresta, no momento exato em que a propriedade pública dos
territórios indígenas corre o risco de ser negada com a permissão da
entrada do agro"-negócio nessas terras.

É isso que chama atenção na canção mais recente ``O império da lei'',
uma canção com motivo ecológico. Nela estão inscritos, num mesmo nó, a
questão climática e a questão ameríndia. A canção metaboliza algo do
filme \emph{Receberia as piores notícias dos seus lindos lábios} (2011)
de Beto Brant, baseado na novela epônima de Marçal Aquino, com Camila
Pitanga no papel principal.


%afastar um pouco da mancha?
\begin{quote}
O império da lei há de chegar no coração do Pará\\
O império da lei há de chegar no coração do Pará\\
O império da lei há de chegar lá\\
O império da lei há de chegar lá\\
Quem matou meu amor tem que pagar\\
E ainda mais quem mandou matar\\
Ter o olho no olho do jaguar\\
Virar jaguar\footnote{Disponível em: \textless{}\emph{https://bit.ly/36gz7iJ}\textgreater{}.}
\end{quote}

A canção se equilibra entre duas fórmulas inconciliáveis: de um lado, um
clichê liberal, ``o império da lei'', sublinhando a necessidade de que
ela, a lei, chegue ao ``coração do Pará'', e puna quem ``matou meu
amor''; e de outro, uma fórmula retirada da obra de Eduardo Viveiros de
Castro, ``ter o olho no olho do jaguar/ virar jaguar''.\footnote{O
  percurso dessa canção nesse texto é sinuoso e merece ser sucintamente
  relatado aqui. A referência me veio de Débora Danowski, que a ouvira
  citada em uma palestra de Pedro Duarte. No ocasião fiquei em dúvida
  se a frase sobre o ``império da lei'' era irônica, ao afirmar o
  pior, no velho estilo tropicalista, ou se Caetano a enunciava \emph{a
  sério}. Eu apostava na interpretação tropicalista, e como tal a inseri
  no meu livro, \emph{O tropo tropicalista} (Rio de Janeiro: Circuito, 2017). José Miguel Wisnik, em uma
  conversa telefônica, me convenceu do contrário. Resultado: tirei a
  referência do livro. Como eu permanecesse em dúvida, e ainda por cima
  quiséssemos ter certeza sobre a referência a Eduardo Viveiros de
  Castro na canção, eis que minha companheira, Cecilia Cavalieiri, em
  intervenção salvadora, resolveu, por iniciativa própria, escrever a
  Caetano, sem revelar quem era. Minha leitura aqui está pautada por
  essa troca de mensagens.} É preciso punir os assassinatos de
ambientalistas da Amazônia, como Chico Mendes, e a irmã Dorothy Stang ---
escreve Caetano em um e"-mail\footnote{Caetano Veloso. E"-mail a Cecilia
  Cavalieri de 15/02/2018.} --- encarnados no livro e na adaptação
cinematográfica pelo personagem do líder religioso e denunciador da
catástrofe ecológica, que acaba assassinado. E isso só ocorrerá no
futuro que a canção descortina e aponta messianicamente: ``O império da
lei há de chegar lá''. O império da lei, o estado de direito, há de
chegar na Amazônia, no interior do Brasil, e fazer com que ``quem matou
meu amor'' pague pelo crime cometido. Ao mesmo tempo que a canção sugere
uma outra solução, tensionada com esta: a metamorfose xamânica de
``devir jaguar'', para citar a fórmula deleuziana da ``diferonça'' de
Viveiros de Castro. A canção se situa então literalmente \emph{entre} o
liberalismo que defende a \emph{rule of law} --- essa a tradução oficial
da expressão ``estado de direito'', que Caetano diz visar com o clichê
``império da lei'' --- de Eduardo Giannetti, que foi quem, diga"-se de
passagem, indicou a Caetano a leitura dos livros de Viveiros de
Castro\footnote{Informação contida no mesmo e"-mail.} --- e a perspectiva,
digamos, anarquista (ou anárquica) e etnográfica que pauta o programa
etnográfico perspectivista. Para este existe uma força essencialmente
``contra o Estado'' --- para lembrar a expressão de Pierre Clastres ---
nos povos ameríndios, uma heterogenidade minoritária, que cabe mal no
molde estatal, que resiste a ele, se pauta por outras leis que não as
dele, embora viva em terras que pertencem legalmente a ele (o estado).
Do outro lado, a democracia liberal, com sua afirmação do respeito às
leis (o estado de direito), reduzido ao respeito mínimo do estado
mínimo, que se resumiria, segundo a doutrina dos pesos e contrapesos, ao
respeito da autonomia dos diversos poderes. Observe"-se \emph{en passant} uma
posição distinta com relação ao estado: o poder da maioria, como
entendemos a democracia representativa, e o poder da minoria, como quer
a etnografia indianista. A contradição se dá, assim, entre uma posição
``menor'', anarquista, anti"-estado, cujo direito territorial depende da
viligância estatal; e uma posição liberal que defende a redução do
estado (ou o estado mínimo), reduzido tão somente ao puro respeito da
lei. A antinomia vem, além disso, de dois Eduardos, nome próprio de
origem anglo"-saxã, \emph{Éadweard}, composto de \emph{ead},
``riqueza, fortuna, próspero'', e \emph{weard}, ``guardião, protetor''.
Cada Eduardo parece então ser guardião de riquezas bastante diferentes,
de nações diferentes. Gianetti é o guardião da riqueza do estado e do
direito, e Viveiros de Castro, é o guardião do direito das nações
ameríndias.

De um lado, portanto, a força imperial do estado, origem ao mesmo tempo
da lei e da ilegalidade, no limite, assassino; de outro, a diferença dos
povos ameríndios e seu devir natureza, com a identificação entre
denúncia da catástrofe climática e existência índia. A canção contém um
dilema que ela não resolve, dilema este, formulado, pelo mesmo
Giannetti, no Programa Roda Viva de setembro de 1996 de Caetano Veloso.
Eis a pergunta: ``será possível o Brasil conquistar a civilização e, ao
mesmo tempo, não perder a sua alma iorubá, selvagem, índia?''\footnote{Disponível em: \textless{}\emph{https://bit.ly/36L5fLp}\textgreater{}.}
O que a canção faz é justapor uma dupla provocação a ambos os lados da
equação, incomodar, e produzir, mais uma vez, e sempre, o incômodo de
uma equivocidade que põe em movimento os dois lados da antinomia que o
move, por meio do salto da diferonça, que ``olho no olho'', resolve a
parada ao virar onça.

Observemos, para terminar, que não se trata mais aqui da repetição pop
tropicalista temperada pela afirmação afetiva, mas de dois sentidos
justapostos, tensionados, que esticam ao máximo a corda da antinomia.
Não há fusão utópica, nem síntese possível: a canção mantém
dialeticamente os dois polos, o império da lei liberal, a aposta na
função estatal da lei que pune e produz justiça, e o devir jaguar
xamânico, que abole o estado, dissolve"-o por dentro e o torna menor. O
que a canção faz é esboçar um mundo impossível em que estas duas
posições pudessem coexistir.

\section{Referências}

\begin{Parskip}
\versal{AB'SÁBER}, Tales. \emph{Ensaio, fragmento. 205 apontamentos de um ano}. São Paulo: Editora 34, 2014.

\versal{ARAÚJO}, Diogo. ``A canção `Coração materno' no tropicalismo''. Disponível em: \textless{}\emph{https://bit.ly/2CjCqYL}\textgreater{}.

\versal{CHAKRABARTY}, Dipesh. ``O clima da história: quatro teses''. Trad. Denise Bottmann et~al. \emph{Sopro}, 92, 07/2013.

\versal{CORRÊA}, Zé Celso Martinez. \emph{Primeiro Ato. Cadernos, depoimentos, entrevistas (1958-1974)}. Ana Helena Carmargo (org.). São Paulo: Editora 34, 1998.

\versal{DANTO}, Arthur. \emph{Andy Warhol}. Trad. Ver Pereira. São Paulo: Cosac Naify, 2013.

\versal{GRANEL}, Gérard. \emph{Traditionis traditio}. Paris: Gallimard, 1972.

\versal{HARAWAY}, Donna. ``Anthropocene, Capitalocene, Plantationocene, Chthulucene:
  Making Kin''. \emph{Environmental Humanities}, vol. 6, 2015.

\versal{MACKAY}, Robin \& \versal{AVANESSIAN}, Armen (eds.). \emph{\#Accelerate\# The Accelerationist Reader}. Falmouth, Reino Unido: Urbanomic, 2014.

\versal{OITICICA}, Hélio. \emph{Aspiro ao grande labirinto}. Rio de Janeiro: Rocco, 1986.

\versal{ROCHA}, Glauber. \emph{Revolução do Cinema Novo}. São Paulo: Cosac Naify, 2004.

\versal{SARTRE}, Jean"-Paul. \emph{L'être et le néant}. Paris: Gallimard, 1943.

\versal{VELOSO}, Caetano. \emph{Verdade tropical}. São Paulo: Companhia das Letras, 2017.

\_\_\_\_\_\_. \emph{O mundo não é chato}. São Paulo: Companhia das Letras, 2005.

\versal{WISNIK}, Guilherme. \emph{Folha explica Caetano}. São Paulo: Publifolha, 2005.

\versal{WARHOL}, Andy. \emph{I'll be your Mirror. The Selected Andy Warhol Interviews}. Kenneth Goldsmith (ed.). Nova York: Carroll \& Graf Publishers, 2004.
\end{Parskip}

\chapter*{Exumando o subdesenvolvimento}
\addcontentsline{toc}{chapter}{Exumando o subdesenvolvimento, \emph{por Diego Viana}}

\begin{flushright}
\emph{Diego Viana}
\end{flushright}

Termo perigoso de recuperar. É como invocar forças ocultas, um fantasma
que há muito deixou de rondar o mundo (e as economias). Em 1967,
\emph{subdesenvolvimento} era um conceito em pleno viço, disparado a
torto e a direito. Mas o meio século que se seguiu lhe foi cruel. Até
seu contraponto, o \emph{desenvolvimento}, foi sequestrado por programas
de governo agressivos, autoritários, destrutivos. O subdesenvolvimento
desapareceu do vocabulário e só emerge, eventualmente, com sentido
pejorativo. O conceito foi esconjurado e enterrado. Em economia, virou
nota de rodapé.

Recuperar a história do subdesenvolvimento, da ascensão à esconjuração,
lança uma estranha luz sobre o destino do próprio desenvolvimento, hoje
sufocado pela assimilação ao industrialismo anacrônico e à supressão de
modos de vida tradicionais. O destino da noção de subdesenvolvimento
permite perguntar: o que fazer do desenvolvimento?

O subdesenvolvimento exumado revela uma estranha potência. Embora tenha
se tornado um jargão batido, o termo parece dotado da capacidade de
embaralhar lógicas. A começar pela teoria econômica, em que a ideia de
que algo possa \emph{ser} subdesenvolvido contém uma dose de afirmação
que destoa clamorosamente da arquitetura em que se insere a noção
ortodoxa de desenvolvimento.

Classicamente, qualquer estado que não seja o pleno desenvolvimento das
forças produtivas é um \emph{defeito} de desenvolvimento, que acomete
toda economia incapaz de aceder às tecnologias de ponta. É um defeito
\emph{do capital}, que tende a ser consertado na medida em que o capital
plenamente desenvolvido investe nas regiões atrasadas, elevando sua
produtividade. O capital que sobra no centro, onde o retorno é menor
porque a concorrência é acirrada, promove nas periferias, onde o retorno
é maior, o fechamento do vão de renda.

Esse raciocínio remete às emanações de Plotino: o capital é como um
Absoluto em relação ao qual os demais entes acumulam imperfeição; é das
emanações do capital (plenamente desenvolvido) que as economias
atrasadas obtêm sua parte no ser, isto é, na produtividade.

A afirmação da positividade do subdesenvolvimento, como um verdadeiro
estado de coisas, atinge essa concepção com uma violência que, no
decurso das teorias do crescimento, no contexto neoclássico em economia,
foi explosiva. Afirmar a relação necessária das duas condições,
desenvolvido e subdesenvolvido, inviabiliza esse quadro, que passa a
parecer ingênuo ou mistificador: não existe ``recuperação do atraso'',
quando a condição não é de atraso, mas de ocupação de um espaço real,
positivo, embora subordinado e empobrecido.

Não é um acaso que, nos meios ortodoxos, falar em subdesenvolvimento
seja anátema \emph{até hoje}. Uma anedota esclarecedora a esse respeito
é contada por Celso Furtado em um de seus livros
autobiográficos:\footnote{Celso Furtado. ``A fantasia organizada''. In:
  \emph{Obra autobiográfica}. São Paulo: Companha das Letras, 2014.} o
esforço do famoso macroeconomista Jacob Viner para implodir as teses de
Raúl Prebisch mal elas foram publicadas, em 1950.\footnote{Raúl Prebisch.
  ``Interpretação do processo de desenvolvimento econômico''.
  \emph{Revista Brasileira de Economia}, v.~5, n.~1, 1951.} Viner se deu ao
trabalho de voar até o Brasil para demonstrar a perfeição formal dos
modelos de desenvolvimento, terminando por dizer que não tem sentido
falar em ``país subdesenvolvido'': não há \emph{positividade} no estado
que está aquém da fronteira tecnológica, ele é apenas imperfeição do
\emph{ser} desenvolvido, distância das emanações do capital.

Desde então, a perspectiva de Viner foi vencedora. Imprimiu"-se na
linguagem (formal, corrente e midiática) o apagamento completo dessa
positividade. Primeiro se adotou ``\emph{developing}'' no lugar de
``\emph{underdeveloped}'', expressando uma teleologia, onde havia uma
relação topológica inerente ao subdesenvolvimento, espaço ocupado e vivo
apesar da subordinação. A consagração dessa estratégia está no passo
seguinte. O ``\emph{developing}'' vira ``\emph{emerging}'': esses países
(digo, mercados) ``emergem'' de onde? Do pântano da negatividade, do
não"-ser que é não"-lugar de impotência. A única potência sendo a
produtividade do capital, esses são lugares que só agora são insuflados
com essa vida.

\section{Política e estética}

A força desarranjadora da noção de subdesenvolvimento transborda o
econômico (cujas fronteiras são arbitrárias). Provavelmente a primeira
teoria a reconhecer um caráter positivo na conjuntura dos países
subordinados foi a do desenvolvimento desigual e combinado de Trótski,
pensada entre 1906 (\emph{Balanços e perspectivas}) e 1930
(\emph{História da Revolução Russa}).

Ainda é uma leitura linear, já que a desigualdade é contrabalançada
pelos ``saltos'' dos retardatários. Trótski conclui que a derrubada
simultânea dos resquícios feudais e do capitalismo são obra da vanguarda
proletária em revolução permanente. Mas a estrutura marxista da economia
mundial é alterada para acrescentar particularidades locais, pipocando
selvagemente no sistema. A realidade dos países periféricos, que depois
seria denominada subdesenvolvimento, é efetiva, cria formas e devires,
reconfigura a história.

Nada parecido com o pântano de impotência do qual um país (digo,
mercado) poderia emergir. O processo histórico, ainda que leve à
superação do modo de produção centrado na valorização do valor, tem uma
indeterminação tal, que assume diversas formas, monstruosas do
ponto de vista do materialismo histórico. O caráter político do processo
traz consigo a constatação de que as formas do subdesenvolvimento são
múltiplas e não estão contidas na lógica do sistema produtivo.

Pulando algumas décadas e saindo da práxis política para a prática
artística, a questão que abre as reflexões de Ferreira Gullar em
\emph{Vanguarda e subdesenvolvimento} é semelhante à que animava
Trótski: se o conceito de vanguarda válido no ``primeiro'' mundo é
aplicável ao ``terceiro'', num país subdesenvolvido como o Brasil. O que
vale para a economia e a política vale também para as artes: se a
relação desse país com o que se espera ser desenvolvimento é fundada na
imperfeição perante um modo de ``ser'' tomado por absoluto, as
possibilidades da vanguarda são umas; se o subdesenvolvimento for uma
posição positiva, em vínculo íntimo e conflituoso com os espaços
desenvolvidos, essas possibilidades são outras. O conceito de
subdesenvolvimento reafirma seu poder performático.

A adoção do termo ``subdesenvolvimento'', para além da intenção dos
primeiros teóricos, foi uma tomada de posição com alcance considerável.
Seu abandono, como se deu historicamente, \emph{idem}. Daí por que a
gênese, na \textsc{cepal}, da \emph{teoria} do subdesenvolvimento causou fortes
reações dentro e fora da profissão de economista. No sentido oposto ao
de Viner, Getúlio Vargas imediatamente identificou nas ideias de
Prebisch e Furtado um arcabouço intelectual para suas políticas de
industrialização dirigida. Daí por diante, toda uma tradição de política
econômica assumiria a alcunha, hoje pejorativa, de desenvolvimentismo.

Na superfície, o que se deduz da teoria do subdesenvolvimento, no plano
econômico, é que entre dois estados positivos, um subordinado e outro
pleno, existe uma transformação possível; não é a revolução permanente,
mas soma ao caráter afirmativo do subdesenvolvimento uma dimensão
performática dos atores, notadamente o Estado, com o planejamento --- mas
só como promotor da concertação de setores da sociedade civil,
principalmente o empresariado industrial.

E como é esse estado pleno? É igual ao dos países centrais? É unívoco?
De onde vem o impulso para a transformação? Quem a conduz? Quem define
seus traços? Como reagem os setores que se beneficiam do
subdesenvolvimento como espaço positivo? Se o subdesenvolvimento é
sistemático, por que se deveria acreditar que os sujeitos da
transformação, setor industrial e classes médias, vão levá"-lo a termo,
em vez de contentar"-se com uma posição confortável, embora subalterna,
na condição subdesenvolvida?

Muitas dessas questões aparecem nos textos que, depois da queda de Jango
e principalmente depois da crise da dívida, tentam entender o que deu
errado. A começar pelo próprio Furtado. O cerne da questão não está em
descrever um estado de coisas. Vive"-se, age"-se, decide"-se, segundo o
pêndulo entre a condição faltosa e a positiva; ou mesmo mais além: o
princípio de que o próprio desenvolvimento é uma ilusão, ou pior, um
impostura. O cerne está na atitude perante a condição subdesenvolvida.

\section{Arte, fome e cooptação}

No campo estético, as questões são semelhantes. Ferreira Gullar traça as
linhas gerais do impasse, nesse período de industrialização e
urbanização aceleradas. Os problemas reverberam muito do que se discutia
nos países centrais: ``onde está o caráter revolucionário? Na adoção de
formas ou nos temas?'' Mas quem primeiro entendeu o aspecto explosivo do
subdesenvolvimento foi Glauber Rocha. Na ``Eztetyka da fome'' (1965),
o cineasta preenche com uma potência ativa os traços
despotencializadores da condição concreta do subdesenvolvimento, aqueles
que a ortodoxia vê como únicos. Ele supera o problema da aplicação do
conceito de vanguarda ao denunciar a estreiteza da comunicação entre a
arte latino"-americana e a europeia, em que só circulam ``mentiras
elaboradas da verdade'', ``exotismos formais que vulgarizam problemas
sociais''.

Glauber fala de uma arte histérica e impotente, que busca dialogar com o
colonizador mas só lhe inspira ``sentimentos humanitários''. Daí a
abertura para a estética da fome, que começa mas não termina na
exposição do miserabilismo, traço que caracteriza o período Jânio e
Jango, como ele diz: ``o período das grandes crises de consciência e de
rebeldia, de agitação e revolução que culminou no golpe de abril''. E
depois do golpe triunfa o cinema ``digestivo'', ao gosto de Carlos Lacerda
e das oligarquias. É a cultura feita para as personagens que C. Wright
Mills descreve: os ``círculos muito pequenos de classes médias
rudimentares'', os ``desventurados eleitos {[}que{]} formam o único
público disponível para os produtos e serviços culturais'' nos países
subdesenvolvidos.\footnote{Charles Wright Mills. ``The Cultural Apparatus''.
In: \emph{The Politics of Truth}. \emph{Selected
  Writings of C. Wright Mills}. John Summers (ed.). Oxford: Oxford University Press, 2008.}

Glauber via claramente a violência inerente a todo movimento de uma
sociedade como a brasileira. Algo que só encontramos em Furtado após o
retorno do exílio. Só nesse momento ele escreve: primeiro
(1980),\footnote{Celso Furtado. \emph{Pequena introdução ao desenvolvimento:
  enfoque interdisciplinar}. Rio de Janeiro: Companhia Editora Nacional,
  1980.} que o desenvolvimento tem uma terceira dimensão, além do
avanço técnico do sistema produtivo e da satisfação de necessidades
básicas, qual seja: a ambição e a ``preparação ideológica'' da sociedade,
particularmente os grupos dominantes; segundo (1984), que uma das
causas do fracasso do desenvolvimento brasileiro no século \versal{XX}, após 50
anos de industrialização, é ``um impasse''\footnote{Celso Furtado. ``Que
  somos?''. In: \emph{Arquivos Celso Furtado 5: ensaios sobre
  cultura e o Ministério da Cultura.} Rio de Janeiro: Contraponto, 2012.}
\emph{necessário} na sociedade brasileira, porque os ganhos de
produtividade foram dirigidos ao consumo das classes altas; terceiro
(2004), que ``a superação do subdesenvolvimento'' é um \emph{processo
político}, e o ``avanço social'' é fruto de ``pressões políticas da
população''.\footnote{Celso Furtado. ``O verdadeiro desenvolvimento''.
  In: \emph{Essencial Celso Furtado}. São Paulo: Penguin Companhia, 2013.}

Vinte anos antes, em pleno período de desenvolvimento, em que deveria
haver a passagem entre dois estados positivos ou da imperfeição à
perfeição, Glauber vai ao cerne do problema político: uma estética da
fome é uma estética da violência, violência do esfomeado, retirante,
favelizado. Na ``Eztetyka do sonho'' (1971), o alvo de Glauber se
evidencia: ele quer esconjurar a oficialização do artista, cooptado para
participar da acomodação com o poder oligárquico, beneficiário da
condição de colonizado, subdesenvolvido.

O cineasta percebe que o subdesenvolvimento é perpetuado pela cooptação
das classes médias, que se acomodam às condições postas por uma elite
inserida no sistema global enquanto elite periférica. Essa acomodação
caracteriza o subdesenvolvimento na teoria cepalina, aprofundada pela
teoria da dependência nos anos 1970. Na oposição pendular entre acomodação
e violência, ou antes, entre a violência oficial, necessária à
acomodação, e a violência da fome quando o famélico age, opera a
dissonância inerente ao processo de desenvolvimento no Brasil do século
\versal{XX}, que também é a dissonância que a teoria do subdesenvolvimento não
consegue resolver. Aqui se situa o impensado do desenvolvimentismo.

\section{Varguismo e conciliação}

Acomodação e conciliação de classes: ressurgem assim os aspectos
econômico e político do problema, vistos antes pelo artista. Acomodação
e conciliação são o cerne da lógica varguista que rege o século \versal{XX}
brasileiro. A revolução de 30 foi acomodadora; a ditadura até 1945
buscou sempre acomodar demandas de grupos industriais e urbanos
nascentes ao poder das oligarquias; o período presidencial de Vargas
começou a não conseguir mais manter a conciliação e inaugurou uma era de
crises. Juscelino se sustentou na conciliação. Jango era conciliador,
até perder o apoio do \versal{PSD}. O plano de estabilização de Furtado, quando
ministro do Planejamento, era conciliador, mas foi bombardeado pela
esquerda e pela direita, que estavam radicalizadas.

A ditadura se serviu de estratégia semelhante. Na economia, sobretudo do
ponto de vista da relação entre Estado e capital, a posteridade a olha
com foco no período de Médici e Geisel. É o período do dito milagre
econômico, em que a ditadura buscou forçar o crescimento industrial e a
ocupação da Amazônia. A ditadura é associada a um Estado tão interventor
que, ao deixar o Senado em 1994, Fernando Henrique discursa associando à
ditadura a lógica varguista do Estado indutor de crescimento. Para \versal{FHC},
a redemocratização \emph{ainda} não tinha eliminado esse resquício, o
varguismo.

Mas o primeiro período da ditadura se desenrolou sob outro signo. O
projeto econômico era liberalizante, conduzido por Roberto Campos. O
varguismo era o inimigo dos conspiradores de 64, mas levou só cinco anos
para que o perfil econômico do regime se transformasse. A linha"-dura
tinha um projeto de Brasil grande, mas existia um arcabouço de
conciliação entre grupos dirigentes e interesses econômicos ao qual
mesmo ela subscreveu.

Voltemos à dissonância inerente ao subdesenvolvimento: violentar"-se pela
acomodação ou reconhecer a violência da ação dos famélicos. Na
perspectiva do poder, outra dissonância age como contraponto: ou a
acomodação com essas potências emergentes, indústria, classes médias e
baixas urbanas, ou a supressão pura e simples de qualquer força que
possa ameaçar a distribuição oligárquica do poder \emph{de facto}. %facto mesmo?

Essas duas dissonâncias explicam muito do que move os movimentos
políticos da história recente do Brasil. É notável que a ditadura
instaurada para reverter o varguismo seja associada, 30 anos depois, ao
próprio varguismo. E também que, mesmo após as privatizações dos anos 1990
e a Carta ao Povo Brasileiro, discursos pela derrubada de Dilma Rousseff
evoquem uma nova ruptura com o desenvolvimentismo\ldots{} herdado do
varguismo!

Termo guarda"-chuva para o desenvolvimentismo, o varguismo expressa algo
ao mesmo tempo \emph{insuportável} e \emph{insuperável} para as classes
dominantes, que transparece em muitos dos nossos conflitos políticos,
econômicos, sociais e estéticos. A dissonância, da perspectiva das
oligarquias, implica que, para os grupos de poder históricos ---
latifúndio e estamentos burocráticos ---, tanto o desenvolvimento, a
industrialização incentivada pelo Estado, quanto as vanguardas
estéticas, expressão da classe média em expansão, são ameaças em
potencial e, por isso, inimigos.

Daí a propensão a suprimi"-los ou torná"-los inviáveis, como se suprimia a
cultura popular no Império e na Primeira República, e como em ambos
esses períodos a política econômica inviabilizou que a industrialização
ganhasse impulso. Mas reencontramos a conciliação: as potências
produtivas e as iniciativas da cultura são administráveis, uma vez
desidratadas. Ou, diria Glauber, que a arte se torna digestiva, e que o
setor industrial absorve os interesses da oligarquia, que o movimento
operário se acomoda aos interesses do patronato, que as classes médias
se contentam com vias muito controladas de manutenção de seu estatuto e
ainda mais controladas de ascensão.

É um jogo perigoso. Qualquer desses elementos acomodados pode receber um
impulso que lhe dê ganas de assumir uma dinâmica própria. Em qualquer
crise, pode ser que as reivindicações ultrapassem o limite do aceitável.
Foi o que aconteceu em 64. Mas também, nos momentos de conflito
distributivo como a crise da dívida e da hiperinflação, nos anos 1980, a
conciliação pode se revelar inviável. Essa tensão reintroduz a
perspectiva da supressão, mas também a das várias formas de violência
ligadas às várias formas da fome, assim como as invenções possíveis. De
onde vem essa possibilidade das invenções, em períodos de violência,
conflito e, historicamente, derrota? Parece vir do espaço que se abre
quando falham os mecanismos da acomodação.

A Tropicália, como o Cinema Novo e outras vanguardas estéticas, surge
nesse momento que está a cavalo entre a derrota (golpe de abril) e a
janela de possibilidade criativa. Foi tratada como ``explosão'' por Celso
Favaretto\footnote{Celso Favaretto. \emph{Tropicália: alegria, alegoria}.
  São Paulo: Ateliê Editorial, 1996.} e ``susto'' por Heloísa
Buarque.\footnote{Heloísa Buarque de Hollanda. \emph{Impressões de
  viagem. \versal{CPC}, vanguarda, desbunde.} São Paulo: Brasiliense, 1979.}
Essas descrições pelo afeto ganham em sentido se contrapostas ao
sentimento do período seguinte: derrota, desilusão, frustração, medo. A
desilusão, fracasso, derrota, parecem conduzir a arte na direção da
ligação direta ao real, no sentido dos determinantes sociais, políticos,
históricos, econômicos.

Se crise e criação estão ligadas à dissonância da oligarquia e do
desenvolvimentismo em forma brasileira, varguista, pode"-se interrogar
também sua relação com o subdesenvolvimento \emph{como afirmação},
\emph{locus} no sistema global, perante uma conjuntura em que a
transformação possível, ``processo de desenvolvimento'', desanda, como
iniciativa política, econômica e, consequentemente, estética. Aí vemos o
grito, a violência, da recusa à acomodação que se mostrou inviável ---
quando Caetano Veloso diz que se nega a ``folclorizar'' \emph{seu}
subdesenvolvimento\footnote{Carlos Acuio. ``Por que canta Caetano Veloso''.
  Revista \emph{Manchete}, 16/12/1967.} --- ou Glauber, novamente:
``tropicalismo é aceitação, ascensão do subdesenvolvimento''.\footnote{Glauber Rocha. ``Tropicalismo, antropologia, mito, ideograma''. In:
  \emph{Revolução do Cinema Novo}. São Paulo: Cosac Naify, 2004.}

Glauber enxerga a violência contida no caráter positivo do
subdesenvolvimento, como os cepalinos viram seu caráter sistemático. O
que fez a descoberta do subdesenvolvimento? Diz Glauber que ela implodiu
o nacionalismo utópico: é impossível ser nacionalista se reconhecemos o
subdesenvolvimento como condição positiva. A implosão tem duas fases:
primeiro com a descoberta do subdesenvolvimento econômico, os termos de
troca desiguais (Prebisch). Depois, entendendo que o subdesenvolvimento
é integral. Glauber se aproxima das dificuldades que Furtado enfrenta:
ao dizer que o cinema brasileiro, tendo constatado a integralidade e
positividade da posição subdesenvolvida, busca superá"-la também
integralmente: ``superar o subdesenvolvimento com os meios do
subdesenvolvimento''.

Para Glauber, na arte isto se traduz em antropofagia e Tropicália. Na
economia, na política, o que seria ``superar o subdesenvolvimento com os
meios do subdesenvolvimento''? A substituição de importações e o
planejamento parecem mais próximos de tentativas de superação do
subdesenvolvimento com meios tradicionais. Neste ponto, não se
diferenciam do projeto de Brasil grande. Têm como horizonte uma
compreensão de desenvolvimento semelhante à ortodoxa, para a qual basta
abrir"-se para investimentos e o capital dá vida como a seiva no xilema.

A teoria é delimitada pelos conceitos habituais: fatores de produção,
excedente, acesso a mercados, penúria de capital. A prática vai um pouco
além, graças ao desenvolvimento local (Sudene) e aos esforços de Darcy
Ribeiro: \versal{U}n\versal{B}, reformas de base e, mais tarde, \versal{CIEP}s. Iniciativas que
foram ou abortadas ou desinfladas depois, em nome da acomodação e da
conciliação.

\section{Os meios do subdesenvolvimento}

Finda a ditadura, Furtado manifesta frustração com o fracasso do
desenvolvimento brasileiro, passando a refletir sobre o movimento
político e a formação cultural das massas. Há uma mudança semântica
sutil quando ele se pergunta, sobre a natureza do subdesenvolvimento, em
1992: ``que é o nosso subdesenvolvimento senão o saldo negativo que nos
deixaram repetidos soçobros na decadência?''.\footnote{Celso Furtado. ``Que Somos?'',
  \emph{op.~cit}.} São reflexões do pensador do subdesenvolvimento, %numero de pagina?
feitas após o impulso de desenvolvimento, na redemocratização, na era
neoliberal e, sobretudo, no contexto em que o próprio termo
\emph{subdesenvolvimento} foi abandonado, tornou"-se anátema, deu lugar
ao ``em desenvolvimento'' e ao ``emergente''.

Os escritos dos últimos vinte anos de Furtado recendem a uma sensação de
derrota que se tornou corrente no Brasil. Ele já chamava de derrota a
virada cambial da segunda metade dos anos 1940, que matou a
industrialização por substituição de importações que tivera lugar
durante a guerra. Analistas do Cinema Novo, da Tropicália e de outros
movimentos dos anos 1960 associam a explosão criativa do período aos
efeitos da derrota sobre o desejo e a inventividade. O mesmo vale para
movimentos dos anos 1970, como o desbunde.

Interrogar a frustração de Furtado pode iluminar a nossa. A sensação de
fracasso que paira sobre nós é semelhante à do economista. Mas há uma
diferença sensível: deixamos de afirmar o subdesenvolvimento,
abandonamos o gesto orgulhoso como a recusa de Caetano a
\emph{folclorizar} o subdesenvolvimento. Abdicamos de nos pautar pela
violência de esfomeado. Incorporamos o discurso do ``emergente'', como se,
efetivamente, a seiva do nosso xilema fosse uma vida presenteada pelo
capital global. Perdemos de vista aquilo que Trótski identificou: a
práxis opera sobre as formas concretas de organização local, perante o
todo do sistema. O que mantém o imobilismo é a posição confortável das
oligarquias, inseridas no sistema global.

Como não pensar na ambiguidade fundamental do período petista, em sua
relação subordinada com o mercado chinês, mas mantendo o discurso
desenvolvimentista, quando lemos o seguinte trecho de \emph{Brasil, a
construção interrompida}: ao longo de todo o período do acelerado
crescimento no Brasil, ``o que permitia aos brasileiros conviver com as
gritantes injustiças sociais era o intenso dinamismo da economia'', o que
parece a Furtado um preço ``exorbitantemente elevado'' para o
desenvolvimento, explicado pela ``obstinada resistência da aliança de
interesses oligárquicos à introdução de reformas modernizadoras das
estruturas''.\footnote{Celso Furtado. \emph{Brasil, a construção interrompida}. São Paulo: Paz e Terra, 1992.}

Mas esse período continha um consenso, lembra o economista: ``interromper
o crescimento econômico não contribuiria senão para agravar os problemas
sociais''. Essa é uma expressão do jogo perigoso, a conciliação em tempos
de movimento.

Vivemos um pastiche dessa acomodação no período petista, com duas
diferenças: primeiro, o discurso do combate às injustiças teve peso
determinante. Segundo, a perspectiva de agir \emph{a partir} do
subdesenvolvimento desapareceu. Introjetamos a noção de ``mercado
emergente'', consagrada na fórmula dos \versal{BRICS}, reunião de ``potências
emergentes''.

Se o subdesenvolvimento se tornou o cadáver que a memória de 1967 nos
faz exumar, o \emph{desenvolvimento} se tornou simulacro, empregado para
descrever o oposto do que se realizava. O termo que surgiu para designar
o confronto com as oligarquias, a emergência de classes médias ativas, a
incorporação de novas habilidades e técnicas, justificou o reforço do
poder oligárquico, a conformação da classe média, a ampliação brutal do
extrativismo.

O sentimento de derrota é escorregadio. É fácil confundir a derrota
informal, mas efetiva, da multiplicidade dos impulsos que viceja no
campo social com a derrota formal, mas não tão efetiva, do partido que
detinha o poder. Com a mudança abrupta de governo, abriu"-se o caminho da
enxurrada obscurantista que atravessamos, operada por forças que, em boa
parte, eram aliadas do partido hoje derrotado (as demais se acomodavam
como adversárias). São duas derrotas. Uma, política, o rearranjo da
acomodação que não deve nada a Vargas, e que Glauber esconjurou. A
outra, social, dos impulsos e inventividades que não foram além da
conciliação. Novamente, vivemos a concretização do poder de supressão
das oligarquias, dormente quando a acomodação opera sem fricções.

Se há algo a resgatar no conceito de subdesenvolvimento, é primeiro seu
caráter afirmativo, o reconhecimento de uma particularidade dentro de um
âmbito sistemático que o reforça. Depois, seu poder de causar
desarranjos, obrigar a repensar modelos. Mas é esse caráter afirmativo
que interdita pensar o desenvolvimento nos mesmos termos do século
passado (ou do governo passado). Desenvolver não é absorver as emanações
do capital, mas constituir um espaço autônomo a partir das forças
presentes e vivas. Glauber deu a palavra: passar além da acomodação e da
conciliação, para que as potências criativas do país periférico não
justifiquem uma inserção de elites no centro global. Subverter o que
entendemos por desenvolvimento. Combater o subdesenvolvimento com os
meios do subdesenvolvimento.

\section{Referências}

\begin{Parskip}
\versal{ACUIO}, Carlos. ``Por que canta Caetano Veloso''. Revista \emph{Manchete}, 16/12/1967.

\versal{FAVARETTO}, Celso. \emph{Tropicália: alegria, alegoria}. São Paulo: Ateliê Editorial, 1996.

\versal{FURTADO}, Celso. \emph{Arquivos Celso Furtado 5: ensaios sobre
  cultura e o Ministério da Cultura.} Rio de Janeiro: Contraponto, 2012.

\_\_\_\_\_\_. \emph{Brasil, a construção interrompida}. São Paulo: Paz e Terra, 1992.

\_\_\_\_\_\_. \emph{Essencial Celso Furtado}. São Paulo: Penguin Companhia, 2013.

\_\_\_\_\_\_. \emph{Obra autobiográfica}. São Paulo: Companha das Letras, 2014.

\_\_\_\_\_\_. \emph{Pequena introdução ao desenvolvimento: enfoque interdisciplinar}. Rio de Janeiro: Companhia Editora Nacional, 1980.

\versal{HOLLANDA}, Heloísa Buarque. \emph{Impressões de viagem. \versal{CPC}, vanguarda, desbunde.} São Paulo: Brasiliense, 1979.

\versal{MILLS}, Charles Wright. \emph{The Politics of Truth}. \emph{Selected
  Writings of C. Wright Mills}. John Summers (ed.). Oxford: Oxford University Press, 2008.

\versal{PREBISCH}, Raúl. ``Interpretação do processo de desenvolvimento econômico''. \emph{Revista Brasileira de Economia}, v.~5, n.~1, 1951.

\versal{ROCHA}, Glauber. \emph{Revolução do Cinema Novo}. São Paulo: Cosac Naify, 2004.
\end{Parskip}



\chapter*{Tudo em volta está deserto}
\addcontentsline{toc}{chapter}{Tudo em volta está deserto, \emph{por Eduardo Jardim}}

\begin{flushright}
\emph{Eduardo Jardim}
\end{flushright}

Quando comecei a escrever \emph{Tudo em volta está deserto}, minha
intenção era considerar os primeiros anos da década de setenta, pois
aquele foi um momento crucial na história recente do Brasil, e,
possivelmente, também na de outros países. Muitos projetos amadurecidos
na década anterior se exauriram, criando uma espécie de vácuo de
expectativas. Por este motivo, ali se prefigurava o cenário dos anos
seguintes --- o dos anos finais da década que durou até recentemente.

Minha visão da época era pouco nítida, mas não pretendia recuperar
apenas meu ponto vista pessoal. Queria recorrer também à documentação do
período e tentar compreender o que se passou. Logo me dei conta de que a
falência das expectativas vivida pelos da minha geração não teve apenas
o sentido da perda das referências, mas foi como um desafio para o
pensamento e também para a poesia. Afinal, foi ali que me formei como
professor de filosofia e foi nesse tempo que convivi com a poeta Ana
Cristina Cesar, de quem me ocupo na última parte desse livro.

Comecei minha pesquisa tratando de um espetáculo que me impressionou na
época, o show \emph{Gal} \emph{a todo vapor}, montado no Teatro Teresa
Raquel, em Copacabana, no final de 1971. Gal cantava lindamente, de
forma emocionada e emocionante, ``Vapor barato'' e ``Mal secreto'', de
Jards Macalé e Waly Salomão, ``Como 2 e 2'', de Caetano Veloso, de onde
tirei o título do livro, ``Hotel das estrelas'', de Macalé e Duda
Machado. Também revelava Luiz Melodia, com ``Pérola negra'', dava
versões de Roberto Carlos e Jorge Ben Jor, naquela época só Jorge Ben,
fazia novas leituras de ``Falsa baiana'', de Geraldo Pereira, de
``Antonico'', de Ismael Silva e de ``Assum preto'', de Humberto Teixeira
e Luiz Gonzaga, e terminava com ``Luz do Sol'', de Carlos Pinto e Waly
Salomão, com o apelo gritado: ``Quero ver de novo a luz do Sol!''

O espetáculo significava uma catarse para todo um grupo que vivia sob a
repressão duríssima de depois do \versal{AI}-5, de dezembro de 1968. Dentro do
teatro lotado, no fundo de um centro comercial, extravasamos nossas
emoções. Ao mesmo tempo, a contenção era também uma marca do show. E
isso provocava uma forma de distanciamento, de recuo, que teria podido
motivar a compreensão do que se passava, não fosse a deriva
contracultural que muitos tomaram, quase sempre de forma desastrosa.
Essa tensão entre liberação e contenção caracterizava também a forma do
espetáculo e a escolha do repertório. Ela tinha a ver com certa
tendência da arte da época, com seu viés construtivo, e também foi um
traço da poética do diretor do show --- Waly Salomão.

Várias coisas estavam em jogo naquele espetáculo. Primeiro, havia aquilo
a que já me referi: o fato de ali se estar vivendo uma experiência
catártica. Além disso, de forma mais ou menos consciente, havia o
propósito de romper com tudo que acontecera antes. Se tomarmos como
referência, por exemplo, o tropicalismo, movimento que representou o que
houve de mais instigante no final dos anos 1960, nota"-se que
\emph{Gal a todo} \emph{vapor} guarda nitidamente distância dele. É como
se os que conceberam o espetáculo já tivessem absorvido as lições do
tropicalismo, não precisassem mais repeti"-las e se sentissem livres para
desbravar novos caminhos. O tropicalismo ainda tinha sido herdeiro do
Movimento Modernista --- basta lembrar a releitura da obra de Oswald de
Andrade feita por seus principais representantes. Os assuntos abordados
por eles ainda eram próximos dos de Mário de Andrade e da antropofagia
oswaldiana: a incorporação da cultura brasileira no cenário moderno
internacional, a integração de elementos diversos formando uma unidade
cultural brasileira, as tensões entre os traços culturais autóctones e
os estrangeiros, entre a dimensão tradicional e a moderna, entre os
componentes erudito e popular da arte. Ora, nenhuma dessas preocupações
esteve presente em \emph{Gal a todo vapor}.

O espetáculo foi uma reação a tempos difíceis, quando tudo em volta
estava mesmo deserto. ``Vapor barato'', a canção principal, faz menção
ao desejo de partir em um velho navio e tinha, com certeza, um sentido
de evasão. Mas isso não era tudo. Havia também a ideia de descortinar
novos cenários, novas possibilidades. Ampliava"-se o leque de
experiências que era preciso considerar, como os costumes, a
sensibilidade e a sexualidade.

Quando se busca situar o show de 1971 relativamente ao que veio antes, é
flagrante o contraste do seu tom sombrio com tudo que tinha acontecido de
exaltado até 1968.

Foi esse jogo de contrastes que me levou de volta a Antonio Callado e a
seu romance \emph{Quarup}, de 1967. O enredo do livro é conhecido. É a
história de Nando, inicialmente o Padre Nando, passada entre 1954 até
pouco depois do golpe de 1964, ambientada em Recife, em um primeiro
momento, em seguida no Rio, na região do Xingu, e, de novo, em Recife,
até o final do livro. Tudo é contado do ponto de vista de quem vivia os
acontecimentos da época em que o livro foi escrito, 1965-1966, momento
que se seguiu ao golpe militar de 1964.

A trama relata a formação de uma expedição de que Nando participa, que
tinha por objetivo encontrar o centro geográfico do país, narra a
própria viagem, o retorno do personagem a Recife, e termina com sua
adesão à resistência armada contra o regime. É também o relato de uma
viagem em busca da plenitude amorosa, encarnada no amor de Nando por
Francisca.

De quantas camadas se compõe um livro com a riqueza ficcional de
\emph{Quarup}?

\emph{Quarup} é um livro de viagem em busca de um graal perdido, como
foi também \emph{Macunaíma}, e tantos outros livros que se empenharam em
definir a identidade nacional.

Apresenta os ideais e sonhos de Callado e de muitos de sua geração. O
sonho da revolução social, que atualizaria uma série de movimentos
presentes na história brasileira, como a república guarani, os
quilombos, Canudos e as então recentes Ligas Camponesas. O livro faz,
deste modo, uma revisão da história brasileira. O do resgate de uma
perspectiva utópica, formulada desde a antiga Grécia, na lenda da
Atlântida, retomada muitas vezes, como pelo Padre Vieira. O livro
imagina a realização no Brasil de uma Atlântida continental. O sonho da
constituição de uma identidade brasileira, com base nos elementos
autóctones, especialmente o índio. Finalmente, o sonho da realização
amorosa plena, representado na esplêndida cena de amor de Nando e
Francisca entre as orquídeas às margens de um braço de rio, no Xingu.

O livro é também um drama com o seguinte percurso: há, primeiro, o
momento de afirmação desses sonhos e ideais. Em seguida, sua perda. A
revolução não se concretiza. Pelo contrário, tinha ocorrido o golpe
militar. O índio que poderia servir de referência para a fixação de uma
personalidade nacional estava ameaçado pelas doenças, pelo progresso, à
beira de ser extinto. O centro do Brasil afinal estava situado em um
imenso formigueiro de saúvas assassinas. O amor tinha se mostrado
irrealizável e não passava de uma quimera. A presença viva do fracasso
também aparece nas agressões da polícia política e na prisão do herói,
bem como no exílio da mulher amada.

Como reação a isso, o drama chega a seu desenlace. Nando toma a decisão
de pegar em armas. Adota o nome de Levindo, o guerrilheiro assassinado
pela polícia que continuou sendo o verdadeiro amor de Francisca, e com a
imagem dela na lembrança, parte para a guerrilha no interior. O livro
expressa a posição de setores da oposição ao regime militar que
defendiam a adesão à luta armada. Esta adesão aparece no livro muito
mais como o resultado da reação à perda dos antigos ideais e às
agressões sofridas.

O que confere a \emph{Quarup} um lugar de destaque na literatura da
época é que ele é atravessado por uma forte tensão entre, de um lado, a
adesão a uma saída política radical, e, de outro, seu questionamento.
\emph{Quarup} é literatura engajada, tem afinidade com as teses
defendidas por Jean"-Paul Sartre no pós"-guerra, muito difundidas,
inclusive no Brasil. Para Sartre, a prosa é fala e, deste modo, já se
situa no âmbito da ação e da intervenção política. Ao mesmo tempo,
\emph{Quarup} narra uma história de deseducação, de despojamento de
todas as certezas. Esse traço questionador seria aprofundado em outros
livros que se seguiram --- \emph{Bar Don Juan} (1971), \emph{Reflexos do
baile} (1976), e \emph{Concerto carioca} (1985).

Depois de passar pela literatura engajada de Callado, pela experiência
de catarse promovida por \emph{Gal -- a todo vapor}, \emph{Tudo em
volta está deserto} se ocupa da trajetória de Ana Cristina Cesar. No
caso de Ana Cristina quase tudo é poesia.

Ana não estava especialmente interessada no engajamento do artista,
também não era militante de qualquer causa, como o feminismo, por
exemplo.

Numa carta, com seu humor característico, se dirigiu à amiga Ana
Candida, que vivia no exterior, no dia da morte de \versal{J.~K.}, em agosto de
1976: ``pela primeira vez em anos se juntaram as multidões cantando
juntas, quase quase uma manifestação. Clima muito agitado no ar. Boatos %Quase quase mesmo?
\& rumores. Há gente comprando mantimentos (minha mãe até) e disseram
que o Geisel teve que convocar o Alto Comando militar para decretar luto
oficial. A manchete do dia é poesia --- você não acha? Especialmente sem
as aspas, e pense na letra da música (vou fazer um quadro)''.\footnote{Ana
  Cristina Cesar. \emph{Correspondência incompleta}. Heloisa
  Buarque e Armando Freitas Filho (org.). Rio de Janeiro: \versal{IMS}, 1999, p.~222.}

Em compensação, Ana se dedicava, de modo até obsessivo, a fazer seus
poemas com versos aparentemente coloquiais, mas que eram o resultado de
muito estudo, de um conhecimento de literatura raro entre seus
companheiros de geração. Isso aparecia também na sua prosa, nas colunas
que escreveu nos jornais, como no semanário \emph{Opinião}, nos
trabalhos acadêmicos e nas traduções. Sua dissertação de mestrado ---
\emph{Literatura não é documento} --- continua sendo uma referência para
a discussão do cinema documentário sobre literatura. Mais tarde
publicada em livro, põe em discussão a própria noção de cinema
documentário. Para Ana, os filmes que mais assumiam um caráter ficcional
no tratamento das obras literárias eram os mais bem"-sucedidos, inclusive
do ponto de vista documental. Uma série de tensões como esta entre
ficcional e documental, entre coloquialidade e construção formal, entre
literatura confessional, pessoal, e impessoal, atravessam, sem solução,
a obra de Ana Cristina Cesar. E é esta a fonte da sua vitalidade.

Ana não apenas não se enquadrava em qualquer militância como também não
se identificava com os ``poetas marginais''. Não era nada
contracultural, não se encaixava em nenhum rótulo. Se tivesse que
definir um lugar para ela, diria que ela estava à margem da margem.

Mas, afinal, o que era poesia para Ana? Ela própria deu a seguinte
definição, em um daqueles momentos frequentes de vacilo da vocação, ao
contrastar seu trabalho como poeta com o de uma amiga pintora:

\pagebreak
%%Afastar um pouco da mancha, assim como os próximos poemas de ana cristina?
\begin{quote}
\emph{Vacilo da vocação}\\[5pt]
Precisaria trabalhar --- afundar ---\\
--- como você --- saudades loucas ---\\
--- nesta arte --- ininterrupta ---\\
--- de pintar ---\\[5pt]
A poesia não --- telegráfica --- ocasional ---\\
me deixa sola --- solta ---\\
a mercê do impossível ---\\
--- do real.
\end{quote}

O conciso poema esclarece muito sobre como Ana entendia a natureza da
poesia. Em primeiro lugar, ela observa que existe um tempo próprio da
poesia, muito diferente do tempo do dia a dia, e inclusive do tempo do
trabalho do pintor. A poesia não é feita linearmente, não é
ininterrupta, ela é ocasional, é instantânea. Além disso, a poesia
desprende, solta dos hábitos mentais e de todos os outros, mas existe um
preço que se paga por isso --- a solidão. Ana está \emph{sola} e
\emph{solta}. De qualquer forma, o ganho é enorme: o poeta se abre ao
contato do extraordinário, vive uma experiência singular que dá acesso
ao impossível. Mas o que é o impossível? É propriamente o real que não
vemos porque estamos entretidos na lida cotidiana. Mas ela, convicta,
afirma: ``À mercê do impossível, do real.''

Disse que tudo era poesia, quase tudo, na vida da Ana Cristina, mas é
preciso acrescentar que a poesia da Ana também era guiada por uma
necessidade de interlocução, e ela achava que isso era, inclusive, o
motivo para se fazer poesia.

De um modo um pouco surpreendente, ela achava que essa busca da
interlocução era característica de uma literatura feminina (não
feminista). Claro que ela não queria dizer que se tratava de uma
literatura feita só por mulheres. Grandes escritores homens --- os que Ana
admirava --- como Guimarães Rosa, também teriam feito literatura feminina.
Em todos esses autores femininos ela reconhecia esta busca de
interlocução, que é, geralmente, interlocução amorosa, mas não apenas.

Para explicar isso, reproduzo dois poemas que exemplificam o que é
poesia feminina --- ``Fogo do final'', de \emph{A teus pés} (1982), com a
referência à famosa passagem de Baudelaire, e ``Contagem regressiva'',
de \emph{Inéditos e dispersos}:

\begin{quote}
(\ldots{})\\
É para você que escrevo, hipócrita.\\
Para você --- sou eu que te seguro os ombros e grito verdades nos
ouvidos, \qb{}no último momento.\\
Me jogo aos teus pés inteiramente grata.\\
Bofetada de estalo --- decolagem lancinante --- baque de fuzil. É só
pra você y que letra tán hermosa. Pratos limpos atirados para o ar.
Circo instantâneo, pano rápido mas exato descendo sobre a tua cabeleira
de um só golpe, e o teu espanto!\footnote{Ana Cristina Cesar. \emph{Poética}.
  São Paulo: Companhia das Letras, 2013, p. 121.}
\end{quote}

De \emph{Contagem regressiva}:

\begin{quote}
(\ldots{})\\
saberias então que hoje, nesta noite, diante desta gente,\\
não há ninguém que me interesse e meus versos\\
são apenas para exatamente esta pessoa que\\
deixou de vir\\
ou chegou tarde, sorrateira, de forma que não posso\\
gritar ao microfone com os olhos presos nos seus olhos\\
baixos, porque não te localizo e as luzes da ribalta\\
confundem a visão,\\
(\ldots{})\footnote{Ibidem, p. 247.}
\end{quote}

\chapter*{Identidade em cena}
\addcontentsline{toc}{chapter}{Identidade em cena, \emph{por Felipe Scovino}}


\begin{flushright}
\emph{Felipe Scovino}
\end{flushright}

Tenho interesse em refletir sobre como as obras de Lina Bo Bardi e Hélio
Oiticica refletiram e problematizaram a representação de uma identidade
vinculada à categoria de ``povo''. Ou melhor, como esses dois
personagens centrais para a ideia de contemporaneidade no Brasil estavam
questionando a forma vertical e hierarquizante como eram vistas as classes
sociais menos abastadas no país. E tendo o museu como meio,
no caso de Bo Bardi, e epicentro de uma revolta, no caso de Oiticica,
que tipo de apreciação sobre cultura e sociedade essa instituição
(museu) estava construindo?

Por volta de 1967, Hélio Oiticica cria um distanciamento seguro dos
conceitos de realismo social, realismo mágico e novo realismo,
apresentando o que compreende como ``arte participante'', expressão
tomada a partir de uma perspectiva de Ferreira Gullar sobre a ideia de
ser artista naquele momento específico. Nesse sentido, como forma de
contextualizar o momento e o que a comunicação pretende expor como
\emph{turning point} no trabalho de Oiticica, seguem as palavras do
artista:

\begin{quote}
{[}\ldots{}{]} O que Gullar chama de participação é, no fundo, essa
necessidade de uma participação do poeta, do artista, do intelectual em
geral, nos acontecimentos e nos problemas do mundo, consequentemente
{[}sic{]} influindo e modificando"-os; um não virar as costas para o
mundo para restringir"-se a problemas estéticos, mas a necessidade de
abordar esse mundo com uma vontade e um pensamento realmente
transformadores, nos planos ético"-político"-social. O ponto crucial
dessas ideias {[}sic{]}, segundo o próprio Gullar: não compete ao
artista tratar de modificações no campo estético como se fora este uma
segunda natureza, um objeto em si, mas sim de procurar, pela
participação total, erguer os alicerces de uma totalidade cultural,
operando transformações profundas na consciência do homem, que de
espectador passivo dos acontecimentos passaria a agir sobre eles usando
os meios que lhe coubessem: a revolta, o protesto.\footnote{Hélio Oiticica. ``Esquema geral da Nova Objetividade''. In: \emph{Escritos de artistas: anos 60/70}. Rio de Janeiro: Zahar, 2006, p.~164.}
\end{quote}

Esse contexto é fundamental para entender uma nova tomada de posição que
alguns artistas, críticos e curadores adotaram tendo como pano de fundo
os primeiros anos da ditadura no país. Estou argumentando sobre o fato
de uma parcela invisibilizada da sociedade ter sua imagem projetada e
problematizada por meio do trabalho de dois artistas, em especial. Mais
que uma denúncia social, as obras de Oiticica e Lina Bo Bardi se
infiltram no espaço do museu por meio do atrito e como uma ação
política, e não panfletária, revelam o povo, especialmente a imagem
de negros, pobres, nordestinos, mulheres e crianças. É outra
possibilidade de se pensar o Brasil, de refletir sobre quem somos e o
nosso papel na seara da construção social e da formação identitária.
Para os dois artistas, não poderia haver mais lacunas nessa
representação social do brasileiro. Isto é o que parece evocar em seus
trabalhos. Cabe explicitar a noção de povo com a qual trabalho e que
deriva do pensamento de Gilberto Velho, no qual ele compreende que povo
abrangeria desde as classes trabalhadoras que mantêm uma rede de
relações compartilhadas em seu território, no campo e na cidade, bem
como um universo heterogêneo de camadas. Lélia Frota explicita melhor
que camadas seriam essas:

\begin{quote}
{[}\ldots{}{]} Pequenos proprietários, boias"-frias, pescadores,
desempregados, semi"-empregados, marginais do mercado de trabalho e de
todos os outros tipos, empregados domésticos, funcionários públicos,
técnicos de nível médio, comerciários, bancários, diversos setores de
camadas médias, moradores de favelas, conjuntos, subúrbio, periferia etc.\footnote{Lélia Frota. \emph{Pequeno dicionário da arte do povo brasileiro}. Rio de Janeiro: Aeroplano, 2005, p.~16.}
\end{quote}

Aliás, os dois momentos, ou estudos de caso, que cito como
\emph{leitmotiv} da comunicação são: o impedimento de Oiticica e de
componentes da Escola de Samba Estação Primeira da Mangueira, todos
vestidos com \emph{Parangolés}, de continuarem no salão do Museu de Arte
Moderna do Rio de Janeiro (\versal{MAM}-Rio) durante a abertura da exposição
Opinião 65, e a montagem da exposição A mão do povo brasileiro,
organizada por Lina Bo Bardi em 1969 no Museu de Arte de São Paulo
(Masp).

Como forma de problematizar e entender esse intervalo de 4 anos entre os
dois estudos de caso que abordo, é importante iluminar o que se passa ---
em termos de conflitos ideológicos e culturais --- na história do Brasil
naquele momento. Trago alguns exemplos pontuais que nos ajudarão a
compreender a relação tensa entre política, arte e instituição. Em 1967,
durante o \versal{IV} Salão de Arte Moderna do Distrito Federal, realizado em
Brasília, foi exibida a obra \emph{Guevara}, de Claudio Tozzi. Mostrado
neste contexto, o retrato de Tozzi do líder cubano, que havia sido
recentemente assassinado, foi parcialmente destruído por um grupo de
simpatizantes do regime militar. Este ataque indicava que arte, quando
exposta publicamente, surge como um alvo ideológico. Essa tendência foi
mais tarde fortalecida por atos de censura do governo contra a \versal{II} Bienal
da Bahia, em 1968, e a Pré"-Bienal de Paris, realizada no \versal{MAM}-Rio em
1969. A polícia militar fechou ambas as exposições e confiscou obras
consideradas ideologicamente suspeitas. Na exposição de 1969, a polícia
também prendeu e deteve a diretora do museu, Niomar Muniz Sodré. Visto
isoladamente, incidentes como esses oferecem somente uma perspectiva
parcial da história da arte produzida durante o período prolongado do
governo militar no Brasil (1964-1985). A questão de como os artistas
operavam em circunstâncias marcadas por extrema repressão não pode ser
respondida somente com referências à censura ou ataques violentos.
Também exige uma consideração de como seria possível para o artista
continuar seu trabalho apesar de tais ações. Como foi possível produzir
trabalhos críticos nesse contexto? Como foi possível também que
experiências coletivas, tais como exposições, continuassem a ser
realizadas? O rápido relato sobre o fechamento e apreensão de obras em
salões nos oferece alguma indicação do risco que era para os artistas
trabalhar neste período e nestas circunstâncias. O jornalista Claudir
Chaves relata em sua crítica intitulada ```Parangolé' impedido no \versal{MAM}'',
publicada no \emph{Diário Carioca}, o que ocorreu na noite de abertura da
exposição Opinião 65:

\begin{quote}
{[}\ldots{}{]} O que causou realmente impacto no grupo, foram os trabalhos
apresentados por Hélio Oiticica, os quais ele denomina de `Parangolé',
onde entram como composição, estandarte, cuba de vidro, tenda de matéria
plástica, capas, fantasias, gente, música, ritmo côr {[}sic{]} e
movimento. Não vamos aqui analisar à luz da crítica nem a tomada de
posição com referência à arte, propriamente dita. Comentaremos o fato de
a direção do \versal{MAM} não permitir a exibição da `arte ambiental' no seu
todo. Não foi possível a apresentação dos passistas, comandados por
Hélio Oiticica, no interior do Museu, por uma razão que não conseguimos
entender: barulho dos pandeiros, tamborins e frigideiras. Hélio
Oiticica, revoltado com a proibição, saiu juntamente com os passistas e
foram exibir"-se no lado de fora, isto é, no jardim, onde foram
aplaudidos pelos críticos, artistas, jornalistas e parte do público que
lotavam as dependências do \versal{MAM}.\footnote{Claudir Chaves. ```Parangolé' impedido no \versal{MAM}''. In: \emph{Opinião 65: 50 anos depois}. Rio de Janeiro: Pinakotheke, 2015, p.~61.}
\end{quote}

Ter sido expulso de sua própria exposição --- e aqui especulo que as
razões podem variar entre a dança e materiais baratos sendo qualificados
como arte, haver um grupo de moradores da Mangueira dentro de um espaço
social e cultural largamente utilizado por classes abastadas, e a música
e um entusiasmo contagiante se contrapondo ao ambiente austero de um
museu --- cria um terreno propício para se pensar no que Oiticica já
articulava e dois anos depois escreve em ``Esquema geral da Nova
Objetividade'':

\begin{quote}
{[}\ldots{}{]} O fenômeno da vanguarda no Brasil não é mais hoje questão de
um grupo provindo de uma elite isolada, mas uma questão cultural ampla,
de grande alçada, tendendo às soluções coletivas. A proposição de Gullar
que mais nos interessa é também a principal que o move: quer ele que não
bastem à consciência do artista como homem atuante somente o poder
criador e a inteligência, mas que o mesmo seja um ser social, criador
não só de obras mas modificador também de consciências (no sentido
amplo, coletivo), que colabore ele nessa revolução transformadora, longa
e penosa, mas que algum dia terá atingido o seu fim --- que o artista
``participe'' enfim da sua época, de seu povo.\footnote{Hélio Oiticica. ``Esquema geral da Nova Objetividade'', op.~cit., p.~165.}
\end{quote}

Focarei a minha crítica na forma como suas proposições foram
documentadas, para que pudessem circular em periódicos e catálogos,
principalmente. Acredito que essa análise encontra imbricações com o
contexto da política brasileira naquele momento, como apresentado no
início da comunicação, e a forma como os artistas criaram micropolíticas
incendiárias para refletir sobre o seu papel no mundo e nessa nova
relação que a arte criava com o meio. Faço menção portanto ao fato de
que parcela significativa das documentações fotográficas dos
\emph{Parangolés} têm como protagonistas amigos do artista, moradores da
Mangueira. Estão lá Mosquito da Mangueira dançando com \emph{Parangolé
P10 Capa 6} e observando o \emph{Bólide Vidro 5, Homenagem a Mondrian}
(1965); o mesmo Mosquito observando o \emph{Bólide Luz 1, Apropriação 3}
(1966), manipulando o \emph{Bólide Plástico 1} (1966) e vestido com o
\emph{Parangolé P4 Capa 1}; Maria Helena da Mangueira vestida com
\emph{P8 Capa 5, Homenagem a Mangueira} (1965); Roseni da Mangueira com
\emph{P5 Capa 2} (1965); Miro da Mangueira bailando com \emph{P4 Capa 1}
(1964), dentre outras imagens. Estão representadas nessas imagens não só
a obra \emph{stricto sensu} mas quem a veste, a manipula, a usa, a
sente. Melhor dizendo, a obra é o conjunto desses fatores: tanto o
\emph{Bólide} e o \emph{Parangolé} com suas respectivas qualidades
formalistas e fenomenológicas, digamos assim, mas também as
representações culturais e sociais que evocam. Quando a obra é
documentada, Oiticica sai de cena, e quem aparece é o povo: a sua
identidade e realidade. Nada muito distante da ideia de realismo pensada
pelo Cinema Novo, onde alguns personagens de filmes, ligados a esse
movimento, não são atores profissionais. Em \emph{Rio Zona Norte} (1957)
de Nelson Pereira dos Santos, filme considerado pelos próprios cineastas
do Cinema Novo como a obra que antecipa as questões neo"-realistas no
cinema brasileiro, Zé Keti --- renomado cantor e compositor de samba,
autor de ``A voz do morro'', canção e título tão sintomáticos para essa
discussão --- é um dos antagonistas. Para percebermos ainda mais esses
laços fecundos entre Cinema Novo e Oiticica, em uma das cenas de
\emph{Câncer} (1968-1972), de Glauber Rocha, presenciamos a atuação de
Oiticica numa cena com atores representando o que poderíamos identificar
como o povo, o bandido e o patronato. Pedindo emprego, o personagem de
Antonio Pitanga (``povo'') é humilhado e subjugado pelo patrão.
Dialogando e se confrontando, os personagens se questionam sobre a ideia
de democracia e justiça social. Eis, de forma teatralizada e ficcional,
a realidade que Oiticica expõe com a sua obra. No filme, assistimos não
só a presença de Oiticica mas também de Tineca, amiga do artista e que
também foi documentada usando suas obras, assim como de passistas da
Mangueira. Em um artigo denominado ``Opinião\ldots{} Opinião\ldots{} Opinião'',
Mário Pedrosa faz uma crítica à Opinião 65 e contextualiza o momento
trazendo o exemplo do Teatro de Arena --- e aqui faço uma triangulação
possível com o Cinema Novo e a obra de Oiticica no tocante a serem modos
de produção cultural que expuseram, guardadas suas especificidades e
intensidades, a condição de renúncia social e econômica de boa parte da
sociedade --- e em especial a canção ``Carcará'', de João do Vale. Segundo
Pedrosa, ``pouca gente ouvia então aquele canto, expressando a realidade
implacavelmente feia, malvada e egoísta da miséria natural e social do
Nordeste, sem ser sacudido por dentro e sem lágrimas nos olhos''.\footnote{Mário Pedrosa. ``Opinião\ldots{} Opinião\ldots{} Opinião''. In: \emph{Opinião 65: 50 anos depois}, op.~cit., p.~55.}

A revolta de Oiticica de certa forma expõe uma ambivalência entre o
registro pictórico \emph{do} povo, segundo a concepção de Gilberto Velho
e Lélia Coelho Frota, e uma ação real e imediata \emph{com} o povo. Por
exemplo, diferentemente da obra de Di Cavalcanti que pintava o subúrbio,
a mulata, as prostitutas, o samba e sua cultura, Oiticica toma para si o
corpo da gente do morro, para usar uma expressão da época que
simbolizava outra concepção sobre povo. Mário Pedrosa tem uma crítica
contundente sobre a função social da pintura de Di:

\begin{quote}
Ele quer colocar a vida artística do Brasil na realidade nacional,
procurar dar"-lhe vida tirada da própria seiva racial, dar sua função
social dentro de nossos limites, procurar fazê"-la unicamente dependente
de um `clima afetivo e cultural tão característico nosso'. Eu gostaria
que Di nos dissesse que `seiva racial' ele fala: da negra? portuguesa?
índia? italiana?\footnote{Mário Pedrosa. ``Um novo Di Cavalcanti''. In:
\emph{Acadêmicos e modernos: textos escolhidos \textsc{iii}}. São Paulo: Editora da
Universidade de São Paulo, 1998, p.~187-189.}
\end{quote}

Pedrosa interroga no mesmo texto que lugar seria o da tradição
brasileira que deveria ser resgatado pela arte. De forma contundente e
com alto teor de acidez, o crítico aponta que a pintura de Di
contraditoriamente ``tem a marca de Paris''. Mário, aliás, foi
importante para Oiticica e sua geração ao pavimentar o terreno
ideológico e artístico. O Brasil a ser visto e pensado nas artes não
poderia ser uma alegoria do aprazível e do apaziguamento, mas um
compromisso, diria, político com o Outro. A frase/conceito ``incorporar
a revolta'' de Oiticica ganha uma potência ainda mais perturbadora e
imediata quando a associamos a esse momento de insurgência, poderíamos
dizer, do seu trabalho. O Oiticica inconformado e que divide esse
incômodo conosco é fruto das suas andanças e vivências na Estação
Primeira da Mangueira. Vendo de perto a miséria, em seus mais diversos
aspectos, daquela comunidade, o artista, especulo mais uma vez, absorve
essa crítica ao seu trabalho. Essa pode ser uma hipótese robusta para a
escolha dos moradores e amigos da Mangueira serem documentados
interagindo com os \emph{Parangolés} e \emph{Bólides}.

A ``Nova Objetividade'' de Oiticica e a sua conclamação a um ``ser
social'' por parte dos artistas e da comunidade ficam claros. Não penso
que Oiticica tivesse um compromisso ou investimento na luta ideológica,
mas estava consciente do seu papel como artista e conciliado em
contribuir com a resistência coletiva. Percebam também que não se
tratava somente de um embate com a censura e as demais imposições do
regime militar mas uma tomada de posição frente aos condicionamentos e
recusas que a sociedade impõe. Daí a atitude de exibir as contradições
sociais e as manifestações populares, ``das quais o Brasil possui um
enorme acervo, de uma riqueza expressiva inigualável''.\footnote{Hélio Oiticica. ``Esquema geral da Nova Objetividade'', op.~cit., p.~165.}
Era o momento de pensarmos sobre a nossa formação social e
política, a permanência de uma larga diferenciação social no tempo
presente e, enquanto artistas, o papel dessa classe frente aos museus e
a sociedade. Como Oiticica escreveu: ``No Brasil (nisto também se
assemelharia ao Dadá) hoje, para se ter uma posição cultural atuante,
que conte, tem"-se que ser contra, visceralmente contra tudo que seria em
suma o conformismo cultural, político, ético, social''.\footnote{Ibidem, p.~167.}
A fragilidade dos materiais que compõem os \emph{Parangolés}
estabelece uma relação intrínseca com a própria estrutura da sociedade:
o quão débil e violenta podem ser as relações inter"-humanas e como nos
colocamos frente aos regimes de alteridade. Enfim, como nos vimos, nos
identificamos e nos relacionamos com o diferente parece ser uma pergunta
recorrente lançada por essas obras.

Incluo ainda nessa análise o icônico \emph{Bólide Caixa 18, Poema Caixa
2, Homenagem à Cara de Cavalo} (1966), cujo interior de seu cubo
semiaberto contem imagens do corpo do referido bandido, momentos depois
de ter sido morto pela polícia. Lá está ele com a face cravada de balas
e com os braços abertos como um crucificado. Tão icônico quanto o objeto
é a imagem de Hélio Oiticica ao lado do \emph{Bólide} apresentando"-o
como uma espécie de documentação sobre o Brasil real, aquele que é
apagado, marginalizado, deixado de lado. Não é uma celebração ao
banditismo mas conforme colocou:

\begin{quote}
{[}\ldots{}{]} Este trabalho representou para mim um `momento ético' que se
refletiu poderosamente em tudo que fiz depois: revelou para mim mais um
problema ético do que qualquer coisa relacionada com estética. Eu quis
aqui homenagear o que penso que seja a revolta individual social: a dos
chamados marginais.\footnote{Hélio Oiticica. ``Cara de Cavalo''. In: \emph{Hélio Oiticica}. Rio de Janeiro: Centro de Arte Hélio Oiticica, 1996, p.~25.}
\end{quote}

Oiticica via ali o humano, o heroico, sem abrir do paradoxo em que Cara
de Cavalo estava situado. Mas interessava ao artista expor também as
condições miseráveis --- em seu mais amplo sentido --- de subsistência de
boa parte da população. Lembro ainda do filme \emph{Semi"-ótica} (1975)
de Antonio Manuel e o \emph{Trio do embalo maluco} (1968) de Lygia Pape
como obras que também contribuíram para a reflexão das condições sociais
e morais respectivamente do bandido e da população negra no país.

Ainda na crítica publicada por Claudir Chaves, o mesmo faz um relato
sobre uma entrevista concedida por Ivan Serpa à Ferreira Gullar para a
revista \emph{Civilização}. Serpa afirma que ``estamos longe de atingir esse
museu ideal, pois ainda hoje a direção dos museus fica contente quando
consegue atrair figuras da elite social, quando devia preocupar"-se em
levar o povo ao museu''.\footnote{Claudir Chaves. ```Parangolé' impedido no \versal{MAM}'', op.~cit., p.~65.} Este argumento parece"-me
ser um vínculo forte entre Oiticica e Lina Bo Bardi. Minha hipótese é
que ambos preocupavam"-se não só em expor ou escancarar um país ruidoso,
áspero e hostil, imerso num contexto de altas taxas de analfabetismo,
mortalidade infantil e pobreza e ainda avassalado pela ditadura, mas
também as suas riquezas, expressas no caráter de invenção do seu povo
(ela tratou de enaltecer o alto grau de astúcia e a artesania das peças
expostas em A mão do povo brasileiro, e nos \emph{Parangolés} estão lá a
costura e entrelaçamento de peles, corpos e musicalidade, sendo que
nenhum dos dois abriu mão de uma crítica social). São os paradoxos do
Brasil ou o ``espanto inexplicado'', como Clarice Lispector identificou
Brasília em sua crônica ``Brasília: cinco dias'', que os interessam.

A mão do povo brasileiro, concebida por Lina Bo Bardi, Pietro Maria
Bardi, Glauber Rocha e Martim Gonçalves, foi a mostra temporária
inaugural do Masp na Avenida Paulista. A exposição é um desdobramento de
outras mostras organizadas pela arquiteta, como Bahia no Ibirapuera,
montada em 1959 durante a \versal{V} Bienal de São Paulo e Nordeste, montada no
Museu de Arte Popular, no Solar do Unhão, Salvador, em 1963 e na
Galleria Nazionale d'Arte Moderna, em Roma, 1965. A mão do povo
brasileiro apresentou um vasto panorama da rica cultura material do
Brasil --- cerca de mil objetos, incluindo carrancas, ex"-votos, tecidos,
roupas, móveis, ferramentas, utensílios domésticos, gaiolas,
maquinários, instrumentos musicais, adornos, brinquedos, objetos
religiosos, pinturas (destacando a participação de Agostinho Batista de
Freitas), esculturas (como as de Mestre Vitalino), objetos
constituidores de um ambiente doméstico (pratos, talheres, mesas, vasos,
santos, adornos) --- desde o final do século \versal{XIX} até aqueles dias. Toda
essa reunião de objetos e tipologias é colocada lado a lado, ressaltando
também o caráter de sincretismo da cultura brasileira, como é o caso da
presença de Agnaldo dos Santos, que esculpia peças de madeira com
figuras divinizadas nos cultos cristãos e africanos. O que fica evidente
é a ideia de manufatura pois são objetos feitos à mão, atravessando
tempos e regiões do país.

A mão do povo brasileiro se insere em um histórico de outras exposições
no Masp, inclusive a pioneira Arte popular pernambucana, em 1949. É uma
exposição considerada de vanguarda no Brasil, país que se acostumava a
receber naquele momento as diferentes ondas de experimentação e novas
produções de todo o mundo via, principalmente, as edições da Bienal de
São Paulo. Ademais, A mão do povo brasileiro continua sendo uma
oportunidade rara de refletir também sociologicamente o lugar do museu
na modernidade. Expor objetos que eram colocados à margem da
historiografia da arte e das exposições leva o público a contestar
noções ortodoxas de `arte popular' e `cultura popular', e mais do que
isso, transmite a possibilidade de reconstruir e reconfigurar as
distintas formas de aparição da história da arte e da cultura no Brasil,
para além do que é consagrado pela história oficial e dominante.

Os cavaletes, criados pela arquiteta um ano antes da mostra, dialogam
intensamente com a expografia, já que Bo Bardi faz uso de madeira
natural como meio de apresentação das obras. Outro gesto que aproxima os
cavaletes das plataformas de madeira, em que boa parte das obras estava
exposta, é a dimensão política que o espectador tem ao perceber a
exposição como uma galeria aberta, translúcida e fluida, oferecendo uma
simultaneidade de visões, perspectivas, acessos e leituras das obras.
Este gesto elimina hierarquias e proporciona uma leitura horizontal dos
trabalhos. A curadoria de A mão do povo brasileiro criou um embate com a
ideia de museu, coleção, público e arte ao valorizar uma produção que
era frequentemente negligenciada pela história da arte. Apresentar os
modos de vida dos brasileiros, seu caráter de invenção baseado na
artesania, a religiosidade, a infância, a cultura do Norte e do Nordeste
do país, cria um gesto radical de descolonizar o museu e repensar os
seus propósitos. A mostra, em certa medida, parte de um caráter de
adversidade para explorar as suas noções próprias e únicas de expressão
e invenção, e o quanto ambas regem o cotidiano dessa sociedade. Essa
adversidade também pode ser entendida pelo fato de que a exposição
Nordeste, em Roma, foi desmontada pouco antes da abertura por ordem da
embaixada brasileira na cidade. Segundo Adriano Pedrosa, ``supõem"-se que
os objetos feitos pela mão do (pobre) povo brasileiro não
representariam, aos olhos europeus, a imagem de um país moderno, em
desenvolvimento --- para a frente ---, a que o nascente regime militar
aspirava''.\footnote{Adriano Pedrosa \& Tomás Toledo. \emph{A mão do povo brasileiro,
1969/2016}. São Paulo: Masp, 2016, p.~32.}

Nesse sentido, acho pertinente a aproximação entre o embate explorado
por Lina Bo Bardi e o caráter de invenção do brasileiro exposto na
``arquitetura favelar'' dos \emph{Penetráveis} de Oiticica e a sua
imposição no espaço da alta cultura do museu, como foi o caso de
\emph{Tropicália} sendo instalada no \versal{MAM}-Rio em 1967. A reunião e
apresentação dos objetos, na mostra de Bo Bardi, se deu por tipologias e
usos --- a cozinha, a religião, o lazer, o trabalho etc. Colocados sobre
tablados de madeira, os objetos, reconhecidos até então como utilitários,
ganharam uma nova função e visualidade. Passam a ser equiparados, e não
apenas simbolicamente, a objetos de arte. A exposição, portanto,
repensou e embaralhou inteligentemente as categorias de arte, artefato e
artesanato, questionando e ao mesmo tempo ampliando o que comumente era
definido de forma segura como objeto de uma coleção de museu de arte.

A exposição apresentou os modos e métodos da criação executados por
artistas, muitas vezes anônimos ou ausentes dos livros de história. São
invenções realizadas com materiais que fogem às especificidades da alta
tecnologia --- sobressaindo"-se a madeira --- e que foram produzidas pela
própria necessidade do homem. Sem o uso de máquinas potentes ou
altamente desenvolvidas, estes objetos falam de uma circunstância
atemporal do brasileiro: o seu caráter de improvisação diante muitas
vezes de uma falta de recursos econômicos, e o quanto essa inventividade
faz parte do nosso cotidiano, vide as gambiarras que nos circundam
constantemente. O que percorreu a seleção desses objetos foi uma
reivindicação do estado de arte popular e a sua própria redefinição,
querendo se afastar de termos pejorativos como ``arte primitiva'' ou
``naïf''. E mais do que isso, trouxe a arte do outro para o diálogo com
e no centro. Enfim, tinha compromisso tensionar o regime de alteridade.

Embora o questionamento sobre identidade no Brasil nas perspectivas de
Oiticica e Bo Bardi estejam ligados a um viés político, sem conotação
panfletária, seus meios de apropriação e reflexão são distintos. Fazem
um uso inteligente e audaz do mesmo meio (arte) como elemento
problematizador, mas distinguem"-se na forma. A documentação fotográfica,
em especial, dos \emph{Parangolés}, e que especulo ser parte intrínseca
da obra de Oiticica e não apenas registro do seu uso, fornece indícios
para identificar à qual lugar, contexto e origem aquela obra pertence.
São amigos do artista, moradores da Mangueira, destituídos de posses e
vinculados intrinsicamente ao carnaval. Essas escolhas, que imagino não
serem ocasionais, criam uma nova forma de entendimento sobre o caráter
amplamente social da obra de Oiticica. A ideia de participação, tão
propalada sobre a sua obra, não é apenas a participação pela
participação, mas especialmente a inclusão de grupos sociais
marginalizados que passam a serem vistos e replicados, por meio de
publicações ou proposições performáticas do artista. É claro que é
preciso relativizar porque o número de exposições, museus e
principalmente publicações de arte que havia nos anos 1960 em comparação
com hoje é muito menor. Contudo, o que exponho é o desejo, e em certa
medida uma estratégia, do artista em fazer circular imagens ou
representações de uma classe social marginalizada dentro de um sistema
(arte) em que sobrevoa um
princípio democrático mas que contem em si uma política segregadora. O
museu passa a ser lugar de debate e questionamento sobre regimes de
alteridade. Um caso exemplar nesse sentido foi a montagem de
\emph{Tropicália}, em 1967, na mostra Nova Objetividade Brasileira.
Terra, brita e plantas tomando o espaço do \versal{MAM}-Rio. Pedaços de madeira
ajudando a sustentar frágeis tecidos que criavam divisórias e
simultaneamente cômodos. É a arquitetura da favela adentrando o espaço
sacralizado do museu e provocando um desconforto generalizado em um
espaço social cujas regras de entrada, permanência e institucionalização
tendiam a ser bem claras. Nesse embate entre arte moderna e o
pós"-moderno, para usar um conceito de Mário Pedrosa, o museu obviamente
é repensado. E a Hélio Oiticica também interessa, acima de tudo,
refletir sobre quem faz uso e produz uma arte definitivamente pública.
Parece"-me claro esse embate proposto pelo artista.

Quero deixar claro que os embates colocados por A mão do povo brasileiro
não se referem exclusivamente a uma discussão sobre arte popular ou
baixa cultura versus objeto de arte assim qualificado pelo museu e pelo
circuito de arte. Trago essa mostra como estudo de caso para demonstrar,
acima de tudo, que havia naquela reunião de objetos um imaginário (ou
``alma'') que alegoricamente representava o trabalho, o corpo e a
identidade de camadas populares negligenciadas. Há uma política de
enfrentamento com o espaço, de trazer para o debate artístico as
idiossincrasias e preconceitos da sociedade. E aqui A mão do povo
brasileiro, guardadas as diferenças entre as duas experiências, se
coloca lado a lado com as propostas de Oiticica. Nesse coletivo de
imagens (de obras e ações praticadas sob a égide da censura e da
repressão), o que importa é expor, refletir e discutir sobre quem somos
e como podemos viver juntos.

\pagebreak

\section{Referências}

\begin{Parskip}
\versal{CHAVES}, Claudir. ```Parangolé' impedido no \versal{MAM}''. \emph{Diário Carioca}, 14/08/1965. In: \versal{PERLINGEIRO}, Max. \emph{Opinião 65: 50 anos depois}. Rio de
Janeiro: Pinakotheke, 2015, p.~61.

\versal{COCCHIARALE}, Fernando \& \versal{FILHO}, César Oiticica. \emph{Hélio Oiticica:
museu é o mundo}. São Paulo: Itaú Cultural, 2010.

\versal{FROTA}, Lélia Coelho. \emph{Pequeno dicionário da arte do povo
brasileiro}. Rio de Janeiro: Aeroplano, 2005.

\versal{OITICICA}, Hélio. ``Cara de Cavalo''. In: \versal{BRETT}, Guy; \versal{FIGUEIREDO}, Luciano; \versal{PAPE}, Lygia et al. \emph{Hélio Oiticica}. Rio de Janeiro: Centro de Arte
Hélio Oiticica, 1996, p. 25.

\_\_\_\_\_\_. ``Esquema geral da Nova Objetividade''. In: \versal{FERREIRA}, Glória \&
\versal{COTRIM}, Cecília. \emph{Escritos de artistas: anos 60/70}. Rio de
Janeiro: Zahar, 2006, p. 154-168.

\versal{PEDROSA}, Adriano \& \versal{TOLEDO}, Tomás (org.). \emph{A mão do povo brasileiro,
1969/2016}. São Paulo: Masp, 2016.

\versal{PEDROSA}, Mário. ``Um novo Di Cavalcanti''. In: \versal{ARANTES}, Otília (org.).
\emph{Acadêmicos e modernos: textos escolhidos \versal{III}}. São Paulo: Ed\versal{USP}, 1998, p.~187-190.

\_\_\_\_\_\_. ``Opinião\ldots{} Opinião\ldots{} Opinião''. In: \versal{PERLINGEIRO}, Max.
\emph{Opinião 65: 50 anos depois}. Rio de Janeiro: Pinakotheke, 2015, p.~55-57.
\end{Parskip}

\chapter*{Tropicalismo: o movimento dos corpos}
\addcontentsline{toc}{chapter}{Tropicalismo: o movimento dos corpos, \emph{por Flávia Cêra}}

%Texto sem referencias

\begin{flushright}
\emph{Flávia Cêra}
%\footnote{Doutora em Literatura pela Universidade Federal de
  %Santa Catarina. Psicanalista, membro da Escola Brasileira de
  %Psicanálise (\versal{EBP}) e da Associação Mundial de Psicanálise (\versal{AMP}).}
\end{flushright}

\begin{quote}
\emph{Uma das passagens mais marcantes para mim do livro \emph{Verdade
tropical}, de Caetano Veloso, é a em que ele conta do encontro, durante
sua prisão, com um capitão que apresentou uma ``sofisticada
interpretação que fazia do tropicalismo. Referiu"-se a algumas
declarações minhas à imprensa em que a palavra} desestruturar
\emph{aparecia, e, usando"-a como palavra"-chave, ele denunciava o
insidioso poder subversivo do nosso trabalho''.\footnote{Caetano Veloso. {Verdade tropical}. São Paulo: Companhia das Letras, 2008, p.~393.}
A terrível execução de
Marielle Franco me fez lembrar desse episódio contado por Caetano.
Naquela noite trágica, Marielle saía de uma roda de conversa cujo nome
era Mulheres Negras Movendo Estruturas. Mover estruturas,
desestruturá"-las, pô"-las em movimento tirando"-as da rigidez que as
pressupõe, antes como agora, é intolerável para a manutenção da
``ordem'', cada vez mais violenta, que segue em marcha no Brasil. Não
tomar isso como mera coincidência me ajuda a ler o que está em jogo
quando estamos diante de um acontecimento: a vida e a luta de Marielle
foram dessas coisas que mudam as perspectivas, que afetam nossos corpos,
que nos abrem o mundo. Este texto, apresentado antes da morte de
Marielle, é uma comemoração dos 50 anos de 1967, e fica também como uma
singela homenagem a essa mulher e à sua luta que conjugava a alegria e a
coragem de quem movia as estruturas. Que saibamos levá"-la adiante.}
\end{quote}

Não teria como falar do tropicalismo, de 1967, sem a forte presença de
um estranhamente familiar. Por um lado, porque muitos dos seus
personagens estão ativos, falando, atuando e por outro, porque o clima,
a atmosfera política de 1967 que nos sobrevoa, talvez nunca tenha deixado
de estar presente, convoca a pensar seus momentos e movimentos. Não se
trata, e quero deixar claro isso antes de mais nada, de pensar em uma
linha progressiva do tempo, embora não deixe de causar um mal"-estar o
fato de, passados 50 anos, não podermos dizer que algumas coisas, coisas
gravíssimas em um sistema político que se quer democrático, tenham
terminado. Essa viva sensação do estranhamente familiar é o que vou
tentar tecer nesse texto.

Tomarei os anos 1960 a partir da mudança definitiva que causou a posta em
cena do corpo e o movimento espantoso da aparição e desaparição dos
corpos que daí sucedeu, e que extrapola o regime simbólico. Não haveria
outro modo de retomar essa inscrição senão com Zé Celso\footnote{José Celso Martinez Corrêa. \emph{Primeiro ato: cadernos, depoimentos,
entrevistas, 1958-1974}. São Paulo: Editora 34, 1998, p.~125.} e
alguns fragmentos do já tão conhecido \emph{Longe do trópico despótico}:

\begin{quote}
Porque 68 foi, acima de tudo, uma revolução que bateu no corpo. Foi um
movimento de ruptura, de descolonização em que a decisão individual era
importantíssima. Independência ou morte. Era o corpo que arriscava, foi
o corpo que arriscou, foi o corpo que avançou, foi o corpo que foi
torturado também. E é o corpo que está até hoje sentindo o frio do
exílio, longe dos trópicos\ldots{} E a experiência da sobrevivência na noite
desses anos, sua memória, está gravada no corpo\ldots{} Qualquer análise que
se queira fazer de 68 terá que partir desse dado.

O corpo social de 68 ainda está preso. Não há anistia para ele. Ainda há
exilados e banidos, e os que ficaram só podem se exprimir caretamente.
Qualquer assunto dessa época será portanto tratado sem a sua componente
decisiva se os discursos não partirem dessa realidade física e tentarem
enquadrar as coisas em escolas, modas, rótulos de militância serão
discursos suspeitos que servirão uma vez mais para se botar a pedra
tumular em cima de uma das experiências mais ricas que o Brasil já teve
em sua história, gérmen, semente de um Brasil futuro. Com todos os erros
e desacordos, a vivência humana desse corpo social rejeitado é decisiva
para se entender tudo --- inclusive e principalmente o tropicalismo.\footnote{Ibidem.}
\end{quote}

Se estamos reunidos aqui pensando o tropicalismo, é porque alguma coisa
--- alguma memória do que aqui, talvez, muitos não tenham vivido ---,
algum sopro de vida, se escreveu em nosso corpo. Vou tentar acompanhar
alguns procedimentos e propostas que sobrevivem para tentar compor um
pouco de possível com o que posso ler na vitalidade tropicalista.

Uma das possibilidades de leitura é a da articulação entre vida e mundo,
entre corpo e arte que guarda consequências políticas profundas de uma
leitura do país feita na contramão de projetos que poderiam disputar o
poder. A ênfase tropicalista estava mais nas fissuras, nos riscos, no
corpo arriscado, que deixava em aberto o que se tentava suturar com a
moral, a família, a religião, a política, etc. Desde a inscrição ``a
pureza é um mito'' na entrada da instalação \emph{Tropicália}, de Hélio
Oiticica, o que se punha em jogo era alguma coisa dos corpos que não
podia ser imediatamente classificada, era estranho e curiosamente
circulava pela televisão. Acho que esse é um dado importante: diria que
era mais porque não se tinha muito nome, porque o que se colocava em
cena era a opacidade dos corpos, suas partes mal"-ditas que os músicos
tropicalistas puderam circular pela televisão. A operação era
sofisticada: tratava"-se mesmo de uma transmissão, de um contágio contra
a política imunitária e sanitarista que tínhamos na época. Era uma
impressão do corpo que não pode ser significada; experiência na qual o
corpo mesmo escapa do controle por seus excessos. Uma espécie de
resistência à ``luz ofuscante do poder totalizante'' e uma aposta nas
brechas, restos (a ``lixeratura'' de Rogério Duarte) capazes de
desabituar: lampejos de imaginação, linhas de fuga. Não era a
transcendência espiritual, mas sim um ``mundo imanente com suas diversas
aparições, camadas, capas, volumes, superfícies, dobras, fissuras,
arestas'', como apontou Waly Salomão sobre Oiticica.

Caetano Veloso definiu o tropicalismo como uma moda em uma conversa com
Augusto de Campos, em 1968, ao responder à pergunta do poeta e crítico
sobre o que seria aquele movimento: ``um movimento musical ou um
comportamento vital, ou ambos? Ambos. E mais ainda: uma moda. Acho
bacana tomar isso que a gente está querendo fazer com o tropicalismo.
Topar esse nome e andar um pouco com ele. Acho bacana. O tropicalismo é
um neo"-antropofagismo''.\footnote{Augusto de Campos. \emph{Balanço da bossa}. São Paulo: Perspectiva, 2008, p.~115.} Em 1969, avaliando sua
instalação Tropicália como ``nova imagem'', Hélio diz que ela se tornou
uma ``palavra"-conceito'', cujos efeitos ele não podia imaginar na época
da sua invenção. A Tropicália se tornou, diz Oiticica:

\begin{quote}
definição de um novo sentimento no panorama cultural geral, ou a síntese
de uma visão cultural específica, de diferentes campos de formas
artísticas em sua manifestação, interrelacionados em suas metas
específicas: o teatro, a música popular, o cinema além das artes
plásticas em toda as suas experiências de vanguarda no Brasil
(principalmente Rio e S. Paulo) encontraram na tropicália uma
identificação sem escopos (\ldots{}) a própria palavra hoje é usada para
definir alguma coisa muito característica, no coletivo; ela virou um
adjetivo, uma moda, cobrindo as áreas mais superficiais, mas também a
reflexão mais profunda do nosso contexto.\footnote{Hélio Oiticica. \emph{Tropicália nova imagem}. Documento 0535/69.
Consulta feita no arquivo do Projeto Hélio Oiticica, 1969a.}
\end{quote}

Então, o tropicalismo era uma nova imagem, uma moda e uma reflexão. Era
preciso vesti"-la com os Parangolés de Oiticica, andar com ela por aí,
como dizia Caetano. Silviano Santiago assinalou que a
imagem que os músicos apresentavam ao público era tão importante quanto
a linguagem. E nisso incluem"-se as roupas, as ambiguidades sexuais, os
cabelos, a dança, etc. Basta lembrar do estranhamento causado por
Caetano no \versal{III} Festival Música Popular da \versal{TV} Record, em 1967, quando ele
sobe ao palco de gola rolê com os Beat Boys argentinos, que tinham
cabelos compridos e guitarras elétricas, para cantar ``Alegria, alegria''
isso ``representava de modo gritante tudo que os nacionalistas da \versal{MPB}
mais odiavam e temiam''.\footnote{Silviano Santiago. ``Caetano enquanto superastro''. In: \emph{Uma
literatura nos trópicos: ensaios sobre dependência cultural}. São Paulo:
Perspectiva, 2000, p.~163.}

Gonzalo Aguilar (2005) ressalta a moda como chave de leitura importante
para o tropicalismo enfatizando o caráter temporal, a imitação como
procedimento e o estímulo erótico próprios da moda articulados à atuação
dos músicos na televisão. Isto causou muitos problemas já que, no
Brasil, durante a ditadura, a censura moral andou ao lado da censura
política. Carlos Fico\footnote{Carlos Fico. ```Prezada censura': cartas ao regime militar''.
\emph{Topoi -- Revista de História} (\versal{UFRJ}), n.~5, 12/2002, p.~255.} conta que uma das narrativas, entre as tantas que
tinham a finalidade de legitimar a repressão, era ``a tese de que a
`crise moral' era fomentada pelo `movimento comunista internacional' com
o propósito de abalar os fundamentos da família, desencaminhar os jovens
e disseminar maus hábitos --- sendo, dessa maneira, a ante"-sala da
subversão''. A corrosão da moral, dos bons costumes, e o desmantelamento
da família eram primeiros passos para se estabelecer uma ``revolução
comunista''. O que se via, então, com os tropicalistas eram imagens que
criavam moda, que transmitiam modos de vida que poderiam ser apropriados
no cotidiano. Os \emph{Parangolés} de Oiticica não eram outra coisa
senão a proposta de uma ``experiência mágica'', transformadora com as
capas, com as capas"-roupas, uma singular aproximação do impróprio, um
modo de com"-viver o íntimo e o estranho. Se concordamos com Emanuele Coccia que ``a roupa é um corpo
transformado em nossa própria pele, é a faculdade de transformar o
impróprio absoluto no absolutamente próprio; e, vice"-versa, de
transferir (alienar) o próprio (enquanto o que há de mais íntimo)
naquilo que lhe é absolutamente estranho'',\footnote{Emanuele Coccia. \emph{A vida sensível}.
Santa Catarina: Cultura e Barbárie,
2010, p.~84.} podemos pensar, a partir daí, uma
política do corpo, que implica o corpo, justamente na sua versão de
gozo, longe da consciência e da conscientização. Daria, assim, para pensar
o tropicalismo como um movimento anti"-pedagógico por excelência porque o
que estava em jogo era um saber"-fazer com esses pedaços de vida que
estão no corpo e que não podem ser educados e que se tentava (e ainda se
tenta), com uma ortopedia cruel, educar. \emph{Eu incorporo a revolta}
ou \emph{Estou possuído}, chamava"-se um \emph{Parangolé} de \versal{H.~O.}, que
seria ``o enigma das outras capas'', e, sobretudo, as \emph{Cosmococas}
e o corpo fragmentado que se apresentava nos textos de Oiticica,
procuravam uma experiência do desconhecido de si no outro, do
inassimilável do outro em si, no mundo e no próprio corpo. E desses
lugares saiam leituras políticas interessantíssimas. Hélio pensou a
subterrânia --- um projeto que reveria e daria continuidade à Tropicália
--- em que apresentava uma leitura do subdesenvolvimento, as ênfases no
sub, no baixo, do corpo, da política:

\begin{quote}
Um pensamento político ou a participação nascem organicamente como a
planta na planta do pé no mundo dos conceitos no do dia a dia: a luta
toda se resume na ascensão de um pensamento não opressivo, de
pensamentosações, para a absorção do que oprime: é o encosta"-na"-parede
longe da encosta, na América do Sul, no Brasil que oprime --- mar e guela
--- amerdicância tem que acabar no sul: de onde vem o mal? De dentro, de
fora? Está em nós? --- participar político é participar na vida: ser
politicamente vivo é estar vivo: aspirar à felicidade: a não"-utopia ---
(\ldots{}) --- pegar nas armas, tirar as amarras, limpar o lugar, o lazer, o
prazer de se cuspir nas medalhas (\ldots{}) --- nos subterrâneos do mundo eu
fico, por entre paredes, sob as gorgetas, embaixo da vida: 3 dias e 3
noites: o limite do desvario.\footnote{Hélio Oiticica. \emph{Subterrânia}. Documento 0494/69. Consulta feita
no arquivo do Projeto Hélio Oiticica, 1969b.}
\end{quote}

Não se tratava de criar um modo de vida para ser seguido, uma fórmula e
suas regras para ser disseminada, nem mesmo se reduzia à transgressão,
ou à provocação. Se o tropicalismo sabia dos usos dos corpos e não
ignorava que a língua afeta os corpos, não me parece que fosse apenas
para a sua exibição, mas um gesto para retirar da clandestinidade, da
bizarria o que deixa cada um sozinho com seu corpo. Era, sobretudo, um
modo de fazer com o que não cabe no esquadro e que, frequentemente,
recebe como resposta rápida e eficaz a moralização. Diria que era
colocar em cena os pontos de opacidade dos corpos, muito mais do que
revelar os segredos dos corpos, e que a questão estética, mas também
ética, era dar um destino para isso que não o desatino da repressão que
só retornava como violência, morte e silêncio e que, hoje em dia, ganha
formas e contornos sutis: a classificação desenfreada, a criação de
transtornos nas suas mais variadas formas, as tentativas assépticas de
transmissão dos saberes, a escola sem partido, por exemplo. Uma política
baseada na prevenção e no controle do ingovernável, uma tentativa de
limitar, de proibir, de censurar, de impor um modo de vida. Um plano de
homogeneização, a monocultura conduzida pela bancada \versal{BBB} (bala, bíblia e
boi). O desenvolvimento do que se considera um subdesenvolvido que anula
a diferença para fazer desenvolver"-se ao seu modo vestido com as mais
belas roupas das boas intenções, da elevação, da limpeza e do bem"-estar.
Este corte entre o bem e o mal, entre o que deve viver e o que deve
morrer, aparece como um processo natural para chegar ao desenvolvido:
dominar para eliminar as contradições, e superar para promover a
igualdade, uma unidade média, de constância, uma espécie de sossego
capaz de neutralizar a experiência em nome de uma realidade total que se
pretende sem restos, sem fora, sem outro, sem mundo.

Mas é preciso, então, fazer um esforço para tentar entender,
transportar, avivar, o tropicalismo em sua dimensão subversiva. Não me
parece algo fácil e é evidente que toda tentativa de generalização é um
fracasso, mas vou arriscar algumas hipóteses sobre os corpos que andam
por aí hoje: não é tão fácil encontrá"-los abertos à subjetivação como se
convocava às participações nas obras, por exemplo, com os Parangolés,
com as experimentações da Lygia Clark; afastou"-se definitivamente a
ideia de um corpo total pensada, por exemplo, nos Parangolés de
Oiticica; não se encontra um fio discursivo de identificações, o que,
catastroficamente, e com uma eficácia perigosa, uma parte da direita
brasileira está tentando restituir; se o inimigo de outrora tinha corpo,
hoje é o mercado, e quem é o mercado? Uma ordem sem discurso, uma ordem
com o discurso publicitário, fundada na violência e na promessa de
felicidade. Por isso que, onde aparecer um mal"-entendido, um corpo que
contenha enigma, ou seja, onde houver qualquer vazio, há de se travar
uma disputa para que se possa manter esse vazio, e a partir dele, criar
outros lugares que não esses velhos conhecidos. De outra parte, há uma
mudança de sociabilidade, nos modos de socialização, mais notadamente
visto nas redes sociais que têm consequências subjetivas e
sociopolíticas importantes que estão para ser pensadas. Há aí novos
modos de vida, novas formas de socializar, bastante diferentes das que
se conhecia. Fabián Ludueña (2016) aborda esse fenômeno como um êxodo
político do real ao virtual que traz consequências inéditas já que o
corpo é o que fica obliterado. Se, por um lado, o corpo sai de cena, o
encontro dos corpos sai de cena ou, mais precisamente, a construção do
corpo que passa pelo encontro com o corpo do Outro pode ficar suspensa,
por outro lado, entra em cena a sobredeterminação e a produção de
fantasmas e semblantes. O que se tem é um esvaziamento dos sentidos, dos
espaços, dos encontros. Ao mesmo tempo, assistimos e participamos de
batalhas (não haveria outro nome para isso) corporais intensas e cada
vez mais violentas em qualquer manifestação e, nesse sentido, podemos
dizer que há uma tendência em apagar o corpo. Não é por acaso, então,
que os movimentos de ocupação ganham a cena. Muito menos se pensarmos
que a força desses movimentos está com jovens adolescentes que, diante
do espaço vazio, do abandono desse espaço tanto discursivo quanto
físico, ocuparam, levaram seus corpos para as escolas em um gesto
bastante diferente da recomposição de um todo e de uma ordem. Eles dão
corpo à política, às suas demandas e tentam criar novos laços, saberes e
práticas; fazem frente a uma lei que só se corporifica no poder da
polícia, que só se materializa com violência. O gesto que eles sustentam
é um verdadeiro esforço de poesia.

Seria importante também pensar em algumas leituras do tropicalismo no
calor do nosso tempo. Destaco a de Nuno Ramos, um fragmento de
seu texto ``Suspeito que estamos\ldots{}''. Ele diz o seguinte:

\begin{quote}
Suspeito que o tropicalismo tenha naturalizado nossa indústria cultural
até um ponto sem retorno, e que o ciclo de conquistas democráticas
provenientes dessa operação tenha já se encerrado há décadas. Suspeito
que perceber o tiquinho de crueldade que haveria em atirar bacalhau nas
pessoas não faça mal nenhum ao país; surpreender um ríspido sargento no
modo como Ivete Sangalo dança e canta também não. Suspeito que acessar
algo de ridículo no \emph{Jornal Nacional} --- a falsa intimidade da dupla, seu
balé de rostos virando para a câmera, a ruga na sobrancelha de William
Bonner, como um aluno estudioso se preparando para começar uma prova, a
gostosíssima Patrícia Poeta descrevendo, e ainda mais com esse nome, a
chegada de um tsunami ou terremoto de nove graus na escala Richter ---
seja uma conquista nacional relevante. Suspeito, no entanto, que nessa
área caminhemos para uma verdadeira hagiografia, unilateral e coletiva
(daí o esforço, essencialmente religioso, de controlar biografias).\footnote{Nuno Ramos. ``Suspeito que estamos\ldots{}''. \emph{Folha de São Paulo}, 28/05/2014.}
\end{quote}

É uma leitura interessante que reúne uma série de outros aspectos que
viriam a reboque com o tropicalismo. É também uma leitura mais
sofisticada do que muitas críticas que o tropicalismo recebeu. Só faria
uma ressalva, ou talvez, uma pergunta: é possível controlar os efeitos
de um movimento? Do tropicalismo mais ainda porque era esse mesmo seu
princípio, seu grande fascínio e terror: ele abriu mão de qualquer
possibilidade de controle, sabia disso, jogava com isso. Apostaria no
seguinte: uma das genialidades do tropicalismo foi saber manter vivo os
espaços dos equívocos, navegar nas entrelinhas, apresentar os impasses,
daí talvez a interpretação de que eles tinham um otimismo ingênuo ou uma
cegueira em relação à realidade política, mas ao contrário, o esforço
era que se imprimisse uma poética da invenção. Nesse sentido, é mais
interessante pensá"-lo como um acontecimento do que como um movimento.
Não haveria, então, como discordar de Zé Celso, com quem comecei e agora
termino esse texto: ``o tropicalismo nunca existiu''. E nunca existiu
porque ele é uma força viva, não é um ideal ou um projeto, ele é capaz
de acontecer a cada vez que lançarmos os dados, que arriscarmos nossos
corpos carregando nossos sonhos e um tanto alegria.

\section{REFERÊNCIAS}

\begin{Parskip}
\versal{AGUILAR}, Gonzalo. \emph{A poesia concreta: as vanguardas na
encruzilhada modernista}. São Paulo: Ed\versal{USP}, 2005.

\versal{CAMPOS}, Augusto de. \emph{Balanço da bossa}. São Paulo: Perspectiva, 2008.

\versal{COCCIA}, Emanuele. \emph{A vida sensível}. Santa Catarina: Cultura e Barbárie,
2010.

\versal{CORRÊA}, José Celso Martinez. \emph{Primeiro ato: cadernos, depoimentos,
entrevistas, 1958-1974}. São Paulo: Editora 34, 1998.

\versal{FICO}, Carlos. ```Prezada censura': cartas ao regime militar''.
\emph{Topoi -- Revista de História} (\versal{UFRJ}), n.~5, 12/2002.

\versal{LUDUEÑA}, Fabián. \emph{Conversação sobre juventude e contemporaneidade}.
Curitiba, 11/2016.

\versal{OITICICA}, Hélio. \emph{Tropicália nova imagem}. Documento 0535/69.
Consulta feita no arquivo do Projeto Hélio Oiticica, 1969a.

\_\_\_\_\_\_. \emph{Subterrânia}. Documento 0494/69. Consulta feita
no arquivo do Projeto Hélio Oiticica, 1969b.

\versal{RAMOS}, Nuno. ``Suspeito que estamos''. \emph{Folha de São Paulo},
28/05/2014.

\versal{SANTIAGO}, Silviano. ``Caetano enquanto superastro''. In: \emph{Uma
literatura nos trópicos: ensaios sobre dependência cultural}. São Paulo:
Perspectiva, 2000.

\versal{VELOSO}, Caetano. \emph{Verdade tropical}. São Paulo: Companhia das Letras, 2008.
\end{Parskip}


\chapter*{Balanços da fossa: o~caso~da~\emph{Revista~Civilização~Brasileira}}
\addcontentsline{toc}{chapter}{Balanços da fossa: o caso da \emph{Revista Civilização Brasileira},\\ \emph{por Fred Coelho}}

\begin{flushright}
\emph{Fred Coelho}
\end{flushright}

\section{I}

O título deste ensaio faz um óbvio jogo de palavras com o famoso livro
de Augusto de Campos, \emph{O Balanço da bossa}. Lançado em março de
1968, trata"-se de uma coletânea de artigos do autor (alguns publicados
na imprensa paulista), além de escritos de Brasil Rocha Brito, de Julio
Medaglia e de Gilberto Mendes. Apesar dos diversos textos sobre o gênero
musical que dá titulo ao trabalho --- a bossa nova --- foram as
entrevistas de Caetano Veloso e Gilberto Gil (junto com Torquato Neto)
feitas no calor da hora de 1968 que deram longevidade ao trabalho. Na
apresentação do livro, Augusto assume duas premissas fundamentais para o
período e seus trabalhos: a ideia de \emph{linha evolutiva}, isto é, uma
perspectiva cronológica e diacrônica da música popular e das artes em
geral (ideia formulada por Caetano Veloso ainda em 1966 e contida, por
exemplo, na proposta de Paideuma dos concretos) e a \emph{Invenção} com
seus ``caminhos imprevisíveis'' como motor de seleção e apreciação dos
trabalhos analisados.

Mesmo assim, é preciso pensar porque a bossa de Augusto se torna aqui a
fossa de uma geração. Cinquenta anos depois, o balanço --- seja a caminho
do mar, seja como exercício crítico --- se torna muitas vezes negativo.
Não pelo que ficou como legado e energia do período, mas sim como
leitura retrospectiva do que não foi possível viver. A ocorrência do
golpe civil"-militar de 1 de abril de 1964 fez com que duas gerações
convivessem entre bossas e fossas, entre evoluções e perplexidades,
entre explosões de corpos e revoluções de vidas. O que farei aqui de
forma breve será uma espécie de mergulho em discursos e debates que
ocorreram entre 1964 e 1967 no âmbito da intelectualidade e de artistas
que, de alguma forma, se sentiram derrotados pelo golpe. Ou, e eis ai a
cisão geracional incontornável do momento, entre derrotados e uma nova
geração que já inicia seu trajeto criativo após o golpe.

O espaço que escolhi para essa leitura crítica foi um dos veículos mais
importantes daquele momento: a \emph{Revista Civilização Brasileira}
(\versal{RCB}). Idealizada e publicada pelo editor Ênio Silveira, a publicação
durou de 1965 até 1968 e foi um marco nos balanços entre velhas e novas
esquerdas, fazendo da própria ideia de debate um caminho de diagnóstico
desse período.

Neste brevíssimo exercício, trarei dilemas do passado para que possamos
jogar um foco de luz sobre os que vivemos no presente. A ideia é
demonstrar, a partir de um recorte mínimo, como o nosso país estava
sendo pensado dentro do debate intelectual de então. Mesmo com um parco
público letrado e leitor, a \emph{Revista} circulou em uma época em que
as definições sobre o que era ``ser brasileiro'' estavam em jogo e
atingiu, ainda em 1965, a impressionante marca de vinte mil exemplares
vendidos. Pensar a cultura, ler os teóricos e dominar as ferramentas da
crítica era um exercício de uma amplitude maior do que podemos imaginar atualmente.

\section{II}

Ao nos debruçarmos sobre os números iniciais da \versal{RCB}, vemos como a
perspectiva do ``balanço'' se faz presente. São diversos os artigos e
textos voltados para tal ideia ou para a retórica do diagnóstico, do
impasse ou da busca de respostas. No seu primeiro número, publicado em
março de 1965, ou seja, um ano depois do golpe, o editorial assinado por
Ênio e chamado de ``Princípios e propósitos'' diz que

\begin{quote}
O povo brasileiro está agora diante de um grande e sério desafio: será
capaz de, superando falhas e contradições, superar também as forças que
se opõem ao desenvolvimento do País, numa linha democrática e
independente? Será capaz de abandonar formulações meramente
especulativas e, através de estudo objetivo de todos os componentes da
realidade nacional, equacionar e depois resolver seus graves problemas?
Terá capacidade para destruir os mitos e clichês que dificultam ou
impedem aprofundamento maior desse estudo?\footnote{\emph{Revista Civilização Brasileira}, nº~1, p.~3.}
\end{quote}

Temos neste trecho palavras que são a senha para o tom da revista:
desafio, superar, abandonar, equacionar, resolver. E a saída, como
indica o autor, é justamente ``o estudo''. Cabia aos intelectuais que a
revista convocava em sua primeira fase --- oriundos em sua maioria dos
anos de 1950, como Ferreira Gullar, Nelson Werneck Sodré, Paulo Francis,
Alex Viany, Álvaro Lins, Moacir Félix, Octavio Ianni ou Florestan
Fernandes --- aprofundar as análises do que deu errado, isto é, de como
as esquerdas e demais grupos progressistas foram derrotados pelos
militares e seus aliados. Vale lembrar que em 1965 ainda não havia no
horizonte a permanência de vinte e cinco anos do novo regime. De certa
forma, ao mesmo tempo em que tal situação de derrota e esparança deixava
os textos de então mais agudos em suas críticas e visavam a retomada do
estado democrático, ela também permeia cada frase com a tinta da
melancolia. Citando ainda o texto referido de Ênio Silveira, ``o golpe
de abril, sendo mero episódio da crise crônica em que nos encontramos,
certamente dificulta, mas por isso mesmo estimula, abre novas
perspectivas e torna inadiável a tarefa que lhes cabe executar''. Para
fechar, o editor anuncia que a \versal{RCB} seria, justamente, ``o veículo em que
esses estudos e pesquisas da realidade nacional serão divulgados''. De
alguma forma, a revista se torna o balanço da fossa por excelência.

Se fizermos um apanhado de alguns dos títulos de artigos que seguem esse
caminho, a lista se torna grande. Apresento aqui alguns para, em
seguida, me deter em momentos emblemáticos do que saiu entre os anos de
1965 e 1967. Já no número 1, de março de 1965, temos ``Política externa
independente: um balanço'' (sem autor); ``Obstáculos políticos ao
crescimento econômico no Brasil'' (Celso Furtado); ``Teatro em 1964: um
balanço'' (A. Veiga Filho); ``Porque parou a arte brasileira'' (Ferreira
Gullar) e ``Música popular: novas tendências'' (Nelson Lins e Barros).
No número 2, de maio de 1965, temos ``1º aniversário do golpe: quem deu,
quem levou, reações possíveis'' (Paulo Francis) e ``A revolução
brasileira e os intelectuais'' (Florestan Fernandes). No número 3, de
julho de 1965, temos ``Brasil de hoje: problemas do futuro com homens do
passado'' (novamente Celso Furtado) e ``Confronto: música popular
brasileira'' (com José Ramos Tinhorão, Luis Carlos Vinhas e Edu Lobo).
No número 4, de setembro de 1965, temos ``Condições e perspectivas da
política brasileira'' (sem autor) e ``Cultura popular: esboço de uma
resenha crítica'' (Sebastião Uchoa Leite).

Vale lembrar que em 1965 Ferreira Gullar publica pela editora
Civilização Brasileira seu livro \emph{Cultura posta em questão}. Pelo
título, vemos que segue o tom de questionário do presente. Gullar
participou intensamente da revista e estava inserido na perspectiva de
balanços e derrota, principalmente por ter sido membro ativo do \versal{CPC} da
\versal{UNE} e, logo depois, do teatro Opinião. No prefácio da primeira edição,
assinado por Leandro Konder, lemos que

\begin{quote}
Erguem"-se hoje, diante de nós, no caminho do desenvolvimento, que apenas
começamos a trilhar, poderosos obstáculos, preconceitos profundamente
enraizados, interesses feridos, privilégios que não querem morrer. Mas
nada disso consegue impedir que certas perguntas se imponham a um
interesse cada vez mais geral.\footnote{Ferreira Gullar. \emph{Cultura posta em
questão: vanguarda e subdesenvolvimento.} Rio de Janeiro: José Olimpyo, 2002, p.~16.}
\end{quote}

Já o próprio Gullar, ainda na introdução do volume, corrobora tal
perspectiva ao afirmar que

\begin{quote}
Vivemos uma época de urgência e, se essa urgência não justifica a
leviandade, impõe um comportamento novo, que é preciso assumir. Faz
parte desse comportamento compreender que só o diálogo aberto e o
esforço comum de pensamento nos permitirão formular e responder às
questões que o momento coloca. Com este livro, pretendemos contribuir
para o debate.\footnote{Ibidem, p.~17.}
\end{quote}

Neste trecho de Gullar, vale reter tanto a ideia de \emph{urgência} de
uma época, quanto a proposta de um ``comportamento novo''. Como sabemos,
meio século depois, o novo chegaria dois anos depois com o chamado
tropicalismo, mas não exatamente como Gullar ou Konder imaginavam.

\section{III}

\begin{quote}
Numa sociedade plenamente desenvolvida o intelectual pode se dar o luxo
de virar a cara e compor sua obra. Ele confia na divisão do trabalho.
Outros estão, mais especificamente, cuidando do país. Um país como o
Brasil é como uma família pobre em que todos fazem de tudo e seria o
cúmulo que exatamente os intelectuais se alheassem (se alienassem, como
se diz agora) da tarefa principal. A crítica pura e simples, ainda que
valorosa, é inútil diante de um espetáculo tão ruim como o Brasil. Nem
dá gosto.\footnote{\emph{Revista Civilização Brasileira}, nº~9-10, p.~338.}
\end{quote}

Esse trecho foi escrito por Antonio Callado em 1966. Com dois anos do
golpe civil"-militar nas ruas, Callado publicava na \emph{Revista
Civilização Brasileira} a resenha sobre o livro de Paulo Francis,
\emph{Opinião pessoal}, e era mais um a deixar claro os dilemas do
intelectual brasileiro daquele período. A ideia do ``engajamento''
político através da produção cultural tornava"-se uma das principais
posturas hegemônicas no país durante a primeira metade dos anos 1960.
Intelectuais de qualquer campo se deparavam em seu cotidiano com o
posicionamento necessário frente a esse problema. A omissão do debate
político"-sociológico e da reflexão sobre o país era cobrada, como mostra
o trecho citado. O resultado da omissão era o estigma da alienação.

Essa hegemonia de um engajamento com perfis pedagógicos e com a
instrumentalização distanciada de ideias como ``povo'' e ``popular'',
porém, começaria a ser questionada a partir de 1964, por intelectuais
que, ao contrário da tradição ensaísta do país, propunham projetos e
ações de áreas estranhas à crítica, à literatura e ao ensaísmo. Glauber
Rocha, por exemplo, se torna uma das forças dessa renovação intelectual
por mudar o eixo das pautas e do debate no período. Era um jovem diretor
que não abria mão do texto (publicava tanto sobre a história do cinema
quanto sobre questões geopolíticas da América Latina) e, principalmente,
da política. Ele radicaliza a postura crítica anti"-colonial no terceiro
mundo, e no Brasil principalmente, e recoloca em novas bases as leituras
acerca de uma arte política. Suas falas e seus escritos não apresentam
visões lastreadas pelos estudos filosóficos e sociológicos do \versal{ISEB} ou
pelas caravanas populares do \versal{CPC} da \versal{UNE}. Para o cineasta, naquele
momento de balanços e cobranças, o dito ``povo brasileiro'' é real,
sente fome e vive no limite da revolta social. Eram os responsáveis pela
criação de uma cultura subversivamente popular sem mediação do
intelectual. A urgência desse povo, desse ponto de vista, era maior que
as dos pensadores.

Trago Glauber Rocha porque foi na mesma \emph{Revista Civilização
Brasileira}, em sua edição de número 3, julho de 1965, que ocorreu a
primeira publicação de ``A estética da fome'', ensaio já amplamente
visitado pelos estudos ao redor do cineasta. Ele marcou o período como a
primeira --- ou a mais visível e contundente --- manifestação de um
intelectual brasileiro frente aos dilemas da relação
colonizador/colonizado. Isso ocorreu quando ``Revolução'' era palavra
corrente em artigos, livros, debates e conspirações nos bares e redações
do país. Glauber expõe em um discurso direto e sintético a
situação"-limite do intelectual brasileiro (e de todo terceiro mundo):
como viver das aspirações, teorias, projetos e decisões das instâncias
externas --- sejam elas comitês da internacional socialista, empresas de
cinema, produtores endinheirados ou gurus intelectuais --- sem sermos
vistos pelo verniz aprisionante do primitivismo ou do ``exotismo''? Como
criar uma cultura afirmativa e nacional, revolucionária e brasileira, se
libertando de premissas que não as nossas, de problemas que não são os
nossos, de uma ideia de ``povo'' que não é o nosso?

As ameaças para a reflexão do intelectual brasileiro se materializam em
duas posturas que, para Glauber, eram falhas e nocivas naquele momento:
a \emph{esterilidade} e a \emph{histeria}. O pensamento brasileiro dos
anos 1950 e 1960 navegava nas águas muitas vezes plácidas do
nacional"-desenvolvimentismo, cuja crença na revolução se dava através da
aliança desenvolvimentista de superação do subdesenvolvimento entre um
operariado esclarecido, um campesinato engajado nas causas populares e
uma burguesia progressista que financiaria e informaria todo o processo
revolucionário. Nesse contexto, o diagnóstico de Glauber demonstrava que
nossos intelectuais se defrontavam com a esterilidade do empenho em
exercícios formais ou meramente comerciais e com a histeria romântica do
anarquismo da ``poesia jovem'' e da indignação vazia, causadora de
equívocos como ``a procura de uma sistematização para a arte popular''
em que ``mais uma vez o paternalismo é o método de compreensão para uma
linguagem de lágrimas ou de mudo sofrimento''. Na esterilidade, o sonho
frustrado da universalização subserviente; na histeria, o devastador
esforço crítico pedagógico para superar a impotência de criar em um país
miserável como o Brasil. A grosso modo, Concretismo e \versal{CPC} eram os dois
modelos paradigmáticos que Glauber combatia. O primeiro, priorizando a
estética pura, a forma como esteio revolucionário. O segundo, propagando
a crença no realismo socialista em que ``povo'', ``burguesia'' e
``classes'' tinham papéis pré"-definidos no desenrolar teleológico da
história rumo à revolução socialista.

Sem meio tom, ``Estética da fome'' é um texto lido em Gênova, Itália,
para a \emph{consciência do outro}, isto é, para o \emph{Colonizador.}
Glauber falava diretamente com os produtores de exotismos, fato raro na
inteligência do país, endossando a violência como ação positiva de
transformação social e como saída, mesmo que limite, para o dilema do
primitivismo. Um engajamento que não pressupõe modelos pré"-estabelecidos
de ação e proposta, mas sim que \emph{marca uma posição}. O intuito não
é mais conscientizar as classes, mas sim intervir através da ação direta
do intelectual e do artista frente a esse dilema.

\section{IV}

Em julho de 1966, a \versal{RCB} apresentou em seu número 7 mais um dos seus
``debates''. Em seu título, permanece ressoando impasses e perguntas de
quem se viu desnorteado pela história. Aos poucos, porém, novos
elementos no campo da arte e da cultura vão obrigando intelectuais e
artistas a ampliar a pauta dos balanços necessários. Como sabemos, as
artes visuais, o cinema, o teatro e, principalmente, a música popular
iniciam seu processo de ruptura com a pauta política do realismo
socialista ou do engajamento cepecista. São obras que passam a inserir o
dado da complexidade entre as polaridades pré"-definidas da Guerra Fria e
impossibilitam \emph{a priori} o maniqueísmo como posicionamento político. O
debate em questão é o famoso ``Que caminhos seguir na música popular
brasileira''. O time de debatedores é amplo e eclético: Flavio Macedo
Soares, Nelson Lins e Barros, Nara Leão, Gustavo Dahl, Ferreira Gullar,
além dos jovens músicos baianos recém"-chegados ao sudeste, José Carlos
Capinam e Caetano Veloso. Ambos, aliás, encontravam"-se em plena gestação
de algo que só em 1967 ganharia corpo definitivo, mas que já se
insinuava em suas palavras. O debate, conduzido por Airton Lima Barbosa,
é iniciado com um pequeno texto que, mais uma vez, corrobora o tom de
balanço e fossa. Cito: ``Em virtude da crise atual da música popular
brasileira, a \versal{RCB} reuniu músicos, compositores, intelectuais e
estudiosos de música popular para um debate''.

A ``crise'' da música popular diagnosticada pelos editores da revista
não vislumbravam nem de longe o que ocorreria no ano seguinte. Vale
pensarmos que, na verdade, o que estava em crise era todo um modelo de
pensamento nacionalista"-popular no âmbito das velozes transformações de
uma cultura de massas mundializada, e não propriamente a música popular.
Vale lembrar também que a própria ideia de popular estava em transição
dramática para as gerações pré"-golpe, saindo de sua perspectiva
folclórica"-romântica para uma perspectiva televisa e pop. Afinal, em
1966 a Jovem Guarda de Roberto, Erasmo e Wanderléa já era uma realidade
incontornável.

Esse debate se tornou marcante também porque é nele que Caetano Veloso
formula a sentença que se torna um marco crítico de sua postura
pré"-tropicalista, isto é, a ideia de uma ``retomada da linha evolutiva''
na música popular a partir da experiência de João Gilberto em 1959 e sua
capacidade de aliar a informação da modernidade musical de seu tempo e a
renovação dos padrões tradicionais, sem abandoná"-los como passado vazio.
O que Caetano reivindica, e que certamente fez Augusto de Campos se
apropriar imediatamente dessa ideia, é uma \emph{possibilidade seletiva}
do repertório da tradição visando a criação como invenção de uma nova
informação. Citando o músico, ``se temos uma tradição e queremos fazer
algo de novo dentro dela não só teremos de senti"-la, mas conhecê"-la. E é
este conhecimento que vai nos dar a possibilidade de criar algo novo e
coerente com ela''. Na frase seguinte, ele arremata dando o contexto da
famosa sentença: ``Só a retomada da linha evolutiva pode nos dar uma
organicidade para selecionar e ter um julgamento de criação''. Ou seja,
a ideia de uma linha evolutiva, apesar de apontar o dado diacrônico e
teleológico da história, é também um processo seletivo que, de alguma
forma, recorta a mesma história em seus momentos que interessam à tal
ideia de evolução. Ela é muito mais uma operação crítica do que um
destino inexorável da qualidade intrínseca da canção popular.

Em outro trecho do debate, Caetano, de alguma forma, aponta a mudança
que sua geração passava em relação a textos como os de Gullar ou
Callado, citados aqui anteriormente. O jovem músico, com 24 anos na
ocasião, já afirmava que não tinha ilusões sobre um ``povo''
unidimensional e necessitado de conscientização política, nem achava que
deveria ser um intelectual com a missão de resolver problemas nacionais.
Citando mais uma vez Caetano, ``sei que a arte que faço agora não pode
pertencer verdadeiramente ao povo brasileiro. Sei também que a arte não
salva nada nem ninguém, mas que é uma das nossas faces''. Por fim,
reafirma a operação crítica como dado inexorável de sua acão artística:
``Me interessa que corresponda o que faço à posição tomada por mim
diante da realidade brasileira''. Uma simples mudança de um ``nós''
pedagógico e culpado para um ``eu'' autônomo e crítico, desembocaria um
ano depois nas transformações que seriam batizadas de tropicalismo.

\section{V}

Os números da \versal{RCB} publicados em 1967 nos mostram, de alguma forma, como
essa visada renovadora de Caetano Veloso irradiava"-se pelas pautas da
publicação. Em dois anos, o tema do ``balanço'' e o diagnóstico das
derrotas é deslocado para a análise dos novos problemas que o país e o
mundo vão adentrando no período. A revista, aliás, se internacionaliza
muito mais e amplia os autores e temas. Em seus cinco números (o
primeiro, 11-12, acaba compilando o último de 1966 e o primeiro de 1967,
demonstrando as dificuldades enfrentadas pelos editores), vemos o
surgimento de artistas como Rubens Gerchman e Antonio Dias, artigos de
Juliet Mitchel dedicados à revolução feminista, textos voltados para o
problema do racismo no Estados Unidos ou trabalhos seminais de Susan
Sontag como ``Marat, Sade, Artaud''. Há também a renovação do pensamento
das esquerdas com textos de Lucien Goldman, Hobsbawn e Althusser. Vale
lembrar que o período também correspondente ao incrudescimento da
censura oficial do regime militar, que cada vez mais demonstrava apetite
para permanecer no poder. O próprio Ênio Silveira, por exemplo, é preso
sete vezes entre 1964 e 1969. Sua trajetória de resistência, suas
escolhas editoriais e sua filiação política ao \versal{PCB} o transformaram em
alvo constante de processos por parte dos militares. Com sua revista,
não seria diferente.

Mesmo abrindo novas frentes a partir de 1967, porém, seus editores
permanecem fiéis à ideia de que apenas a rebeldia, para retomar o ensaio
de Leandro Konder publicado justamente no número 15, de setembro de
1967, não desembocaria na superação da fossa. No editorial do número 13,
comemorando dois anos da publicação, há o reconhecimento das
transformações, mas há também a necessidade de seguirem a \emph{Tarefa
de} ``organizar o movimento''. Assumem que há uma realidade brasileira
ainda em jogo, mas agora ``sempre em movimento''. Apesar de entenderem
cada vez mais as transformações fora dos esquematismos típicos da geração
que formara a primeira fase da revista, insistiam que o intelectual
tinha uma missão. Citando o editorial,

\begin{quote}
O que é importante é não esquecer que sem indagar, a qualquer preço,
pela verdade dessas realidades, a função do intelectual perde a sua
capacidade criadora e desce ao nível dos atos em que o homem avilta em
si a humanidade inteira ao aviltar"-se na consciência de um definido
dever que o redima. Tarefa crescentemente difícil, o importante é que
isto é o que viemos tentando --- com muitos tropeços, dúvidas e erros ---
durante esses dois anos de luta em um período que toda a Nação conhece,
e durante qual o manifesto apoio de camadas sociais mais lúcidas do
nosso país foi o principal alento que tivemos.\footnote{\emph{Revista Civilização Brasileira}, nº~13, p.~3.}
\end{quote}

Ainda aqui, ressoa a tarefa, o dever e a missão coletiva do intelectual.
A transformação apontada por Caetano de um ``nós'' para um ``mim'', de
um ``Vamos'' para um ``vou'' ou do popular para o pop, não seria
imeditamente aceita. Talvez não tenha sido até hoje para alguns setores.
Mas não deixa de ser sintomático que, em 1968, o tema da cultura de
massas e suas armadilhas imperialistas tenha sido constante nos últimos
números da revista.

Concluindo: Roberto Schwarz, em 1987, afirmava no ensaio ``Nacional por
subtração'' que ``a cada geração a vida intelectual brasileira parece
começar do zero''. Em 1967, três anos após o golpe, o balanço da fossa e
a renovação da mesma por outra geração assume a sugestão de Schwarz.
Começar do zero não como terra arrasada, mas como necessidade de
renovação crítica por conta de transformações radicais que, antes de
1964, apareciam cirstalinas em sua visada política e cultural.

Em suas ``escritas políticas'' do \emph{Grau zero de escrita,} Roland
Barthes afirma que ``Não há dúvida que cada regime possui a sua escrita,
cuja história ainda está por fazer''. Entre 1964 e 1967, o novo cenário
político brasileiro colocou para os seus intelectuais de forma drástica
e traumática novos regimes de escrita e de pensamento crítico. E se a
história de tais regimes sempre estará por se fazer, o balanço de sua
fossa serve de alerta dentre as novas gerações cuja leitura
retrospectiva pode cada vez mais se aproximar daquele momento dramático
pela lente absurda de nosso presente. Talvez, se ainda há missões para
intelectuais, a mais preemente é entender que urgências e diganósticos
precários se tornaram parte de nosso cotidiano na luta contra a linha
evolutiva do conservadorismo brasileiro.

\section{Referências}

\begin{Parskip}
\versal{BARTHES}, Roland. ``Escritas políticas''. In: \emph{O grau zero da
escrita}. São Paulo: Martins Fontes, 2000.

\versal{GULLAR}, Ferreira. \emph{Cultura posta em questão: vanguarda e
subdesenvolvimento}. Rio de Janeiro: José Olimpyo, 2002.

\versal{SCHWARZ}, Roberto. ``Nacional por subtração''. In: \emph{Que horas são?.}
São Paulo: Companhia das letras, 1987.

\emph{Revista Civilização Brasileira}, números 1-13 (1965-1967).
\end{Parskip}


\chapter*{1967: exercícios experimentais de liberdade}
\addcontentsline{toc}{chapter}{1967: exercícios experimentais de liberdade, \emph{por Imaculada Kangussu}}

\begin{flushright}
\emph{Imaculada Kangussu}
\end{flushright}


\section{1967, ano da \emph{Tropicália}}

Em 1967, Hélio Oiticica apresenta \emph{Tropicália} na exposição ``Nova
Objetividade Brasileira'', realizada no Museu de Arte Moderna do Rio de
Janeiro. \emph{Tropicália}, o termo criado por Hélio, dá nome à música
de Caetano Veloso considerada como ``música inaugural'' e ``matriz
estética'' do tropicalismo, por Celso Favaretto, para quem a canção
configura um painel histórico, uma ``metaforização'' do Brasil, e, em
suas palavras,

\begin{quote}
desenha uma situação contraditória, um contexto em desarticulação,
presentificando as indefinições do país, em que indiferenciadamente
convivem os traços mais arcaicos e os traços mais modernos. Com uma
operação de bricolagem, o Brasil emerge da montagem sincrônica de fatos,
eventos, citações, jargões e emblemas, resíduos, fragmentos.\footnote{Celso
Favaretto. \emph{Tropicália, alegoria, alegria.} Cotia: Ateliê, 2007,
  p.~63. Este ensaio, como também o interesse filosófico que desenvolvi
  pela obra de Hélio Oiticica, foram tão profundamente influenciados
  pela leitura dos escritos de Celso Favaretto que qualquer
  agradecimento se faria supérfluo.}
\end{quote}

A canção foi gravada no disco também nomeado \emph{Tropicália ou Panis
et circencis}, visto por Celso Favaretto como a ``suma tropicalista'',
fruto do trabalho coletivo do ``grupo baiano'': Caetano, Gilberto Gil,
Gal Costa, Torquato Neto, Capinan, Mutantes, Rogério Duprat, Tom Zé e
Nara Leão.

\emph{Tropicália}, o nome, chegou a Caetano Veloso indiretamente,
através da mediação de Luiz Carlos Barreto, diretor de fotografia de
\emph{Vidas secas} e \emph{Terra em transe}, entre outros, que, ao ouvir
a canção ainda sem título, conta seu autor,

\begin{quote}
sugeriu ``Tropicália'', por causa, dizia ele, das afinidades com o
trabalho do mesmo nome apresentado por um artista plástico carioca, uma
instalação (na época ainda não se usava o termo, mas é o que era) que
consistia num labirinto ou mero caracol de paredes de madeira, com areia
no chão para ser pisada sem sapatos, um caminho enroscado, ladeado de
plantas tropicais, indo dar, ao fim, num aparelho de televisão ligado,
exibindo a programação normal. O nome do artista era Hélio Oiticica, e
era a primeira vez que eu o ouvia.\footnote{Caetano Veloso.
  \emph{Verdade tropical.} São Paulo: Companhia das Letras, 1997, p.~188.
  \emph{Tropicália}, escreve o autor"-cantor, ``justificou para mim a
  existência do disco, do movimento e de minha considerável dedicação à
  profissão que ainda me parecia provisória: era o mais perto que eu
  pudera chegar do que me foi sugerido por \emph{Terra em transe}'' (Ibidem, p.~187). Vale lembrar que o filme de Glauber também é de 1967.}
\end{quote}

A obra \emph{Tropicália}, cuja descrição Caetano ouviu, é um ambiente
formado por caminhos --- de areia e pequenas pedras de brita ---
ornamentados com plantas, poemas e até araras, através dos quais o
público pode chegar a dois \emph{Penetráveis}, pequenos barracões, cuja
estrutura remete àquela das construções feitas na favela, nos quais as
pessoas são encorajadas a \emph{penetrar}: no penetrável \versal{PN}2
(\emph{Pureza é um mito}) e no \versal{PN}3 (\emph{Imagético}), onde uma
televisão permanece ligada transmitindo a programação em tempo real. A
forma da obra revela"-se apenas a quem a percorre, exige a experiência
espaço"-temporal, é impossível abarcar todas as suas partes
simultaneamente com o olhar. Hélio Oiticica fala sobre a experiência de
\emph{Tropicália}, em entrevista de 15 de maio de 1968, a Mário Barata:

\begin{quote}
O resultado, para mim, foi de absoluto sucesso quanto às possibilidades
e às ocorrências aí verificadas: para entrar em cada \emph{Penetrável},
era o participador obrigado a caminhar sobre areia, pedras de brita,
procurar poemas por entre as folhagens, brincar com araras etc. --- o
ambiente criado era obviamente tropical, como que num fundo de chácara,
e, o mais importante, havia a sensação de que se estaria de novo
\emph{pisando a terra}. Esta sensação, sentia eu anteriormente ao
caminhar pelos morros, pela favela, e mesmo o percurso de entrar, sair,
dobrar ``pelas quebradas'' da \emph{Tropicália}, lembra muito as
caminhadas pelo morro.\footnote{Hélio Oiticica. \emph{Aspiro ao grande
  labirinto}. Rio de Janeiro: Rocco, 1986, p.~99.}
\end{quote}

A sensação de estar ``de novo \emph{pisando a terra}'' revela"-se
essencial quando se busca um descondicionamento social, diz o artista
na mesma entrevista. Segundo Hélio, dois elementos importavam na
\emph{Tropicália}: o primeiro, o trabalho do artista de criar um
ambiente para provocar experiências sensoriais distintas daquelas
vividas no prosaísmo cotidiano; o segundo referia"-se ao próprio
comportamento do público baseado na experiência do contato direto com
tal ambiente. Pode"-se assim perceber \emph{Tropicália} como
\emph{objeto} voltado a levar o \emph{sujeito} a experiências físicas,
sensoriais, distintas daquelas cotidianas. O objeto buscado por Hélio
não é mais a obra acabada, e sim o que traz em si uma proposta relativa
à participação do sujeito.

Propositadamente não tecnológica, a ambientação visava a volta à terra,
em um exercício experimental para além da absorção do sujeito pela
avalanche de informações e imagens contemporâneas. Ao justapor elementos
extremos, a terra a ser pisada e o aparelho de televisão, por exemplo, o
artista fixa contradições em uma síntese imagética e, sem buscar
superá"-las, evita a fixação de uma realidade fechada, definida e
definitiva. \emph{Tropicália} é uma obra de transformação, pela qual,
declara Oiticica, deseja"-se chegar ``à pura disponibilidade criadora, ao
lazer, ao prazer, ao mito de viver, onde o que é secreto agora passa a
ser revelado na própria existência, no dia a dia''.\footnote{Ibidem,
  p.~100.}

A intenção assumida pelo artista é a de compor uma objetivação da imagem
brasileira, visando derrubar o mito universalista então adotado nas
manifestações da vanguarda nacional. Seu desejo é objetivar uma imagem
brasileira através da ``devoração'' de símbolos da cultura nacional.
\emph{Tropicália} propõe a apresentação de uma imagem brasileira em
linguagem universal.

A proposta foi bastante bem sucedida. \emph{Tropicália} foi escolhida
para inaugurar uma nova galeria da Tate Modern, em Londres, em 17 de
junho de 2016, quase 50 anos (nos quais houve intensa e variadíssima
produção de arte contemporânea) depois de sua primeira apresentação.
Como parte do acervo da Tate collection (desde junho de 2007), a obra
permanece em exposição até dezembro de 2018. Segundo notícia publicada
no jornal \emph{The Guardian}, nos três primeiros dias da exposição, a
Tate Modern recebeu 143 mil visitantes, o dobro do número usual, o que
levou inclusive à retirada das araras.\footnote{Mark Brown. ``Tate
  modern removes macaws as visitor numbers soar''. \emph{The
  Guardian}, 20/05/2016.} Em 2017, o Whitney Museum, em Nova York, para
inaugurar suas novas instalações, também exibiu \emph{Tropicália},
dentro da grande retrospectiva \emph{Hélio Oiticica: To Organize
Delirium}.

Considerado um dos artistas mais originais do século \versal{XX}, Oiticica
realiza trabalhos que acordam o corpo, os sentidos, os sentimentos, e
convidam a uma posição mais ativa: transformam o espectador em
participante, registra o catálogo da exposição. ``O mito da
tropicalidade é muito mais do que araras e bananeiras'', escreve o
artista, ``é a consciência de um não condicionamento às estruturas
estabelecidas, portanto altamente revolucionário em sua totalidade.
Qualquer conformismo, seja intelectual, social, existencial, escapa à
sua ideia principal''.\footnote{Hélio Oiticica. \emph{Aspiro ao grande
  labirinto}, op.~cit., p.~109.} A proposta foi considerada uma ``estética do
envolvimento''.\footnote{Benedito Nunes. \emph{Lygia Clark e Hélio
  Oiticica}. Rio de Janeiro: Funarte, 1987, p.~42.}

\emph{Tropicália}, tanto o conceito quanto a obra têm desdobramentos
essenciais nas obras chamadas ``tropicalistas''. Ambas apresentam
aspectos distintos, e mesmo discrepantes, das produções culturais,
desconstruindo qualquer linguagem voltada à totalização. Os processos e
os procedimentos são mais importantes que os temas. ``O tropicalismo
evidenciou o conflito das interpretações do Brasil, sem apresentar um
projeto definido de superação dos antagonismos; expôs a indeterminação
da história e das linguagens, devorando"-as'', observa Favaretto,
``ressituou os motos da cultura urbana e industrial, misturando os
elementos arcaicos e modernos, ressaltando os limites das polarizações
ideológicas no debate cultural em curso''.\footnote{Celso Favaretto. \emph{A
  invenção de Hélio Oiticica}. São Paulo: Ed\versal{USP}, 2000, p.~100.}

Na ausência de privilégios valorativos entre produções culturais
antagônicas, na indistinção entre experimentalismo radical e crítica da
cultura, Oiticica percebe um caráter revolucionário implícito nas
criações dos músicos tropicalistas. Em sua perspectiva, o tropicalismo
não visa ser mais um movimento isolado, mais um ``ismo'', e sim ``um
estado geral cultural'' capaz de unir díspares como Zé Celso, os poetas
concretos paulistas, Glauber Rocha, Oswald de Andrade, e mais. Os
músicos baianos, considera o artista, ``promoveram a maior tarefa
crítica de nossa música popular, inclusive cabe a eles a tarefa da
desmistificação, na música, do bom gosto como critério de
julgamento''.\footnote{Hélio Oiticica. ``O sentido de vanguarda do grupo
  baiano''. \emph{Correio da manhã}, 24/11/1968.}

\section{1967, ano do supra"-sensorial}

Em dezembro de 1967, Hélio Oiticica escreve um breve ensaio onde
apresenta o conceito, que vinha desenvolvendo desde o início da década
de 1960, o de \emph{supra"-sensorial}.\footnote{Hélio Oiticica. ``Aparecimento do
  supra"-sensorial na arte brasileira''. In: \emph{Aspiro ao grande
  labirinto}, op.~cit.} O \emph{supra"-sensorial} proposto é a tentativa de criar
obras, para além das bases tradicionais da pintura e da escultura, que
culminem em objetos abertos à mobilidade. Em outras palavras, a proposta
fundamental das obras ligadas ao supra"-sensorial consiste em criar novas
condições de experiências, através de proposições abertas; e em provocar
novas formas de percepção, através do envolvimento participativo do
espectador. Com isso, a posição de espectador passivo é transformada na
de participante ativo. Oiticica tinha em vista ultrapassar o objeto como
finalidade última da expressão estética que, a partir dessa perspectiva,
é o meio e passa a ter como fim experiências voltadas ao comportamento
participante do público. As proposições lançam mão de qualquer objeto ---
caixas, palavras, ambientes --- de olho na resposta do participador,
tendo como alvo comum transformar o participante através da experiência
sensorial.

O supra"-sensorial visa ir além do movimento de estímulo e reação,
próprio da arte e limitado ao sensível, e abrir o participante para o
dilatamento da sensibilidade mesma (como acontece em estados
alucinógenos induzidos por drogas), e também da consciência, através de
experiências provocadas pelo objeto capaz de levar a um novo modo
perceptivo, a uma transformação na percepção. Não se trata apenas de
criar uma estética do objeto ou do ambiente --- que permanecem
importantes e mesmo fundamentais --- e sim de lidar com a objetividade
das proposições.

O desenrolar da ideia do supra"-sensível tem como antecedente a proposta
da ``Nova Objetividade'', voltada à produção de objetos capazes de
abarcar proposições criativas abertas ao exercício imaginativo dos
indivíduos. Trata"-se de um contraponto no diálogo com o ``Novo
Realismo'', proposto por Schemberg, cuja caracterização mais ampla
inclui Gerchman (``realismo do fato significativo''), Antonio Dias
(``realismo do absurdo''), Wesley Duke Lee (``realismo estrutural''),
Sérgio Ferro e Waldemar Cordeiro (``realidade"-imagem'').\footnote{Celso Favaretto.
  \emph{A invenção de Hélio Oiticica}, op.~cit., p.~158-161. Cf. Hélio Oiticica.
  ``Situação da vanguarda no Brasil''. In: \emph{Aspiro ao grande
  labirinto}, op.~cit.; Mario Schemberg. ``Um Novo Realismo''. In: \emph{Objeto na arte: Brasil anos 60}. São Paulo: \versal{FAAP}, 1978; Waldemar
  Cordeiro. ``Realismo: musa da vingança e da tristeza''. In: \emph{Objeto na arte: Brasil anos 60}, op.~cit.} O ponto crucial de distinção entre a
``Nova Objetividade'' e o ``Novo Realismo'' é a ideia de ``arte
participante''. Conforme pode"-se ler no ``Esquema geral da Nova
Objetividade'', ao artista cabe propor objetos capazes de provocar a
participação, de modo a operar transformações na percepção e na
consciência do público, ``que de espectador passivo dos acontecimentos
passaria a agir sobre eles, usando os meios que lhe coubessem: a
revolta, o protesto, o trabalho construtivo para atingir a essa
transformação, etc.''\footnote{Hélio Oiticica. ``Esquema geral da Nova
  Objetividade''. In: \emph{Aspiro ao grande labirinto}, op.~cit., p.~94.} Na
experiência estética proposta, é o participante quem articula os
elementos colocados pelo artista.

Na situação opressiva imposta pelo regime autoritário e violento da
ditadura militar, desde o dramático golpe de 1964, Oiticica adverte:

\begin{quote}
Hoje, o que quer que se faça, qualquer que seja nossa \emph{démarche},
se formos um grupo atuante, realmente participante, seremos um grupo
\emph{contra} coisas, argumentos, fatos {[}\ldots{}{]} No Brasil (nisto
também se assemelharia ao Dadá) hoje, para se ter uma posição cultural
atuante, que conte, tem"-se que ser contra, visceralmente contra tudo que
seria em suma o conformismo cultural, político, ético, social.

Para finalizar, quero evocar ainda uma frase que, creio, poderia muito
bem representar o espírito da Nova Objetividade {[}\ldots{}{]} ei"-la: \versal{DA
ADVERSIDADE VIVEMOS!}\footnote{Ibidem, p.~98.}
\end{quote}

E do conceito de Nova Objetividade, criado em 1966, nasceu a ideia de
\emph{Tropicália}, revela o artista ao retomar o conceito, em dezembro
de 1967:

\begin{quote}
O conceito de Nova Objetividade não visa, como pensam muitos, diluir as
estruturas, mas dar"-lhes um sentido total, superar o estruturalismo
criado pelas proposições da arte abstrata, fazendo"-o crescer por todos
os lados, como uma planta, até abarcar uma ideia concentrada na
liberdade do indivíduo, proporcionando"-lhe proposições abertas a seu
exercício imaginativo, interior --- esta seria uma das maneiras,
proporcionada neste caso pelo artista, de desalienar o indivíduo, de
torná"-lo objetivo no seu comportamento ético"-social. O próprio ``fazer''
da obra seria violado, já que o verdadeiro ``fazer'' seria a vivência do
indivíduo.\footnote{Hélio Oiticica. ``Aparecimento do supra"-sensorial na arte
  brasileira'', op.~cit., p.~103.}
\end{quote}

O artista deixa bem claro que não visa um novo condicionamento para o
público, como acontece às vezes com a chamada ``arte política'', voltada
a substituir um condicionamento por outro, e sim à diluição de todo
condicionamento. Ou ainda, a derrubada de todo condicionamento para
abrir espaço à busca, e talvez ao encontro, da liberdade individual,
``através de proposições cada vez mais abertas visando fazer com que
cada um encontre em si mesmo, pela disponibilidade, pelo improviso, sua
liberdade interior, a pista para o estado criador''.\footnote{Ibidem,
  p.~102.} Na experiência proposta pelo objeto, o espectador experimenta
o jogo livre, a expressão do próprio corpo, um mergulhar em si mesmo e a
abertura para o novo.

As obras ligadas à Nova Objetividade dirigem"-se aos sentidos visando,
através deles, a percepção total para ``levar o indivíduo a uma
``supra"-sensação'', ao dilatamento de suas capacidades sensoriais, à
descoberta de seu centro criativo interior, da sua espontaneidade
expressiva adormecida, condicionada ao cotidiano''.\footnote{Ibidem,
  p.~104.} Com isso, entra em jogo a ordem social, ética, política e todo
o universo simbólico no qual o indivíduo se movimenta. A adaptação ao
que é dado implica o abandono da própria interioridade, em favor do que
é oferecido e, com isso, o encolhimento da imaginação. O desejo de
ultrapassar tal situação implica abrir o participador para ele mesmo,
para uma dimensão deixada de lado em função do comportamento imposto
pela estrutura social em curso.

Hélio tem consciência do caráter transformador de sua proposta e das
reações que ela pode provocar. Em suas palavras:

\begin{quote}
O próprio problema da liberdade, do dilatamento da consciência do
indivíduo, redescobrindo o ritmo, a dança, o corpo, os sentidos, o que
resta, enfim, a nós como arma de conhecimento direto, perceptivo,
participante, levanta de imediato a reação dos conformistas de toda
espécie, já que é ela (a experiência) a libertação dos prejuízos do
condicionamento social a que está submetido o indivíduo. A posição é,
pois, revolucionária no sentido total do comportamento {[}\ldots{}{]} A arte
já não é mais instrumento do domínio intelectual, já não poderá ser
usada como algo ``supremo'', inatingível, prazer do burguês tomador de
uísque ou do intelectual especulativo.\footnote{Ibidem, p.~105.}
\end{quote}

Da arte do passado, considera Hélio, restará apenas o que puder ser
experimentado como emoção direta, o que for capaz de mover o indivíduo
para além do condicionamento opressivo, imposto pelo comportamento
exigido no cumprimento das tarefas do dia a dia, oferecendo"-lhe novas
dimensões formais aptas a encontrarem resposta no comportamento do
participante. Nesta perspectiva, artistas não são mais apenas criadores
de obras a serem contempladas e sim motivadores para a criação, a ser
realizada com a participação dinâmica daqueles a quem, antes, estava
reservado o papel de mero espectador, de quem apenas olha. A busca do
supra"-sensorial, escreve o artista, ``é a \emph{descoberta da vontade}
pelo ``exercício experimental da liberdade'' (Pedrosa), pelo indivíduo
que a elas se abre''.\footnote{Ibidem, p.~105.} A participação é
fundamental, trata"-se de experiências ``que não se limitam à visão, mas
abrangem toda a escala sensorial, e mergulham de maneira inesperada num
sujeito renovado, como que buscando as raízes de um comportamento
coletivo ou simplesmente individual, existencial''.\footnote{Hélio Oiticica.
  ``Situação da vanguarda no Brasil''. In: \emph{Aspiro ao grande
  labirinto}, op.~cit., p.~111.}

\section{1967, voltando atrás para ir às raízes: o~estado~de~invenção}

Em 1964, Hélio desenvolve a denominada \emph{proposição vivência} com a
realização do \emph{Parangolé}, considerado o ponto crucial e definidor
de uma posição específica no desenvolvimento teórico de suas
experiências. \emph{Parangolé} é uma espécie de capa vestível construída
com camadas de tecidos variados, alguns cujo uso é inusitado em peças de
vestuário, e de cores diversas que se revelam à medida que quem a veste
se movimenta. O ato de ``vesti"-lo'' inicia a participação do corpo na
obra, levada adiante por o \emph{Parangolé} incitar o desejo de
movimentar"-se e provocar o movimento da obra. E, pelo fato de vestir a
obra, o corpo passa a participar dela: não há mais separação entre corpo
e obra.

O \emph{Parangolé} instaura ``um sistema que desata a fantasia'',
escreve Favaretto, ``descontínuo de atividades, agencia
estruturas"-percepções, que revelam uma outra ordem do simbólico: o
comportamento''.\footnote{Celso Favaretto. \emph{A invenção de Hélio
  Oiticica}, op.~cit., p.~107.} A improvisação e a expressividade corporal,
provocadas pelo convite ao gesto e ao ritmo, são percebidas como
introdutórias à descoberta da capacidade de criação, do corpo como
expressão. Os \emph{Parangolés} interferem no comportamento de quem os
veste, \emph{Parangolés} realizam ``uma interferência contínua e de
longo alcance, que se poderia alçar nos campos da psicologia, da
antropologia, da sociologia e da história''.\footnote{Hélio Oiticica. ``Bases
  fundamentais para uma definição do Parangolé''. In: \emph{Aspiro ao
  grande labirinto}, op.~cit., p.~65.} Conforme Hélio revela em entrevista a Ivan
Cardoso,

\begin{quote}
O \emph{Parangolé} não era, assim, uma coisa para ser posta no corpo,
para ser exibida. A experiência da pessoa que veste, para a pessoa que
está de fora, vendo a outra se vestir, ou das que vestem simultaneamente
a coisa, são experiências simultâneas, são multiexperiências. Não se
trata, assim, do corpo como suporte da obra; pelo contrário, é a total
``incorporação''. É a incorporação do corpo na obra e da obra no corpo.
Eu chamo de ``in"-corpo"-ração'' {[}\ldots{}{]} \emph{Parangolés} foram o
caminho para a descoberta do que eu chamo de ``estado de invenção''. Não
se trata de ficar nas ideias. Não existe ideia separada do objeto, nunca
existiu, o que existe é a invenção.\footnote{Hélio Oiticica. ``A arte
  penetrável de Hélio Oiticica'', entrevista a Ivan Cardoso. \emph{Folha
  de São Paulo}, 16/11/1985, p.~48.}
\end{quote}

A obra requer a participação corporal: ao revestir o corpo, pede que ele
se movimente e mesmo que, de alguma forma, dance. Segundo Oiticica, com
os \emph{Parangolés} aconteceu a descoberta das estruturas
``comportamento"-corpo'', e tudo passou a girar em torno do ``corpo
tornado dança''.\footnote{Hélio Oiticica. ``Carta a Torquato''. In: \emph{Os últimos dias de paupéria}. São Paulo: Max Limonad, 1982,
  p.~278.} O mero ato de vestir a obra implica uma transformação
expressiva e corporal do participante, essa transmutação seria a
``característica primordial da dança''.\footnote{Hélio Oiticica. ``Anotações
  sobre o \emph{Parangolé}''. In: \emph{Aspiro ao grande labirinto},
  p.~70.} E a dança é a revelação plena do mundo, um ``ato total do eu'',
considera o artista, ``o que seria para Nietzsche a embriaguez
dionisíaca é na verdade uma lucidez expressiva da imanência do
ato''.\footnote{Ibidem, p.~74.}

Neste mesmo texto, em sua última parte, denominada ``Posição ética'',
Oiticica revela que, mais que uma obra, \emph{Parangolé} nomeia um
``Programa ambiental'', voltado à liberdade moral, a derrubar mesmo
todas as morais que tendem ao conformismo estagnizante, a opiniões
estereotipadas e conceitos não criativos. Em suas palavras,

\begin{quote}
A liberdade moral não é uma nova moral, mas uma espécie de antimoral,
baseada na experiência de cada um: é perigosa e traz grandes
infortúnios, mas jamais trai a quem a pratica: simplesmente dá a cada um
o seu próprio encargo, a sua responsabilidade individual; está acima do
bem, do mal, etc. Deste modo estão como que justificadas todas as
revoltas individuais contra valores e padrões estabelecidos: desde as
mais socialmente organizadas (revoluções, p.~ex.) até as mais viscerais e
individuais (a do marginal, como é chamado aquele que se revolta, rouba
e mata) {[}\ldots{}{]} Na verdade, o crime é a busca desesperada da
felicidade autêntica, em contraposição aos valores sociais falsos,
estabelecidos, estagnados, que pregam o ``bem"-estar'', a ``vida em
família'', mas que só funcionam para uma pequena minoria.\footnote{Ibidem, p.~81-82.}
\end{quote}

O programa do \emph{Parangolé} é fortalecer as manifestações marginais
voltadas à ``felicidade utópica'', como as de Antônio Conselheiro,
Lampião, Cara de Cavalo: só um mau caráter poderia ser contra eles e a
favor de quem os destruiu, reflete Hélio. Pouco disposto a aceitar a paz
enquanto houver escravo e senhor, o artista aposta na vitalidade da
experiência criativa, em oferecer ao público a chance de deixar de ser
passivo e de participar da atividade criadora. Segundo a conclusão
fundamental da ``Posição ética'', ``sobrepujando todas as deficiências
sociais, éticas, individuais, está \emph{uma necessidade superior em
cada um de criar, fazer algo que preencha interiormente o vácuo que é a
razão mesma dessa necessidade} --- é a necessidade de realização,
completação {[}sic{]} e razão de ser da vida''.\footnote{Ibidem,
  p.~83 (grifo do autor).} A vitalidade criativa, individual e
coletiva, é percebida como índice das possibilidades de criação de uma
nova realidade.

Oiticica revela ter quebrado o cerco pequeno"-burguês no qual, por uma
série de condicionamentos, se encontrava amarrado, ao descobrir outras
pessoas que, como ele, ``pensam e fazem, querem comunicar, construir''.
E, na mesma carta (15/10/1968) à Lygia Clark que traz esta revelação,
radicaliza a escolha das margens:

\begin{quote}
Hoje, recuso"-me a qualquer prejuízo de ordem condicionante: faço o que
quero e minha tolerância vai a todos os limites, a não ser o da ameaça
física direta: manter"-se integral é difícil, ainda mais sendo"-se
marginal: hoje sou marginal ao marginal, não marginal aspirando à
pequena burguesia ou ao conformismo, o que acontece com a maioria, mas
marginal mesmo: à margem de tudo, o que me dá uma surpreendente
liberdade de ação --- e para isso preciso apenas ser eu mesmo segundo meu
princípio de prazer.\footnote{Hélio Oiticica \& Lygia Clark. \emph{Cartas
  1964-1974}. Rio de Janeiro: \textsc{ufrj}, 1998, p.~44.}
\end{quote}

Ainda nessa carta, Hélio escreve à Lygia que ao ler \emph{Eros e
civilização}, de Marcuse, percebe ``que tinha razão''\footnote{Ibidem, p.~44.}
ao escolher colocar"-se à margem. Em outra carta do mesmo ano
(08/11/1968), o autor refere"-se novamente a Marcuse ao afirmar a
participação do público na obra como uma atividade desrepressiva que,
``como diria Marcuse, libera o Eros reprimido por atividades
repressivas''.\footnote{Ibidem, p.~72.} A participação, capaz de
desconcertar e liberar forças imprevisíveis é salientada como altamente
revolucionária e como a grande inovação proposta por ambos, Hélio e
Lygia. Distintamente do que acontece na Europa, com uma civilização
saturada que parece secar a imaginação, o Brasil é uma síntese de raças,
povos e costumes, escreve o artista, diante da qual ``o europeu fala mas
não fala tão alto, a não ser nos meios universalistas acadêmicos, que
não são criação cultural, mas sim arremedo. A criação, já mesmo em
Tarsila e principalmente Oswald de Andrade, possui uma carga subjetiva
que muito difere do racionalismo europeu''.\footnote{Ibidem,
  p.~73.} Para Hélio, esta constatação não é o bastante, é necessário
também sonhar um mundo novo, para evitar que o futuro seja a repetição
do existente ou ainda pior do que este. E, voltando ao filósofo alemão,
observa:

\begin{quote}
Para Marcuse, os artistas, filósofos, etc. são os que têm consciência
disso ou ``agem marginalmente'' pois não possuem classe social definida,
mas são o que ele chama de ``desclassificados'', e é nisso que se
identificam com o marginal, isto é, com aqueles que exercem atividades
marginais ao trabalho produtivo alienante: o trabalho do artista é
produtivo, mas no sentido real da produção"-produção, criativo, e não
alienante como os que existem em geral numa sociedade capitalista.
Quando digo ``posição à margem'' quero algo semelhante a esse conceito
marcuseano: não se trata da gratuidade marginal ou de querer ser
marginal à força, mas sim colocar no sentido social bem claro a posição
do criador, que não só denuncia uma sociedade alienada de si mesma, mas
propõe por uma posição permanentemente crítica, a desmistificação dos
mitos da classe dominante, das forças da repressão, que além da
repressão natural, individual, inerente à psichê {[}sic{]} de cada um,
são a ``mais"-repressão'' e tudo o que envolve a necessidade dessa
mais"-repressão.\footnote{Ibidem, p.~74-75.}
\end{quote}

Hélio sugere que Lygia leia Marcuse --- e Frantz Fanon, \emph{Lés damnés
de la terre}\ldots{} Ele mesmo continua leitor do filósofo crítico,
mencionado novamente na correspondência do ano seguinte: ``no livro mais
recente que escreveu, Marcuse tem um capítulo em que propõe uma
``sociedade biológica'' como desrepressiva, baseada na comunicação
direta em cadeia, o mesmo que pensei e já havia escrito''.\footnote{Ibidem,
  p.~121.} Lido no calor da hora, trata"-se do \emph{Ensaio sobre a
libertação} (\emph{An Essay on Liberation}), publicado em 1969, onde
Marcuse escreve um capítulo relativo à possibilidade de uma
fundamentação biológica para o socialismo, ``A Biological Foundation for
Socialism''.\footnote{É este o nome do primeiro capítulo em Herbert
Marcuse. \emph{An Essay on Liberation}. Boston: Beacon Press, 1969.}

\section{Mora na filosofia}

A quem me acompanhou até aqui, proponho a leitura de reflexões do
próprio Marcuse, cuja influência pode ser percebida --- é mesmo
explicitada, como se viu --- nas instigantes proposições de Hélio. Também
Caetano revela conhecer \emph{Eros e civilização}, (pelo menos o título)
quando diz que, enquanto escrevia \emph{Verdade tropical}, pensava em
chamá"-lo de \emph{Boleros e civilização}, em suas palavras, ``um velho
trocadilho meu de 1968 (que estaria na contracapa de um disco meu que não
fiz porque a prisão interrompeu meus planos de composição), como piada
com o famosíssimo título \emph{Eros e civilização}, de
Marcuse''.\footnote{Caetano Veloso, em sua página no facebook, 13/11/2017. O livro \emph{Verdade tropical} foi nomeado a partir
  da canção ``Vereda tropical'', bolero do compositor mexicano
  Gonzalo Curiel (1904-1958), gravado por Ney Matogrosso, em 1991. O
  bolero também nomeou uma novela da Globo em 1984.}

O mencionado \emph{Ensaio sobre a libertação} é a tentativa de
configurar recusas ao estado de coisas em curso. Marcuse apresenta as
existentes resistências tanto à obscena --- apesar de visível ---
exploração (do ser humano e da natureza) levada a cabo pelo poder
massivo e quase hegemônico do capitalismo corporativista, quanto à
dominação pela força militar, pela burocracia repressiva, pelas
ideologias introjetadas nas e pelas massas.

Diante dessas estratégias de poder no final da década de 1960 (e ainda
hoje, \emph{mutatis mutandi}), Marcuse considera o impulso de
libertação, que percebe presente na emergência de novos valores e novas
aspirações no meio sociocultural da época, como alternativa \emph{par
excellence} capaz de resistir e de recusar as formas de poder em curso.
Seria irresponsável supervalorizar as possibilidades de vitória desta
força pulsional de libertação, adverte o filósofo, ``mas os fatos estão
aí, fatos que são não somente os símbolos como também as encarnações da
esperança''.\footnote{Herbert Marcuse. ``Preface'', op.~cit.,
  p.~viii.} Cabe então à teoria crítica reexaminar as pré"-condições
existentes para transformações sociais qualitativamente distintas
daquelas dadas.

Neste rumo, na perspectiva de Marcuse, são imprescindíveis mudanças
qualitativas nas necessidades, na infraestrutura dos indivíduos que
compõem a infraestrutura da sociedade. A questão colocada pelo filósofo
é a seguinte: como satisfazer as próprias necessidades individuais (a
tautologia visa à ênfase) sem reproduzir --- no jogo entre aspirações e
satisfações --- a aparelhagem exploradora que perpetua a servidão. Ainda
que sem definição, o filósofo percebe a transformação no ar.
Necessidades e satisfações bastante distintas, antagonistas mesmo,
daquelas em curso foram expressas por estudantes e ativistas no final da
década de 1960. Através da visível expressão de mudança interior, pode
ser percebida a base pulsional para a libertação --- a pulsão vital por
liberdade, bloqueada pela longa história da sociedade de classes. O
caminho da libertação inclui, pensa Marcuse, um organismo vivo incapaz
de submeter"-se ou mesmo de adaptar"-se ``às performances competitivas
requeridas para o bem"-estar sob dominação, não mais capaz de tolerar a
agressividade, brutalidade e feiúra do modo de vida estabelecido. A
rebelião teria então alcançado raízes na própria natureza, na biologia
do indivíduo''.\footnote{Ibidem, p.~5.} Só a partir deste
fundamento metas concretas, bem como estratégias de lutas políticas,
podem ser configuradas, considera o filósofo.

Diante do fato de as capacidades de transformação, desenvolvidas pelos
avanços tecnológicos e intelectuais, permitirem ultrapassar a moldura da
exploração, na qual permanecem confinadas, Marcuse julga possível
superar a servidão voluntária, ``voluntária porque introjetada nos
indivíduos'',\footnote{Ibidem, p.~6.} através de práticas capazes
de alcançar as raízes da infraestrutura humana tendo em vista uma
radical ``transvaloração dos valores''. Tais práticas implicam ``uma
quebra das familiares e rotineiras formas de ver, ouvir, sentir e
compreender as coisas'':\footnote{Ibidem, p.~6.} assim a
transformação encontra"-se ancorada na estrutura pulsional. Conhecimento
e tecnologia têm de mudar suas direções e finalidades, ``têm de ser
reconstruídas de acordo com uma nova sensibilidade --- com as demandas
das pulsões de vida''.\footnote{Ibidem, p.~19.} Com uma \emph{gaya
scienza}, escreve Marcuse.

\emph{An Essay on Liberation} propõe um ``\emph{ethos} estético'', de
acordo com o qual a oposição entre matéria e intelecto, imaginação e
razão, poesia e técnica seria invalidada --- como acontece nas obras de
arte capazes de harmonizar estes opostos. Seria este o \emph{ethos}

\begin{quote}
de homens e mulheres que não precisassem mais ter vergonha de si porque
superaram o sentimento de culpa: aprenderam a não se identificar com os
falsos pais que construíram e toleraram e esqueceram os Auschwitzs e os
Vietnans da história, as câmaras de tortura de todas as inquisições
seculares e eclesiásticas, os guetos e os templos monumentais das
corporações, e que adoraram a mais alta cultura desta realidade. Se e
quando homens e mulheres agirem e pensarem livres desta identificação,
terão quebrado a corrente que enlaça pais e filhos de geração a geração.
Não terão redimido os crimes contra a humanidade, mas terão se tornado
livres para interrompê"-los e prevenir seu recomeço.\footnote{Ibidem,
  p.~25. A superação da ``vergonha de si'', mencionada no início da
  citação, refere"-se ao aforismo 275, ``\emph{Qual o emblema da
  liberdade alcançada} --- Não mais envergonhar"-se de si mesmo'', de
  Friedrich Nietzsche. In: \emph{A gaia ciência}. Trad. Paulo César de Souza. São
  Paulo: Companhia das Letras, 2001, p.~186.}
\end{quote}

Com o \emph{ethos} estético, valores, verdades e ordem estética --- até
então monopolizados, segregados e preservados à parte na esfera da
cultura --- podem voltar"-se à práxis e permear a dimensão material, em um
movimento denominado por Marcuse de ``dessublimação da arte'' --- cujo
propósito é figurar materialmente o que permanecia na instância do
sublime e da sublimação. A imaginação assume seu papel criativo na
prática transformadora. Os que se rebelam, acredita Marcuse, ligam a
libertação à dissolução da percepção cotidiana ordinária e desejam
sentir, ouvir e ver de um novo modo.

As palavras de Marcuse sobre esta nova sensibilidade, percebida nas
ações estudantis, artísticas e intelectuais no final da década de 1960,
podem ser estendidas à obra de Oiticica:

\begin{quote}
Ela emerge na luta contra violência e exploração onde esta luta é
realizada por modos e formas de vida essencialmente novos: negação do
\emph{Establishment}, de sua moralidade e cultura; afirmação do direito
de construir uma sociedade na qual a abolição da miséria e da labuta
termina em um universo onde o sensual, o brincalhão, o calmo e o belo
transformam"-se em formas de existência, e assim, na \emph{Forma} da
própria sociedade.\footnote{Herbert Marcuse. ``Preface'', op.~cit., p.~25.}
\end{quote}

A proposta de Hélio é a de liberdade, liberdade de invenção do lado do
artista, experiência de objetos sem bula, sem instruções de uso, aptos a
atravessarem os condicionamentos sofridos pelo espectador e proporem uma
experiência de jogo livre, de mergulho sem rede, sem \emph{a priori}.
Através da experiência estética, ``o indivíduo tem a chance de desejar
ser livre de cooptação e da opressão {[}\ldots{}{]} por meio de seu próprio
corpo, sem depender das questionáveis estruturas governamentais ou de
revoluções, e de seus resultados inerentemente imprevisíveis''.\footnote{Catherine
Dawson. ``Maquinaria íntima: um exame marcuseano sobre a obra \emph{A
  casa é o corpo}, de Lygia Clark''. Revista \emph{\versal{ARTEFILOSOFIA}}, n.~18,
  2015, p.~208.}

%Desejo finalizar este breve ensaio com as palavras de Marcuse,
%encontradas no final do prefácio de \emph{An Essay on Liberation}:

Finalizo este breve ensaio com as palavras de Marcuse:

\begin{quote}
Jovens militantes sabem, ou sentem, que o que está em jogo é
simplesmente suas vidas, a vida de seres humanos que foi transformada em
joguete nas mãos de políticos e gerentes e generais. As (\emph{the})
rebeldes querem tirá"-la dessas mãos e fazer valer a pena vivê"-la; elas e
eles (\emph{they}) perceberam que isto ainda é possível hoje, e que sua
realização precisa de uma luta que não pode mais ser contida por regras
e regulamentos de uma pseudo democracia em um mundo orwelliano livre
(\emph{Free Orwellian World}). A elas, a eles, eu dedico este
ensaio.\footnote{Herbert Marcuse. ``Preface'', op.~cit., p.~x.}
\end{quote}

\section{Referências}

\begin{Parskip}
\versal{BROWN}, Mark. ``Tate modern removes macaws as visitor numbers soar''. In:
\emph{The Guardian}, 20/06/2016, consultado em \textless{}\emph{www.theguardian.com}\textgreater{}.

\versal{DAWSON}, Catherine. ``Maquinaria íntima: um exame marcuseano sobre a obra
\emph{A casa é o corpo,} de Lygia Clark''. Tradução de Thiago Reis. Revista
\versal{ARTEFILOSOFIA} (\versal{UFOP}), n.~18, Ouro Preto, 2015.

\versal{FAVARETTO}, Celso. \emph{A invenção de Hélio Oiticica}. São Paulo: Ed\versal{USP},
2000.

\_\_\_\_\_\_. \emph{Tropicália, alegoria, alegria.} Cotia: Ateliê,
2007.

\versal{MARCUSE}, Herbert. \emph{An Essay on Liberation}. Boston: Beacon Press,
1969.

\versal{NETO}, Torquato. \emph{Os últimos dias de paupéria}. São Paulo: Max
Limonad, 1982.

\versal{NIETZSCHE}, Friedrich. \emph{A Gaia Ciência}. Trad. Paulo César de Souza.
São Paulo: Companhia das Letras, 2001.

\versal{NUNES}, Benedito. \emph{Lygia Clark e Hélio Oiticica}. Rio de Janeiro:
Funarte, 1987.

\versal{OITICICA}, Hélio. \emph{Aspiro ao grande labirinto}. Rio de Janeiro:
Rocco, 1986.

\_\_\_\_\_\_. ``O sentido de vanguarda do grupo baiano''.
\emph{Correio da manhã}, 24/11/1968.

\_\_\_\_\_\_. \& \versal{CARDOSO}, Ivan. ``A arte penetrável de Hélio
Oiticica'', entrevista a Ivan Cardoso. \emph{Folha de São Paulo}, 16/11/1985.

\_\_\_\_\_\_. \& \versal{CLARK}, Lygia. \emph{Cartas 1964-1974}. Rio de
Janeiro: \versal{UFRJ}, 1998.

\versal{PECCININI}, Dayse. \emph{Objeto na Arte: Brasil anos 60}. São Paulo:
\versal{FAAP}, 1978.

\versal{VELOSO}, Caetano. \emph{Verdade tropical.} São Paulo: Companhia das
Letras, 1997. \enlargethispage{\textheight}
\end{Parskip}

\chapter*{Aqui é o fim do mundo: Candeias,~Coni~Campos,~1967}
\addcontentsline{toc}{chapter}{Aqui é o fim do mundo: Candeias, Coni Campos,~1967,\\ \emph{por Juliano Gomes}}

\begin{flushright}
\emph{Juliano Gomes}
\end{flushright}

\section{I}

1967 é um ano decisivo para a história do cinema brasileiro. Ano de
filmes como \emph{O caso dos irmãos Naves}, de Luis Sérgio Person,
\emph{Garota de Ipanema,} de Leon Hirzsman, \emph{Opinião pública}, de
Arnaldo Jabor, \emph{El justicero}, de Nelson Pereira dos Santos, filmes
que tiveram repercussão a longo prazo e que produziram reverberações no
debate cultural da época e constam na literatura de referência sobre
cinema brasileiro até os dias de hoje. Boa parte do que se entende como
cinema brasileiro moderno ganha contorno na filmografia deste ano.

Definitivamente, pode"-se afirmar com tranquilidade que se trata do ano
onde o evento central é o lançamento de \emph{Terra em transe}. O
terceiro longa de Glauber Rocha não só ocupava as manchetes de jornais
ainda antes do lançamento (temperado pelo episódio da exibição do filme
no Festival de Cannes daquele ano), mas foi um dos grandes
acontecimentos culturais daquela década e subsequentemente. Sem dúvida,
se trata de um dos filmes brasileiros mais analisados em artigos na
imprensa e na academia até hoje. A vasta literatura que o faz perdurar
ainda está em produção até os dias de hoje, e, com razão, seu fôlego
como obra está longe de esgotar. A restauração razoavelmente recente deu
ainda novo fôlego às pesquisas. Além disso, nossa história sociopolítica
insiste em eternos retornos absolutamente literais, como por exemplo o
golpe em Dilma Roussef em 2016, que faz daquela operística ficção algo
cada vez mais próximo de um documentário.

\emph{Terra em transe} é pedra fundamental nos debates sobre o
``fenômeno 1967'', onde uma série de acontecimentos muito próximos
sintetiza uma marca de mudança nos debates culturais brasileiros. O
lançamento do filme, mais a exposição Nova Objetividade Brasileira no
\versal{MAM}-\versal{RJ} e os festivais da canção da \versal{TV} Record em São Paulo formam uma
espécie de trinca que resume uma extrema agitação na vida cultural e
artística do país, que se estende para bastante além dos domínios de
maior visibilidade. A sequência destes eventos e sua visibilidade ímpar
formam uma espécie de arquipélago que evidencia que certas movimentações
no campo cultural brasileiro, que vinham se formando num momento
anterior, estão tomando uma escala nunca antes vista e consequentemente,
tomando de assalto uma razoável porção do debate público. Naquele momento, agora
distante, a arena da comunicação pública era marcada por um
número razoável de publicações que se interessavam por este debate na
imprensa, fato que, cinquenta anos depois, parece uma conjuntura
distante.

A trinca de ases Glauber Rocha--Hélio Oiticica--Caetano Veloso tem como
linha comum a atuação transversal que transcende os domínios mais
imediatos dos seus campos de trabalho e atravessa a cultura brasileira
através de ensaios, intervenções e aparições midiáticas, formando uma
reserva imensa de pensamento sobre o Brasil e seus limites no tocante ao
debate cultural, repertório esse que não cessa nem cessará de ecoar. Uma
nova imagem do artista intelectual, crítico e agitador cultural, parece
se formar, ativando um contexto midiático que se torna favorável,
através de programas de \versal{TV}, artigos de jornal, debates públicos,
performances, assim como uma rede de colaboradores que dá capilaridade
ao movimento.

Podemos dizer que uma das marcas dessa intensa movimentação, que tem
1967 como marco, é a inclusão decisiva da violência\footnote{O livro de
  Frederico Coelho \emph{Eu, brasileiro confesso, minha culpa e meu
  pecado} é decisivo na caracterização da Tropicália como movimentação
  que traz a violência no centro de suas ações.} como traço constitutivo
da ruptura corrente na discussão sobre cultura e sociedade brasileira. A
violência como matéria poética, como mote, como motriz das ações, como
energia revoltosa que se maneja nos trabalhos. Os eventos centrais do
fenômeno 1967, incluindo aí também a encenação do \emph{Rei da vela} por José
Celso Martinez Corrêa, no teatro Oficina em setembro de 1967, trazem à
tona essa exigência da prática artística como uma redistribuição poética
da violência, constitutiva da condição brasileira, subdesenvolvida.

\section{II}

Voltando mais especificamente ao cinema, este pequeno ensaio tem como
desejo tentar colocar à baila dois filmes deste caro ano, que geraram
menos combustão subsequente, para tentar buscar neles faíscas do magma
cultural. Ligados de maneiras distintas à linha mais visível dos
acontecimentos, dois filmes que ficaram razoavelmente fora das maiores
discussões da época e dos períodos posteriores parecem interessantes
para se trazer para conversa hoje no ímpeto de ampliar as linhas que
brotam desse contexto.

No Rio de Janeiro, o baiano Fernando Coni Campos realiza seu segundo
longa"-metragem: \emph{Viagem ao fim do mundo}. A produção bastante
independente, demorada e cheia de contratempos, talvez seja um motivo
para a pouca repercussão do lançamento do filme. Mas, em resumo, a fita
tem um enredo simples: uma viagem de avião entre Rio e São Paulo, mais
ou menos em tempo real (do que a viagem levava em 1967), acompanhada
pelo ponto de vista de alguns dos personagens que estão no avião. Esta
aí a síntese da ação do enredo. Contudo, o filme usa tal situação para
construir um conjunto de relações e associações entre signos de ordens
variadas, criando muitas vezes uma espécie de colagem que resulta num
inventário cultural digressivo da época.

Os capítulos do romance de Machado de Assis \emph{Memórias póstumas de
Brás Cubas} chamados ``O delírio'' e o ``Senão do livro'' são as
bases para um intenso exercício de montagem de texto e imagem que nunca
deixa ou abandona completamente o núcleo narrativo do avião, mas faz das
digressões mutantes seu modo de ação principal. A ampla presença
literária da banda sonora tagarela se combina com uma absoluta
irreverência no uso de seus materiais, sejam eles observações de Simone
Weil sobre a fé, cinejornais sobre a Segunda Guerra Mundial, peças
publicitárias ou gags \emph{nonsense}. Coni Campos realiza nesta espécie
de ficção ensaística um raro exercício de colagem de materiais
extremamente heterogêneos onde nunca nos situamos numa região claramente
erudita nem no campo de uma superficialidade dos materiais publicitários
e massivos. O filme quer exercitar justamente esta dimensão comum que
torna possível a aproximação de elementos que têm circulações
distantes no campo da cultura.

Curiosamente, estão lá canções dos baianos ``nucleares'' Caetano Veloso
e Gilberto Gil: ``Alegria, alegria'', ``Soy loco por ti America'',
``Tropicália'' e ``Onde Andarás''. Desde a montagem de abertura do filme,
que combina as canções ao texto de Machado, as inserções musicais têm no
cancioneiro tropicalista sua marca evidente. O que surpreende é que a
incorporação de canções que estão no centro da tropicália musical não
fez com que o filme fosse abordado na literatura sobre este repertório
cultural de maneira significativa. Trata"-se de uma obra que parece não
só carregar estes ``selos de pertencimento'' à movimentação da
Tropicália, mas que adota procedimentos comuns como a colagem, a mistura
irreverente de repertório massificado e erudito e uma não filiação a
gêneros pregressos, essa ``identidade dinâmica''.\footnote{Bernardo
  Oliveira usa essa expressão no texto ``Rastros dos trópicos'', sobre a
  Tropicália, disponível em: \textless{}\emph{https://bit.ly/2Cr28KE}\textgreater{}.}
Talvez de maneira mais clara que \emph{Terra em transe} por exemplo, com
sua épica falta de humor e ironia.

Coni Campos cria uma incessante produção de fluxos e séries que às vezes
emanam da memória de personagens, ou de uma revista de variedades, de um
livro (como o exemplar de bolso do \emph{Brás Cubas} que o personagem de
Fábio Porchat compra no início do filme, criando uma duplicidade de
sentido para a presença do texto machadiano no filme, entre a narração
externa e a diegese interna do enredo). As vinhetas vão se combinando
umas às outras voltando periodicamente ao espaço"-tempo da viagem de
avião. \emph{Viagem ao fim do mundo} opera por estes mergulhos
inveterados num certo campo de possíveis que emana das pessoas e
objetos, onde nos dá a impressão de que tudo parece interessar, tudo é
material de trabalho, tudo é Brasil. Tal voracidade indica um certo
gosto pelo aleatório, por um manejo do que não tem centro definido, que
se caracteriza justamente pelo gesto dos exercícios de variação que ou
emanam ou desembocam nos elementos da situação de base da viagem. Seu
fundamento é justamente performar essas variações inesperadas, essas
derivas aberrantes.

A metalinguagem está presente já na primeira parte do filme e
estabelece uma ambiguidade, uma correspondência, entre livro e filme
através dos trechos da \emph{Ortodoxia} de Chesterton: ``Pois se este
livro é uma brincadeira, ele é uma brincadeira contra mim mesmo'', diz a
narração. Logo em seguida, vemos um personagem folhear justamente o
livro de Chesterton na livraria do aeroporto. Entretanto, o texto é
falado antes desta cena e dura muito além, criando a ambivalência
entre um narrador \emph{on} ou \emph{off}.\footnote{\emph{Narrador off} é
  aquele que narra os fatos sem participar da história, não sendo nem
  mesmo um personagem. Enquanto o \emph{narrador on} seria aquele que
  faz a narração de dentro da cena, na diegese do filme.}
O trânsito do filme entre materiais e meios (fotos, cinejornais,
televisão, canções, livros, propagandas) é reforçado pela reflexividade
das narrações onde mais de uma vez se fala na primeira pessoa, e esta
diz ``este livro\ldots{}''. \emph{Viagem ao fim do mundo} é um filme, é um
livro, uma canção, uma propaganda e uma fotomontagem. Todo seu trajeto é
marcado por esta tensão entre uma força centrípeta e outra centrípoda.

O filme tem como ação central operar e reoperar sínteses, associações,
conexões, improvisos entre material cultural diverso, formando uma vivaz
etnografia do seu próprio tempo. Esta se forma simultaneamente em duas
vias: seja diretamente nos elementos que mostra ou na maneira como o
faz, voraz, com uma espécie de fome de tudo, que o exercício do
ecletismo, na prática, une em um fio comum entre o que
parecia possuir matrizes diversas em sua origem.

\section{III}

Às margens do rio Tietê, o cineasta, fotógrafo, ex"-militar e
caminhoneiro Ozualdo Candeias realiza em 1967 um dos mais intrigantes
filmes brasileiros: \emph{A margem}. Este seu primeiro longa narra duas
estranhas ``histórias'' de amor que acontecem à beira do rio Tietê, num
cenário de ruínas e com personagens que vivem à margem do processo
urbano e industrial da cidade de São Paulo. Candeias dá ao filme um
tratamento extremamente peculiar: pouquíssimos diálogos, personagens
deambulatórios, e um recorrente uso da câmera subjetiva\footnote{Quando
  um filme simula através dos ângulos de câmera o olhar de um dado
  personagem.} como ferramenta de criação de dinâmica e intenção na
montagem. A trilha sonora executada pelo Zimbo Trio dá o tom das
situações que variam entre o mitológico, o onírico e o mais mundano.

Os personagens de Candeias não são intelectuais, estudantes, políticos,
artistas, ou mesmo pessoas de classe média. Os corpos desta narrativa
são corpos que vivem realmente à margem do sistema: prostitutas, pessoas
em situação de rua, seres completamente desgarrados, esfarrapados, para os quais
uma perspectiva de inserção ou inclusão na sociedade formal não está em
vista --- nem é uma questão que o filme sugere. Não está em vista. Carros,
caminhões e estrada são panos de fundo do filme. Não há ninguém que faça
parte da circulação hegemônica da sociedade paulistana entre os
personagens do filme: nenhum contraponto. Ninguém com sobrenome, nem
mesmo com nome próprio. Não há uma individualidade definida para os
personagens. São aparições onde parece prevalecer justamente um certo
trânsito entre as subjetividades: trocas de roupas entre eles, trocas de
situações. Características evidentes de uma modernidade cinematográfica
expressa numa caligrafia incomum.

Não há no desenrolar do filme progresso ou avanço. Aquelas vidas parecem
presas ao espaço marginal do rio, em um deserto aquoso povoado por
ruínas, barrancos, barracos e precários elementos de urbanização, como
uma ponte de madeira sobre o Tietê que funciona como um dos principais
espaços do filme. Toda ação do filme está marcada por uma força de
retorno cujo centro é a morte.

Uma estranha tragédia pobre se desenha. Todos os afetos ali parecem
destinados à morte. Um homem branco de terno esfarrapado e furado, uma
negra vestida de noiva à procura de algo, um homem de cabeça raspada com
gestos exagerados e infantis, uma mulher de peruca loura e traços mestiços,
são os polos principais desta dimensão criada pelo filme. Todos ao redor
de uma ponte, à beira do rio, próximos a uma favela, uma ponte de
madeira, as ruínas de uma igreja antiga, ladeados pelas grandes vias da
marginal Tietê.

O paralelo possível com a barca de Caronte, que conduz os mortos na
mitologia grega, confirma a perspectiva temporal da empreitada de
Candeias. Não se trata de um tempo da atualidade, mas uma dimensão
repetitiva, difusa, incerta, assim como o encadeamento dos
acontecimentos e da trilha sonora. \emph{A margem} trata de um certa
dimensão desesperada onde uma perspectiva de saída daquele mundo não se
coloca nem mesmo como questão pensável. Candeias quer etnografar um
determinado campo afetivo possível diante dessas condições materiais e
existências de quem não tem materialmente quase nada.

À margem do espaço por excelência do desenvolvimento brasileiro, Ozualdo
Candeias concebe um drama praticamente mudo e interessado na vida do
pobre, daquele que não participa do imaginário coletivo sem estar
associado aos sentidos de pena ou compaixão do espectador. Em Candeias,
os desgarrados são pessoas que durante o filme passam por experiências
variadas, que não estão na imagem para figurar o ``outro de classe'',
nem para denunciar a miséria socioeconômica do país. É claro que uma
dimensão de precariedade econômica se torna presente no que o filme
mostra, entretanto, claramente seu objetivo é ter essa miséria como dado
inicial e não como ponto de chegada.

Não por acaso o filme concentra seu investimento formal no jogo
intercambiável de perspectivas. A questão de partir de outros pontos de
observação, incomuns no cinema, é essencial no exercício proposto por
Candeias. Literalmente ver pelos olhos de quem tem como horizonte um
cotidiano de pequenos bicos, quem se vira com quase nada, onde o
imaginário homogeneizante típico do cinema da época ao imaginar este
outro de classe não se impõe. O que se coloca como central são os
desencontros amorosos de seus personagens, em formas de caminhadas,
desencontros, alucinações e morte. \emph{A margem} constrói, acima de
tudo, um estado, uma atmosfera.

\section{VI}

Uma primeira imagem parece unir estes dois filmes de matizes tão
distintas. O exercício literário e composicional de Coni Campos e a
mitopoética mundana de Candeias têm como figura de expressão a ideia do
fim do mundo. No filme carioca, já nos seus créditos iniciais, sob o
título do filme, um homem comemora num desfile de carnaval esfuziante: o
fim do mundo como celebração, como algo a ser buscado e desejado. Na
\emph{Viagem} definitivamente se combinam estes números sem finalidade,
na medida em que o filme não se ocupa de maneira alguma a se estruturar
a partir da criação de ganchos narrativos de expectativa. Não há fim nem
razão para o que acontece. O que parece ser celebrado é justamente esta
ordem, ou uma maneira de organizar o possível que se esgota, e que o
filme quer instituir outra maneira. Seu assunto parece ser um modo de
vida do século, uma vida da classe média que vivencia as mudanças na
sociedade do consumo, com seus maridos, viagens de avião, um imaginário
de classe afinal.

Em Candeias, os homens e mulheres do fim do mundo são o material
narrativo. Uma terra desgastada. Corpos sem morada. Acontecimentos
súbitos como a aparição de um grupo de jovens que dança iê iê iê
completamente do nada, à beira do rio, e depois some. Eventos
aleatórios, afetos interrompidos, um entrelugar da metrópole tornado
arena de um conjunto de vidas sem nome, sem destino nem origem. Não se
sabe bem se é um sonho ou pesadelo o que estamos vendo. Sabe"-se que há um
mundo ali na tela, que não podemos dizer que não é o nosso, apesar de
seu funcionamento peculiar.

Candeias antecipa a característica atmosférica do cinema das últimas
três décadas, na qual personagens falam pouco e um valor de ambiência
se sobrepõe a uma narrativa mais organizada. O mutismo, que domina o
cinema de arte a partir dos final dos anos 1980, já domina aqui. A
caracterização dos personagens se dá por um investimento na
fisicalidade, na fotogenia e não num investimento psicológico. A
deambulação característica dos cinemas modernos ganha aqui uma inflexão
mais bruta, de saltos súbitos, de ângulos de câmera expressionistas, de
expressões bestificadas porém altamente complexas, pois seu sentido não
se encerra. O apocalipse é aqui, podemos concluir.

Já Coni Campos nessa sua colagem suicida e inveterada cria um dos
capítulos mais elaborados de uma ensaística brasileira no cinema.
Completamente ignorado na literatura sobre o assunto, o caleidoscópio
inveterado do cineasta baiano opera um dos mais radicais e impuros
mergulhos numa enunciação híbrida, reflexiva, de modalidades mutantes,
entre a ficção prosaica, o cinejornal, a investigação filosofia, o
chiste e a meditação religiosa. O fim do mundo de Coni Campos é este
espaço das matérias do mundo, onde tudo parece ter perdido sua origem e
necessita do uso para ter vida. Viver é operar por montagem deste imenso
e belo aterro sanitário que é a vida subjetiva nas metrópoles. A
narração do filme emula um certo estado de ``pinto de lixo'', um
entusiasmo febril, patológico, em busca de um sentido que nunca vem, mas
que não embasa um niilismo.

\section{V}

Um investimento no delírio como matéria caracteriza os dois filmes. Um
primado da razão não se impõe em suas superfícies. Nem a típica moldura
sociológica, nem uma análise clara da cultura da época se formam aqui.
Se trata de investir numa espécie de irracionalismo libertador que não
se torna afinal regressão, mas transgressão criadora. Não são filmes que
podemos plenamente entender, mas sim sentir o prazer de suas
movimentações, de seus encadeamentos opacos e rítmicos, cada um com
velocidades muito específicas.

Uma certa celebração da desordem, de um manejo expressivo do caos os
constitui. Se trata afinal de imaginar violentamente outras composições
e formas de vida. Uma política da imaginação prevalece como gesto de
desobediência, violando justamente um campo de possíveis que ali se
impunha. Violando uma discursividade corrente, ocupada dos ``problemas
do país''. Levando a um ponto ainda mais intenso a investida contra o
racionalismo, como ruptura descolonial. Um sugestão de uma poética
brasileira, de um delírio de pobre, encontra semente nesta filmografia.
Assim como em canções de outro imenso criador da época, Jorge Ben, o
exercício do desvario criador é ferramenta política de primeira ordem na
medida em que concebe futuros impossíveis, não condicionados por um
campo de expectativa pré"-formulado.

\section{VI}

As sobrevivências de 1967 produziram uma centralidade de um certo
repertório ligado a um grupo de artistas geniais já citados aqui.
Entretanto, algo parece ter sido nublado neste meio século. O que se
impunha ali como método de intervenção não era exatamente aquelas
canções específicas ou filmes, mas justamente o gesto de perceber uma
conjuntura hegemônica e atuar violentando"-a. 1967 e seus marcos apontam
justamente para os gestos de intervenção à ordem. Fetichizar seus
artistas e obras parece inócuo como lição. É preciso perceber onde o
gesto vive hoje. Não é perguntar se a arte hoje é melhor ou pior, mas
entender onde ela está se ativando, em que contexto.

E o que esta digressão aqui deseja é justamente ativar este prazer do
fim do mundo, essa ritualização de uma desordem libertadora, que
inerentemente sugere outra caligrafia, outro ambiente afetivo e
plástico. Redistribuir a violência é encontrar estabilidades e
deliciosamente as implodir nos trabalhos. É dançar sobre as ruínas. Pois
não podemos construir o que não podemos imaginar primeiro.

\section{Referências}

\begin{Parskip}
\versal{ASSIS}, Machado de. \emph{Memórias póstumas de Brás Cubas}. Disponível em:
\textless{}\emph{https://bit.ly/34CRr3S}\textgreater{}. Acesso em: 08/07/2018.

\versal{CAMPOS}, Fernando Coni. \emph{Cinema -- sonho e lucidez}. Rio de Janeiro:
Azougue Editorial, 2003.

\versal{COELHO}, Frederico. \emph{Eu, brasileiro confesso, minha culpa e meu
pecado.} Rio de Janeiro: Civilização Brasileira, 2010.

\versal{OLIVEIRA}, Bernardo. \emph{Rastros dos trópicos}.\\ Disponível em:
\textless{}\emph{https://bit.ly/2PZeMZw}\textgreater{}.
\end{Parskip}


\chapter*{Brasil por multiplicação\footnote{Este título faz referência a
  um ensaio de Roberto Schwarz, intitulado \emph{Nacional por
  subtração}, publicado em 1986 no livro \emph{Que horas são?} (São Paulo: Companhia das
  Letras, 1987). Na verdade, além deste, outros dois ensaios desse
  autor marcaram a escrita deste texto e a preparação desta exposição, a
  saber: \emph{Política e cultura, 1964-1969,} publicado em 1978, e
  \emph{Verdade tropical: um percurso em nosso tempo}, publicado em
  2012. Hélio Oiticica por conta de seu referido texto, é a inspiração
  principal, mas Roberto Schwarz, Mario Pedrosa e Caetano Veloso são
  importantes referências artísticas, críticas e conceituais por trás
  desta curadoria.}}

\addcontentsline{toc}{chapter}{Brasil por multiplicação, \emph{por Luiz Camillo Osorio}}

\begin{flushright}
\emph{Luiz Camillo Osorio}
%\footnote{Professor na \versal{PUC}-Rio, pesquisador
  %\versal{CNP}q e curador do Instituto \versal{PIPA}. Gostaria de agradecer a Madalena Vaz
  %Pinto, Lia Rodrigues, Felipe Chaimovich, Pedro Duarte, Marta Mestre e
  %Sergio Martins pelas conversas ao longo da preparação deste Panorama e
  %da escrita deste texto. Além das trocas com todos os artistas
  %participantes que foram fundamentais para transformar o projeto em uma
  %exposição. Por fim, a toda a equipe do \versal{MAM}-São Paulo, em especial a
  %Paula Amaral, Patricia Lima, Ana Paula Santana, Renato Salem e Rafael
  %Itsuo, além dos designers Barbara Szanieck e Felipe Taborda e, dos
  %arquitetos responsáveis pela museografia, Felipe Tassara e Iara Ito.}
\end{flushright}

\epigraph{``No Brasil há fios soltos num campo de possibilidades: por que não
explorá"-los?''}{Hélio Oiticica}

Este texto foi originalmente escrito para acompanhar o catálogo da
exposição homônima, que deu título ao 35º Panorama da Arte Brasileira,
realizado no \versal{MAM}-\versal{SP} em 2017. Ele foi terminado antes da abertura da
exposição. Digo isso pois não é mencionado nele nada do
episódio de intolerância radical detonado pela participação
inocente de uma criança na performance \emph{La Bête} de Wagner
Schwarz realizada

\noindent{}no dia da abertura.\footnote{Quem quiser ler uma discussão mais aprofundada deste
  episódio, escrevi dois artigos a respeito, a saber: \emph{Revista
  Jacarandá}, nº 6, Rio de Janeiro, 2018; \emph{Revista
   Concinnitas}, v.~19, n.~33 , 2018, p.~197-208.} Só menciono este fato pois ele ocupou e tomou de 
assalto todo o desdobramento da exposição. Estes acontecimentos
inesperados, por mais indesejados que sejam, têm sua função política e,
porque não, pedagógica. Revelou, acima de tudo, os retrocessos
civilizatórios em que estamos perigosamente mergulhados. Tais fatos
deram a esta revisitação de 1967 --- proposta por este seminário (e agora
livro) aqui da \versal{PUC}, assim como deste diálogo com o ``Esquema geral''
de Oiticica do Panorama --- uma ressonância agudamente atual. Os tempos
passam e as adversidades se renovam.

Em 1967, Hélio Oiticica escreveu um texto determinante para se pensar a
arte e o Brasil. Intitulado ``Esquema geral da Nova Objetividade'', há
nele um desenho panorâmico da cena artística àquela altura e dos
desafios a serem enfrentados.\footnote{No ensaio \emph{Que horas são?}, de Roberto Schwarz, é discutido o trauma da influência externa como um problema mal resolvido da cultura
  brasileira. Nossa dependência econômica e política em relação aos
  centros de poder do capitalismo ocidental complicava qualquer apelo a
  uma identidade cultural. A perspicácia do autor desmontava, à esquerda
  e à direita, a procura por uma brasilidade essencial. Sem essência
  identitária, todavia, o Brasil deve ser tomado, sem qualquer tom
  apologético, enquanto um experimento, já na origem, pós"-identitário ---
  e isso tem um valor político inestimável neste momento de tensões
  migratórias pesadas. Não obstante sua enorme contribuição ao debate,
  cremos que a leitura do tropicalismo feita por Schwarz fica por demais
  dependente de um repertório teórico que não lida com as ambivalências
  alegóricas inerentes a uma arte política cabível após o esgotamento
  das grandes narrativas revolucionárias. A luta política não tem mais
  modelos totalizantes à mão, e os conflitos constituem"-se no interior
  dos espaços instituídos, atravessados pelas contradições do mercado e
  visando a produção de formas heterogêneas de arte e de vida. O que se
  perde do ponto de vista de uma ruptura com o sistema instituído,
  ganha"-se enquanto aposta radical na pluralidade democrática e no
  atrito inerente à exposição das diferenças. As \emph{ideias estão
  sempre fora de lugar} e o que resta é a defesa inexorável dos espaços
  onde pulse algum \emph{exercício experimental de liberdade.}} Escrito em um momento politicamente tenso,
com desalentadoras perspectivas de futuro, para dizer o mínimo, ele
destaca seis características da arte brasileira: (1) vontade
construtiva; (2) tendência para o objeto; (3) participação do espectador
(corporal, tátil, semântica); (4) abordagem e tomada de posição em
relação a problemas políticos, sociais e éticos; (5) tendência para
proposições coletivas; (6) ressurgimento e novas formulações do conceito
de antiarte.

Conceber um Panorama da Arte Brasileira em 2017 tendo este texto como
inspiração deve ser visto como uma homenagem e um desafio. Muita coisa
mudou de lá para cá, não obstante minha convicção de que suas linhas
gerais ainda nos orientam de alguma maneira. Ali naquele contexto, há
cinquenta anos, junto com \emph{Terra em transe, Rei da vela,
tropicalismo,} abriu"-se um horizonte novo para o debate artístico e
político no Brasil. A utopia moderna e o sonho revolucionário
dissolveram"-se, obrigando a arte a repensar forças e formas de
enfrentamento ao poder instituído. Ambivalência e resistência
irmanaram"-se, gostemos ou não. Montar uma exposição hoje com artistas
contemporâneos (de diversas linguagens e campos de atuação, indo da
arquitetura à dança), a partir desse texto, deve ser tomado como uma
espécie de ensaio curatorial sobre o momento histórico do Brasil e seus
vínculos com o passado e o futuro --- no sentido de uma experimentação de
montagem que vê a articulação entre as obras enquanto uma narrativa
fluida de singularidades que, através das relações propostas, se
potencializam no conjunto.

Uma pergunta, ainda atual, perpassava a escrita do ``Esquema geral'' de
Oiticica: como apostar em uma relação nova entre singularidade local e
inserção global. No caso da cultura brasileira --- e isso foi colocado de
modo muito original pela geração tropicalista sob a influência da
antropofagia ---, nossa singularidade foi sendo construída pela mistura
de diferentes matrizes culturais. Ou seja, não temos uma essência
própria, uma marca de origem a ser depurada de qualquer contaminação
indesejada, vivemos da apropriação constante do outro, somos uma colagem
de influências que não para de se transformar. Como escreveu Oiticica,
estamos sempre ``à procura de uma caracterização cultural, no que nos
diferenciamos do europeu com seu peso cultural milenar e do americano do
norte com suas solicitações superprodutivas''.

Mario Pedrosa, em um texto sobre Oiticica publicado no ano anterior ao
``Esquema geral'', já apontava para uma transição de período histórico, para
um esgotamento das premissas básicas da arte moderna: a
autorreferencialidade dos valores plásticos e a projeção utópica em
direção ao futuro. Agora, diz ele, ``nessa fase de arte na situação, de
arte antiarte, de arte pós"-moderna dá"-se o inverso: os valores
propriamente plásticos tendem a ser absorvidos na plasticidade das
estruturas perceptivas e situacionais''.\footnote{Mario Pedrosa.
  ``Arte ambiental, arte pós"-moderna, Hélio Oiticica''. In: \emph{Dos
  Murais de Portinari aos espaços de Brasília}. São Paulo: Ed.
  Perspectiva, 1981, p. 206.} Há uma integração da obra em seu ambiente
cultural, incorporando novas materialidades, absorvendo e deslocando
constantemente elementos da história da arte, do cotidiano e da cultura
popular e de massas. Não se trata de redução nas expectativas
experimentais ou de captura da arte na lógica do consumo ou do
espetáculo --- mas de redefinir os critérios pelos quais se julgam as
formas de experiência e intervenção da arte dentro do sistema integrado
da cultura. Por um lado, a mediação do mercado e das redes de
comunicação parece inevitável para a circulação e difusão (e, porque
não, democratização) da produção cultural. Por outro, a aceleração e a
consequente dispersão da sensibilidade cotidiana, somadas à
homogeneização, via capital, dos valores de legitimação do trabalho e da
subjetividade, tendem a limitar as expectativas da produção
artística.\footnote{Seria interessante percebermos aqui como um crítico
  comprometido até à medula com o projeto moderno, como é o caso de
  Mario Pedrosa, e que sempre esteve vinculado a uma leitura libertária
  do marxismo, não era refratário às contaminações propostas naquele
  momento tropicalista. Se mudavam os critérios de ajuizamento da arte,
  mudavam também as expectativas em relação à dimensão crítica da arte.
  É dentro desse contexto que surge a defesa do fazer artístico enquanto
  exercício experimental de liberdade.}

É dentro dessa situação ambivalente, de mais visibilidade e menos
atenção, de intensificação sensorial e espetacularização dos afetos, que
a arte pós"-moderna de Oiticica vai operar. Não quero entrar no mérito da
discussão sobre pós"-modernismo. O que me interessa é que, diante daquele
contexto da década de 1960, no qual as obras de Oiticica, Clark e Pape
ganham uma inflexão ao mesmo tempo mais experimental e mais
transdisciplinar, um crítico como Pedrosa, acompanhando atentamente os
acontecimentos artísticos e políticos, sente necessidade de rever os
termos que pautavam a crítica modernista. Sem abrir mão do atrito, o
fazer artístico incorpora afetos mais excitados e dispersivos. Dispersão
não significa necessariamente redução ou diluição. É aí dentro que a
vontade construtiva deve atuar e se articular com a pulsação da
antiarte.

A dimensão ambiental que Pedrosa via em Oiticica pressupõe que as obras
de arte atuem no interior de uma situação cultural complexa e
atravessada por contradições de todo tipo. Uma nova ecologia começava a
se desenhar para a arte em escala global. A aproximação com a energia do
samba ou de práticas terapêuticas experimentais, para falarmos de casos
nossos conhecidos, é parte dessa dinâmica --- e isso ganhou
desdobramentos os mais variados de lá para cá, o que tentamos explicitar
neste Panorama. ``Ambiental é para mim a reunião indivisível de
todas as modalidades em posse do artista ao criar --- as já conhecidas:
cor, palavra, luz, ação, construção etc., e as que a cada momento surgem
na ânsia inventiva do mesmo ou do próprio participador ao tomar contato
com a obra''.\footnote{ Hélio Oiticica. ``Programa ambiental''. In:
  \emph{Aspiro ao grande labirinto}. Rio de Janeiro: Rocco, 1986, p.~78.}
O signo visual ganha sentido no interior de um jogo de linguagem que
incorpora elementos culturais estranhos ao regime puramente artístico. A
fenomenologia da percepção começava a assumir as tensões de uma
sociabilidade conflituosa inscrita nos corpos e nas subjetividades. As
linhas de fuga possíveis não contam com um fora que adviria de um corte
revolucionário e nem com a projeção utópica de uma sociedade livre do
capital e do mercado. Isto não implica renúncia ou acomodação política,
o que muda são os termos em que se dão os enfrentamentos.

``Assumir ambivalências não significa aceitar conformisticamente todo
esse estado de coisas; ao contrário, aspira"-se então a colocá"-lo em
questão. Eis a questão''.\footnote{Hélio Oiticica. ``Brasil diarréia''.
  In: \emph{Hélio Oiticica: Encontros}. Rio de Janeiro: Azougue, 2009, p.
  116.} Segue a questão. A dimensão construtiva deve saber dos limites
de sua vontade de transformação social. Ao mesmo tempo, a contrapelo dos
limites, a arte deve seguir mobilizando formas de sentir e pensar
heterogêneas, assim como resistir à transparência comunicativa
produzindo atritos nos códigos de reconhecimento da linguagem cotidiana.
De lá para cá, a ambivalência cresceu.

A despeito da falência da ideia de progresso e de uma avassaladora crise
urbana e ambiental, ainda resiste uma vontade construtiva entre nós. Uma
construção que se sabe frágil, mas crucial para enfrentar os riscos de
uma informalidade desagregadora. Mais uma vez, Oiticica nos dava, já em
1963, pistas para reconfigurarmos a compreensão de construtividade, tão
cara a partir do concretismo e que àquela altura, no contexto da
problemática pós"-moderna apontada por Pedrosa, ganhava outras variáveis.
``São os \emph{construtores}, construtores da estrutura, da cor, do
espaço e do tempo, os que acrescentam novas visões e modificam a maneira
de ver e sentir, portanto os que abrem novos rumos na sensibilidade
contemporânea (\ldots{})''.\footnote{Hélio Oiticica. ``A transição da cor
  do quadro para o espaço e o sentido de construtividade''. In:
  \emph{Aspiro ao grande labirinto,} op.~cit., p.~55.} A construção
opera sobre os rumos da sensibilidade, sobre o modo como vemos, falamos,
pensamos, sem que isso implique descompromisso político ou qualquer tipo
de alienação diante dos desafios sociais. Justamente o contrário: a arte
pós"-moderna e pós"-utópica se faz política, deslocando nossas formas de
estarmos no mundo, de vivermos temporalidades heterogêneas e produzirmos
territórios de compartilhamento de experiências menos restritivos. Isto
implica também procurar abrir brechas nas instituições. Falidas as
promessas utópicas e os modelos hegemônicos de colonização do futuro,
será no território agônico do presente, constituído por várias camadas
temporais combinadas, que se produzirão as diferenças que nos fazem
acreditar em um mundo comum menos homogêneo e mais oxigenado.

A precisão formal aliada à precariedade material garante às instalações
de Fernanda Gomes uma vocação simultaneamente estética e ética. A
apropriação do descartável e a fragilidade com que um gesto constrói um
momento de forma são uma lição para um mundo que só pensa a produção sob
a lógica do consumo e da destruição. Este mesmo gesto se desdobra no
corpo"-bicho de Wagner Schwartz (\emph{La Bête}), que vai se moldando no
contato com o outro, expondo seu corpo ao contato e ao gesto do outro
que ao manusear o corpo, se sente manuseado. Deslocando para a escala
ampliada do espaço urbano, temos as intervenções em zonas urbanas
precarizadas, como na Rocinha ou na comunidade de Manguinhos, realizadas
pelo arquiteto Jorge Mario Jáuregui (\emph{Encontros e alianças}) sempre
em diálogo franco com a comunidade que participa das decisões e escolhas
sobre como transformar o espaço urbano em que vivem, trabalham,
circulam. Inserindo a discussão proposta por este Panorama a partir do
texto do Oiticica dentro do histórico de seu trabalho na cidade,
Jáuregui ressalta que ``o desafio de articular a cidade dividida entre
formal e informal implica dar visibilidade e oportunidade para a
emergência de pessoas sem importância, permitindo novos encontros e
alianças''.\footnote{Texto enviado em troca de e"-mails na preparação da
  curadoria.} Em cada uma dessas poéticas mencionadas a construção não
se assume de fora, não intervém sem comprometer o outro, a própria vida.
Fazer com o outro e junto ao outro.

É a partir dessa visão de construtividade situada, integrada ao ambiente
em que se insere, que vemos também uma crescente abertura do fazer
artístico para problemas sociais, éticos e políticos, ou seja, para um
engajamento, nada simplificador, que acredita nas brechas em que a arte
quer se infiltrar para tentar mudar as coisas --- sabendo"-se que querer
mudar não basta e que sua impotência pode ter desdobramentos
imprevistos. O modo como as poéticas atuam diante desses desafios de
ordem extra"-estética varia enormemente. O que se convencionou chamar de
\emph{artivismo} deve ser compreendido dentro do campo alargado da crise
política, em que as formas de participação e intervenção buscam
alternativas perante o engessamento da democracia representativa.
Olhando sob outro ângulo, dentro daquilo que se poderia nomear como a
zona de atrito inerente à arte, muito do que se tem defendido como
engajamento político acaba por constranger a indeterminação do
experimental. Esta discussão que em 1967 tinha como polos de
enfrentamento, dentro do campo progressista, de um lado, os artistas que
queriam manter e radicalizar a experimentação, de outro, os que
defendiam um programa nacional"-popular. Esta tensão, hoje, guardadas às
diferenças, situa"-se nas delimitações dos lugares de fala. À arte
caberia participar dos processos de afirmação identitária, da construção
e potencialização de vozes que se mantiveram, via opressão de toda
ordem, sempre caladas. Evidentemente, tais compromissos são fundamentais
e louváveis. Muito do que de mais contundente foi feito na defesa de
minorias teve na produção artística um instrumento de luta
imprescindível. Todavia, isso não deve ser compreendido como impedimento
de experimentações pós"-identitárias, em que a arte e o gesto poético
produzem devires imprevistos e abrem campos de diálogo"-tradução onde
antes havia apenas ruído e exclusão. O que se defende para a arte é a
possibilidade, mais ainda, o compromisso com a indeterminação radical
diante de toda e qualquer identidade fixa e a fratura no controle do
endereçamento. Não interessa, \emph{a priori}, definir quem fala e a direção do
que é dito. ``Os fios soltos do experimental são energias que brotam
para um número aberto de possibilidades''.\footnote{Hélio Oiticica.
  ``Experimentar o experimental''. In: \emph{Hélio Oiticica: Encontros},
  op.~cit., p. 109.} Que se multipliquem os atores sociais com a defesa
sem trégua dos espaços de afirmação das vozes minoritárias
historicamente excluídas. Conflito, dissenso e liberdade experimental
devem conviver a partir daquilo que Oiticica impunha no ``Esquema geral''
como obrigação do artista e do intelectual engajados: ``a necessidade de
abordar esse mundo com uma vontade e um pensamento realmente
transformadores, nos planos ético"-político"-social''.\footnote{Hélio Oiticica.
  ``Esquema geral da Nova Objetividade''. In: \emph{Aspiro ao grande labirinto}, op.~cit.}

Uma das possibilidades que se abriram a partir daquele momento, ou seja,
há exatos cinquenta anos, foi uma ampliação da noção de participação
através do deslocamento do gesto criativo em direção ao outro.
Participar implicava convocar. Para além da participação no sentido
semântico, comunicativo e sensorial abordados por Oiticica, o que foi se
definindo mais recentemente foi uma nova convocação: não mais falar em
nome do(s) outro(s), mas convocá"-lo(s) como força(s) criadora(s). Seja
na Mangueira, seja em workshops na Sorbonne, os artistas buscavam na
energia popular e na pulsão coletiva dos corpos, formas de articular
processo e obra; a arte e seus processos abertos passam a ser um espaço
de mobilização e disseminação de novas possibilidades coletivas. Novos
agenciamentos e cumplicidades atravessam o processo criativo e atuam
sobre uma multiplicidade de corpos. Neste aspecto, a arquitetura (\versal{RUA}
arquitetos e Jorge Mario Jáuregui) e a dança (Marcelo Evelin e Wagner
Schwarz) presentes neste Panorama, apontam para poéticas
coletivas que interferem em corpos e circuitos de alta conectividade, e
seus resultados não se fecham em um significado específico, mas se
multiplicam em intensidades e funções constantemente redefinidas. O
coletivo \emph{Mão na Lata}, criado por Tatiana Altberg e atuando na
comunidade da Maré no Rio de Janeiro é um exemplo de agenciamentos
criativos que articulam imagens, corpos e textos na produção de
subjetividades que se afirmam poderosamente à revelia da adversidade
absurda que grita nas periferias brasileiras. O fora e o dentro, a
comunidade e a casa, articulam"-se de forma intensa nas cenas construídas
por estes jovens que se lançam ao mundo destemidamente. A fabulação é
uma poderosa arma de invenção de si, liberada pela força criadora do
coletivo.

Desdobrando esses processos de criação coletiva, os diagramas e leituras
apresentados por Ricardo Basbaum (\emph{conversa"-coletiva: nova
objetividade/nova subjetividade}) integram palavra, cor, linhas, gestos,
vozes e audição através de performances coreografadas na conjunção de
texto e grupo. Intensidades e propagações. Eu, o outro, o comum. A mesma
abertura processual em que o contato com o outro vai produzindo
agenciamentos poéticos inesperados é percebida no projeto de Cadu
(\emph{Soy Mandala)} mostrado neste Panorama. Atuando durante
meses junto a uma comunidade de costureiras mexicanas, ele participa e
se aproxima, através de uma atividade lúdica, à dança, que integra a
costura dos corpos à costura de uma mandala. Já Bárbara Wagner e
Benjamin de Burca (\emph{Como se fosse verdade}) convocam pessoas,
quaisquer, para compartilharem sonhos e músicas, fabularem sobre si
mesmos, imaginarem outros mundos menos achatados na mera sobrevivência.
A estética cafona é parte da liberdade exercitada como resistência às
imposições do bom gosto convencional. Desde a \emph{Tropicália} que a
pureza se tornou um mito. A cafonice como pulsação vital é um aceno para
a produção de singularidade. Isso aparece também em Leandro Nerefuh
(\emph{Uma} \emph{breve história da banana na história da arte}), cuja
ironia das associações somada à agressividade dos padrões estéticos, faz
brotar um riso nervoso que desestabiliza padrões normativos que
capilarmente determinam formas de poder.

O vídeo e xilogravuras de José Rufino (\emph{Insolentia}), combinando
arquivos de uma memória do trabalho e da opressão no que sobrou de uma
velha usina de açúcar redireciona a questão para uma geografia de afetos
contraditórios. O Brasil urbano se desloca para o sertão nordestino e a
formas de opressão e luta diferentes. Uma beleza áspera cintila no meio
dos escombros que um dia deram unidade a um sistema de trituração de
corpos. As práticas artesanais deslocam"-se do trabalho na usina para as
xilografias e destas para a madeira trabalhada por Marcelo Silveira
(\emph{Manuais de liêdo}). Temporalidades heterogêneas submergidas na
velocidade urbana despontam nas feiras populares e nos hábitos
impregnados que resistem à margem de nossa modernização conservadora.
Possibilidades, apesar da desigualdade, combinam"-se incansavelmente
neste Brasil que é contínua multiplicação de mundos --- a um só tempo
precários e potentes, pós"-modernos e medievais. Clarice Hoffmann e
Lourival Cuquinha (\emph{Macunaíma coloral)} vão atrás de cores que
retratam tipos raciais pouco ortodoxos e socialmente desvalidos. A
auto"-designação racial em uma sociedade mestiça implica um mosaico de
adjetivações que buscam dar conta de tonalidades difusas. A diversidade
cromática não impede o racismo instituído, que exclui sem hesitações. Da
pele da usina arruinada, passando pelos corpos explorados, chegamos
outra vez à madeira trabalhada por Marcelo Silveira, que é pele, folha,
texto e daí às varandas do \versal{RUA} arquitetos. A varanda é a
institucionalização do puxadinho, o elo entre a casa e a rua, o convívio
e a privacidade. O que sobra da artesania popular e da improvisação
cotidiana abre brechas em uma normatividade burocratizada, reinventa"-se
na expressão de imaginários represados cuja circulação atual acaba
sugerindo formas de vida menos homogeneizadas. Do mesmo modo, no filme
de Karim Ainouz e Marcelo Gomes (\emph{Compasso)}, o velho que dança
sozinho em um fim de festa popular, combina melancolia e resistência que
se inscrevem em um corpo que é pura sabedoria rítmica. O carnaval e o
futebol são imagens clichês do Brasil, mas dentro deles há camadas
sensoriais inexploradas para além do afeto barato da alegria induzida. A
forma de mostrar essas imagens, em um corredor saturado de azul, nos
obriga a chegar perto dos pequenos monitores com os filmes ao mesmo
tempo em que somos levados para longe, para uma luz onírica e
artificial. O perto e o longe, o pertencer e o não"-pertencer se
articulam nesta instalação de Marcelo e Karim.

Propositalmente, muitas poéticas neste Panorama reverberam este
debate/cruzamento entre ética, política e arte, sendo que fazem isso a
partir de um contexto local, atravessado por especificidades de uma
formação cultural problemática --- na qual, como mostram exaustivamente
os textos do crítico literário Roberto Schwarz, o discurso liberal
conviveu e convive, sem medo de ser feliz, com uma realidade
escravocrata. Felicidade trágica, portanto. Explicitar as diferenças,
enfrentar as complexidades, construir dissensos que passam pela
performatização daquilo que aparece, mas não tem visibilidade
reconhecida, são estratégias poéticas recorrentes que buscam rotas de
escape no interior da captura institucional. Em mais uma volta da
ambivalência, o que se percebe é que muito da vontade construtiva foi
canalizada, principalmente depois do golpe militar e do \versal{AI}-5, à revelia
do interesse comum em uma atrofia desenvolvimentista, que não só esgotou
impiedosamente nossas riquezas naturais como concentrou poder e dinheiro
de maneira desavergonhada. Modelo este desastrosamente repetido na
segunda parte dos governos do \versal{PT}. A transformação da construção em
progresso sem apreço pelo singular foi mais uma volta da nossa tão
batida aplicação de modelos esgotados e sem nenhum vínculo com as
demandas locais, ou seja, a exploração infinita de ideias fora de lugar.
Os projetos realizados para este Panorama por Beto Shwafaty
(\emph{\versal{IPO}: unidade estética, distribuição econômica}) e Romi Pocztaruk
(\emph{\versal{BOMBABRASIL}}), levam ao limite do absurdo os descaminhos do
desenvolvimentismo através de leituras críticas do que foi feito de
Brasília, da Petrobras e da usina nuclear de Angra. A vontade
construtiva de muitos reduzindo"-se à construção da vontade de poucos ---
progresso como expropriação, geometria do desastre ambiental. A estética
de ambos os projetos, em sua frieza bruta, explicita nossas
contradições.

A vídeo"-instalação de Dora Longo Bahia (\emph{Brasil x Argentina}),
atravessando as queimadas na floresta amazônica e o degelo na Patagônia
expõe uma espécie de des"-razão sublime que nos faz antecipar a própria
ideia, mais real que nunca, de um fim do mundo que nos espera de braços
abertos. A calma com que a bola vai passando de pé em pé é própria dessa
tonalidade afetiva melancólica e cínica que nos imobiliza por dentro, no
ritmo lento da adaptação perversa e do pesadelo ambiental. Trazendo a
floresta para dentro do \versal{MAM}-São Paulo, João Modé (\emph{Land}) ocupa o
espaço"-estufa onde fica a \emph{Aranha} de Louise Bourgeois ---
temporariamente fora do museu para restauração. Aí dentro algumas
esculturas da coleção convivem em um jogo de estranhamento no qual as
linhas divisórias entre natureza e cultura ficam borradas. Quanto da
arte é vida e quanto da vida natural é artifício? Devemos, de uma vez,
assumir os cruzamentos e os híbridos, construir com a natureza sem
tomá"-la como objeto manipulável.

As contradições existem e devem ser assumidas. A arte está mergulhada aí
dentro e procura, como sempre, brechas. Se o desastre é o destino do
progresso, cabe à imaginação nos fazer saltar para fora dessa rota e
procurar olhar em outras direções. Se vivemos um presente amplo de
contemporaneidades, como diz o teórico Hans Ulrich Gumbrecht, devemos
radicalizar essas multi"-temporalidades para buscar novas formas de vida.
A arte, em sua potência indeterminada, constantemente transformada em
impotência poderosa, pode ser uma aliada, desencavando potencialidades
esquecidas e imaginando outros pequenos mundos possíveis. O sintoma de
uma vida na adversidade com que Oiticica termina seu texto não deve ser
tomado como algo conjuntural ou provisório, mas como nossa condição
mesma de vida, calcada em subjetividades e sociabilidades precárias que
se mantêm de pé sempre por um triz e só podem se mobilizar no contato
com o outro. A adversidade é nossa condição. Saber dela, viver com ela e
a partir dela, nos fará menos presunçosos em nossa vontade de potência e
mais aptos a encontrar formas de habitar o mundo e cuidar dele. Os
corpos que dançam em \emph{Apêndice} --- projeto especial do coreógrafo
Marcelo Evelin (Demolition Incorporada) para esse Panorama --- revelam
justamente a exaustão e o mal"-estar diante da agressividade imposta pela
combinação entre hiper"-produtividade, precarização da vida e
intolerância crescente diante das diferenças.

Neste aspecto, trazer Ibã Huni Kuin para realizar o projeto Parede junto
ao 35º Panorama da Arte Brasileira é indicativo de que outras
temporalidades e imaginários podem conviver dentro de nossa
contemporaneidade acelerada. Tempos menos apressados e mais atentos ao
outro (humano, animal, vegetal, divino) que buscam desencadear novos
modos de viver singular tão necessários se quisermos seguir adiante.
Segundo o antropólogo Amilton Mattos, que acompanha de perto, na
Universidade Federal do Acre, o movimento \versal{MAHKU} dos Huni Kuin, há, nesse
projeto, a disposição de aprender com o outro, enfatizando a arte de
prestar atenção e o saber dos povos da floresta que os Huni Kuin trazem
como herança longínqua. Um saber do futuro.\footnote{Parágrafo
  apresentado junto ao projeto Parede com Ibã Huni Kuin durante o 35º
  Panorama da Arte Brasileira.}

Reunir em uma exposição, que se pretende um Panorama da Arte Brasileira,
desde a concretude da intervenção arquitetônica até a fluidez da dança,
passando pelo audiovisual, pela escultura, pela fotografia e pela
palavra, mais que explicitar a diversidade da cena contemporânea, em que
a divisão de meios expressivos e de disciplinas parece obsoleta, busca
ressaltar a multiplicidade de tempos que compõem nosso momento histórico
a contrapelo e junto à homogeneidade globalizada. O tempo do corpo que
dança, da palavra escrita e da imagem projetada respondem a formas de
percepção e de experiência plurais. Simultaneamente, é parte de nosso
desafio articular os diferentes imaginários que se contaminam e se
multiplicam no Brasil entre a cidade e a floresta, as comunidades
periféricas e os centros cosmopolitas, entre o caos, a indeterminação e
o mito.

Misturar poéticas conflitantes, trazer outras vozes e gestos para dentro
das instituições que constroem as narrativas hegemônicas, revelar
antagonismos e diferenças, tudo isso é parte de uma ideia de
Panorama e de uma discussão sobre o Brasil, tendo o texto de
Oiticica de 1967 como iluminação profética vinda do passado. Isso, no
exato momento em que o Brasil vive uma de suas piores crises de
identidade, quando a promessa de futuro virou uma terrível distopia que
constrange as possibilidades do presente. Mais uma vez, sem medo das
contradições, parece propício colocar a pergunta sobre o Brasil. Quais
as nossas possibilidades enquanto cultura que se sabe, ao mesmo tempo,
periférica e continental, moderna e arcaica? Pergunta tão mais urgente
quão mais longe estamos da ideia de uma modernidade enquanto projeto
inacabado. O Problema"-Brasil é um desafio e uma miragem: aparece como
promessa de alegria, mas escapa quando vamos em sua direção. E, a cada
passo, parece que vai para mais longe. Entretanto, não dá para virar as
costas; há que se encarar a miragem, ao mesmo tempo ilusória e real,
fazendo deste enfrentamento o caminho para nos tornarmos menos
assombrados com nossa assustadora incompetência coletiva. Esta exposição
é um ensaio de possibilidades poéticas cuja montagem articula desejos e
afetos que não se reduzem às necessidades funcionais do presente. A arte
é o espaço disponível para ampliarmos o campo do possível.

\section{Referências}

\begin{Parskip}
\textsc{oiticica}, Hélio. \emph{Aspiro ao grande labirinto}. Rio de Janeiro: Rocco, 1986.

\_\_\_\_\_\_. \emph{Hélio Oiticica: Encontros}. Rio de Janeiro: Azougue, 2009.

\textsc{pedrosa}, Mario. \emph{Dos Murais de Portinari aos espaços de Brasília}. São Paulo: Ed. Perspectiva, 1981.

\textsc{schwarz}, Roberto. \emph{Que horas são?}. São Paulo: Companhia das Letras, 1987. \enlargethispage{\textheight}
\end{Parskip}

\chapter*{A hora e a vez do sendo}
\addcontentsline{toc}{chapter}{A hora e a vez do sendo, \emph{por Marcia Sá Cavalcante Schuback}}


\begin{flushright}
\emph{Marcia Sá Cavalcante Schuback}
\end{flushright}

\emph{1967, meio século depois} --- a proposta do seminário mobiliza
muitas questões.
A primeira é o modo como uma data mobiliza a hora e a vez do nosso
agora. Agora, cinquenta anos depois de 1967. Mas 1967 também foi um agora,
cinquenta anos depois de 1917, a data da Revolução Russa. São muitos,
porém, os modos de contar. 1967 foi também um ano antes de 1968, 1968 ele
mesmo cinquenta anos depois do final da primeira guerra mundial; 1967, um
ano antes de 1968, ano marcante de múltiplas liberações e violentas
repressões. Para alguns, um ano antes da exceção repressiva
institucionalizada; para outros, um ano antes da liberação
desinstitucionalizadora. Para muitos, um e outro, um no outro.

Contar a hora e a vez de nosso agora como cinquenta anos depois de 1967
pode ser entendido de muitos modos também. ``Meio século depois'':
significa dizer que somos a sequência e a consequência de 1967? De certo
modo sim porque somos sempre (numa) uma história. Mas justamente por 1967
não ser mais, já ser um passado e desse modo uma herança, somos também a
interrupção e a inconsequência de 1967. Nessa conta, somos devolvidos não
só para uma sequência causal. Somos devolvidos sobretudo para uma data
energética, para uma data que não é meramente um ponto numa sequência
temporal mas uma energia, uma energia de liberação. Somos devolvidos
para a liberação de uma energia. Energia, energia --- o tom fundamental
em toda ``alegria, alegria'', em todo sofrer, sofrer. Hoje, cinquenta
anos depois, num agora de esperanças arrebentadas, pode"-se bem entender
porque mobilizar a hora e a vez do agora à luz dessa energia.

1967, meio século depois --- no contexto brasileiro, apresenta uma conta
menos redonda entre 1922 e 1967, entre antropofagia e Tropicália. 1967, como
Caetano Veloso formulou, é o ano que marca a enunciação do
``tropicalismo como neo"-antropofagismo''. É também o ano inserido na
década que marca a conta ainda menos redonda entre concretismo e
neoconcretismo, até mesmo a conta entre São Paulo e Rio, lembrando o
modo como Mario Pedrosa a formulou. É o marco de uma historicização
estética e de uma estética historicizante. A história contra a história
transparece, história versus estória. Não mais só o devorar do outro mas
também e muito o devorar de si mesmo, a antropofagia da antropofagia,
uma meta"-antropofagia e autofagia.\footnote{Peço licença para remeter ao
  meu texto ``Antropofagia da brasilidade''. In: \emph{Olho a olho}. Rio
  de Janeiro: 7Letras, 2010.} Sem dúvida pode"-se discutir o sentido
dessa auto"-antropofagia, se esta deve ser entendida como realização ou
acabamento do projeto antropofágico ou já como uma espécie de
``dissidência'' da antropofagia. É possível argumentar no sentido de
ambas as alternativas serem pertinentes pois esse acabamento se fez
extraviando"-se de si; afinal antropofagia é por definição já um extravio
e uma extravagância.

Se o antropofagismo --- ou seja, a metáfora fulgurante do modernismo
brasileiro --- se pronunciou por manifesto, o tropicalismo, o
``neo"-antropofagismo'' --- não lê tanto manifesto. Assim surge a
pergunta: quem lê tanto manifesto? Essa pergunta faz sentido sobretudo
hoje, que formulado em termos eloquentes de ``meio século depois'',
deixa transparecer uma operante nostalgia dos manifestos e manifestações
dos sonhos da vanguarda.

Se filosofia faz algum sentido não é de modo algum pela acumulação do
capital do seu saber ou por um tipo de resposta --- a resposta
fundamentada, cheia de certezas, sejam elas universais e cabais ou
locais e dogmáticas. Se filosofia faz sentido é precisamente por tornar
respostas questões e não por dar respostas a questões. Gostaria assim de
dar uma chance à filosofia e ficar na pergunta --- ``quem lê tanto
manifesto?'' e sugerir que essa pergunta colocou em questão todos os
elementos dessa mesma pergunta; colocou em questão o ``quem'', o ``lê'',
o ``tanto'' e o ``manifesto''. É uma pergunta grávida de perguntas. A
primeira refere"-se a

\begin{enumerate}[label=\alph*]
\item \emph{Quem} --- Quem é quem? Não só ``quem'' fala, não só de que lugar se
fala, mas de onde se pergunta sobre quem --- ou seja, qual o lugar da
fala sobre um lugar da fala. De onde fala a exigência de que toda fala
tenha um lugar e seja a fala de um quem? \emph{Quem lê tanto manifesto?}
expressa um questionamento do quem e de sua subjetividade.

A segunda busca

\item \emph{Ler} --- não só isso ou aquilo, o manifesto e o que não está
manifesto, mas ler o ler, e diz assim respeito à atenção ao ato de
leitura, à leitura como ato. \emph{Quem lê tanto manifesto?} expressa um
questionamento das leituras não só dos manifestos mas sobretudo da
leitura como manifestação, ou seja, do que seja um ato de leitura.

A terceira questiona

\item \emph{Tanto} --- não mais a quantidade, não mais a qualidade ou a
oposição entre quantidade e qualidade, erudito e popular, nacional e
internacional, nativo e o gringo, mas o tanto como questão de acentos: o
acento grave da história, o acento agudo do presente, o acento
circunflexo do possível, parafraseando Paul Celan. \emph{Quem lê tanto
manifesto?} expressa um questionamento mais de acentos do que pesos e
medidas.

E por fim, a pergunta pelo

\item \emph{Manifesto} --- \emph{não mais} enquanto um texto de prescrição,
um projeto de dever"-ser, que, desde o \emph{Manifesto comunista}, acompanha as
inúmeras variações modernistas desse manifesto nos manifestos estéticos,
que são manifestos por tornarem manifesta e pública a relação entre
estética e política e a linha tênue que separa a estetização da política
e a politização da estética (lembrando a distinção feita por Benjamin em
conversa com Brecht). Não mais. O tropicalismo diz porém esse ``não
mais'' ao manifesto de maneira unicamente múltipla. Diz mudando os
acentos. Diz ``não mais não mais''; diz acentuando os acentos --- pois
acentua ``não mais não mais'' às vezes dizendo ``não'', ``mais'', às
vezes, ``mais não'', e por ai afora, incluindo até os absurdos mais
preciosos da língua portuguesa que diz ``pois não'' para dizer sim, e
``não é não'' para dizer não. \emph{Quem lê tanto manifesto?} expressa
um questionamento do discurso e da linguagem.
\end{enumerate}

À guisa de ensaio, pode"-se dizer que o tropicalismo transforma a
antropofagia por meio de um \emph{glissando}.\footnote{Sobre o modo como
  estou usando aqui o termo \emph{glissando}, cf. meu artigo ``Literatura,
  filosofia e utopia: o espaço da antropofagia''. In: Pedro Duarte (org.). \emph{O que nos faz
    pensar}, n.~38, 06/2016, \versal{PUC"-RJ}.}
Um \emph{glissando} de manifesto para manifestação. 1967, nessa data fica
manifesto que não basta ler manifesto, essa forma de dizer não ao
infesto e sim a um projeto de ser. Fica manifesto que não basta dizer
não. Fica manifesto que é preciso dar tempo e abrir espaço para que o já
manifesto se manifeste. É tempo e hora de manifestações --- políticas,
existenciais, estéticas, artísticas, físicas, metafísicas. Neo"- (de
neo"-antropofagismo) descobre outros étimos, o neo de neon, o neo como
anagrama de noe, de não é? De ``por que não?''.

1967 --- Nesse mesmo ano, Guy Debord lança a \emph{Sociedade de
espetáculos}, McLuhan, \emph{O meio é a mensagem}, Derrida,
\emph{Gramatologia}, \emph{Escritura e diferença}, \emph{A voz e o fenômeno},
Horkheimer, a \emph{Crítica da razão instrumental}, Roland Barthes,
\emph{A morte do autor}, e Martin Luther King, \emph{Para onde vamos
daqui: caos ou comunidade?} Em muitos lugares é a hora e a vez da
manifestação. Poesia grafita a rua; parangolés na rua, para fora, tudo
tem de ir para fora, pois o corpo do tempo deve se desnudar. Mas como
entender essa urgência de manifestação? Cinquenta anos depois,
mergulhados numa espécie de quarta feira de cinzas das manifestações,
quando as artes ambientais são instaladas e assim interiorizadas pelo
capitalismo dos museus e das galerias, pelo vampirismo do mercado, o que
resta dessa energia da manifestação? Mas será mesmo essa a pergunta
adequada para a hora e a vez de nosso agora? Ou não será mais adequado
perguntar como manifestar essa energia que, voltando"-se contra si mesma,
aparece esgotada, explorada, violentada, torturada, arrebentada, mas
ainda uma energia, uma energia minguante? Como manifestar a lua
minguante dessa energia? Como pensar a manifestação da energia minguante
da criação?

Manifestação não é só ir às ruas a favor ou contra uma causa.
Manifestação é tornar manifesto o que está manifesto. Na manifestação, a
realidade se manifesta, colocando à flor da pele a realidade sensível.
Na manifestação, o que é e está sensível se sensibiliza. Existe uma
confusão na filosofia acerca do que significa manifestação. Durante
séculos, se acreditou --- filosofia e ciência estão alicerçadas em credos
--- que por trás das coisas há significações, sentidos ocultos,
fundamentos e razões e que é preciso desviar os olhos das coisas para
ver as suas significações, filosóficas e científicas, metafísicas e
racionais. As coisas foram entendidas como manifestações de
significações ocultas, como aparências de essências. A virada do século
\versal{XIX} para o \versal{XX} foi vertiginosa, pois tornou"-se manifesto que as coisas
não tem significação, mas sim existência, lembrando os versos caieiros
de Pessoa, que por trás das coisas não há mistério algum, que as coisas
é que são mistério. A tarefa seria devolver às coisas o seu valor de
mistério e enigma --- tarefa gigantesca de Clarice. A tarefa, des"-eificar
as coisas reificadas. Coisa não é em si. Coisa é manifestação.

Quando Marx expõe a sua teoria do fetiche das mercadorias, quando
Nietzsche propõe uma genealogia do sentido, quando Freud elabora a
teoria do inconsciente, o que aparece não é um sentido oculto das coisas
mas o modo como a sociedade, a cultura, a história, a filosofia, a
ciência ocultam o manifesto, o modo como a sociedade suicida a criação,
como diria Artaud. A tarefa passa a ser desocultar as ocultações,
aprender a desaprender, tornar manifesta a manifestação. A tarefa passa
a ser cair na real onde já sempre se está, de onde não há saída, pois só
o real é entrada"-saída. Mas não no sentido de uma alienação no dado, e
sim de um estranhamento dentro do modo como o dado se dá, como os fatos
se fazem, como o aparecer aparece. Pois se cabe falar de ser é somente
no sentido de aparecer. Ser é aparecer. Mas aparecer só aparece
desaparecendo e se escondendo no que aparece. Seria mais preciso dizer
ser é des"-aparecer. Quando alguém chega, com esse alguém chega
igualmente o chegar. A questão é como ver, dizer, pensar o aparecer no
que aparece, o chegar no que chega, o ver no que se vê, o dizer no que
se diz. Em questão está como ver, dizer, pensar, verbalmente,
gerundivamente. Como pensar o sendo de dentro do sendo, sem prender"-se
ao dogmatismo das posições e sem desmaterializar"-se no pensamento vazio
de um devir.

Pensar --- ou seja, sentir, ver, dizer, escutar o sendo de dentro do
sendo, o aparecer aparecendo, a manifestação se manifestando, é ver uma
não"-coisa em todas as coisas vistas. Não"-coisa não deve ser entendido de
forma alguma como outra coisa, transcendente ou transcendental, absoluta
ou fundamental, em si ou para si. Não"-coisa tem mais o sentido de
\emph{probjeto}, usando o termo de Hélio Oiticica, no sentido de coisa %%Probjeto mesmo?
se dando como coisa de maneira a dar conjuntamente o dar"-se, no sentido
de coisa"-movimento, se movimentando e movimentando. A manifestação que
toda coisa é só se deixa apreender no instante em que tocar e ser
tocado, sentir e sentido, ver e ser visto não se dissociam, expõem"-se
como um só nó. Um nó erótico. Em questão, em jogo, está a manifestação
nua, um aprendizado de desaprender as ideias de ser"-em"-si e para"-si das
essências, dos preconceitos, das predeterminações, que separam o ver do
que se vê. Pensamento da manifestação não pode se distinguir da
manifestação do pensamento --- do pensamento acontecendo como pensamento,
do dizer o dizendo, da visão vendo o não ver do ver, da escuta escutando
os seus silêncios.

Cinquenta anos depois de uma constelação de visões, escutas, falas,
escritas, políticas, estéticas e teorias da `manifestação da
manifestação', da performance, happenings --- ou falando com nossos
próprios botões de Rosa --- acontecências, testemunhamos um quebra das
teorias. Vivemos hoje uma \emph{teoriclastia}. Uma teoriclastia que como
toda clastia, como toda quebra, se dá por confrontação de paradigmas
cuja tendência é neutralizar o elã crítico do pensamento. Assistimos
hoje, por um lado, a antropologização e etnocentralização de toda teoria
--- antropologia visual, musical, filosófica, etc\ldots{} --- e por outro, a
transhumanização e o transhumanismo das teorias: de um lado, a busca do
antes do homem --- o arcaico enraizado --- e de outro, a busca do depois do
homem --- o epígono desarraigado. De um lado, o paradigma da
de"-subjetivação e por outro o das novas subjetividades e subjetivismos.
De um lado, o paradigma da des"-essencialização e por outro uma
re"-essencialização de todos os gêneros, tipos e espécies. De um lado, um
discurso contra a identidade e, por outro, políticas de identidade. Essa
luta teórica espelha a dinâmica neoliberal do capitalismo, a dinâmica da
monetarização do real, que é a dinâmica de des"-ontologização e
re"-ontologização sem fim --- tudo tem de deixar de ser o que é para poder
vir a ser qualquer coisa com vistas a ser usado, abusado, explorado em
qualquer lugar, a qualquer preço, em qualquer momento. Os especialistas
chamam isso de flexibilidade do capital e se apropriam do vocabulário da
liberdade para descrever essa dinâmica de extorsão e apropriação de cada
um. Por um lado, tudo tem hora e vez; por outro lado, nada tem hora e
vez. Tudo vira nada e o nada vira tudo.

Se durante séculos e milênios, os modos de ver foram guiados pela
oposição entre afirmação e negação, prós e contras, de um lado e de
outro, é preciso agora aprender com os cegos distinções muito mais sutis
entre o mesmo e o mesmo, entre coisa e coisa. Cinquenta anos depois,
fica claro como é preciso desenvolver uma nova arte das distinções,
capaz de distinguir o ``corrido, contínuo do incessar'' (final do
``Pirlimpsiquice'' de G. Rosa) dos discursos de flexibilidade e não
simplesmente dos discursos da permanência. Distinguir opostos é fácil.
Difícil é separar alhos de bugalhos, é distinguir isso disso. Para
tanto, é preciso desenvolver olhos e ouvidos para o cada um de cada um e
seus infinitos modos de doação finita.

No momento de hipertensão que vivemos, a manifestação da energia
minguante de 1967 chama a hiper"-atenção ao sendo, à necessidade de não ser
engolido, contaminado e devastado pela lógica do dinheiro que é bem mais
rebuscada do que a lógica do cálculo. A lógica do dinheiro é a lógica da
ambiguização de todo sentido, que torna bom em ruim, feio em bonito,
ruim em bom, bonito em feio, crédito em débito, débito em crédito e por
ai afora. É a lógica que neutraliza toda diferença, pois um vira outro e
outro vira um. É a lógica que neutraliza toda diferença mediante uma
incessante diferenciação, uma diferenciação que indeferencia. Dizer
``você é um outro'' é, ao mesmo tempo, reconhecer a diferença
``\emph{você} é um outro'' e excluir ``você é um \emph{outro}''. Dentro
dessa lógica, todo dizer se volta contra si mesmo. E a diferença entre
uma política da diferença e a do apartheid tende a se anular.

A dinâmica de ambiguização de sentido que vem nos consumindo é o
exercício de uma tremenda censura, a censura do sentido, \emph{sensura}
com s, (esse termo foi usado pela primeira vez pelo escrito francês
Bernard Noël),\footnote{Bernard Noël. \emph{Le château de cène}. Paris:
  Gallimard, 1990, p.~158.} operada pelo próprio sentido. É a
\emph{sensura} da busca de sentido operada quando a própria linguagem
esvazia a linguagem, quando a imagem satura a imagem, quando a crítica
torna a crítica acrítica, quando o próprio pensamento impede de pensar
ao reproduzir sem fim o já pensado, quando a informação substitui o
saber. É a sensura gerada pelo conformismo midiático, que é um meio
poderoso de censura internalizada e interiorizada. Ou em outras
palavras, censura digerida, que nem mais necessita da pedagogia de
`estudos dirigidos'.

Nessa hipertensão provocada pela ambiguização do sentido, a energia
minguante de 1967 exige uma hiper"-atenção à dinâmica do sentido. Pede
``outras palavras, outras palavras''. Pede escuta, muita escuta e não a
mimesis da mimesis, a cópia da cópia, a recolonização operada pelas vias
da decolonização.

Em tudo isso, o mundo se depara com o desafio de existir sem
prender"-se ao protofascismo das grandes formas e figurações (na era dos
populismo pálidos, dos fascismos imitativos e dos fanatismos
anacrônicos, o pensamento de Lacoue"-Labarthe mostra"-se cada vez mais
atual). Se arte é hoje uma necessidade política, ela o é menos como meio
de expressão política e mais por ser um aprendizado do vir à forma, da
existência sem forma, desafiada a encontrar uma forma para o sem forma
para não resvalar no perigo de erigir o fascismo de formas fechadas e
concluídas. Em questão está uma política da sutileza --- que não se furta
porém à necessidade de ser enfática e contundente. Em jogo está uma
política dos esboços, dos rabiscos e rascunhos, da resistência de traços
efêmeros e pontilhados de respirações, essas que andam na corda bamba do
sendo, na atenção redobrada ao mais perto e apertado, ao mais estreito
por onde passam os fios de nossas angústias e de nossas alegrias
inesperadas. Agora no mundo talvez seja a hora e a vez do sendo.

\chapter*{1967: meio século depois (23/11/17)} 
\addcontentsline{toc}{chapter}{1967: meio século depois (23/11/17), \emph{por Marília Rothier Cardoso}} 

\begin{flushright}
\emph{Marília Rothier Cardoso} %\footnote{Departamento de Letras, \versal{PUC-RJ}.}
\end{flushright}

\section{DE MANIFESTOS E DE MARGENS}

O ano de 1967 inseriu"-se na história acumulando a energia resgatada de
acontecimentos decisivos de datas anteriores. Deixou memória de
resistência e inventividade. Experimentou soluções artísticas radicais
em repúdio ao autoritarismo; ousou exibir corpos e mentes livres para
desafiar a moral conservadora. Seu legado nos interpela a toda hora.
Hoje, precisamos urgentemente que sua crueldade alegre sobreviva. Em
1967, os artistas resistiam às arbitrariedades do estado ditatorial
lançando propostas instigantes nas imagens e palavras das obras que
apresentavam. O público, perplexo ou atento, participava dessa atividade
afirmativa intensa. Nunca é demais revisitar a memória desses
espetáculos.

Depois de um incêndio em seu teatro, o grupo Oficina, de São Paulo,
decidiu reinaugurar a série de montagens experimentais, características
de sua trajetória, com a encenação de \emph{O rei da vela}. José Celso
Martinez Corrêa, o diretor revolucionário ainda em atividade, explica
sua escolha:

\begin{quote}
O problema era o do `aqui e agora'. E o `aqui e agora'
foi encontrado em 1933 n'\emph{O rei da vela} de Oswald de Andrade. (\ldots{}) E \emph{O
rei da vela} (viva o mau gosto da imagem!) iluminou um escuro enorme do
que chamamos realidade brasileira, numa síntese quase inimaginável.
(\ldots{}) \emph{O rei da vela} acabou virando manifesto para comunicarmos no
Oficina, através do teatro e do antiteatro, a `chacriníssima' realidade
nacional''.\footnote{José Celso Martinez Corrêa. In: \emph{O rei da vela}. São Paulo: Difusão Europeia do Livro, 1967, p.~45-46.}
\end{quote}

A descoberta da
rigorosa contemporaneidade da escrita de Oswald, pronta a ser
reapropriada como gesto inaugural, reporta"-se, de imediato, ao
\emph{Manifesto antropófago}, que propõe, como contribuição da América à
transformação das relações econômico"-políticas do mundo, a prática
selvagem da devoração ritual do inimigo --- aquele cuja força resistente
o tivesse destacado na luta. Em termos estratégicos, naquele final da
década de 1960, a utopia nacional"-modernista já se mostrava vencida. Fosse
pelo conservadorismo das plateias de teatro, fosse pela agressividade
repressora da classe dirigente do país, o teatro de Oswald de Andrade
jazia inerte, esquecido numa edição de 1937. Cabia reviver o rito
ancestral e devorar sua energia crítica. O momento pedia a exibição, em
imagens grosseiras, da violência opressiva que vinha marcando as
relações sociais, desde sempre, no país. José Celso preside a realização
do rito como manifesto político, incorporando à arte dramática, que
praticava, as manifestações marginais à solenidade dos palcos --- o
teatro de revista, o circo, a opereta bufa, o programa de televisão.

Quando se reavalia esse espetáculo"-emblema do Teatro Oficina a partir de
texto abandonado ao esquecimento, deve"-se assinalar que, efetivamente,
foi nos anos 1960 que a escrita de Oswald de Andrade efetivamente começou
a circular, liberando, com seu rigoroso mau gosto anarquista, as forças
criativas de seus sucessores. Foram os autores da poesia concreta que,
num gesto potente de flexibilização da racionalidade iluminista,
incluíram Oswald, ao lado de Mallarmé, em seu \emph{paideuma} e
produziram ensaios competentes obrigando a reedição das obras
experimentais, esgotadas há décadas. Dessa perspectiva, Oswald de
Andrade inclui"-se entre os artistas"-pensadores de 1967. Sua proposta
filosófico"-política da antropofagia foi divulgada, explícita ou
indiretamente, de modo voluntário ou inconsciente pelo ativismo
artístico dos vários grupos da época, cada um com sua singularidade ---
concretismo, neoconcretismo, Cinema Novo, tropicalismo, poesia marginal.
Ressalte"-se que, nos idos de 1967, quando a tarefa do pensamento,
decididamente, dissemina"-se pelo corpo, a prática antropofágica
evidencia sua sobrevivência. Mais do que isso, torna"-se programática; e
as almas"-corpos, escolhidas para o banquete, habitam as margens --- da
ordem capitalista, das metrópoles, do poder estabelecido à força, da
própria cultura ocidental. Na encenação de \emph{O rei da vela}, assiste"-se
o espetáculo da baixa antropofagia: a linhagem dos Abelardos, pequenos
empresários da vela e da usura, devora a pobreza de seus clientes e
aceita a posição cômoda de deixar"-se devorar pelos grandes usurários
internacionais. Na arte de Oswald e de José Celso, essa caricatura da
devoração moderna é transvalorada na alta antropofagia, que se nutre da
energia dos pobres --- bichos encerrados na jaula dos inadimplentes,
pelo pequeno empresário/domador --- que a cenografia ritual resgata, sem
idealizar. Se a solenidade da escrita de um Euclides da Cunha já
incorporava o ``sertanejo'' como ``um forte'', o humor oswaldiano,
ingerido pelo Oficina, vai incluir esses viventes das margens,
exercitando uma linguagem que os toca porque se constrói com expressões
apropriadas de seu uso. Na utopia dos anos 1960, a complexidade da
vanguarda inclui a fala do povo. Vindos, ainda, da burguesia, os
artistas"-pensadores mais argutos, a exemplo de Guimarães Rosa, compõem
suas escritas --- verbais, plásticas ou cênicas --- com os fragmentos da
tradição oral recolhida nas periferias, onde sobrevivem, mais evidentes,
os resíduos de culturas arcaicas não"-ocidentais. Hoje, certamente,
ocorre desdobrar os significados negativos da ``vela'' --- o atraso, a
sexualidade machista --- numa espécie de resgate dos objetos do passado
de uso não"-predador.

Em seu prefácio a \emph{O rei da vela} para a reedição que acompanhou a
montagem inaugural, José Celso Martinez Corrêa insiste que ``a peça é
fundamental para a timidez artesanal do teatro brasileiro de
{[}então{]}, tão distante do arrojo estético do cinema novo''.\footnote{José Celso Martinez Corrêa. \emph{O rei da vela}, op.~cit., p.~49.} De fato, quando o Oficina passou a
revolucionar os palcos do país, a plateias do cinema já tentavam
absorver o impacto de filmes experimentais. \emph{Terra em transe}, o
experimento que Glauber Rocha exibe em 1967, já é o terceiro de sua
filmografia, sem contar curtas"-metragens. Como cineasta, crítico e
historiador do cinema brasileiro, Glauber tem grande afinidade com
Oswald de Andrade, nas posturas desconcertantes e incômodas, na
militância quase anarquista que se confronta com o marxismo sistemático,
na tendência a uma estética de gestos brutos, avessa ao bom gosto
convencional. A exemplo de Oswald, Glauber também lançou dois manifestos
de impacto. O mais conhecido e notável pelo modo incisivo com que
defende uma estratégia para o cinema político do terceiro mundo é
``Eztetyka da fome'', apresentado durante a \emph{\versal{V} Rassegna del
Cinema Latino"-Americano}, em Gênova, janeiro de 1965, sob o patrocínio
do \emph{Columbianum}.

Assim como o prefácio"-manifesto de José Celso, o texto escrito por
Glauber está marcado por forte agressividade, ora tática, ora raivosa.
Envolvidos com artes dispendiosas, que dependem do retorno das plateias
e enfrentam o convencionalismo burguês tanto quanto as restrições de
eventuais agências financiadoras, os diretores de cena apontam os
entraves à livre experimentação no campo de sua atividade: em termos
locais, têm de enfrentar a incompreensão do público, sempre na
expectativa do entretenimento; nos termos das relações internacionais,
esbarram com a persistência do colonialismo que explora e paternaliza as
regiões subdesenvolvidas. Ambicioso, como artista crítico, e radical,
como militante, Glauber atribui, ao que considera o inescapável
``condicionamento econômico e político'' da América Latina, o
``raquitismo filosófico'' e a ``impotência''.\footnote{Glauber Rocha. \emph{Revolução do Cinema Novo}. São Paulo: Cosac Naify, 2004, p.~64.}
Dirigindo"-se a interlocutores europeus, procura apresentar o trabalho do
Cinema Novo à distância dos produtos acolhidos na qualidade de
primitivos ou exóticos. Pretende exibir seu filme, \emph{Deus e o diabo na
terra do sol}, e outros com temáticas e tratamentos afins, como
testemunhos das carências do povo, aí presente. Participando da mostra
italiana, enuncia ``a trágica originalidade do Cinema Novo diante do
cinema mundial''. Afirma que ``nossa originalidade é nossa fome e nossa
maior miséria é que esta fome, sendo sentida, não é compreendida''.\footnote{Ibidem, p.~65.}
Com uma contundência comparável à do \emph{Manifesto antropófago},
desqualifica os filmes brasileiros, cujo cenário é ocupado por
``paisagens tropicais'' que tentam ``disfarçar'' tanto a pobreza local
quanto a ``indigência mental dos cineastas''.\footnote{Ibidem.} Em tom exaltado,
descreve a produção do Cinema Novo, alheia ao modelo iluminista da
estética autônoma: ``estes filmes feios e tristes, estes filmes gritados
e desesperados onde nem sempre a razão falou mais alto'' são a evidência
de ``que a fome não será curada pelos planejamentos de gabinete e que os
remendos do tecnicolor não escondem mas agravam seus tumores''. A
economia restrita e tensa de tais montagens aparentemente improvisadas
justifica"-se porque ``a mais nobre manifestação cultural da fome é a
violência''.\footnote{Ibidem.}

Mas o tom esquemático desse discurso de denúncia política resume apenas
um lado do pensamento complexo e auto"-avaliador de Glauber Rocha. Em
1971, num seminário equivalente, desta vez, na Universidade de Colúmbia,
o diretor lança sua contrapartida --- ``Eztetyka do sonho'' --- também ``a
respeito de arte e revolução'', a que o ``tema da pobreza'' está
``ligado''.\footnote{Ibidem, p.~248.} A veemência do tom não impede a avaliação astuta
das circunstâncias --- a ``arte revolucionária'' acaba, frequentemente
``rejeitada pela esquerda e instrumentalizada pela direita''.\footnote{Ibidem, p.~249.}
Certo de que ``os sistemas culturais atuantes, de direita e de esquerda,
estão presos a uma razão conservadora'', aponta o ``sonho'' como a
dimensão revolucionária que se pode alcançar apenas pela arte. É
incisivo e desafiador quando propõe: ``A revolução é a \emph{anti"-razão}
que comunica as tensões e rebeliões do mais \emph{irracional} de todos
os fenômenos que é a \emph{pobreza}''.\footnote{Ibidem, p.~250.} Entre o lançamento dos
dois manifestos, Glauber filmou \emph{Terra em transe}, divulgado a partir
de 1967, trabalho que deve ter"-lhe proporcionado a oportunidade de levar
mais longe sua crítica aos fundamentos da ordem ocidental, seja no plano
da economia capitalista, seja a partir das referências epistemológicas
de base judaico"-cristã. Na construção desse filme é que examina,
diretamente, os possíveis sucessos e fracassos da atividade
revolucionária na América Latina. Da denúncia da ``fome'', cujas
consequências são observadas através do cangaço --- enfrentado
problematicamente por Antônio das Mortes, o sertanejo justiceiro a
serviço dos padres e coronéis ---, passa à avaliação
(histórico"-imaginária) de uma experiência de política transformadora,
posta em questão pelo intelectual urbano, Paulo Martins. Numa cena
impactante, onde um trabalhador pobre apresenta suas reivindicações, o
intelectual de esquerda reprime sua fala e o agride. Aí se expõe a
fragilidade dos projetos de governo que idealizam o povo mas temem sua
força reprimida e agem de modo a detê"-la. A partir desse episódio, o
filme acompanha toda a ambiguidade dos movimentos revolucionários e sua
tendência a reorganizar"-se no mesmo modelo de Estado repressor. Com suas
imagens belas e desnorteantes, \emph{Terra em transe} torna"-se a exploração
cruel de um território de esperança, fracasso e dúvidas. O hibridismo
barroco de seu estilo de narração labiríntica é acompanhado, em várias
sequências, por ``música afro"-brasileira''\footnote{Glauber Rocha. \emph{Roteiros do terceyro mundo}. Rio de Janeiro: Alhambra/Embrafilme, 1985, p.~287.} e
evidencia, em cenas"-chave, a contaminação pelos ritos católicos
africanizados. Na perplexidade desse experimento estético"-político,
insinua"-se a brecha por onde a voz do faminto possa fazer"-se ouvir, em
seu próprio tom, com a peculiaridade de seus resíduos selvagens
finalmente percebidos. A ``sensibilidade afro"-índia''\footnote{Glauber Rocha. \emph{Revolução do Cinema Novo}, op.~cit., p.~251.} seria o veículo de compreensão da fome e possível interferência em
sua longa história.

No teatro e no cinema, as propostas de responder, com encenações
revolucionárias, aos desequilíbrios históricos da sociedade brasileira e
à crise política, instaurada em 1964, levaram a soluções instigantes,
desencadeando avanços irreversíveis na prática de seus respectivos
espaços --- palco e tela. Já o campo das artes plásticas, onde se
debatiam políticas internas à vanguarda estética que, pouco a pouco, se
ampliavam para questões éticas e socioeconômicas mais amplas, passou a
operar com maior radicalidade --- explodiu os limites da pintura e
escultura. Nesse sentido, a trajetória de Hélio Oiticica é exemplar. Seu
próprio testemunho da atividade desenvolvida parte da ``desintegração do
quadro'' e avança, mostrando que esta equivale à ``desintegração da
pintura'' e conduziu à ``abolição da estrutura significante'' até
atingir o que passou a considerar ``a invenção pura''.\footnote{Hélio Oiticica. In: \emph{A invenção de Hélio Oiticica}. São Paulo: Ed\versal{USP/FAPESP}, 2000, p.~47.}
A crescente rebeldia, que tomou
conta da década de 1960, flertou, insistentemente com a periferia, seja
dos cânones, seja da ordem estabelecida. Assim, para pôr"-se à margem
tanto da história quanto do mercado da arte, o inventor crítico faz"-se
intermediário no resgate dos saberes e valores daqueles que não
encontraram lugar na organização social.

Esta atividade mediadora foi capital na transformação dos conceitos,
materiais e práticas de seu trabalho. Os avanços, na carreira de Hélio
Oiticica, revelam"-se instigantes pois se apresentam paradoxais: quanto
mais rigorosa se tornava sua verve classificatória e nomeadora, mais
violentos foram seus gestos de romper expectativas e desafiar o senso
comum. Frequentador simultâneo de ateliês de arte, da seção de
entomologia do Museu Nacional e do Morro da Mangueira, o artista
combinava, com sua astúcia criativa, as três ordens de aprendizagem. No
desenvolvimento de uma etapa decisiva para a plena instauração de seu
``Programa ambiental'', construiu ``bólides'' e ``parangolés''. Ambos
são penetráveis, através dos quais pesquisou o efeito da cor e da
textura, atraindo o espectador para a função de participante da arte,
que passaria a usufruir com todos os sentidos. Os ``bólides'', feitos de
madeira e/ou vidro, eram manipuláveis e exigiam movimentos do corpo para
que se percebessem, por frestas, suas divisões internas. Dentre todos,
destaca"-se o \emph{Bólide Caixa 18, Homenagem a Cara de Cavalo} (1966),
evidência conhecida do empenho de Hélio em ``homenagear a revolta
individual social'',\footnote{Ibidem, p.~131.} em
tom de indignação contra o assassinato, pela polícia, do jovem bandido,
um de seus amigos do morro. E foi lá na Mangueira, integrado à escola de
samba, que desenvolveu sua forma mais atraente de ``antiarte
ambiental''--- os ``parangolés''. Mais forte do que qualquer homenagem,
o conjunto de diversos tecidos coloridos, cujo efeito só se produz
quando portado por corpos que dançam, trouxe, em várias oportunidades
dos anos 1960, o mais novo conceito de arte (ou antiarte) para confronto
com o museu e o mercado. O heroísmo político --- atribuído por Oiticica
--- àqueles que se arriscam na contravenção, como grito de protesto, e a
performance rítmica, de que só são capazes os passistas do samba, foram
conceitos preparatórios de ``Tropicália'' --- instalação de 1967 ---,
onde se concretizou seu programa ambiental. Trata"-se, nas palavras do
próprio inventor, da ``primeiríssima tentativa consciente, objetiva, de
impor uma imagem obviamente `brasileira' ao contexto atual da vanguarda
e das manifestações em geral da arte''.\footnote{Ibidem, p.~137.}
Resultante da combinação de dois \emph{penetráveis} ---
\versal{PN}2 \emph{Pureza é um mito} (1966)  e \versal{PN}3 \emph{Imagético} (1966, 1967) ---
a instalação confronta a cena tropical popular à tecnologia e convida o
observador a percorrer seu labirinto, pisando areia e brita, entre
plantas e araras, capas de ``parangolé'' e poemas"-objeto, para
encontrar, no fim, uma televisão ligada. Por proporcionar,
simultaneamente, essa variedade de sensações e afectos desencontrados, %Afectos mesmo?
``Tropicália'' tornou"-se um emblema. Sua apresentação, inesperada,
heteróclita e desconcertante do contemporâneo nacional, disseminou a
ótica tropicalista pelas manifestações político"-culturais das décadas de
1960 e 1970 e repercute em nosso imaginário até hoje.

Como a ``Tropicália'' emblemática, o ano de 1967 sobrevive --- nosso
contemporâneo --- nesse 2017 de crise e perplexidade. A ação mediadora de
seus artistas mais argutos e anárquicos se apresenta, agora, como passo
necessário ao protagonismo da presença marginal, começado há pouco e
ainda muito incipiente. Mas, por certo, não fossem os pobres
inadimplentes enjaulados, no cenário do Teatro Oficina, os famintos
reivindicadores rechaçados, em \emph{Terra em transe}, o cadáver do Cara de
Cavalo e os passistas alegres dançando os ``parangolés'', nos jardins do
\versal{MAM}, talvez não tivéssemos os saraus de poesia, os \emph{Cadernos
negros}, Akins Kinté, os contos de Allan da Rosa e a \versal{FLUPP}. Por isso,
como outra potência inventiva dos anos 1960, vale lembrar a atividade de
Arthur Bispo do Rosário, entre tábuas, vidros, objetos de metal e
plástico, desfiando seu uniforme de interno e bordando mantos na cela
forte da Colônia Juliano Moreira.

Começada nos anos 1940, 1950, a coleção de objetos produzidos por Bispo
mostra"-se, paradoxalmente, arcaica, intempestiva e paralela aos
movimentos antiarte do final dos anos 1960. Mas o que destaca e
singulariza essa trajetória é que seu desenvolvimento autodidata
desconhece qualquer referência estética, tendo sido desenvolvida com
plena dedicação por um apelo místico. Conforme os registros que se
conseguiu recolher, em 1938, numa primeira visão, o antigo marinheiro
teria sido convocado pela divindade para dar conta de todos os
``materiais existentes na Terra para o uso do homem''.\footnote{Bispo do Rosário. In: \emph{Bispo do Rosário e os 90 anos da Colônia Juliano Moreira}. Rio de Janeiro: Azougue, 2016, p.~188.} Lacônico e enigmático, o
escolhido anuncia ``Eu vim''.\footnote{Paulo Herkenhoff. In: \emph{Arthur Bispo do Rosário}. Rio de Janeiro: Réptil, 2012, p.~141.}
A partir daí, recolhido à instituição psiquiátrica, Bispo
não se descuidou de reunir objetos, construir miniaturas, organizá"-los
em painéis e bordar, em estandartes e outros suportes, as cifras
correspondentes a essas amostras dos seres e coisas, bem como dos nomes
daqueles escolhidos que, ele --- o enviado --- deveria conduzir aos
céus. Preparando"-se para este apocalipse, passou a cercar seu corpo e
seu cotidiano de interno com a pompa possível, representada em fardões,
mantos e no dossel que, enfeitado, cobria sua cama"-nave. Nesse trabalho
artesanal, tão ousado quanto minucioso, condensam"-se as reminiscências
arcaicas de sua vila sergipana e as imagens de peças eruditas e da
cultura de massa, observadas nas revistas que, eventualmente, lhe caíam
nas mãos. Contemporâneo das oficinas de arteterapia --- entre as quais o
conhecido ateliê da Dra. Nise da Silveira ---, sempre se esquivou da
frequência a esses espaços. Sua parca biografia registra os anos 1960 como
o marco da dedicação completa ao trabalho para que fora assinalado.
Desde 1964, nunca mais saiu da Colônia Juliano Moreira e, em períodos de
isolamento e jejum, multiplicava sua obra. Quando deixou que expusessem
alguns de seus objetos, no \versal{MAM}, integrando a exposição coletiva ``À
margem da vida'', recusou"-se a ir visitá"-la. Na ocasião, teria dito:
``Meus olhos não estão preparados para ver aquilo''.\footnote{Luciana Hidalgo. \emph{Arthur Bispo do Rosário, o senhor do labirinto}. Rio de Janeiro: Rocco, 2011, p.~137.}
Tendo sido construída à margem dos cânones estéticos, dos
valores do mercado e do ativismo de resistência aos mesmos, a arte de
Bispo do Rosário, ainda que, inevitavelmente domesticada pelas
instituições, apresenta"-se incontornável, com sua força de invenção,
mais incisiva que qualquer manifesto.

\pagebreak

\section{Referências}

\begin{Parskip}
\versal{ANDRADE}, Oswald de. \emph{O rei da vela}. Prefácio de Sábato Magaldi e
artigo crítico de José Celso Martinez Corrêa. São Paulo: Difusão Europeia do Livro, 1967.

\versal{ARAÚJO}, Emanoel et al. \emph{Arthur Bispo do Rosário}. Wilson Lázaro (org. e curadoria). Rio de Janeiro: Réptil, 2012.

\versal{CAMPOS}, Marcelo (org.). \emph{Bispo do Rosário e os 90 anos da Colônia Juliano Moreira}. Rio de Janeiro: Azougue, 2016.

\versal{FAVARETTO}, Celso. \emph{A invenção de Hélio Oiticica.} São Paulo: Ed\versal{USP/FAPESP}, 2000.

\versal{HIDALGO}, Luciana. \emph{Arthur Bispo do Rosário, o senhor do labirinto}.
Ed. ampliada. Rio de Janeiro: Rocco, 2011.

\versal{ROCHA}, Glauber. \emph{Revolução do Cinema Novo}. São Paulo: Cosac Naify, 2004.

\_\_\_\_\_\_. \emph{Roteiros do terceyro mundo}. Orlando Senna (org.). Rio de Janeiro:
Alhambra/Embrafilme, 1985.
\end{Parskip}


\chapter*{Uma arte subdesenvolvida}
\addcontentsline{toc}{chapter}{Uma arte subdesenvolvida, \emph{por Moacir dos Anjos}}

\begin{flushright}
\emph{Moacir dos Anjos}
\end{flushright}

\section{Subdesenvolvimento como condição}

Em 1961, os compositores Carlos Lyra e Chico de Assis musicaram a peça
teatral \emph{Um americano em Brasília}, de Chico de Assis e Nelson Lins
e Barros, tendo como destaque a canção ``O subdesenvolvido''. Lançada em
disco no ano seguinte como parte da coletânea \emph{O povo canta} --- uma
produção do Centro Popular de Cultura (\versal{CPC}), instituição ligada à União
Nacional dos Estudantes (\versal{UNE}) ---, a música ironizava a continuada
dependência econômica, política e cultural do Brasil em relação a outros
países. De letra extensa, ``O subdesenvolvido'' dissertava, em cada uma
de suas muitas estrofes, sobre um período da história do país --- de
colônia portuguesa aos então dias recentes, quando se fazia mais
presente a influência dos Estados Unidos na vida dos brasileiros ---,
apontando as renovadas formas de exploração de seu povo em benefício do
capital estrangeiro. Ao final de cada um desses blocos, repetia"-se o
refrão da música, acusatório, insistente e provocador: ``era um país
subdesenvolvido, subdesenvolvido, subdesenvolvido''. O grande sucesso de
``O subdesenvolvido'', principalmente entre estudantes secundaristas e
universitários, foi seguido, logo após o golpe militar de 1964, da
proibição de ser executada publicamente e da retirada de circulação do
disco em que fora gravada.

Não é à toa que o conceito de subdesenvolvimento tenha informado uma das
mais populares canções de protesto em um momento de grande efervescência
política e cultural no Brasil como foi o início da década de 1960.
Embora de origem mais remota no pensamento econômico, a ideia de
subdesenvolvimento ganha conteúdo novo e mais força no pós"-Segunda
Guerra Mundial, como resultado dos esforços institucionais e
intelectuais feitos para pensar a reconstrução do mundo em um ambiente
inteiramente mudado, no qual se confundiam devastações e oportunidades.
Ambiente onde não cabia mais muito daquilo que era antes tomado como
dado imutável, tal como as desigualdades nos níveis de vida observadas
entre partes diversas do planeta. E foi no âmbito da Comissão Econômica
para a América Latina e Caribe --- \textsc{cepal}, órgão criado pelas Nações
Unidas em 1948 e sediado no Chile, que o termo subdesenvolvimento
adquiriu não somente maior densidade, mas também significados renovados,
passando a pautar parte relevante do pensamento econômico e social
daquelas regiões. Mudança que é em boa parte devida às elaborações
teóricas, sempre coladas à observação do que de fato ocorria ali, dos
economistas Raúl Prebisch, argentino, e Celso Furtado, brasileiro.

Em termos amplos, a maior novidade da concepção de subdesenvolvimento
dos integrantes da \textsc{cepal} (os chamados cepalinos), talvez tenha sido o
abandono de uma visão teleológica da história que era corrente à época,
segundo a qual o subdesenvolvimento (com todas as carências materiais e
humanas que o definiam) seria um estágio a ser inevitavelmente superado
pela aceleração continuada do crescimento econômico. Abandonava"-se,
portanto, a ideia --- de algum modo reconfortante --- de que o
desenvolvimento seria o destino natural dos países subdesenvolvidos, uma
vez que seus produtos e rendas crescessem sustentadamente ao longo dos
anos, ainda que não houvesse previsão certa para essa mudança.
Abdicava"-se da crença de que, embora tivessem se atrelado à dinâmica
capitalista um pouco atrasados, seria questão de tempo para que os
países então subdesenvolvidos se ombreassem aos já desenvolvidos em
satisfação de necessidades antigas e também das novas, constantemente
recriadas por um sistema econômico que promovia e dependia de um consumo
massivo.

Em vez de se conformarem com a concepção vigente de que existiriam
etapas a serem alcançadas e fatalmente vencidas em momento incerto, os
cepalinos fizeram, naquele momento, um empenho concertado para entender
o subdesenvolvimento dos países latino"-americanos e caribenhos não como
um estágio, mas como uma \emph{condição}. Condição cuja origem remontava
ao início da colonização daquela região por países europeus, no século
\textsc{xvi}, mas que ganhara contornos mais precisos a partir de finais do século
\textsc{xviii}, com a acelerada industrialização da Europa ocidental e consequente
expansão das economias capitalistas para além de suas fronteiras, em
busca de efetivar suas aumentadas possibilidades de ganho material. Tal
deslocamento de capitais alcançou, com toda a violência física e
simbólica que o caracteriza, regiões já ocupadas com sistemas econômicos
seculares de natureza pré"-capitalista, criando ali estruturas
\emph{dualistas} nas quais se reproduziam, de modo articulado, setores
que obedeciam a critérios econômicos distintos.\footnote{Celso Furtado.
  \emph{Teoria e Política do Desenvolvimento Econômico}. São Paulo:
  Editora Nacional, 1968, p.~154.} Aos países
colonizados/subdesenvolvidos cabia, no mais das vezes, a oferta de
matéria"-prima barata extraída por mão de obra majoritariamente
escravizada, enquanto aos países colonizadores/desenvolvidos restava o
processamento e comercialização --- interna e externa --- dos bens
produzidos.

Essa dualidade interna replicava, no interior dos países
subdesenvolvidos, a existência, no plano mundial, de um centro e de uma
periferia articulados por uma situação de subordinação. Centro e
periferia entendidos não somente como expressão das diferentes posições
geopolíticas ocupadas pelos países a dado momento, mas como partes
orgânicas de um sistema que as reproduzia conjuntamente em ritmos
distintos, repondo e ampliando, continuamente, as desigualdades
econômicas e sociais que constituem e definem esse arranjo. Restava
implícito, nessa tipificação, o entendimento de que o funcionamento da
economia mundial engendrava e mantinha relações de \emph{dependência}
entre países, aspecto que seria, nas décadas de 1960 e 1970, explorado e
atualizado por diversos economistas e sociólogos
latino"-americanos.\footnote{Entre os trabalhos mais influentes sobre
  essa dependência sistêmica, encontra"-se: Fernando Henrique Cardoso e
  Enzo Faletto. \emph{Dependência e desenvolvimento na América Latina:
  ensaio de interpretação sociológica.} Rio de Janeiro: Zahar, 1970.}

Os cepalinos compreendiam o subdesenvolvimento, portanto, como um
processo histórico \emph{autônomo} que orientava, desde aquele passado
remoto e ainda persistindo no século \textsc{xx}, a dinâmica econômica e social
de países localizados em partes variadas do mundo. No Brasil, como em
outros lugares, esse arranjo ganhou complexidade ao longo do tempo, com
a gradual capacitação de indústrias locais para atender parte da demanda
por bens de consumo e, em seguida, por equipamentos de reposição
necessários à continuada valoração da riqueza, antes trazidos já prontos
dos chamados países centrais --- mecanismo depois conceituado como de
``substituição de importações''. Mesmo nessa configuração mais complexa,
porém, em que a economia subdesenvolvida é diversificada o bastante para
produzir, internamente, parte dos bens de consumo e de capital
necessários à geração de riqueza, havia limites ao rompimento daquela
relação de dependência. Limites estabelecidos pelo fato de que a
incorporação dos ganhos ali gerados permaneciam desigualmente
distribuídos, cabendo a maior parte deles às empresas estrangeiras que
comandavam as principais cadeias produtivas daquelas economias, sendo
transferidos, direta ou indiretamente, aos seus países de origem.

É preciso atentar, contudo, que esse diagnóstico --- aqui apenas esboçado
--- não implicava a imutabilidade da condição do subdesenvolvido.
Justamente por chamar a atenção para a historicidade da constituição das
relações de dependência entre países --- ou seja, ao promover um esforço
para se compreender como se forjou, ao longo do tempo, a condição do
subdesenvolvido ---, tal concepção enfatizava também a possibilidade de
intervenção nessa realidade para mudá"-la. Mudança que não aconteceria
apenas como resultado de um maior crescimento nominal dessas economias,
como era antes pensado, sendo necessário, para além disso, transformar
radicalmente seu perfil produtivo e social, criando"-se uma autonomia
relativa dos países periféricos em relação aos centrais em observância a
interesses nacionais específicos. Em função desse entendimento, a \textsc{cepal}
desenvolveu um sofisticado instrumental de planejamento e de intervenção
em economias subdesenvolvidas para superar sua condição de subordinação,
processo no qual o Estado, obviamente, desempenharia um papel de
fundamental importância.

Muitas das formulações teóricas e históricas associadas a essa nova
concepção de subdesenvolvimento e sua superação foram elaboradas entre o
início da década de 1950 e meados da década seguinte. No Brasil, são
várias as decisões políticas e econômicas que, tomadas ao longo desses
anos, exibem, de modo menos ou mais evidente, indícios dessa nova
maneira de pensar o lugar da chamada periferia do mundo no contexto
internacional. Entre estas, incluem"-se tanto a construção de Brasília
como os ambiciosos projetos de industrialização que marcaram o período,
os quais tiveram, entre seus efeitos mais imediatos, o crescimento
urbano e a consolidação de uma nova estrutura de classes no país, bem
como o estímulo a migrações internas. É também nesse ambiente de
transformações que ressurge, com força, uma questão regional no Brasil,
reconhecimento da existência de um centro e de uma periferia
perversamente articulados dentro do próprio país.

Mesmo as estratégias políticas hegemônicas no campo das esquerdas no
início da década de 1960 incorporavam, no Brasil, o diagnóstico ali
implícito, na medida em que focavam mais em uma aliança com a burguesia
nacional para fazer frente ao poder das economias centrais (ou ao
\emph{imperialismo}, para usar o termo então em voga) do que na
agudização dos patentes conflitos de classe internos do país. A
centralidade da questão fundiária no período --- em particular, a
necessidade da reforma agrária --- talvez seja exemplar dessa situação
complexa e algo ambígua, já que era considerada, em textos e em ações,
simultânea ou alternadamente, como necessidade de modernização
capitalista e/ou como reparação de desigualdades ou injustiças.

As mudanças no modo de diagnosticar o subdesenvolvimento e na maneira de
posicionar"-se frente à essa condição com o intuito de subvertê"-la
repercutiram em várias concepções e ações culturais e artísticas no
Brasil da década de 1960. Assim como na economia e na política, também
nesses campos as ideias adquiridas do centro começaram a ser mais
fortemente questionadas por sua inadequação à realidade nacional,
passando a ser entendidas no contexto de uma dependência sistêmica que
subordinava, material e simbolicamente, alguns muitos países a outros
poucos.

\section{Uma vanguarda subdesenvolvida}

Um dos mais conhecidos desses esforços de reposicionamento crítico foi o
empreendido pelo poeta e ensaísta Ferreira Gullar ao discutir a ideia de
\emph{vanguarda} artística no contexto do subdesenvolvimento. Escrevendo
em 1969 --- embora retomando e revendo reflexões e posicionamentos de
quase uma década ---, o escritor enfatiza a condição de dependência que
define o Brasil e outros tantos países considerados subdesenvolvidos,
bem como a inexistência de equivalência cultural perfeita entre estes e
aqueles tidos como desenvolvidos. Em reação a ambiente assim
sobre"-determinado por uma relação de subordinação sempre reposta, as
concepções de vanguarda artística deveriam, segundo defendia o autor,
corresponder a problemas e necessidades diferentes daqueles observados
nos países desenvolvidos, tendo que forçosamente levar em conta a
\emph{questão nacional}. É nesse contexto que Ferreira Gullar sugere que
``a definição de arte de vanguarda num país subdesenvolvido deverá
surgir do exame das características sociais e culturais próprias a esse
país e jamais da aceitação ou da transferência mecânica de um conceito
de vanguarda válido nos países desenvolvidos''.\footnote{Ferrreira Gullar.
  \emph{Vanguarda e subdesenvolvimento. Ensaios sobre arte}.
  Rio de Janeiro: Civilização Brasileira, 1984, p.~78 (1ª edição de
  1969).}

Valendo"-se dessa chave interpretativa, o ensaísta critica ou aplaude
diversos movimentos culturais e artísticos das décadas de 1950 e 1960 no
Brasil em função de sua menor ou maior capacidade de levar em conta a
integração subordinada do país à dinâmica ditada pelos países
desenvolvidos, inclusive no âmbito da cultura. Sua objeção,
retrospectiva, ao concretismo, e seu elogio ao neoconcretismo --- dois
dos mais destacados movimentos de vanguarda artística do Brasil entre
meados da década de 1950 e início da seguinte --- fazem parte desse
raciocínio mais amplo. Enquanto o concretismo estaria desconectado da
realidade brasileira, emulando questões formais pertencentes e
pertinentes a uma vanguarda internacional, o neoconcretismo estaria
buscando maneiras originais de lidar com uma situação de dependência,
mesmo sem abrir mão de outras questões que também movessem seus
integrantes. É esta, talvez, a contribuição mais genérica e relevante
desse texto de Ferreira Gullar para se pensar os conceitos de vanguarda
e de arte próprios à condição do subdesenvolvido: defender o que chamava
de ``caráter nacional da expressão estética''. Nacional, diz o escritor,
``não por ser `nacionalista' ou regionalista, ou folclórica, ou exótica;
mas por ser a expressão concreta, particular, do universal no âmbito de
uma cultura determinada''.\footnote{Ferreira Gullar. \emph{Vanguarda e subdesenvolvimento. Ensaios sobre arte}, op.~cit., p.~95.}

Expressão do particular que se traduziu, não poucas vezes ao longo da
década de 1960, em uma aproximação da dita vanguarda artística
brasileira de temáticas e procedimentos então associados à \emph{cultura
popular}. Aproximação que ecoava aquela produzida pela vanguarda
política da época, notadamente nos anos que precederam o golpe militar
de 1964, incorporando e transformando, a partir de referências trazidas
de outros cantos, modos de falar e de fazer próprios à população que
mais sofria os efeitos perversos e persistentes do subdesenvolvimento no
Brasil. E embora esse avizinhamento tenha sido obstruído, no âmbito
propriamente político, pela ruptura institucional daquele ano, ela
perdurou no campo artístico, ainda que se refazendo o tempo inteiro, até
quase o final da década, quando os efeitos do Ato Institucional n. 5
(\versal{AI}-5) --- repressão, censura e tortura --- a desmantelaria quase
totalmente em espaços de criação.\footnote{Roberto Schwartz. ``Cultura
  e política, 1964-1969''. In: \emph{As ideias fora do lugar: ensaios
  selecionados.} São Paulo: Penguin Clássicos/Companhia das Letras,
  2014, p.~8 (texto originalmente publicado em 1970).} A despeito das
variadas ênfases e matizes observadas nesse conturbado momento do país,
essa proximidade entre o campo do que é genericamente tido como cultura
popular e a arte então considerada de vanguarda é um traço que atravessa
parte significativa da produção associada ao campo das artes visuais, do
cinema e da música no período. Uma produção na qual o subdesenvolvimento
assume não somente uma conotação de falta, mas também de potência de
transformação que um entendimento particular do mundo pode possuir; que
traz em si o germe de uma mudança política, social e sensível. Uma noção
de arte subdesenvolvida, portanto, que assume o caráter paradoxal e
contraditório do ambiente onde é produzida.

\section{Uma estética da fome}

No ensaio ``Cinema: trajetória no subdesenvolvimento'', escrito no
início da década de 1970, o crítico de cinema Paulo Emílio Sales Gomes
sugere que os principais movimentos do cinema brasileiro podem ser
entendidos como diferentes respostas à \emph{condição} de
subdesenvolvimento a que o Brasil estaria submetida. Condição que é
afirmada logo na abertura do texto da seguinte maneira: ``o cinema
norte"-americano, o japonês e, em geral, o europeu nunca foram
subdesenvolvidos, ao passo que o hindu, o árabe ou o brasileiro nunca
deixaram de ser. Em cinema, o subdesenvolvimento não é uma etapa, um
estágio, mas um estado: os filmes dos países desenvolvidos nunca
passaram por essa situação, enquanto os outros tendem a se instalar
nela''.\footnote{Paulo Emílio Sales Gomes. ``Cinema: trajetória no
  subdesenvolvimento''. In: \emph{Cinema:
  trajetória no subdesenvolvimento}. São Paulo: Paz e Terra, 1996, p.~85 (texto originalmente publicado em 1973).}

Uma característica do subdesenvolvimento brasileiro, contudo, diz o
crítico, é ser fruto de uma formação colonial que provocou, como algo
inerente à sua lógica e dinâmica, o quase extermínio dos povos nativos
que viviam nas terras que conformariam o Brasil. Como resultado não
planejado desse etnocídio, teria havido uma gradual e progressiva
indistinção cultural entre quem é o ``ocupante'' e quem é o ``ocupado''
no país. O ocupado sobrevivente desse aniquilamento colonizador --- em
sua maior parte descendentes de imigrantes voluntários e de escravizados
trazidos à força ao Brasil --- teria sido assim forjado quase à
semelhança do ocupante, embora mantendo"-se subordinado a este. Ancorado
nesse raciocínio, Paulo Emílio Sales Gomes conclui que nada seria de
fato estrangeiro para o brasileiro, pois tudo já o é.\footnote{Ibidem, p. 89-90.} Nesse contexto, o
cinema nacional teria tido que lidar com a ambivalente questão de ser e
de não ser o \emph{outro} estrangeiro: de confundir"-se com o outro ou de
reinventá"-lo desde um ponto de vista próprio. O Cinema Novo --- movimento
que eclode, amadurece e se exaure na década de 1960 --- foi, segundo o
autor, uma das respostas ao estado de subdesenvolvimento do Brasil. Mas
ao contrário de movimentos anteriores, especialmente a \emph{chanchada},
dominante durante as duas décadas precedentes, o Cinema Novo não teria
buscado o apaziguamento dessa ambígua condição, buscando, ao contrário,
confrontá"-la criticamente a partir da visão do ocupado. Confronto tanto
temático, enfocando as extremas desigualdades observados no campo e nas
cidades do país e quase ausentes da produção cinematográfica nacional,
quanto formal, desafiando as convenções de roteiro, atuação, fotografia,
edição e direção do cinema mundialmente hegemônico da época.\footnote{Ibidem, p.~99-104.}

Tais questões estão presentes, de maneira enfática, em alguns dos
primeiros filmes de Glauber Rocha, o cineasta a quem mais diretamente se
associa o caráter transgressor do Cinema Novo; notadamente, ainda que de
modos muito diversos, em \emph{Deus e o diabo na terra do sol} (1964) e
\emph{Terra em transe} (1967). Em ambos, o diretor disse ter buscado se
afastar quer da ``esterilidade'' formal, quer da ``histeria''
humanitária que acometeria tantos artistas brasileiros, seja no cinema
ou em outras áreas.\footnote{Glauber Rocha. ``Eztetyka da fome''. In:
  \emph{Revolução do Cinema Novo}. São Paulo: Cosac
  Naify, 2004, p. 64 (texto originalmente publicado em 1965).} Se a
esterilidade formal era o resultado anódino da replicação de soluções
criativas adequadas para situações em tudo diferentes da vivida pelo
brasileiro, a histeria humanitária implicava a idealização de um povo
que, justamente por ser ``{[}d{]}oente, faminto e analfabeto'', seria
complexo o bastante para se deixar afetar por processos simplistas de
conscientização que o tratavam como incapazes de pensar por si
mesmos.\footnote{Glauber Rocha. ``O Cinema Novo e a aventura da
  criação''. In: \emph{Revolução do Cinema Novo}, op.~cit., p. 132 (texto
  originalmente publicado em 1968).} Concordando com Ferreira Gullar
sobre a necessidade de considerar as particularidades da vanguarda
artística no subdesenvolvimento, Glauber Rocha advogava ser preciso
expor o público a ``um novo tipo de cinema: tecnicamente imperfeito,
dramaticamente dissonante, poeticamente revoltado, sociologicamente
impreciso como a própria sociologia brasileira oficial, politicamente
agressivo e inseguro como as próprias vanguardas políticas brasileiras,
violento e triste, muito mais triste que violento, como muito mais
triste que alegre é o carnaval''.\footnote{Ibidem, p.~133.}

Diante do desafio que colocava para o cinema nacional (e para ele
próprio, portanto) --- fazer filmes que não ignorassem as características
mais marcantes de um país subdesenvolvido e que as tomassem, além disso,
como tema e modelo de criação ---, Glauber Rocha traduziu suas
inquietações em uma série de textos escritos ao longo da década de 1960
e início da seguinte. Em um dos mais conhecidos, intitulado ``Eztetyka
da fome'', escrito em 1965, deu destaque a um elemento então
constitutivo do subdesenvolvimento --- a fome --- concedendo"-lhe, além do
poder de sintetizar o sofrimento de muitos dos que vivem sob uma
condição subordinada no mundo, o papel de operador central na luta para
subverter as desigualdades que produzem e preservam a falta do que
comer.\footnote{Glauber Rocha. ``Eztetyka da fome'', op.~cit., p.~63-67.}

A centralidade da fome em países subdesenvolvidos já havia sido apontada
e estudada, nas décadas de 1940 e 1950, pelo médico e geógrafo Josué de
Castro. Sua extensa pesquisa sobre o assunto, introduzida no livro
\emph{Geografia da fome} (1946), permitiu evidenciar a complexidade das
causas de uma alimentação insuficiente ou inadequada em regiões pobres
do mundo. Para o estudioso, a fome deveria ser entendida não somente
como um fenômeno agudo ou como uma ``fome total'', característica de
áreas de miséria extrema ou sujeitas a contingências extraordinárias;
haver"-se"-ia de levar em conta, igualmente, o fenômeno da ``fome
parcial'', fruto da falta continuada de nutrientes imprescindíveis à
vida, a qual lentamente mata populações que a eles não têm
acesso.\footnote{Josué de Castro. \emph{Geografia da fome}. Rio de
  Janeiro: Civilização Brasileira, 2011, p.~18 (1ª edição de 1946).} Fome que a escritora e favelada Carolina Maria de
Jesus afirmava, em seu livro \emph{Quarto de despejo} (1960), afetar até
mesmo os sentidos da visão, fazendo com que o faminto enxergasse tudo em
volta com uma cor somente: o amarelo.\footnote{Carolina Maria de Jesus.
  \emph{Quarto de despejo. Diário de uma favelada}. São Paulo: Ática,
  2014, p.~44 (1ª edição de 1960).} E para que se
entendessem as causas daquela fome ``oculta'', Josué de Castro dizia ser
necessário considerar não somente fatores geográficos, mas,
principalmente, fatores associados aos sistemas econômicos e sociais
que, de modos vários, privavam populações inteiras da alimentação
necessária à sua sobrevivência.\footnote{Josué de Castro. \emph{Geografia da fome},
op.~cit., p.~34.}

Para além ou por causa de seus efeitos e origens perversos, Glauber
Rocha assume a fome como catalizador de experiências do
subdesenvolvimento que poderia dar distinção à produção nacional, em
contraposição à tendência ``digestiva'' presente em um cinema que se
contentaria em imitar códigos cinematográficos vindos de fora. Defendia
fazer filmes ``feios e tristes'', pois para o cineasta somente uma
cultura da fome poderia, paradoxalmente, minar aquilo que a gera. E a
mais nobre manifestação cultural da fome, dizia ele, é a violência:
``uma estética da violência antes de ser primitiva é revolucionária, eis
aí o ponto inicial para que o colonizador compreenda a existência do
colonizado''.\footnote{Glauber Rocha. ``Eztetyka da fome'', op.~cit., p.~66.} Somente através de uma estética da
violência as razões da fome poderiam ser entendidas e também atacadas. É
entre a fraqueza que a fome causa e a potência violenta de uma estética
da fome que Glauber Rocha enxerga o núcleo do projeto do Cinema Novo.
Intervalo paradoxal entre fraqueza e potência que evoca a frase com que
o artista Hélio Oiticica conclui seu texto ``Esquema geral da Nova
Objetividade'', escrito em 1967, mesmo ano de lançamento de \emph{Terra
em transe}, o ``manifesto prático da estética da fome'' de Glauber
Rocha: ``da adversidade vivemos!''.\footnote{Hélio Oiticica. ``Esquema
  geral da Nova Objetividade''. In: \emph{Hélio Oiticica. Museu é o mundo}. Rio de Janeiro: Beco do
  Azougue, 2011, p.~101 (texto publicado originalmente em 1967).} Esta
seria a condição do subdesenvolvido.

\section{Estamos com fome}

Em ``Esquema geral da Nova Objetividade'' --- ensaio que acompanhava a
mostra Nova Objetividade Brasileira, exibida no Museu de Arte Moderna do
Rio de Janeiro em abril de 1967 ---, Hélio Oiticica resume o que
enxergava como características principais da arte brasileira de
vanguarda, sendo a primeira e mais ampla delas aquilo que chamou de
``vontade construtiva geral''. Para o artista, a \emph{vontade
construtiva} que identificava em trabalhos de criadores diversos do país
se traduzia em tentativas de caracterizar, culturalmente, o que seria
próprio a um país subdesenvolvido. A partir desse traço da produção
artística que lhe interessava e importava, propunha atualizar e
radicalizar a \emph{estratégia antropofágica} do modernista Oswald de
Andrade --- ``defesa'' e ``arma criativa'' contra o domínio exterior ---
para que se estabelecesse, no Brasil, um ``estado criador geral''.
Alerta, contudo, para o fato de que, diferentemente do que ocorrera no
passado com as então chamadas vanguardas artísticas do país, o que
identifica como vontade construtiva geral estaria ancorado em um gradual
fortalecimento do envolvimento de artistas em questões políticas, éticas
e sociais, principalmente após o golpe de 1964.\footnote{Ibidem, p. 88.}

Embora estivesse refletindo, nesse texto, sobre um grupo de artistas do
qual se sentia próximo em graus variados, há, nas reflexões de Hélio
Oiticica, o desejo de compreender e expressar, primeiramente, aquilo que
ele próprio estava produzindo. E entre suas criações incontornáveis
desse período, encontram"-se os \emph{Parangolés}, trabalhos que
sintetizavam o programa criativo do artista --- arte como
\emph{experiência} no mundo --- e que assumiam, na maior parte das vezes
(mas não sempre ou necessariamente), o formato de ``capas'' feitas com
tecido e outros materiais encontrados no cotidiano ordinário, as quais
deveriam ser carregadas ou vestidas em situações diversas de movimento
corporal para existirem como trabalhos. Para Hélio Oiticica, o parangolé
desmanchava os limites entre aquilo que é proposto como ato criativo e a
emergência de significados para tal gesto por meio da participação ativa
do ``outro'', borrando distinções estanques entre artista e espectador.
Mais do que um termo que identificava um conjunto de trabalhos que
partilhavam certos atributos, o parangolé seria, principalmente, um
conceito que revelava os limites do entendimento convencional de arte,
no qual algo é criado apenas para a ``contemplação'' alheia. Como
corolário dessa posição, passa a sugerir que o lugar de efetivação desse
tipo de trabalho não poderia ser mais a ``exposição'', propondo, em seu
lugar, a noção de ``ambiente''.\footnote{Hélio Oiticica. ``Anotações
  sobre o parangolé''. In: \emph{Hélio Oiticica. Museu é o mundo}, op.~cit., p.~82.}

Em 1967, Hélio Oiticica produziu uma série de parangolés que pela
primeira vez incorporavam frases escritas ou afixadas neles, três dos
quais denominou de ``capa protesto''. O primeiro desses parangolés
(\emph{P15 -- Capa 11}) trazia escrita a frase"-lema ``da adversidade
vivemos'', a qual afirma a capacidade de responder criativamente às
dificuldades de sobrevivência material no Brasil ao mesmo tempo em que
assinala um posicionamento vitalmente contrário (e de resistência ativa,
portanto) ao cerceamento do direito de cada um afirmar um ponto de vista
distinto, liberto de constrangimentos políticos, econômicos, estéticos
ou morais. O segundo (\emph{P17 -- Capa 13}), exibia a frase ``incorporo
a revolta'', em sintonia com a necessidade de ``fundamentar a vontade
criativa no campo político"-ético"-social'', característica definidora,
segundo o artista, da vanguarda de um país subdesenvolvido como o
Brasil; dava relevo, além disso, ao corpo como lugar privilegiado para o
exercício daquela fundamentação. Já o terceiro e último desse conjunto
de parangolés de protesto feito em 1967 (\emph{P18 -- Capa 14})
estampava a frase ``estamos famintos'', numa alusão evidente à condição
de falta que é própria do subdesenvolvido mas, igualmente, como sugeria
Glauber Rocha em seu texto ``Eztetyka da fome'', à potência emancipadora
que um corpo faminto adquire.

1967 foi ainda o ano da realização de \emph{Tropicália}, um dos mais
importantes trabalhos na trajetória de Hélio Oiticica. Incluído na
mostra Nova Objetividade Brasileira ---, foi a primeira aparição, na obra
do artista, do que ele chamava de ``ambientes'', ou espaços estruturados
para a participação do ``outro''. \emph{Tropicália} foi fruto direto da
vivência de Hélio Oiticica nos morros do Rio de Janeiro e, em
particular, na comunidade da Mangueira. A experiência de andar pelas
``quebradas'' de favelas, esgueirando"-se por entre barracos e becos, era
emulada, nesse ambiente construído, por meio da presença de
\emph{Penetráveis} --- labirintos de madeira e tecido com passagens
estreitas e articulados por caminhos de areia ou brita. As casas
improvisadas que havia nos bairros pobres que frequentava --- respostas
originais a uma impossibilidade e atestado das desigualdades de acesso a
moradias --- eram invocadas, ademais, nos arranjos construtivos que davam
forma às criações ali reunidas.\footnote{Para uma discussão extensa
  sobre a relação entre os modos construtivos empregados nas favelas e a
  obra de Hélio Oiticica, ver Paola Berenstein Jacques. \emph{Estética
  da ginga. A arquitetura das favelas através da obra de Hélio
  Oiticica}. Rio de Janeiro: Editora Casa da Palavra/\versal{RIOARTE}, 2001.}
Articulados por meio de elementos da natureza (além da areia e brita,
folhagens e araras vivas) e da cultura locais, esses \emph{penetráveis}
integravam um ambiente capaz de oferecer, àqueles que o percorressem
descalços, a sensação de estar ``pisando a terra'', evocando e
partilhando o que, segundo o depoimento de Hélio Oiticica, ele mesmo
sentia ao caminhar por entre as vielas dos morros e favelas.\footnote{``Quando
  eu ando ou proponho que as pessoas andem dentro de um \emph{Penetrável} com
  areia e pedrinhas\ldots{} eu estou sintetizando a minha experiência da
  descoberta da rua através do andar\ldots{} do espaço urbano através do
  detalhe, do andar\ldots{} do detalhe síntese do andar\ldots{}''. ``Depoimento de
  Hélio Oiticica para Ivan Cardoso, janeiro de 1979''. In: \emph{Hélio
  Oiticica. Encontros}. Rio de Janeiro: Beco do Azougue, 2009, p.~231.}

Tudo somado, havia em \emph{Tropicália} a vontade de propor, em
contraposição à ``avalanche informativa e imagética'' que a sociedade
moderna impunha desde fora, um retorno a experiências basilares de vida,
uma estratégia para se descondicionar de um contexto social alienante.
Em seu conjunto, os elementos presentes no ambiente buscavam recuperar
uma sensação vivida e apresentá"-la simbolicamente ao outro.\footnote{``Entrevista
  a Guy Brett. Londres, fevereiro de 1969''. In: \emph{Hélio
  Oiticica. Encontros}, op.~cit., p. 60.} Anos
depois, o artista viria a identificar esses procedimentos condensados em
\emph{Tropicália} com um processo de ``mitificação''
consciente do cotidiano.\footnote{Hélio Oiticica. ``Anotações sobre o
  parangolé'', op.~cit., p. 75.}
Mitificação que não se confundia, entretanto, com mera celebração do
primitivo ou do precário, expondo, por meio da construção de uma imagem
fragmentada de brasilidade, atritos e incoerências da vida do
país.\footnote{Para uma discussão sobe o processo de construção de uma
  ``imagem'' de Brasil em \emph{Tropicália}, ver Sérgio Bruno Martins.
 ``Hélio Oiticica. Mapping the constructive''. \emph{Third Text}, v.~24, n.~4, p.~409-422, 07/2010.}

\section{Uma criança sorridente, feia e morta}

Um dos principais marcos do Cinema Novo --- talvez seu ápice e início de
fim --- é, concordam críticos e historiadores, \emph{Terra em transe}, de
Glauber Rocha. Filme que enfrenta, ou ao menos expõe, os impasses de uma
prática política pretensamente revolucionária na América Latina em
meados da década de 1960, mas que também apresenta o conservadorismo
cultural à época dominante e a renovada presença estrangeira no Brasil,
características do ambiente após a tomada do poder à forca pelo
militares, em 1964. Não é de estranhar, portanto, que o então jovem e
politizado compositor Caetano Veloso tenha sido profundamente afetado
por \emph{Terra em transe} no processo de criação de ``Tropicália'',
música que abre seu primeiro disco individual (\emph{Caetano Veloso}),
lançado em 1968. Segundo seu autor, ``Tropicália'' foi o mais perto
que pode chegar, no âmbito da música, do que lhe foi sugerido por
\emph{Terra em transe}\footnote{Caetano Veloso. \emph{Verdade
  tropical}. São Paulo: Companhia das Letras, 1997, p.~187.} --- articulação
contraditória entre um acolhimento crítico dos arcaísmos culturais,
sociais e políticos vigentes no período ditatorial e uma valoração de
formas, imagens e gestos próprios de um outro ideário em gestação. Em
``Tropicália'', o ``monumento'' inaugurado ``no planalto central do
país'' pelo narrador da canção é ``bem moderno''; mas ``não tem porta'',
e sua entrada ``é uma rua antiga feia e torta''. Há ainda, na letra da
canção, a justaposição, sem hierarquias, de bossa e palhoça, de roseiras
e urubus e de samba e bang"-bang, reforçando o caráter dualista e
desigual próprio dos países subdesenvolvidos. Países que combinam, em
aberto paradoxo, situações de atrelamento ao novo --- aviões, caminhões,
``cinco mil autofalantes'' --- e de manutenção do antigo --- condensada na
imagem de ``uma criança sorridente, feia e morta'' que ``estende a
mão'', certamente em súplica por comida. O moderno não acaba a fome,
antes a reproduz em alguns cantos do mundo.

``Tropicália'', a música influenciada por \emph{Terra em transe}, é
considerada o marco inicial do tropicalismo, movimento que, de
acordo com o ensaísta Roberto Schwarz, construiu uma \emph{alegoria} do
Brasil --- exposição dos anacronismos do país ``à luz branca do
ultramoderno''.\footnote{Roberto Schwarz. \emph{As ideias fora do lugar: ensaios selecionados}, op.~cit., p.~24.}
Alegoria que resulta, como propõe o pesquisador Celso Favaretto, da
desatualização proposital de imagens que seriam próprias de um país
moderno por meio de montagens inusitadas e com frequência paródicas,
mostrando"-as como indicadores às avessas de um Brasil ainda
culturalmente arcaico.\footnote{Celso Favaretto. \emph{Tropicália:
  alegoria, alegria.} Cotia: Ateliê Editorial, 2000, p.~48
  (1ª edição de 1979).} Estratégia criativa que era,
conforme Caetano Veloso defendeu à época, tentativa de superar o
subdesenvolvimento ``partindo exatamente do elemento `cafona' da nossa
cultura, fundido ao que houvesse de mais avançado industrialmente, como
as guitarras e as roupas de plástico''.\footnote{Caetano Veloso.
  ``Acontece que ele é Baiano''. Entrevista concedida a Décio Bar. Revista \emph{Realidade}, ano \versal{III}, n.~33, 12/1968, p. 197.} Produção
musical, portanto, que, diferentemente do Cinema Novo e, em menor
medida, da produção de Hélio Oiticica no mesmo período, se distanciava
do que era considerado ``cultura popular'' no Brasil para reinventar o
país a partir da deglutição de um variado ideário pretensamente moderno
que se consolidava ali.

Mas se ``Tropicália'', a canção, sofreu reconhecida influência do
filme de Glauber Rocha, seu título foi tomado emprestado do trabalho
homônimo de Hélio Oiticica, revelando uma segunda proximidade de
entendimentos da situação então vivida no país, a despeito das
diferenças em procedimentos criativos. Sem ter qualquer conhecimento
sobre a obra do artista quando compôs a música, Caetano Veloso foi
alertado por um amigo que as articulações de temas e formas condensadas
naquela canção de alguma maneira se relacionavam com o ambiente que
Hélio Oiticica exibia, àquele momento, na mostra Nova Objetividade
Brasileira. Atraído pela descrição da obra do artista que lhe fora feita
e sem outro nome para dar à música já pronta, Caetano Veloso terminou
por batizá"-la como ``Tropicália'' poucos meses depois, quando
finalmente a registrou em disco.\footnote{Caetano Veloso. ``Acontece que ele é Baiano'', op.~cit., p.~188.}

\section{Antropofagia do faminto}

Para além das aproximações quase fortuitas entre artistas que se
posicionaram, de modos distintos, em relação a um tempo e a um lugar
partilhados, é possível apontar a existência de algo que atravessa e
articula as obras de Caetano Veloso, Glauber Rocha e Hélio Oiticica ---
ou tropicalismo, Cinema Novo e Tropicália --- nesse período, também
guardadas, obviamente, suas distâncias e diferenças. No texto
``Tropicalismo, antropologia, mito, ideograma'', escrito em 1969,
Glauber Rocha sugere que o mais significante da produção dos
tropicalistas e cinemanovistas, feita sob as restrições próprias do
subdesenvolvimento e a partir da potência que essa situação encerra --- a
partir da fome, esse elemento que enfraquece e ao mesmo tempo fortalece
quem a sente ---, seria o \emph{ponto de vista} adotado por músicos e
cineastas diante de algo que os constrangia e simultaneamente os
desafiava. Seria, portanto, menos uma questão de repertório ou de
conceituação e mais uma questão de afirmação de um lugar de locução
específico, determinado pela condição do subdesenvolvido. Como resume o
autor, quando ``o país descobriu o subdesenvolvimento, o nacionalismo
utópico entrou em crise e caiu''; ou seja: quando ficou evidente que o
Brasil não estava seguindo uma rota que necessariamente levaria ao
padrão de desenvolvimento já usufruído por outros países, só restaria
aos brasileiros ``superar o subdesenvolvimento com os meios do
subdesenvolvimento''. O importante, para Glauber Rocha, seria assumir
uma ``atitude diante da cultura colonial que não é uma rejeição à
cultura ocidental'', mas que gera uma procura e um resultado estético
originais e que se querem emancipadores.\footnote{Glauber Rocha.
  ``Tropicalismo, antropologia, mito, ideograma''. In: \emph{Revolução do Cinema Novo},
  op.~cit., p. 150.} Uma atitude que se confunde com uma
antropofagia em contexto de fome.

Essa \emph{atitude} é a grande novidade do Cinema Novo, assim como é a
novidade do tropicalismo. É a novidade também de algumas produções no
campo das artes visuais, entre as quais se destacava, àquele momento, a
obra de Hélio Oiticica. Atitude que está indissoluvelmente ligada ao
reconhecimento do subdesenvolvimento como condição, e não como etapa
passageira. Condição que é obstáculo e, em simultâneo, oportunidade de
gerar pensamento discursivo e performativo original, exigindo, daqueles
que vivem sob ela, uma postura ativa de entendimento e de confronto,
caso se deseje um dia superá"-la. Que exige uma ``vontade construtiva
geral''. Tal como, de outra maneira e no contexto da formulação de
políticas econômicas, sugerira Celso Furtado já na década de 1950.
Postura de criação que reconhece o \emph{outro} hegemônico (seja ele
quem ou o que for) e o transfigura a partir do contexto político, ético
e social em que cada um se encontre, caracterizando o local de um jeito
novo e único. Atitude que continuamente subverte limites dados --- sendo
um método de construção, portanto --- e que é, nesse contexto, mais
valorada como gesto do que como invenção formal, de modo que o processo
criativo importa, por vezes, mais do que o resultado. Vontade
construtiva geral que traduz a busca por uma singularidade emancipadora
da experiência do subdesenvolvimento.

\section{Diarréia}

A proposição de uma atitude crítica e criativa diante de um estado de
coisas que oprime vai de novo aparecer, de modo transformado, em
trabalhos e textos produzidos entre o final da década de 1960 e o início
da seguinte. Momento em que a experiência do tropicalismo se exaure, com
dois de seus principais protagonistas --- Caetano Veloso e Gilberto Gil
--- no exílio, o Cinema Novo se enfraquece como projeto coletivo de
invenção de imagens e Hélio Oiticica passa a maior parte do tempo fora
do país. Momento de ainda maior fechamento político, com a instauração
do Ato Institucional n. 5 (\versal{AI}-5) e o início do período de mais brutal
repressão aos que contestavam a ditadura militar no Brasil. É o momento
em que se quer interditar, à força, a vontade construtiva que
caracterizara tantas criações nos anos anteriores. Diante do cerco, a
saída de vários artistas é dar nova expressão a um traço característico
de uma produção que articulava, desde a posição do subdesenvolvido,
temas e formas os mais distintos, forjando uma maneira nova de
identificar"-se com a complexidade do país. Em um contexto abertamente
regressivo e violento, o que era mistura de proximidade e distância
passa a ser, cada vez mais, mistura de atração e repulsa. Na
impossibilidade de exprimir o hibridismo antropofágico que caracterizava
suas produções de um modo irônico ou festivo, artistas passam a
expressá"-lo, menos ou mais explicitamente, como \emph{abjeção}, conceito
que, no campo disciplinar da psicanálise, remete a uma condição em que o
sujeito tem certezas postas em cheque e significados partilhados entram
em colapso, provocando uma sensação de alerta e tensão face a esse
desamparo.\footnote{Julia Kristeva. \emph{Powers of horror. An essay on
  abjection.} Nova York: Columbia University Press, 1984.}

Tal virada criativa ajuda a compreender, como exemplo de outros tantos,
a emergência de trabalhos como os que o artista Artur Barrio faz no
início da década de 1970, em que centenas de sacos plásticos com restos,
dejetos e fluidos de naturezas as mais diferentes, mas comuns à vida
ordinária de qualquer um, são lançados em pontos diversos da cidade,
espelhando a indistinção da natureza dos itens contidos nos sacos na
indiferenciação dos lugares escolhidos para sua distribuição.
Confrontados com sacos cheios de coisas que, embora reconhecíveis na sua
banalidade, não se encaixam simbolicamente umas nas demais, além de não
deverem supostamente estar onde foram colocados, os passantes
experimentam o desmanche das categorias classificadoras que definem o
que é ou não lixo, o que é alimento ou matéria inorgânica, o que atrai o
interesse ou o afasta. São postos em contato, mesmo que momentaneamente,
com um ato de resistência a uma ordem social autoritária e
artificialmente regradora.

Ao transformar a alegria pulsante e emancipadora característica da
produção artística de meados da década de 1960 em situações de abjeção,
alguns artistas (além de Artur Barrio, Anna Maria Maiolino, Antonio
Dias, Antonio Manuel, Lygia Clark, Lygia Pape, Paulo Bruscky e outros
mais) fazem confluir e precariamente conviver, na materialidade instável
de seus trabalhos, o apreço por seu lugar de vida e a rejeição a muito
do que ele representava naquele momento de cerceamento à expressão.
Procedimentos paradoxais em que não se discernem mais as diferenças
entre ``simpatia e desgosto'' sobre a condição do Brasil,\footnote{Roberto Schwarz. \emph{As ideias fora do lugar: ensaios selecionados}, op.~cit., p.~26.} colocando"-se em antítese ao slogan
\emph{dicotômico} do governo da ditadura à época --- ``ame"-o ou deixe"-o''
---, endereçado a todos que discordavam dos rumos dados então ao país.

Também Hélio Oiticica se aproximou, a seu modo, da ideia de abjeção como
estratégia de resistência política. Em textos cuja sintaxe é cada vez
mais fragmentada e o léxico é constantemente inventado --- talvez uma
maneira de acentuar um ponto de vista ou um sotaque próprio à condição
do subdesenvolvido ---, Hélio Oiticica cunha a ideia de que o Brasil
seria a ``subterrânia'' do mundo. E embora descreva a subterrânia como
``o sub desenvolvido embaixo da terra como rato de si mesmo'',\footnote{Hélio Oiticica.
  ``Subterrânia''. In: \emph{Hélio Oiticica. Museu é o mundo}, op.~cit., p.~145.} não há nessa caracterização uma conotação puramente
negativa. Há, antes (ou em paralelo), a vontade de reafirmar a posição
precária desde onde se vive e se constrói algo; desde um país cada vez
mais submetido a uma lógica conservadora e violenta, interna e
externamente. Frente a um ambiente que constrange a emergência e a
consolidação de uma vontade construtiva geral, propõe assumir"-se uma
posição crítica que considere, com renovado vigor, as ambivalências e
contradições de viver e de criar diante das adversidades próprias à
condição do subdesenvolvido. Posição que não quer negar a ``condição
colonialista'' a que o Brasil está submetido nem tampouco conservá"-la.
Posição crítica que quer, ao contrário, ``assumir e deglutir os valores
positivos dados por essa condição'', dessa forma combatendo o sentimento
de impotência que ela instaura e construindo algo que não existia
ainda.\footnote{Hélio Oiticica. ``Brasil diarreia''. In: \emph{Hélio Oiticica. Museu é o mundo},
op.~cit., p.~163.}

Hélio Oiticica chama a atenção, contudo, para o fato de a formação
cultural do Brasil ser \emph{diarreica}, resultado de um fluxo de
excrementos expelidos de um corpo social que tem fome e que come tudo o
que encontra à frente. A fome, talvez, de que falava Glauber Rocha:
aquela que se sacia somente através da violência apressada dos famintos.
Diante dessa singularidade, diz o artista, haveria duas possibilidades:
ou recalcar a diarreia e produzir ``a prisão de ventre nacional''
(metáfora para o que há de mais regressivo no país), ou ``mergulhar na
merda'' para construir outra possibilidade de Brasil.\footnote{Ibidem,
p.~163.} Uma construção subterrânea que talvez fosse, parece
sugerir o artista, uma estratégia possível de combate à condição do
subdesenvolvimento no ambiente politicamente restritivo e repressivo da
época.

\section{Merda}

Considerar tudo isso tendo em vista o Brasil de agora implica também
pensar em quais sentidos essas formulações poderiam manter"-se artística
e politicamente válidas. Se é certo que o termo subdesenvolvido
raramente é encontrado em livros de economia atuais, essa mudança não se
deve ao fim das desigualdades estruturais entre países diversos, nem
tampouco das que existem internamente a tantos espaços nacionais. Tais
desigualdades continuam sendo o tempo inteiro repostas, condenando
milhões de pessoas a uma condição de falta em meio à riqueza que as
relações entre esses mesmos países gera. Falta, inclusive, de comida
acessível a todos, levando a \versal{FAO} (Organização das Nações Unidas para
Agricultura e Alimentação) a criar, em 1990, um Mapa da Fome, no qual
indica, a cada ano, ``em quais países há parte significativa da
população ingerindo uma quantidade diária de calorias inferior ao
recomendado''. Mapa do qual o Brasil saiu, pela primeira vez, em 2014, e
para o qual há riscos cada vez maiores de retornar.\footnote{``Como o
  Brasil saiu do Mapa da Fome. E por que ele pode voltar''. Disponível em:
  \textless{}\emph{https://bit.ly/2PVScAY}\textgreater{}.}
A razão de o termo subdesenvolvido não ser mais largamente empregado
como antes repousa, paradoxalmente, em ser de novo hegemônica, em parte
relevante da academia e na mídia, a ideia de que o subdesenvolvimento é
uma etapa a ser naturalmente vencida pelos países mais pobres por meio
de um esforço continuado de produção. Setores da academia e da mídia
que, por serem espaços de poder assentados no ideário e no léxico
neoliberais, preferem a designação de país \emph{emergente} à de país
\emph{subdesenvolvido}. Denominação que remete, em disfarçado retrocesso
conceitual, a uma \emph{situação} passageira de inferioridade de alguns
países frente a outros, e não a uma \emph{condição} que somente pode vir
a ser superada através de amplas mudanças estruturais no sistema
econômico vigente. Nesse contexto, talvez pensar o Brasil como um país
subdesenvolvido (e não como um país emergente) seja uma forma possível
de resistência a essa condição. Não mais com a quase irônica resignação
contida na canção de protesto feita no início da década de 1960. Mas
resgatando, daquela mesma década e do início da seguinte, uma vontade
construtiva em meio à condição de subalternidade --- externa e interna ao
país. Reinventando, para os tempos recentes, a capacidade de imaginar o
impossível no meio da merda.

\pagebreak

\section{Referências}

\begin{Parskip}
\textsc{cardoso}, Fernando Henrique \& \textsc{faletto}, Enzo. \emph{Dependência e desenvolvimento na América Latina: ensaio de interpretação sociológica.} Rio de Janeiro: Zahar, 1970.

\textsc{castro}, Josué de. \emph{Geografia da fome}. Rio de Janeiro: Civilização Brasileira, 2011.

\textsc{fao} -- Organização das Nações Unidas para
Agricultura e Alimentação. ``Como o Brasil saiu do Mapa da Fome. E por que ele pode voltar''. Disponível em:
  \textless{}\emph{https://bit.ly/2PVScAY}\textgreater{}.

\textsc{favaretto}, Celso. \emph{Tropicália: alegoria, alegria.} Cotia: Ateliê Editorial, 2000.

\textsc{filho}, César Oiticica (org.). \emph{Hélio Oiticica. Museu é o mundo}. Rio de Janeiro: Beco do Azougue, 2011.

\_\_\_\_\_\_; \textsc{cohn}, Sergio; \textsc{vieira}, Ingrid (org.). \emph{Hélio
  Oiticica. Encontros}. Rio de Janeiro: Beco do Azougue, 2009.

\textsc{furtado}, Celso. \emph{Teoria e Política do Desenvolvimento Econômico}. São Paulo:
  Editora Nacional, 1968.

\textsc{gomes}, Paulo Emílio Sales. \emph{Cinema: trajetória no subdesenvolvimento}. São Paulo: Paz e Terra, 1996.

\textsc{gullar}, Ferrreira. \emph{Vanguarda e subdesenvolvimento. Ensaios sobre arte}. Rio de Janeiro: Civilização Brasileira, 1984.

\textsc{jacques}, Paola Berenstein. \emph{Estética da ginga. A arquitetura das favelas através da obra de Hélio Oiticica}. Rio de Janeiro: Editora Casa da Palavra/\versal{RIOARTE}, 2001.

\textsc{jesus}, Carolina Maria de. \emph{Quarto de despejo. Diário de uma favelada}. São Paulo: Ática, 2014.

\textsc{kristeva}, Julia. \emph{Powers of horror. An essay on abjection.} Nova York: Columbia University Press, 1984.

\textsc{martins}, Sérgio Bruno. ``Hélio Oiticica. Mapping the constructive''. \emph{Third Text}, v.~24, n.~4, p.~409-422, 07/2010.

\textsc{schwartz}, Roberto. \emph{As ideias fora do lugar: ensaios selecionados}. São Paulo: Penguin Clássicos/Companhia das Letras, 2014.

\textsc{rocha}, Glauber. \emph{Revolução do Cinema Novo}. São Paulo: Cosac Naify, 2004.

\textsc{veloso}, Caetano. \emph{Verdade tropical}. São Paulo: Companhia das Letras, 1997.

\_\_\_\_\_\_. ``Acontece que ele é Baiano''. Entrevista concedida a Décio Bar. Revista \emph{Realidade}, ano \versal{III}, n.~33, 12/1968.

\end{Parskip}

\chapter*{Corpo presente? Algumas reflexões sobre televisão, música e corpo no fim dos anos 1960}
\addcontentsline{toc}{chapter}{Corpo presente?, \emph{por Paulo da Costa e Silva}}

%Texto com problemas de bibliografia. 

\begin{flushright}
\emph{Paulo da Costa e Silva}
\end{flushright}

\epigraph{``Eu não acho que os anos 1960 tenham sequer começado. Acho que os anos
1960 duraram talvez quinze ou vinte minutos na mente de alguém. Eu os vi
se moverem muito, muito rapidamente na direção do mercado. Não acho que
tenham sequer existido.''}{Leonard Cohen\footnotemark}
%\end{quote}

\footnotetext{Trecho retirado da
  entrevista de Leonard Cohen publicada no livro \emph{Songwriter's on
  songwriting}. Paul Zollo (org.). \versal{EUA}: DaCapo Press, 2003.}

\section{\versal{TV} e liberação do corpo}

É difícil exagerar a influência da televisão na dinâmica veloz das
mudanças que marcaram os anos 1960. Com a televisão um novo regime de
difusão cultural e comportamental tomou forma. São famosas as fotos de
lares norte"-americanos, já em meados dos anos 1950, em que o aparelho
televisor ocupa um lugar de honra na sala, desbancando o velho rádio. As
fotos são um prenúncio do que mais cedo ou mais tarde aconteceria em
quase todo o mundo Ocidental. O núcleo familiar passa a se organizar ao
redor desse novo ente. A televisão de algum modo ampliou e continuou o
rádio. Mas trouxe a novidade decisiva da imagem em movimento. Já
conhecida do cinema, a imagem em movimento se tornou finalmente presente
nos recessos do próprio lar, mediando a vida social com força cada vez
maior.

Era uma questão de tempo que a televisão liberasse a propagação de uma
corporalidade nova. Trata"-se de um veículo mais do que adequado para
isso, com velocidade e penetração inéditas então. Para isso bastava que
houvesse uma ``corporalidade nova'' nos anos 1960. E havia. O conluio da
imagem com o som tornou também mais evidente que tipo de corporalidade
era detentora do poder. Era mais do que constatar de que era um ``homem
branco de tal ou tal fisionomia''. A imagem em movimento, captada de
modo mais instantâneo e natural do que em geral faz a película, e também
de modo mais efêmero e pedestre, fez enxergar, com talvez maior nitidez
do que antes, que havia uma ``retórica corporal'' do poder. Eram gestos,
ritmos, posturas, maneiras de franzir o cenho e de olhar, movimentos
insuspeitos dos lábios, formas de se relacionar com outros corpos que se
associavam a certa representação imaginária do poder, que solicitavam o
reconhecimento de um lugar específico na hierarquia social. Tudo isso
parece ter ficado mais claro através desse novo espelho social que o
mundo técnico produzia. Mas ao mesmo tempo em que a televisão reforçava
o gesto de confirmação da ordem social, ela também ampliava o alcance de
gestos que contrariavam essa ordem. Seu potencial disruptivo,
revolucionário --- no sentido mais amplo que o termo pode ter --- equivalia
ao seu potencial conservador de mantenedora da ordem.

Na verdade, o poder já havia expandido sua corporalidade com a chegada
do rádio, a partir da materialidade da voz. A voz já revela um corpo. A
mensagem veiculada pelas palavras é inevitavelmente enquadrada, tingida,
condicionada, pelo organismo vivo que a emite. O ouvinte capta de modo
inconsciente uma série de informações que lhe permitem contextualizar o
que é dito. Imaginamos quem é o dono da voz --- sabemos se é uma mulher ou
um homem, ponderamos a idade, o vigor físico, o estado de humor,
percebemos ironias e intuímos falsidades, captamos notas involuntárias
de alegria ou de ressentimento. Ou seja, a voz já revela um corpo, e não
admira que os grandes líderes políticos da primeira metade do século \versal{XX}
tenham sido, em geral, grandes vozes. Figuras como Fidel, Hitler,
Churchill e Carlos Lacerda foram grandes faladores. O poder se dirigia
majoritariamente aos ouvidos. Com a televisão isso é de algum modo
ampliado.

É curioso refletir sobre o papel decisivo que as imagens terão na
eclosão da contracultura nos anos 1960. É justamente na brecha dessa
mudança de um regime mais aural para outro, com predominância visual,
que a contracultura encontrará um terreno no qual florescerá com
inesperados vigor e rapidez. Há uma sequência do recente documentário
\emph{No intenso agora} (2018), de João Moreira Salles, em que tal transição
aparece com muita nitidez. Ela envolve o general"-presidente Charles De
Gaulle, símbolo da resistência francesa na Segunda Guerra, e os jovens
líderes estudantis que comandaram as barricadas de maio de 1968.
Querendo transmitir uma mensagem apaziguadora, uma mensagem de algum
modo restauradora da ordem, De Gaulle faz uma aparição na tevê para
desejar aos cidadãos franceses um feliz natal. Embora ainda seja a mesma
voz reconfortante da Resistência Francesa, fazendo novamente um apelo à
coesão social, a imagem televisiva revela impiedosamente um homem já
velho e bastante fragilizado. Mais do que isso: a imagem destoava
radicalmente da energia alegre, explosiva, que emanava dos
acontecimentos que viravam de cabeça para baixo a vida da capital
francesa. A televisão parecia envelhecer ainda mais a geração do general
De Gaulle. Sua imagem exibia uma aparência antiquada para a ânsia de
mudança dos estudantes e para o novo plano simbólico. O meio televisivo
--- ele próprio exalando juventude --- parecia exigir algo mais dinâmico.

É como se naquele momento, no final dos anos 1960, o poder estabelecido
não soubesse ainda como ocupar devidamente o novo espaço aberto pela
televisão. Ancorado em velhos ritos e hierarquias, esse poder parece o
oposto da espontaneidade solicitada por programas transmitidos ao vivo e
captados por aparelhos no espaço protegido e informal da casa. Vendo
entrevistas de jovens expoentes desse mesmo período, vemos como se dão
bem com o novo meio, como casam com ele. Basta ver, no mesmo
documentário de João Salles, as falas televisivas do líder estudantil
Daniel Cohn"-Bendit; ou, em outros filmes de época, a naturalidade das
entrevistas de Caetano Veloso, Gilberto Gil e Roberto Carlos durante o
Festival da Canção da Record de 1967 (``Uma noite em 67'' [2010], Renato
Terra e Ricardo Calil); ou as maravilhosas tiradas de Bob Dylan ou dos
Beatles diante de repórteres quase sempre engravatados nos anos 1960. A
juventude do meio se adequa de modo formidável à juventude dos novos
protagonistas da vida cultural.

O filósofo George Steiner escreveu que nos anos 1960 ocorreu certa
``musicalização da cultura''. Não vivi essa época, mas parece que a vida
pulsava ao som de canções. Compositores do calibre de John Lennon, Bob
Dylan e Chico Buarque tornaram"-se verdadeiros mensageiros culturais.
Pegando emprestado o título de um dos livros do teórico da canção Luiz
Tatit, parece que o século \versal{XX} foi ``O século da canção'',\footnote{Luiz Tatit.
\emph{O século da canção}. São Paulo: Ateliê Editorial, 2004.} e que nesse
mesmo século nenhuma década foi tão intensamente habitada pelas canções
quanto a década de 1960. Talvez isso tenha a ver com a recuperação do
valor da emoção, um valor por muito tempo recalcado pela grande onda
racionalista. O trabalho da mente racional pode ser distraído, ou
distorcido, pelos imperativos causais do corpo. Eis a moldura básica de
uma longa tradição que ganha nova força a partir de Renée Descartes e da
Revolução Científica dos séculos \versal{XVII} e \versal{XVIII}. A apologia da razão e o
desprezo da emoção se tornaram a base da educação moral e da ordem
social no Ocidente moderno. Basta pensar na razão ordenadora que
classifica e hierarquiza o mundo na \emph{Encyclopédie} de Diderot e
D'Alembert, o mais importante acontecimento cultural do século \versal{XVIII}. Ou
na preferência pela cor preta dos burgueses dos século \versal{XIX}, de que fala
Walter Benjamin. Ou nos corpos femininos literalmente encaixotados em
espartilhos que ofereciam uma restrição física aos movimentos
expressivos dos quadris. Cria"-se literalmente uma cultura de repressão
das emoções, uma cultura que pretende discipliná"-las, colocá"-las sob o
jugo da razão; uma cultura marcada por tremenda autoconsciência e
autocontrole.

A geração de 1960 basicamente recuperou o valor da emoção, e a música
cantada (a ``musicalização da cultura'', de que fala Steiner), seria,
justamente, uma forma de discurso regida pela emoção --- enquanto o
discurso sem a melodia do canto seria justamente o \emph{logos} que
permanece quando você retira a emoção. A palavra cantada é tal que não
pode ser dissociada do corpo, de sua concretude física. Não à toa os
principais heróis geracionais dos anos 1960 foram poetas"-cantores, que
logo se tornaram também ídolos televisivos. Guardavam algo dos antigos
aedos gregos, porta"-vozes dos deuses, tal como descritos por Platão no
\emph{Íon}; ou dos \emph{vates} da antiga Roma, a um só tempo poetas e
videntes, enxergando através da neblina de mudanças. No Brasil do fim
dos anos 1960 havia nada menos do que sete programas de música popular
na televisão. Hoje não temos nenhum. Foi ali que despontaram alguns de
nossos principais ícones culturais do século \versal{XX}. A música popular se
tornou em grande medida um \emph{acontecimento televisivo}, capaz de dar
uma outra amplitude, por exemplo, ao gesto de Caetano repetindo o ``por
que não?'' ao final de ``Alegria, alegria'' no Festival da Canção de
1967.

\section{Tropicalismo e corpo}

Aqui talvez seja apropriado falar sobre os tropicalistas. E por um
motivo básico: foram eles que melhor souberam explorar o novo meio
televisivo. De fato, a televisão se revelou o suporte ideal para a
``exposição destabuizada do corpo'',\footnote{O termo é usado por
  Guilherme Wisnik no livro que fez sobre Caetano Veloso para a série
  ``Folha Explica''. Guilherme Wisnik. \emph{Caetano Veloso (Folha
  Explica)}. São Paulo: Publifolha, 2005.} parte fundamental da proposta estética do grupo
encabeçado por Caetano e Gil. Se Carmen Miranda era uma das grandes
musas do tropicalismo foi porque, além do tom paródico de pastiche
nacional emanado por sua figura, ela já havia prenunciado o potencial do
elo entre música popular e corporalidade. Carmen foi a primeira a migrar
do rádio para o registro da imagem em movimento (no seu caso, o cinema).
Já sendo um exímia cantora de samba, ela constrói a imagem estilizada e
exótica da baiana com \emph{tutti"-frutti hat}, que a tornaria a atriz
mais bem paga de Hollywood e a brasileira mais famosa do século \versal{XX}. Era
um modelo, uma inspiração, para que os tropicalistas explorassem
artisticamente, de modo consciente e com os novos recursos técnicos, um
dos traços definidores da vida brasileira: a centralidade do corpo. Numa
das passagens mais reveladoras de \emph{Verdade tropical}, a autobiografia
de Caetano, ele conta da conversa que teve, quando estava preso, com um
alto funcionário do exército. Homem culto e lúcido, este deu a entender
que considerava muito mais perigosa a atitude libertária do grupo
tropicalista --- com suas roupas estranhas, seus cabelos longos, suas
danças e movimentos dionisíacos --- do que as canções de protesto com suas
letras ``subversivas''. O perigo revolucionário estava muito mais no
corpo do que nas letras. Foi o corpo, o comportamento corporal, que foi
punido com o exílio.

E não se tratava apenas de uma questão de liberdade comportamental para
os filhos da classe média --- desejosos de se conectarem aos movimentos da
juventude internacional, querendo se livrar das amarras constrangedoras
das antigas noções de respeitabilidade. Não. Parece que o corpo no
Brasil sempre foi constituído na tensão entre dois registros: o registro
oficial, que tem por modelo a ``respeitabilidade'' do mundo ``europeu
civilizado'', e o registro popular, mais livre e anárquico, fundado em
matrizes não europeias, e que aparentemente não se cristalizou num
código moral definido. Esses registros descrevem universos sociais
distintos, mas que, no contexto flexível da sociabilidade brasileira, se
interpenetram continuamente. Basta pegar as histórias suburbanas de
Nelson Rodrigues para ver como esses registros corporais se cruzam o
tempo todo. Ou \emph{O cortiço}, de Aluísio Azevedo; ou ainda \emph{Dom
Casmurro}, de Machado de Assis --- na clássica figura do agregado que
reúne num único indivíduo dois registros corporais absolutamente
distintos, como bem colocou Roberto Shwarz.

No Brasil o corpo é um índice marcado pela ambiguidade: parece
potencialmente livre, uma vez que não foi inteiramente condicionado, ou
adestrado, pela moralidade cristã de repúdio ao corpo; ao mesmo tempo,
parece ter sido coisificado ao extremo, uma vez que foi colocado a
serviço da dinâmica escravista. Ele oscila entre esses dois polos: um
hora promessa de liberdade e prazer, com grandes reverberações
existenciais; outra hora simples refugo descartável a ser usado e
consumido de acordo com a lógica extrativista do escravocrata. E é
evidente que a exposição visual do corpo na televisão reforçou a
percepção do abismo de classes no Brasil, uma vez que coexistem
corporalidades inteiramente distintas no interior da sociedade
brasileira. De fato, a televisão deu maior visibilidade aos padrões
discriminatórios que organizam as sociedades, e que eram justificados
como princípios hierárquicos fundamentais. Talvez o aguçamento dessa
percepção tenha sido um dos fatores (entre outros) que contribuíram para
levar artistas como Hélio Oiticica e Lygia Clark a abandonarem o
formalismo adotando o corpo como principal suporte de suas obras. Em
Hélio Oiticica a dimensão antropológica do corpo é ainda mais nítida.
Sua aproximação do universo do samba é fundada na percepção de que o
samba não é apenas um estilo musical, mas um complexo corporal. Cultura,
tradição e mito estão inscritos na carne, qual uma tatuagem. São
vivenciados não como imagem a ser pensada, reflexão abstrata ou peça de
museu, mas como experiência concreta do corpo. É ali que a tradição
efetivamente vive. É ali que tem se dado o movimento profundo da
experiência brasileira, em todas as suas complexidades e diferenças.
Ali, no que José Miguel Wisnik denominou como ``lugar fora das ideias''
(invertendo a fórmula de Roberto Shwarz das ``ideias fora do lugar''):
``o vetor insconsciente por meio do qual o substrato histórico e atávico
da escravidão se reinventou de forma elíptica, artística e
lúdica''.\footnote{José Miguel Wisnik. \emph{Veneno remédio: o futebol e o Brasil}. São Paulo:
  Companhia das Letras, 2008, p.~405.} É possível que a entrada decisiva da
televisão no imaginário brasileiro tenha expandido ou gerado um novo
tipo de percepção desse lugar.

\section{Um paradoxo: o corpo retorna, mas o corpo some}

Há um termo em inglês que descreve um dos efeitos que o mundo
contemporâneo impõe sobre nossa experiência do real: \emph{disembodiment} ---
que poderia ser traduzido por algo como ``descorporificação''. Tenho
esbarrado com esse termo em textos que pensam a nova experiência
subjetiva num contexto onde proliferam telas e realidades virtuais, onde
a excitação contínua da hiperconexão se tornou a norma. E onde, apesar
de tudo isso, há um profundo sentimento de vazio e insatisfação. O
\emph{disembodiment} representa um desligamento do próprio corpo. Há uma
distinção crucial entre sentir a vida sendo vivida estritamente na
cabeça (o mundo sendo tomado como uma realidade puramente conceitual) e
a vida vivida no corpo e através do corpo (o corpo sendo tomado como
nosso modo primário de conhecimento, entidade inteligente em
si).\footnote{Para uma discussão densa e contemporânea sobre o
  \emph{disembodiment} ver o excelente livro do acadêmico inglês Iain
  McGillchrist sobre os dois hemisférios cerebrais: \emph{The master and
  his emissary -- the divided brain and the making of the Western world}. Estados Unidos: 
Yale University Press, 2009.} Residir primordialmente na cabeça é se aproximar do mundo
objetificando"-o; é se colocar em posição de dominação e controle.
Domesticamos o mundo filtrando"-o através dos nossos conceitos --- fazemos
um mapa mental que nos permite manipular o mundo, tomar posse dele,
submetê"-lo a nossas agendas e vontades. Por outro lado, habitar
plenamente o corpo é redescobrir nossa conexão profunda com o mundo. Não
estamos mais acima dele, como manipuladores, numa posição de dominação e
controle; estamos agora incrustados dentro dele, postos em relação de
interdependência com tudo o que existe --- pessoas, espaços
arquitetônicos, animais, vegetais, todo o mundo natural. De fato, os
anos 1960 voltam a pensar a humanidade como estando inserida no mundo
dos processos naturais, rompendo com a longa tradição de filósofos e
cientistas naturais que defendiam pontos de vista que apartavam a
humanidade da natureza. O pensamento mais holístico volta a ganhar
espaço em detrimento da visão fragmentária.

Há uma perspectiva existencial diferente acompanhando tal mudança de
pensamento. É o percurso sugerido pelo Vedanta e pelas religiões
orientais, no caminho para a iluminação. É também o que se depreende
como sendo um dos fundamentos da \emph{Ética} de Espinoza e da
\emph{áskesis} sugerida pela filosofia de Nietzsche: uma libertação da
prisão conceitual e de sua ilusão de separação, que nos leva ao corredor
de espelhos de pensamento que se desdobra em outro pensamento, e um
retorno ao corpo, ao fato primordial da existência, à pura presença.
Nesse sentido, a tarefa do pensamento seria não apenas a de colocar em
xeque sua pretensa autonomia e realidade própria, mas de reconduzir ao
corpo.

Acontece, contudo, algo curioso nesse período de transição --- instala"-se
uma tensão fundamental que traz um novo colorido para tais mudanças. E a
questão é basicamente a seguinte: a onda de liberação corporal que marca
os anos 1960 se dá no momento da transição social para um regime do
espetáculo de massa. Heidegger já havia colocado que ``o fato de que o
mundo se torna imagem é o que distingue a essência da era moderna''. Da
imagem do mundo, passamos ao mundo"-imagem. Mas os anos 1960 experimentam
uma outra volta no parafuso na redução do mundo à imagem. E quem captura
esse estado no momento mesmo em que parece emergir é Guy Debord, com
\emph{Sociedade do espetáculo}, de 1967. Senão, vejamos o que diz o
pensador francês logo na primeira parte de seu clássico ensaio: ``Toda a
vida das sociedades nas quais reinam as modernas condições de produção
se apresenta como uma imensa acumulação de espetáculos. Tudo o que era
vivido diretamente tornou"-se representação''. Em outro momento ele
escreve: ``Quando o mundo real se transforma em simples imagens, as
simples imagens tornam"-se seres reais e motivações eficientes de um
comportamento hipnótico''. Em belíssima formulação, o espetáculo será
descrito como ``o sol que nunca se põe no império da passividade
moderna'', com sua ``multidão crescente de imagens"-objetos''.\footnote{Trechos
  retirados respectivamente dos fragmentos 1, 13, 15 e 18 da parte \versal{I} de
  \emph{A sociedade do espetáculo -- comentários sobre a sociedade do
  espetáculo}, de Guy Debord. Rio de Janeiro: Contraponto, 1997.} O pensamento de Debord faz lembrar uma
cena de \emph{No intenso agora}, na qual, tentando captar as revoltas de
1968, a câmera já não filma diretamente as ruas: filma as imagens da rua
pela televisão, ou seja, a imagem da imagem\ldots{}

Foi desse modo que a visão --- o mais frio dos sentidos, aquele que é
capaz de maior desengajamento, o mais anticorporal --- passou a dominar
tudo. Talvez seja possível traçar uma comparação dos anos 1960 com os
dias atuais; uma linha que estabelece o desenvolvimento de uma
civilização dominada cada vez mais por telas --- por representações e
imagens mediadas. Parece que, em certo sentido, nunca fomos tão visuais.
A tendência, desde então, é voltar a residir primordialmente na cabeça.
Recuar dos sentidos, como se fossemos repelidos pelo corpo.
\emph{Disembodied}.

Uma contradição paira então no ar: a ferramenta potencialmente
libertária que seria a televisão, acaba por ser tornar o próprio
circuito ideológico. Pouco importa o conteúdo veiculado nela: o meio é a
mensagem. E não seria essa uma intuição já presente na instalação
\emph{Tropicália}, de Hélio Oiticica, obra pioneira realizada também no ano
de 1967 (de onde veio o termo que batizou o movimento de Caetano e Gil)?
\emph{Tropicália} pode ser concebida como um percurso no qual o corpo é
acionado em toda a sua riqueza perceptiva (por meio de sensações
variadas) para, logo depois, sumir diante da imagem de um televisor no
fim da instalação (penetrável). E o próprio Oiticica parece ter
confirmado essa leitura numa de suas descrições da obra: ``parecia"-me,
ao caminhar pelo recinto, pelo cenário da Tropicália, estar dobrando
pelas `quebradas' do morro, orgânicas como a arquitetura fantástica das
favelas --- outra vivência, a de `estar pisando a terra' outra vez. Ao
entrar no penetrável principal, depois de passar por diversas
experiências táteis"-sensoriais, abertas ao participador que cria aí o
seu sentido imagético através delas, chega"-se ao final do labirinto,
escuro, onde um receptor de \versal{TV} está em permanente funcionamento: é a
imagem que devora então o participador, pois é ela mais ativa que o seu
criar sensorial''.\footnote{A citação de Hélio Oiticica é datada de março
  de 1968 e foi incluída na coletânea \emph{Tropicália: uma revolução
na cultura brasileira (1967-1972)}. Carlos Basualdo (org.). São Paulo: Cosac Naify, 2007.}

%\pagebreak

\section{Referências}

\begin{Parskip}
\textsc{basualdo}, Carlos (org.). \emph{Tropicália: uma revolução na cultura brasileira (1967-1972)}. São Paulo: Cosac Naify, 2007.

\textsc{debord}, Guy. \emph{A sociedade do espetáculo -- comentários sobre a sociedade do
espetáculo}. Rio de Janeiro: Contraponto, 1997.

\textsc{mcgillchrist}, Iain. \emph{The master and his emissary -- the divided brain and the making of the Western world}. Estados Unidos: Yale University Press, 2009.

\textsc{tatit}, Luiz. \emph{O século da canção}. São Paulo: Ateliê Editorial, 2004.

\textsc{wisnik}, Guilherme. \emph{Caetano Veloso (Folha Explica)}. São Paulo: Publifolha, 2005.

\textsc{wisnik}, José Miguel. \emph{Veneno remédio: o futebol e o Brasil}. São Paulo: Companhia das Letras, 2008.

\textsc{zollo}, Paul (org.). \emph{Songwriter's on songwriting}. \versal{EUA}: DaCapo Press, 2003.
\end{Parskip}

\chapter*{Eles sabem o dia de amanhã, mas eu quero seguir vivendo}
\addcontentsline{toc}{chapter}{Eles sabem o dia de amanhã, mas eu quero seguir vivendo,\\ \emph{por Pedro Duarte}}

%Texto sem referências bibliográficas

\begin{flushright}
\emph{Pedro Duarte}
\end{flushright}

No livro \emph{Verdade tropical}, de 1997, Caetano Veloso defende"-se de
críticas dirigidas a ele nos anos 1960, acusando seus autores de nunca
discutirem ``temas como sexo e raça, elegância e gosto, amor ou forma''.
Era como se o debate em torno do conflito econômico entre burguesia e
proletariado obliterasse a visão de outras formas de embate na
sociedade. O marxismo dominante na interpretação da esquerda sobre o
Brasil deixava à sombra, para Caetano, diversos problemas do país sobre
os quais o tropicalismo queria lançar luz. Isso explica o sentimento de
que o filme \emph{Terra em transe}, de Glauber Rocha, ao pôr sob
suspeita a crença na energia revolucionária do povo, não fosse apenas um
``fim de possibilidades'' mas também um ``anúncio de novas tarefas''
para os tropicalistas, que no mesmo ano do filme --- isto é, 1967 ---
apareciam e transformavam o cenário musical e cultural brasileiro. Sua
``anarquia comportamental'' tinha a ver justo com isso. Os cabelos
selvagens, a dança erótica e as roupas de plástico de Caetano ao se
apresentar no \versal{III} Festival Internacional da Canção da Globo, em 1968,
situavam precisamente a ação artística de transformação em um campo da
cultura além ou aquém da luta de classes dualista, quer dizer, não
coincidente com ela. O choque estético era, ao mesmo tempo, moral.
Incidia sobre costumes que, na linguagem da época, seriam caretas e que
eram igualmente distribuídos na esquerda e na direita políticas.

Escute"-se a canção ``Eles'', do primeiro disco solo de Caetano,
gravado ainda em 1967 e lançado em 1968. Os versos dirigem"-se
criticamente a ``eles'', mas eles não são só os militares que governavam a
ditadura em vigor no Brasil desde 1964 e nem a burguesia dona dos meios
de produção que explorava economicamente os trabalhadores. Eles são os
indivíduos conservadores da sociedade brasileira. Eles mantêm costumes
tradicionais e não toleram nada diferente. ``Eles têm a certeza do bem e
do mal, alegres ou tristes, são todos felizes durante o Natal''.
Portanto, o que define ``eles'' é a certeza, ou seja, a ausência de
dúvida ou questionamento, e especialmente a certeza moral sobre o que
são o bem e o mal, o certo e o errado no comportamentos das pessoas. Nem
tomam a sério sua alegria ou sua tristeza, pois garantem a protocolar
felicidade nos rituais sociais, como o Natal.

No final da década de 1960, a ditadura no governo do Brasil
intensificava os mecanismos opressivos, como censura e tortura. Isso
dificultou o acolhimento do sentido político que tinha a crítica moral
da Tropicália, na medida em que ela tirava do governo a exclusividade de
alvo a ser abatido. Não eram só a tirania do Estado pela violência ou a
exploração burguesa pela economia que entravam em jogo, mas também a
repressão geral da sociedade pelo conservadorismo --- coisas diferentes
porém com vínculo entre si. No entanto, no sufoco da ditadura, vários
artistas taxaram a Tropicália de alienada. Era como se o
conservadorismo, que se distribui entre as variadas classes econômicas
sem distinção e se instala mais no cotidiano das pessoas do que em
medidas governamentais, fosse uma questão de menor relevância, ao menos
diante do contexto dos anos 1960 no Brasil. Logo, o tratamento de temas
como sexualidade e raça, conforme diria Caetano, ficariam secundários.
Na sua canção de 1967, eles ``têm medo da maçã'', e não da guerrilha
urbana ou do povo revolucionário. Ou seja, têm medo do pecado original,
do sexo e da nudez, em resumo, da desrepressão moral da sociedade. Logo,
``eles'' não são só os militares e os burgueses, mas todos os
conservadores. Essa ideia suscitava polêmica, pois afrontava tanto
governantes políticos e privilegiados econômicos quanto seus detratores,
uma vez que existiam conservadores nos dois lados.

Exemplar da polêmica gerada pelo tropicalismo ao desfazer dualismos foi
a reação do diretor de teatro Augusto Boal, que defendera o maniqueísmo.
Para ele, interpretações fora da estrutura dicotômica seriam suspeitas.
``Os repetidos ataques ao maniqueísmo partem sempre de visões
direitistas que desejam'', Boal reprovava, ``instituir a possibilidade
de uma terceira posição, da neutralidade, da isenção, da equidistância,
ou de qualquer outro conceito mistificador'', e concluía que afinal
``sabemos que existe o bem e o mal, a revolução e a reação, a esquerda e
a direita, os explorados e os exploradores''. Note"-se que o emprego
gramatical da primeira pessoa do plural na frase, nós, define"-se
exatamente pela qualidade que, na canção de Caetano, definia a terceira
pessoa do plural, ``eles'': são os que têm certeza do bem e do mal. Para
Boal, há apenas duas posições no mundo, que viram uma moral, pois aí há
certo e errado: o bem e o mal universais, absolutos. Reduz"-se qualquer
outra possibilidade à má"-fé, pois ``na verdade'', se falássemos
sinceramente, admitiríamos: ou se está de um lado ou do outro. Não há
liberdade para uma nova alternativa, só se aceita a posição igual ou
oposta à sua. O resto é um mero jogo de cena falso para camuflar
reacionarismo ou abstencionismo.

Essa certeza do bem e do mal é precisamente, portanto, o que a
Tropicália criticava, inclusive o Teatro Oficina, do diretor Zé Celso
Martinez Corrêa, ao qual se opunha Boal. Na música ``Eles'',
encontramos uma espécie de recenseamento de todos os princípios dessa
moralidade, tratados com a devida ironia na letra e na forma de cantá"-la
por Caetano. Temos ali um machismo monogâmico para o qual ``só há um
galo em cada galinheiro'', ou seja, em cada família ou comunidade pode
haver muitas mulheres, mas apenas um homem que manda. Temos ainda a
ética regular do trabalho para a qual ``mais vale aquele que acorda
cedo'' e que mesmo assim esconde, muitas vezes, que em ``farinha pouca
meu pirão primeiro''. Pois há certo cinismo nessa moralidade
superficial, desde escamotear alegria ou tristeza para que prevaleça a
pacífica felicidade natalina até o estímulo ao trabalho como sacrifício,
que obriga a acordar cedo mas não hesita em competir com quaisquer
artifícios disponíveis para vencer quem está ao lado. Em todos os casos,
o que se exclui é, sempre, o princípio do prazer. Ele é o grande
reprimido. ``E não há amor como o primeiro amor'', pois é ele ``que é
puro e verdadeiro''. O ideal religioso da pureza rege a experiência
amorosa, e não a prática do prazer. No final da década, a Tropicália ---
como movimentos culturais que ocorriam dos Estados Unidos até a França
--- contrapunha"-se à repressão moral do prazer no seio da civilização.

O prazer é o tempo do hoje, é o tempo do agora. Diferente do que
acontece com o dinheiro, que ``eles guardam'', o prazer não pode ser
contabilizado. Não há como acumulá"-lo, como se faz com o capital. O
prazer é o tempo do instante sem par e sem comparação. Ele é o presente.
Caetano canta que ``eles sabem o dia de amanhã'', ``eles sempre falam no
dia de amanhã'' e ``eles amam os filhos no dia de amanhã''. Eles estão
sempre postergando, acumulando, contabilizando. Tudo fica para amanhã.
Nada é hoje, o tempo de quando somos presentes, vivos. Reprimir o prazer
é reprimir o tempo. Reprimir o prazer é subordinar o presente ao futuro
pré"-determinado. ``Eles desde já querem ter guardado, todo o seu passado
no dia de amanhã'', canta Caetano. Há uma ansiedade por fechar e
concluir o tempo que já quer guardar o passado no futuro, sem nem
experimentar a sua passagem. Ou seja, a repressão moral do prazer era,
por extensão, uma repressão temporal do presente. Se ``eles sabem o dia
de amanhã'', contudo, ``eu quero seguir vivendo''. É o que canta Caetano
em ``Alegria, alegria'', música do mesmo disco feito em 1967 e
lançado já em 1968. O presente é esse gerúndio: vivendo. Não é o ponto
temporal que exclui o passado e o futuro. É este vivendo de passado,
presente, futuro.

O tropicalismo, assim, afastava"-se das canções de protesto daquela
época, pois elas eram firmes na direção do futuro, do amanhã. É possível
comparar um e outras por suas diferentes imagens da maneira de andar
pela cidade. Em ``Pra não dizer que não falei de flores'', de Geraldo
Vandré, a caminhada tem direção definida de futuro. Há um objetivo
final, há um \emph{telos}, isto é, ela se alinha àquela teleologia que,
para alguns filósofos modernos, determinava o tempo histórico e
explicava as revoluções. Em ``Aroeira'', Vandré fala ``no dia que já
vem vindo, que esse mundo vai virar''. Estão aí futuro e revolução, um
bem atrelado à outra: o dia por vir e o mundo virado. Em ``Alegria,
alegria'', Caetano tematiza um passeio mais tranquilo. Era 1967. Lançara
mão, para abrir a composição, da mesma palavra que, um ano depois,
consagraria Vandré: ``caminhando''. O feitio, porém, é outro. O eu"-lírico
da canção, ``caminhando contra o vento, sem lenço e sem
documento'', é desprovido de identidade. Diz ``eu vou'', mas não
esclarece para onde ou por quê. O narrador se compraz em só deslindar o
que vê e faz no presente confuso das cidades: olha bancas de revista,
fotos e nomes, e toma uma Coca"-Cola. O fim algo apoteótico da música não
o é por prever uma virada do mundo e um futuro certo. O fim nem é, a
rigor, uma afirmação, mas uma pergunta --- por que não? Paulo Eduardo
Lopes, que analisou diferentes empregos da expressão ``caminhada'' na
música popular, percebeu que esta ```caminhada' do sujeito não leva a
nenhuma progressão''.

Para finalizar a comparação, cabe sublinhar que a caminhada de protesto
era coletiva e a tropicalista, individual. ``Somos todos iguais, braços
dados ou não, nas escolas, nas ruas, campos, construções'', escutamos em
``Pra não dizer que não falei de flores''. ``Eu vou'', ouvimos em
``Alegria, alegria''. Sou eu quem vou. Nisso, os tropicalistas
opunham"-se ao achatamento das diferenças individuais que, muitas vezes,
era solicitado na época em nome de um bem maior na política: fosse ele a
nação, a revolução ou o futuro. Para a sensibilidade da Tropicália, a
liberdade era valor a ser defendido mesmo diante daqueles que, em nome
dela, só aceitavam o programa de ação que lhes parecia adequado para
realizá"-la. Sua suspeita diante da esquerda brasileira originava"-se na
tolerância de parte dela com experiências socialistas, como a soviética
de Stálin ou até a cubana de Fidel, que perseguiram, censuraram e
mataram no presente em nome do futuro de igualdade. Conforme diria uma
canção de Caetano que surtiu o maior embate direto na trajetória dos
tropicalistas com o público dos festivais que assegurara originalmente
boa parte de seu sucesso: é proibido proibir. Não ao não. Que resulta em
sim.

Numa página arguta, Heloísa Buarque de Hollanda situou o tropicalismo
dentro da ``crise de uma perspectiva de futuro'' conjugada à ``noção de
revolução marxista"-leninista''. É que o projeto futuro acalentado
politicamente no fim da década de 1960 pela esquerda tinha feições
revolucionárias advindas da filosofia de Marx. Os tropicalistas
afastaram"-se dessa perspectiva. Mesmo que falassem da revolução de seu
movimento, o substantivo era empregado na acepção corrente de
transformação radical e geralmente aplicado só para a história da
cultura, ou seja, sem o compromisso com um programa de futuro para toda
a sociedade. No roteiro fictício da contracapa do disco \emph{Tropicália
ou Panis et circencis}, rechaçava"-se o lema ``o Brasil é o país do
futuro''. Caetano e Capinan diziam que esse gênero caíra de moda, ``no
Brasil e lá fora: nem ideologia nem futuro''. Na época, práticas
revolucionárias davam sinais de autoritarismo e burocratização, como na
União Soviética, mas a própria ideia de revolução perdia seu encanto
pela pretensão de uma ``colonização do futuro'' (para aproveitar a
expressão conhecida de Octavio Paz). Os tropicalistas rebelaram"-se
contra toda submissão do presente ao futuro, o que os distanciou do
ideal revolucionário à direita e à esquerda.

Essa rebeldia era, do ponto de vista do marxismo, condenável. Tome"-se o
exemplo de Leandro Konder, uma figura de proa da intelectualidade de
esquerda no Brasil, que acusava a ``inocuidade'' da atitude de
``rebeldia'' que podia ser vista, por exemplo, no filme \emph{Terra em
transe}, de Glauber Rocha. Para ele, a rebeldia, por si mesma, ``não
basta para que a conduta humana se torne plenamente libertária: é
preciso que o inconformismo se exerça na direção certa''. Escrevendo no
calor da hora, exigia portanto justamente aquilo que, conscientemente,
os tropicalistas evitavam: conduzir a crítica iconoclasta para um fim
certo e uma direção fechada de futuro. O movimento dos tropicalistas
estava espantando, por assim dizer, um ``fantasma da revolução
brasileira'' à sua volta, como o denominou o pesquisador Marcelo
Ridenti. Empregando seus termos, ``o fantasma que a esquerda brasileira
tem de superar é o das revoluções projetadas; derrotadas, no entanto,
pela força da contrarrevolução''. Referia"-se ao empuxo vitorioso do
golpe de 1964 que pôs fim aos projetos revolucionários anteriores. Nos
anos 1960, o tropicalismo fez o luto pela morte desses projetos e estava
desassombrado em relação ao fantasma. Isso tornou árduo seu entendimento
na época. O próprio Ridenti notou a mistura de sentido revolucionário na
forma estética e de sentido anárquico na mensagem política, o que para
muita gente, contudo, soava apenas reacionário.

Não se tratava, para os tropicalistas, de tomar o poder por uma
revolução, seguindo a tradição intelectual marxista quanto à práxis
política. O próprio Marx, ao discutir \emph{A ideologia alemã}, escreveu
que ``somente com uma revolução a classe que derruba detém o poder de
desembaraçar"-se de toda a antiga imundice''. Na frase, fica claro que
existe uma classe, o proletariado, que toma o poder da outra, a
burguesia. Nas canções tropicalistas, porém, a crítica dirige"-se menos
ao poder dos donos dos meios de produção sobre aqueles que vendem a
força de trabalho e mais às ``pessoas da sala de jantar'', como dizia
\emph{Panis et circenses}, ou seja, a um conjunto conservador de valores
morais que perpassava todas as classes sociais. Embora pudessem parecer
frívolas ou secundárias para parte da intelectualidade marxista, as
preocupações que os tropicalistas queriam colocar em pauta diziam
respeito não só à pobreza e à exploração, mas à homossexualidade, ao
machismo, ao racismo, ao divórcio, à pílula anticoncepcional, à
liberdade estética. Em suma, tratava"-se de relativizar a dicotomia moral
entre bem e mal, o certo e o errado --- proveniente de valores
tradicionais que normatizavam genericamente a vida das pessoas,
independentemente de seus desejos particulares.

Isso é claro no disco"-manifesto \emph{Tropicália ou Panis et circencis},
já de 1968, mas espalha"-se por muitas obras tropicalistas. ``Eles'',
de 1967, antecipa o esquema de \emph{Panis et circenses} (a canção foi
grifada de forma distinta do título do disco no qual está incluída).
``Eles'' estão ``em volta da mesa, longe do quintal'': são as pessoas da
sala de jantar, encarnam o conservadorismo moral repressor ---
especialmente para jovens rebeldes. Os tropicalistas atacavam essa
certeza moral tradicional da direita e da esquerda, de ricos e pobres
(todos igualmente concentrados ``no dia de amanhã'', como repete a
canção ``Eles'', sugerindo a crítica à prioridade do futuro sobre o
presente). O bem e o mal serviam como categorias universais para julgar
as condutas dos outros, desrespeitando a individualidade irredutível de
cada um no cotidiano da vida social. Variações dessas metáforas das
pessoas em volta da mesa e na sala de jantar constam ainda na canção
``Deus nos salve esta casa santa'' (gravada por Nara Leão, composta
por Caetano com Torquato) ou na engraçada ``Namorinho de portão'' (de
Tom Zé) que fala de um ``bom rapaz, direitinho''. Nesses casos todos, o
poder questionado é menos o do grande capital econômico e mais o das
pequenas moralidades familiares. Havia um outro conflito tomando forma
nesses anos: em vez da luta de classes, que evidentemente não sumira,
passava a ganhar relevo a luta entre as gerações, nos costumes
cotidianos.

No \versal{III} Festival Internacional da Canção da Globo, a acusação que Caetano
faz no discurso de ``É proibido proibir'', quando enfrenta o público
de jovens que o vaiava, era que, embora quisessem tomar o poder,
provavam"-se tão intolerantes em arte quanto os integrantes do Comando de
Caça aos Comunistas em política, que pouco antes tinham espancado atores
da peça de teatro \emph{Roda viva} (de Chico Buarque e dirigida por Zé
Celso Martinez Corrêa, em São Paulo). Lembre"-se que os jovens no
festival, além dos ataques verbais, atiraram de fato objetos no palco.
Caetano aproveitou e provocou: se vocês forem em política como são em
estética, estamos feitos. Essa crítica tropicalista explicitava o
dogmatismo espraiado na cultura brasileira, e não só concentrado no
governo ditatorial (cujo autoritarismo agradava boa parte da sociedade
que zelava pela ordem). Ele penetrava desde a arte até a família, desde
as instituições até a sexualidade, e assim em diante. Não foi apenas por
causa dos laços do governo militar com o grande empresariado do Brasil
que a nossa ditadura nacional teve um quinhão civil, como tem insistido
o historiador Daniel Aarão Reis, mas também porque ela era apoiada por %leitura equivocada de Aarão
grande parte da população, em nome de valores morais tradicionais e da
manutenção de costumes de bem. Lembre"-se que em 1964 houve a Marcha da
Família com Deus pela Liberdade, mobilizada contra uma suposta ameaça
comunista.

Na década de 1990, a minissérie \emph{Anos rebeldes}, de Gilberto Braga,
expôs o desencontro das afinidades políticas e geracionais da época dos
tropicalistas. No enredo, os protagonistas João Alfredo e Maria Lucia,
vividos ali por Cássio Gabus Mendes e Malu Mader, apaixonam"-se. Ele se
engaja politicamente na luta contra a ditadura, adere à guerrilha urbana
e acabará exilado. Só que antes achara grande afinidade de ideais com o
seu sogro, pois ambos são de esquerda e de orientação socialista. O
contato simpático nos assuntos coletivos da política, entretanto, será
quebrado adiante pela discordância nos problemas privados da moral,
marcados pela distância geracional. Quando o pai descobre na bolsa da
filha uma cartela de pílulas anticoncepcionais, sua reação imediata é de
desaprovação. Como se sabe, o surgimento das pílulas foi saudado porque
contribuía para uma independência e liberação sexual das mulheres, que
poderiam, ao transar, evitar a gravidez sem auxílio do homem. Essa
passagem da série de televisão da Rede Globo explicitava que a crítica à
sociedade brasileira na década de 1960 nem sempre era a mesma,
dependendo se o que estava em jogo era política ou moral, governo ou
costumes, velhos ou jovens, Estado ou cultura, marxismo ou sexualidade.

Nesse ponto, os tropicalistas pareciam muito afinados com os estudos do
filósofo alemão Herbert Marcuse. Em 1955, ele publicara um livro de
relevância extrema para os jovens daquele momento: \emph{Eros e
civilização}, no qual conjugava o marxismo à psicanálise. Referia a
crítica ao capitalismo não somente à economia, mas também ao excesso de
repressão ao prazer individual (eros) exigido para a constituição social
coletiva (civilização). Marcuse denunciava, ali, a castração do prazer,
da fantasia, da arte e da sexualidade como o problema central da vida da
época, tendo em vista escritos tardios de Sigmund Freud. Identificava a
ausência de liberdade como fruto do progresso. Parafraseando o conceito
de mais"-valia de Marx, falava de uma mais"-repressão. Sua tese era que,
embora certa repressão de pulsões individuais fosse incontornável para a
vida em conjunto, na sociedade atual teríamos uma repressão maior do que
essa, fundando uma moralidade tão cheia de proibições que tornaria
difícil a felicidade. O tropicalismo --- nas canções e nos gestos, nas
músicas e nas atitudes --- buscava um alargamento do campo da
experimentação individual da imaginação para além desses limites
impostos pela repressão excessiva identificada por Marcuse. Ela não
estaria presente somente nos aparelhos de Estado, mas também em
instituições como a família patriarcal monogâmica, ou seja, nas
``pessoas da sala de jantar'' e no ``bom rapaz direitinho'', cantados
pelos tropicalistas e espalhados pela sociedade brasileira.

Na mesma época do tropicalismo, outro filósofo expandia também, numa
espantosa sintonia, o conceito de poder frente à tradição marxista. Seu
nome era Michel Foucault. Ele defendia que a exploração do trabalho e o
capital eram uma relação de poder importante, mas não a única. Dizia
claramente que as teorias do Estado tampouco esgotavam as relações
sociais de poder. Questionava, assim, o protagonismo teórico e prático
da luta de classes entre burgueses e proletários, chamando atenção para
outros conflitos e outros personagens: ``as mulheres, os prisioneiros,
os soldados, os doentes nos hospitais, os homossexuais'', observaria
Foucault já em 1972, ``iniciaram uma luta específica contra a forma
particular de poder, de coerção, de controle que se exerce sobre eles''.
É que o poder estaria em toda parte: disseminado, microfísico,
espalhado. O poder é descentralizado e o Estado, mesmo em uma ditadura,
jamais o esgota. Ele é exercido, por exemplo, nas relações familiares,
raciais ou de gênero, justamente aqueles assuntos que os tropicalistas,
segundo Caetano, quase nunca viam discutidos mesmo no espectro político
de esquerda do Brasil que era filiado ao marxismo.

Nesse contexto, a emergência do tropicalismo em 1967 foi uma novidade na
cultura do Brasil. Simultaneamente, fazia o triste luto pela derrota dos
ideais revolucionários e comemorava o alegre jogo de outras
possibilidades abertas. O crítico Idelber Avelar descreveu a primeira
parte desse processo, relativa ao luto, como ``alegorias da derrota''
(embora não tivesse em vista o tropicalismo nem os intérpretes que o
classificaram como alegórico, mas a literatura latino"-americana
pós"-ditatorial). Identificou que aquele momento foi o traumático fim da
época heroica das artes modernas na qual a produção de símbolos
estéticos justificava a atuação de ``figuras literárias fundacionais que
apresentavam sua escrita como momento inaugural em que contradições de
natureza social, política e econômica podiam ser finalmente
resolvidas''. No final da década de 1960, perdia"-se assim o referente
totalizante da simbolização do país e de seu futuro. ``Sei também que a
arte não salva nada nem ninguém'', dizia Caetano em 1966, ``mas que é
uma de nossas faces''. (Ele mesmo, porém, seria uma figura ambivalente
nesse aspecto, pois a força poética de sua obra e o carisma de sua
personalidade deixaram nele se projetar a expectativa da solução de
contradições que, embora estivessem só sendo apresentadas, pareciam ali
poderem se redimir em um artista).

O que chama a atenção no tropicalismo é que esse luto pelo fracasso dos
projetos fechados de futuro foi experimentado, ao mesmo tempo, como um
gozo alegre por tudo que se abria a partir daí. Pois aquele engajamento
que concedia à arte o papel de porta"-voz da massa ou de anunciadora do
futuro prendia as suas transformações num processo teleológico
previsível e determinado, mesmo que às vezes destinado ao comunismo.
Impunha uma conclusão futura para a crítica presente. Livre desse
esquema, surgia o espaço poético de jogo e de liberdade da alegoria. Daí
a alegria tropicalista. Isso se explica pois tal ``linguagem alegórica
extrai sua profusão de duas fontes que se juntam num mesmo rio de
imagens'', o que foi apontado muito precisamente por Jeanne Marie
Gagnebin: ``da tristeza, do luto provocado pela ausência de um referente
último; da liberdade lúdica do jogo que tal ausência acarreta para quem
ousa inventar novas leis transitórias e novos sentidos efêmeros''. Os
urubus e os girassóis, da canção ``Tropicália'', são o luto e o jogo:
morte e vida, tristeza e alegria. Era 1967. Meio século depois, o Brasil
não parece ter escapado dessa terrível ambiguidade que o condena e o
salva. Entre a tristeza e a alegria, caminhamos.

\chapter*{A \versal{TV} brasileira e a \emph{barbárie tecnizada}}
\addcontentsline{toc}{chapter}{A \versal{TV} brasileira e a \emph{barbárie tecnizada}, \emph{por Regina Mota}}

\begin{flushright}
\emph{Regina Mota} %\footnote{Pesquisadora independente.}
\end{flushright}

\epigraph{``Estou aqui para confundir, eu não vim para explicar.''}{\emph{Dom} Abelardo Chacrinha}

Eu nasci com a televisão brasileira, e por isso ela é algo tão familiar
quanto um liquidificador, ou outro eletrodoméstico qualquer. Mas o seu
início no Brasil é épico. Precisou de um jagunço nordestino ir à América do Norte
e trazê"-la para o país no lombo do seu burro. Se algo existiu no Brasil
quase que simultaneamente ao resto do mundo foi a \versal{TV}. Não havia qualquer
infraestrutura nem regulamentação e, com o mesmo espírito desbravador,
Chateaubriand\footnote{Chateaubriand começou a pensar em fazer televisão
  já em 1947. Foi aos Estados Unidos procurar David Sarnoff, fundador da
  \versal{RCA} (Radio Corporation of America), e queria comprar a \versal{TV} colorida,
  que estava em experimentação. Assim, implantou a primeira televisão da
  América Latina. Walter Clark \& Fernando Barbosa Lima. \emph{\versal{TV} ao vivo. Depoimentos}. São Paulo: Brasiliense, 1988.} colocou ``a coisa'' no ar, em
setembro de 1950. Comprou 200 aparelhos, espalhou pela cidade de São
Paulo e da sua antena transmitiu os primeiros sinais. Obviamente que, da
noite para o dia, as pessoas não teriam como assistir aos poucos
programas e canais existentes, mas logo foi providenciada a produção de
televisores. Mesmo que apenas uma minoria conseguisse comprá"-los, podia
compartilhá"-los com vizinhos e amigos. Eu mesma vi televisão em torno
dos 6 anos de idade, ou seja, em 1959, na casa do meu vizinho de cima.

Aquele Brasil \emph{bossa nova} tinha todas as condições criativas para
explorar o novo meio, e foi isso mesmo que aconteceu. A televisão surgiu
na cultura quando outros meios como cinema, teatro, artes plásticas,
música etc. já tinham passado por várias vanguardas e viradas. Portanto,
estavam muito avançados nos anos 1950 e 1960. Desse ponto de vista, a
televisão ficava a alguns passos atrás, porque, apesar de tudo que a
técnica ensejava, seu domínio era primitivo com respeito aos gêneros,
linguagem e mesmo à comunicação. A busca por uma associação aos meios
existentes pareceu o caminho mais evidente, sobretudo para as televisões
públicas europeias, que optaram por levar o drama ao vídeo. Os \versal{EUA}, que
já tinham uma indústria de entretenimento desenvolvida, viu na televisão
a possibilidade de seu incremento.

A emergência de câmeras cinematográficas 16mm, leves e portáteis, e os
gravadores suíços \emph{Nagra} permitiram que a agilidade das filmagens
em externas praticamente inventassem uma nova linguagem, tanto para o
cinema como para a televisão. Isso se deu com os chamados \emph{Cinema
Direto} (Grupo Câmera Viva) e \emph{Cinèma} \emph{Verité} (França), cuja
maioria dos produtores e cinegrafistas eram egressos da produção
jornalística televisual. Desde sempre, a transmissão ao vivo de eventos
e acontecimentos era o que melhor cabia na tela da televisão, por
coincidir com o seu próprio processo de produção aberto e em andamento,
mostrando o desenrolar da vida.\footnote{Regina Mota. \emph{A épica eletrônica de Glauber. Um estudo sobre cinema e televisão}. Belo Horizonte: \textsc{ufmg}, 2001.}

A nossa sorte é que a \emph{barbárie tecnizada} no Brasil estava no seu
auge, em todos os campos. Cito aqui Oswald de Andrade, para quem a
reabilitação do primitivo permitia ultrapassar a visão que opõe e traça
um percurso evolutivo do bárbaro (homem natural) ao civilizado. Isso
seria possível com a emergência de um novo termo, o ``bárbaro
tecnizado'', resultado inovador da transubstanciação do segundo no
primeiro, pelo rito antropofágico, em uma operação criativa que alterna
e ignora o impasse do atraso. Oswald constrói, na sua teoria, uma
perspectiva anticêntrica, antiexclusivista, projetando a revalorização
do homem natural, que se produz contra os quadros esclerosados do homem
histórico, do homem civilizado, do homem vestido, enfim, do homem
cartesiano.\footnote{Maria Cândida Ferreira Almeida. ``Só a antropofagia nos une''. In: \emph{Estudios y otras practicas intelectuales
latinoamericanas en cultura y poder}. Venezuela: \textsc{clacso}, Consejo
Latinoamericano de Ciencias Sociales, 2002, p.~9.} Maria Cândida chama atenção para o aspecto
anti"-hierárquico prefigurado na operação metafísica, que liga o rito
antropofágico à transformação do tabu em totem, do valor oposto ao
favorável, da vida como devoração pura. Segundo ela,

\begin{quote}
\ldots{} a antropofagia, enquanto conceito, apresenta uma face
produtiva, diversa da pura destruição com que costuma aparecer no
discurso ``civilizado'' sobre a ``barbárie'', que utiliza o ato canibal
como signo da violência máxima. Sob a perspectiva oswaldiana e selvagem,
a antropofagia preconiza uma espécie de transubstanciação na qual aquele
que é o devorador se altera no devorado: ``trata"-se apenas da
transformação do tabu em totem, isto é, do limite da negação em elemento
favorável''. A ``morte'' e ``devoração'' do outro recria o próprio;
dentro dessa perspectiva, o discurso ressentido das relações coloniais
torna"-se discurso produtivo de identidades.\footnote{Ibidem, p.~4.}
\end{quote}

O que é, a meu ver, particular nos anos 1960 é a grande interação entre
os artistas, organizados ou não em movimentos, como era comum nessa
época. Eu costumo dizer que o nosso modernismo começa no final do século
\versal{XIX} e vai até o final dos anos 1960, em um contínuo, apesar de várias
rupturas. Isso porque o centro das questões já enunciadas por Machado de
Assis e narradas por Euclides da Cunha ganham autoconsciência com Mário
e Oswald, e dizem respeito à descoberta do país. Da sua diversidade, dos
sotaques, das manifestações culturais, dos ritmos, dos vários tipos de
conflitos, de beleza e riqueza. Os modernistas, com seus manifestos
bem"-humorados, nos entregaram uma nova visão que ainda não existia no
Brasil e que foi radicalizada pelos artistas da Tropicália. A
tropicalidade sempre esteve no horizonte dos escritores, pintores,
músicos, dramaturgos, cineastas e pensadores dos anos 1920, que
plantaram o germe do nosso autorreconhecimento.

A televisão vai absorver como uma esponja esses acontecimentos culturais
e a disponibilidade de mão de obra jovem de alto nível intelectual e
artístico para a sua invenção. Influência do Cinema Novo, da música
popular, da literatura, do teatro e das artes plásticas e gráficas.
Enfim, das diversas audiovisualidades como de um filme como \emph{Terra
em transe}, ou da peça de Oswald de Andrade, \emph{O rei da vela},
encenada por José Celso Martinez em 1967, pela primeira vez.
Contribuíram ainda o traço dominante da oralidade presente na cultura
brasileira e o circo, já bastante aculturado e popular disseminador de
novidades pelo interior do país.

Por isso a televisão brasileira não se filiou a nenhum modelo específico
de televisão. Fruto ou não da precariedade, a televisão aqui se fez a si
mesma, com os elementos artísticos e culturais disponíveis, com um novo
repertório, que trazia como elemento estranho a transmissão ao vivo. Não
havia nem condições nem o desejo de traduzir programas americanos, por
isso a \versal{TV} brasileira surge com uma linguagem própria, revelando
talentos. Isso se deu na \versal{TV} Difusora, a primeira que depois se
transformou na \versal{TV} \versal{TUPI}. Segundo Walter Clark, ``foi uma televisão muito
importante, mas talvez muito avançada dentro da perspectiva que as
próprias condições técnicas permitiam''.\footnote{Walter Clark \& Fernando Barbosa Lima. ``Um pouco de história e de reflexão''. In: \emph{\textsc{tv} ao vivo. Depoimentos}, op.~cit., 1988.}

Imaginem o Nelson Rodrigues trabalhando na \versal{TV} Rio, escrevendo uma novela
improvisada, dirigida pelo Sérgio Brito e estrelada por Fernanda
Montenegro. E ainda Miéle e Bôscoli redigindo e dirigindo um musical e
Fernando Barbosa Lima criando novas formas de informar e fazer pensar.
Ali tudo era um aprendizado. Cada uma das emissoras que foram surgindo
improvisava uma equipe técnica vinda geralmente do rádio, além de
artistas, escritores ligados ao teatro, cinema, artes plásticas,
jornalismo e literatura para produzir o conteúdo. Esses pioneiros
entenderam na experimentação as características do meio, fizeram acordos
com anunciantes para bancar seus projetos e iam ao ar ao vivo.

Diferentemente dos outros países da Europa e dos Estados Unidos, a
televisão brasileira praticamente não utilizou a película como forma de
incrementar o conteúdo por falta de estrutura e custos. Por isso,
investiu no caráter direto das transmissões, o que exigia muita
criatividade. No lugar de cenários, o uso da luz e de grafismos sugeriam
profundidade em estúdios improvisados. Assim que surgiu o videotape, as
emissoras adotaram a tecnologia, mas não podiam se dar ao luxo de fazer
um acervo de material e por isso quase não temos imagens e sons das duas
primeiras décadas da televisão no Brasil, porque as fitas eram apagadas
para ser reutilizadas. Isso ocorreu até bem mais tarde, já na época do
formato U"-Matic, das câmeras portáteis da Sony, que representaram uma
revolução na produção de externas.

A televisão era uma novidade que estava sendo consumida por uma classe
de telespectadores imersos na cultura revolucionária dos anos 1950 e
1960 e, portanto, era preciso atender às expectativas, já que ela não
tinha ainda se ``popularizado''. Isso só vai ocorrer depois do golpe de
estado em 1964, quando os militares resolveram investir nas
telecomunicações, permitindo o desenvolvimento do modelo de redes de
televisão e a hegemonia do conteúdo produzido no Sudeste para o resto do
país, em cabeças de rede.

Mesmo que sejamos críticos ao universo televisual, somos obrigados a
admitir que temos, se não a melhor, uma das melhores televisões do
mundo, e isso se deveu a esse grande laboratório que foram as emissoras
de televisão aberta da década de 1950 e 1960 e seus inventores. O
processo aberto em andamento da televisão, como característica própria,
somou"-se à liberdade de criação e experimentação junto à diversidade das
emissoras, ao contrário do que se estabeleceu no modelo concentrador,
inaugurado pela Rede Globo.

Apenas para se ter uma ideia, na década de 1950, quando iniciaram as
emissões, até o final da década de 1960, tínhamos a \versal{TV} Tupi, a pioneira
de Chateaubriand, que ficou no ar até 1980, quando pegou fogo. A segunda
foi a \versal{TV} Paulista, de 1951 a 1965, quando foi vendida à Rede Globo. A \versal{TV}
Excelsior, do Banco Simonsen, ficou no ar de 1959 a 1970, quando foi
retirada pelos militares. A \versal{TV} Record, que não tinha nada a ver com os
bispos, iniciou suas transmissões em 1953 e está no ar até hoje, e foi
palco dos grandes festivais de música que uniram corações e mentes de
todo o país. Mas a grande criadora, o laboratório de invenção da
televisão brasileira foi a \versal{TV} Rio, por onde passaram os melhores
escritores, atores e diretores durante os anos de 1955 a 1975. A Globo,
Bandeirantes, \versal{TVE} e \versal{TV} Cultura de São Paulo também foram criadas nos
anos 1960, já no período da ditadura, e praticamente a seu serviço.

Os militares investiram pesado em redes tecnológicas e, com a
inauguração, em 1968, da \versal{TELSTAR} --- Rede de Enlaces de micro"-ondas ---,
são criados 18.000 quilômetros de enlaces, permitindo as redes nacionais
de televisão. O \emph{Jornal Nacional} da \versal{TV} Globo estreia como primeiro
programa em rede nacional. Em 1969, é inaugurada a estação ligada ao
satélite, que permite transmissões internacionais, e o Brasil vê ao vivo
a chegada do homem à lua.

Gostaria de lembrar também alguns nomes que compuseram as faces da
televisão brasileira: os produtores, criadores e inventores da \versal{TV} e os
pensadores, que construíram uma crítica e uma referência para o seu
estudo. Entre eles, estão alguns pioneiros como Walter Clark, Fernando
Barbosa Lima e Carlos Alberto Lofler, Walter Avancini, Dias Gomes e
Walter Durst, Abelardo Barbosa. Entre os críticos, estão Arlindo
Machado, Gabriel Priolli, Décio Pignatari, Arthur da Távola e Washington
Novaes.

A questão para esses bárbaros tecnizados era o próprio, o Brasil, e
ninguém representou isso tão bem quanto Abelardo Barbosa, como descreve
Décio Pignatari:

\begin{quote}
Abelardo ``Chacrinha'' Barbosa é o nosso primeiro grande palhaço
televisual. Ele não é o palhaço de circo na televisão, não: ele é o
palhaço da televisão, aquele que soube somar o rádio, a praça pública, a
multidão, o circo e o teatro de variedades para obter um espetáculo
televisual único em todo o mundo. No espaço circense do Chacrinha, gente
e cenografia se confundem, nunca se separam. Daí a impressão de festa
contínua que transmite, daí o calor humano que irradia. Nesse espaço, a
câmera contínua exercia uma função das mais importantes. Ela e o
Chacrinha pareciam estar sempre brincando, fazendo micagens um ao outro,
fugindo ou se procurando, tentando pregar peças um ao outro --- e
envolvendo o telespectador, que passava a fazer o papel de
palhaço"-parceiro do Velho Guerreiro, embolado na multidão formada pelo
auditório, os membros do júri, as chacretes e os candidatos a qualquer
coisa que fosse ``a maior da América do Sul''\ldots{} Tudo parecia girar num
turbilhão maluco, grotesco, popular, cômico, rabelaisiano --- uma
gigantesca gargalhada televisual.\footnote{Décio Pignatari. \emph{Signagem da televisão}. São Paulo: Brasiliense, 1988, p.~13.}
\end{quote}

Também no horizonte das outras artes estava a televisão como produtora
de uma nova visão sobre o mundo, ou se fazendo, ela mesma, o próprio
mundo. Isso se evidencia no movimento neoantropofágico da Tropicália.
Nas obras de Hélio Oiticica, Glauber Rocha e Caetano Veloso, que
demarcaram o território, ressurge o brilho da luz neon, a corporalidade
excessiva e o barroquismo próprio das emissões, extraído para compor
obras de arte originais, fechando o ciclo da criação autônoma da arte
brasileira, que sabia do seu rumo desde o início do século \versal{XX}.

No final dos anos 1970, no ocaso do período de 15 anos de exceção,
Glauber Rocha recria na televisão, sob a batuta de Fernando Barbosa
Lima, o ensaio democrático das aberturas políticas. O programa
\emph{Abertura}, da Rede Tupi de Televisão, foi ao ar de fevereiro de
1979 a julho de 1980 e contou com a participação ruidosa do diretor em
aproximadamente 30 de suas emissões.\footnote{Regina Mota. \emph{A épica eletrônica de Glauber. Um estudo sobre cinema e televisão}, op.~cit.} Livre de
intermediação, de roteiros, falando diretamente para o público, Glauber
Rocha construiu, na tela da televisão brasileira, uma épica eletrônica,
ao esgarçar a possibilidade estética do meio, didatizando
brechtianamente a sua potência política e noética. Como em seus filmes,
criou uma tipologia de personagens reais que representavam a grande
maioria dos brasileiros urbanos, como o retirante Severino e Brizola, o
favelado, ambos atuando como seus intercessores, revelando limites e
potencialidades de se romper as distâncias sociais: ``Já pensaram o
Severino senador?''

Sempre dificultando a ocorrência de uma única interpretação, Glauber
ficcionalizava, inserindo elementos do real, montando o quadro ao vivo,
se batendo contra a câmera ou com a luz. Exercício raro na tão
controlada mídia eletrônica. O \emph{Abertura} foi seu último testamento
modernista tropicalista antropófago, no qual devora o meio pela operação
da diferença, do conflito e da alteridade. Do outro lado, o
telespectador reconhecia a opacidade da \versal{TV} e participava do transe e da
possibilidade de mover sua consciência.

Apesar de não ter feito escola, a intervenção de Glauber Rocha na
televisão, além de didatizar o fim da censura prévia, serviu de exemplo
e inspiração para a nova geração de videorealizadores que surgiu na
década de 1980, com seus bem"-humorados quadros da sempre surpreendente
realidade brasileira, como o \emph{Olhar Eletrônico}, \emph{\versal{TVT}udo},
\emph{Intervídeo}, entre outros, que ocuparam com alguma independência
espaços nas televisões Gazeta, Bandeirantes e Manchete.

\pagebreak

\section{Referências}

\begin{Parskip}
\versal{ALMEIDA}, Maria Cândida Ferreira. ``Só a antropofagia nos une''. In: \versal{MATO},
Daniel (org.). \emph{Estudios y otras practicas intelectuales
latinoamericanas en cultura y poder}. Venezuela: \versal{CLACSO}, Consejo Latinoamericano de
Ciencias Sociales, 2002.

\versal{ALMEIDA}, Maria Cândida Ferreira. (2002) ```Só me interessa o que não é meu':
a antropofagia de Oswald''. Disponível em: \textless{}\emph{https://bit.ly/32nfeDz}\textgreater{}.

\versal{ALMEIDA}, Cândido; \versal{MACEDO}, Cláudia; \versal{FALCÃO}, Ângela (org.). \emph{\versal{TV} ao vivo. Depoimentos}. São Paulo: Brasiliense, 1988.

\versal{BARBOSA LIMA}, Fernando; \versal{MACHADO}, Arlindo; \versal{PRIOLLI}, Gabriel.
\emph{Televisão e vídeo}. Rio de Janeiro: Zahar, 1985.

\versal{CLARK}, Walter; \versal{BARBOSA LIMA}, Fernando. ``Um pouco de história e de
reflexão''. In: \versal{ALMEIDA}, Cândido; \versal{MACEDO}, Cláudia; \versal{FALCÃO}, Ângela (org.). \emph{\versal{TV} ao vivo. Depoimentos}. São Paulo: Brasiliense, 1988.

\versal{MOTA}, Regina. \emph{A épica eletrônica de Glauber. Um estudo sobre
cinema e televisão}. Belo Horizonte: \versal{UFMG}, 2001.

\versal{PIGNATARI}, Décio. \emph{Signagem da televisão}. São Paulo: Brasiliense, 1988.
\end{Parskip}

\chapter*{Realismo, anti-realismo e meta-realismo em~Glauber~Rocha}
\addcontentsline{toc}{chapter}{Realismo, anti-realismo e meta-realismo em Glauber Rocha,\\ \emph{por Rodrigo Nunes}}

\begin{flushright}
\emph{Rodrigo Nunes}
\end{flushright}

Como se sabe, ``a realidade nacional'' é uma das grandes consignas da
cultura brasileira dos anos 1960, e um problema constante para as artes
desde o sonho eufórico da ``revolução brasileira'' do governo João
Goulart até sua repressão nos anos de ditadura. No entanto, se há uma
coisa que a permanência desta consigna \emph{como problema} ao longo do
período demonstra, é que ``a realidade'' não é algo necessariamente
evidente. Não é apenas que a realidade a que se referiam os Centros
Populares de Cultura e a Tropicália, para ficar nos dois exemplos mais
óbvios, eram bastante distintas. Se a realidade \emph{faz problema} para
as artes, é sobretudo porque a questão de \emph{como dar a ver} esta
realidade, e de como este dar a ver pertence ou se distingue desta mesma
realidade, e pode ou não funcionar como \emph{intervenção} nela, é ela
mesma objeto de disputa. Neste texto, pretendo explorar a elaboração
desenvolvida por Glauber Rocha em torno deste problema na segunda metade
dos anos 1960; e, mais exatamente, a maneira como se articulam, em seu
pensamento e em seu cinema deste momento, realismo, anti"-realismo e algo
que poderíamos chamar de \emph{meta"-realismo}.

O ponto de partida da elaboração glauberiana pode ser situado numa
matriz comum à arte engajada do período. Esta se estabelece na última
metade da década de 1950 a partir de uma ruptura que se dá simultaneamente
no cinema de Nelson Pereira dos Santos, antecessor imediato do Cinema
Novo, e no Teatro de Arena --- que, pela mão de figuras"-chave como
Oduvaldo Viana Filho e Chico de Assis, desembocará no início dos anos 1960
no Centro Popular de Cultura da \versal{UNE}. Os dois marcos aqui seriam, no
cinema, \emph{Rio 40 graus}, de 1955; e, no teatro, \emph{Eles não usam
black"-tie}, 1957. Podemos definir esta matriz comum, que em seguida se
desenvolverá em vias divergentes, como uma \emph{estética da desestetização}.

De imediato, desestetização significava, contra a estética transplantada
do Teatro Brasileiro de Comédia ou dos estúdios Vera Cruz, a ideia de
que a arte de um país subdesenvolvido não deve emular nem o conteúdo nem
a forma da arte produzida do mundo desenvolvido, mas incorporar sua
diferença --- seu caráter singular como experiência do subdesenvolvimento
--- tanto no nível da forma quanto no do conteúdo. A ideia, aliás, era
que podia haver entre os dois níveis um feedback positivo: à medida em
que a realidade brasileira se tornasse objeto, a arte brasileira se
descolonizaria, se libertando da necessidade de imitar a cara, o
imaginário, os estilos e \emph{production values} de fora. A melhor
linguagem, e a mais verdadeira, era, num certo sentido, aquela
\emph{para a qual o dinheiro dava}: não se tratava apenas de criar uma
nova linguagem para lidar com o conteúdo da realidade nacional, mas
deixar em certa medida que estas formas fossem ditadas por esta
realidade, isto é, pelas condições de produção e relação com o público
enfrentadas por artistas buscando fazer uma arte nova e independente num
país subdesenvolvido.

A centralidade desta problemática para Glauber se expressa não apenas
(como salta aos olhos) na prática de seu cinema, mas também em suas
reflexões sobre o ofício de cineasta:

\begin{quote}
Vi um filme cubano sobre a revolução (\ldots{}) feito em estúdio, com
iluminação e camponeses vestidos. Perguntei ao realizador: ``Como é que
o Senhor me faz um filme sobre a revolução campesina em estúdio,
encenado? No momento em que o Senhor coloca esses camponeses em estúdio
e em cena, no momento em que o Senhor começa a dramatizar essa realidade
segundo dados culturais, ficcionais, acadêmicos, o Senhor não está
fazendo um filme revolucionário, está tratando um tema de esquerda com
um ponto de vista de direita''.\footnote{Glauber Rocha. ``Cinema
  verdade''. In: \emph{Revolução do Cinema Novo}. São Paulo: Cosac Naify,
  2004, p.~75-76.}
\end{quote}

No \versal{CPC}, esta estética da estetização se desdobraria num caminho que
conduzia à desvalorização da forma em favor do conteúdo, da expressão em
favor da comunicação. Quando Vianinha e Chico de Assis abandonaram o
Teatro de Arena em favor do \versal{CPC}, foi por entender que a experiência do
primeiro chegara a um limite: ainda que agora se tivesse começado a
falar \emph{do} povo, seguia"-se fazendo"-o dentro de um teatro burguês,
para uma plateia burguesa, sem chegar aos maiores interessados em
conhecer aquele novo teatro e o que ele tinha para dizer. A partir daí,
então, tratar"-se"-ia de chegar àquela parte da população que os circuitos
normais da produção artística não alcançavam, diretamente na rua ou
através de sindicatos ou das Ligas Camponesas. Seja pela imposição da
necessidade de falar a um público alheio a preocupações formais, seja
pelo ritmo de premência que regia a produção, a pesquisa formal era não
apenas deixada de lado como vista com boa dose de desconfiança. A
expressão tinha de dar lugar à comunicação.

Este caminho, para Glauber, era claramente insuficiente. Ao desprezar o
problema da \emph{representação} da realidade --- ao minimizar o problema
da forma, e, portanto, tratar o conteúdo (``a realidade'') como algo
cuja representação ia de si ---, a linguagem cepecista acabava
incorporando acriticamente na sua representação elementos desta mesma
realidade, reproduzindo"-os. Em seu ``primarismo'' cheio de ``boa
consciência'', esta ``arte populista'' lançava mão das próprias ``formas
de alienação da cultura contemporânea''.\footnote{Glauber Rocha. ``O
  Cinema Novo e a aventura da criação''. In: \emph{Revolução do Cinema Novo}, op.~cit., p.~132.}

\begin{quote}
O artista populista sempre diz ``não sou intelectual estou com o povo, a
minha arte é bonita porque comunica etc\ldots{}.''. Mas comunica o quê?
Comunica em geral as próprias alienações do povo. (\ldots{}) O populismo vai
a estas fontes e as devolve ao povo sem nenhuma interpretação. O povo
recebendo na cara a comicidade epidérmica do subdesenvolvimento, acha
genial sua própria desgraça e morre de rir. Daí o sucesso da chanchada,
toda ela fundada sobre o pitoresco miserabilista do caboclo ou da classe
média.\footnote{Ibidem. É interessante observar que a falta de
  distanciamento crítico da realidade retratada, que Glauber associa
  àquilo que identifica como ``populismo'', fora também exatamente
  aquilo que a direção do \versal{CPC} da União Nacional dos Estudantes lamentara
  no filme coletivo \emph{Cinco vezes favela}, estopim da ruptura entre
  cepecistas e cinemanovistas. Ver, neste sentido, o ``Relatório do
  Centro Popular de Cultura'' incluído como apéndice em Jalusa
  Barcellos. \emph{\versal{CPC} da \versal{UNE}. Uma história de paixão e consciência}.
  Rio de Janeiro: Nova Fronteira, 1994, p.~451-453.}
\end{quote}

Tratava"-se, ao contrário, de lançar mão do cinema ``\emph{como método e
não como ilustração}'', ``\emph{como revelação e não como descrição do
óbvio}''.\footnote{Glauber Rocha. ``O padre e a moça''. In: \emph{Revolução Cinema Novo}, op.~cit., p.~79 (grifo no original).} Na contramão do revisionismo posterior que
tenta reduzir o diferendo entre Cinema Novo e \versal{CPC} a uma simples disputa
entre ``dirigismo pecebista'' e ``liberdade criativa'' --- entre
engajamento e arte ---, Glauber afirma em 1965:

\begin{quote}
Para mim, existe um método válido de análise que é o método marxista,
isto é, um método de abstração para uma análise histórica de um
fenômeno. Não creio que no momento exista um outro método de
desmistificação, senão a aplicação de um método dialético claro,
objetivo, desestetizante, e estético na medida em que ético.\footnote{Glauber
  Rocha. ``Cinema verdade'', op.~cit., p.~75.}
\end{quote}

A questão é, justamente, que este método, sendo ``de abstração'', exige
o contrário de um realismo ingênuo ou de um naturalismo; a realidade, ao
invés de ser aquilo que ``fala por si'', é o que precisa ser indagado,
questionado para além da superfície. Fora durante a filmagem de
\emph{Barravento} que Glauber tivera seu primeiro contato com obra de
Bertolt Brecht, e foi a partir deste encontro fundamental que ele
começou a desenvolver seu cinema ``épico"-didático'', a ponto de declarar
sem vacilo em 1975: ``meu modelo dramatúrgico permanece
Brecht''.\footnote{Glauber Rocha. ``\emph{Filmcrítica}''. In: \emph{Revolução do Cinema Novo}, op.~cit., p.~300.}
A épica"-didática é a síntese que ele oferece para a tensão entre
comunicação e pesquisa formal; é uma linguagem que ao mesmo tempo educa
e conscientiza, desvelando as raízes dos problemas sociais, e expressa o
impulso de sua transformação através de uma linguagem nova,
não"-alienante, e por si mesma revolucionária. Uma coisa não pode existir
sem a outra: o didático, sozinho, ``gera informação estéril e degenera
em consciência passiva das massas e boa consciência dos intelectuais'';
o épico, sozinho, ``gera romantismo moralista e degenera em demagogia
histérica''.\footnote{Glauber Rocha. ``Revolução cinematográfica''. In: \emph{Revolução do Cinema Novo}, op.~cit., p.~100.}

A referência ao marxismo como ``método de abstração'' parece"-me dever
ser interpretada à luz do duplo movimento, analítico e sintético, que
caracteriza a épica"-didática glauberiana. Seu primeiro momento,
analítico, consiste em depurar o real até encontrar suas articulações,
isolando as variáveis de um problema ou situação, estabelecendo os dados
de sua estrutura fundamental: ``um nível de abstração que é filosófico e
radical'',\footnote{Glauber Rocha. ``\emph{Filmcrítica}'', op.~cit., p.~301.} onde,
``mesmo se as estruturas são complexas'', ``se chega a conclusões
simples''.\footnote{Ibidem, p.~299.} Mas, em seguida, chega"-se ``à liberdade
de constituir os pensamentos, de vesti"-los e fazê"-los falar
historicamente''; este é o momento sintético, onde a ``estrutura
psico"-econômica de um certo tema''\footnote{Ibidem.} é transformada em
imagens que a expressam. Chega"-se à ``realidade'', portanto, não pela
fidelidade à sua superfície, mas primeiro abstraindo"-a e partindo"-a em
seus componentes, nas forças e estruturas que a constituem; e em
seguida, reconstituindo"-a em uma realidade ``suplementar'', onde aquelas
estruturas, forças e tendências aparecem de maneira mais condensada,
\emph{saturada}, como num ideograma. A quebra com o realismo
convencional conjura o choque de uma realidade exposta não em sua
superfície, mas nas suas ``condições de vida'',\footnote{Walter Benjamin.
  ``What is Epic Theatre?''. In: \emph{Illuminations}. London: Pimlico,
  1999, p.~147.} nas forças (políticas, econômicas, sociais, psíquicas)
que a compõem.

Exemplos deste procedimento abundam na obra"-prima de 1967, \emph{Terra
em transe}, a começar pela própria metáfora do transe que estrutura a
narrativa, estado extremo em que os indivíduos aparecem movidos por
atavismos e determinismos históricos que eles não controlam. Inclui"-se
aí também todo o jogo de analogias, metonímias e substituições que
percorre o filme: Diaz (Paulo Autran) ``coroado'' e ``pai da pátria'',
sublinhando a relação do personagem com a estrutura social e psíquica
legada pelo colonialismo; a bandeira negra do conservadorismo,
simbolizando sua pulsão de morte subterrânea; os cartazes em branco e as
bocas muda do populismo, bem como o choque deste diante do povo
``verdadeiro'', que acaba por se resolver em morte.\footnote{Talvez o
  caso mais extremo do momento sintético, aliás, seja exatamente a cena
  que se segue a esta morte, na qual Paulo Martins (Jardel Filho) é
  confrontado pelos estudantes interpretados por Francisco Milani e
  Paulo Cesar Pereio. Aqui, Glauber opta por reduzir todo o diálogo a
  duas únicas frases repetidas à exaustão --- ``o seu anarquismo!'',
  ``a sua irresponsabilidade política!'' ---, o que resume à perfeição
  todo e qualquer outro diálogo que poderia ter ocorrido naquele
  momento.}

Trata"-se de dar a ver um real mais profundo (mais abstrato, mais
sintético) que qualquer ``realidade'' superficial. De onde, justamente,
a necessidade de romper com as convenções narrativas psicologizantes do
drama realista burguês para substituí"-lo por uma épica onde a imbricação
entre a história e o indivíduo possa manifestar"-se de maneira direta. A
grandeza dos maiores trabalhos de Glauber --- e poucos deles são tão
felizes neste sentido quanto \emph{Terra em transe} --- reside na
capacidade de encontrar a imagem justa para dar conta, ao mesmo tempo,
do que há de mais essencial em uma situação e de todas as determinantes
complexas que nela confluem.

Ao mesmo tempo, seu método é uma complicação do de Brecht. Para ambos, o
objetivo do estranhamento era sempre tornar visível a oposição entre a
falsa aparência de naturalidade que cerca ``o real'' e seu caráter
profundo de fato histórico (e portanto, uma vez visto como tal,
transformável). Nas imagens justas do cinema de Glauber (``metáforas
lancinantes'', para usar a frase de Haroldo de Campos sobre Oswald de
Andrade), joga"-se, no entanto, com diferentes escalas de tempo e
diferentes graus de consciência que uma ``mera'' \emph{tomada de
consciência histórica}, no sentido marxiano, talvez não baste para
dominar. É aí que aparece, particularmente, a relação complexa de
Glauber com o místico, que pode tanto ser um efeito de superfície
ideológico a ser superado por uma visão materialista da história quanto
uma força material profunda que aparenta possuir sua eficácia causal
própria --- o elemento que complementa e complexifica o materialismo
histórico, mas não sem ao mesmo tempo problematiza"-lo e coloca"-lo em
dúvida. Daí, então, a dimensão ao mesmo tempo mítica e histórica, causal
e profética de personagens como o Santo, Antônio das Mortes e Corisco,
em \emph{Deus e o diabo na terra do sol}; o magma psíquico encarnado no
Diaz de \emph{Terra em transe}; ou o misticismo que se cancela ao mesmo
tempo em que se verifica, em \emph{Barravento} (onde o desenlace da
narrativa passa pela realização da profecia que se pretendia
desmistificar).

Abstraída no conteúdo, a diversidade concreta da realidade --- a ganga
bruta sensível sob a qual se esconde aquilo que cabe ao artista analisar
e sintetizar --- retorna num outro nível, que poderíamos chamar de
\emph{metacomunicativo}; de onde minha sugestão da expressão
\emph{meta"-realismo} para discutir este aspecto da poética glauberiana.

O termo ``metacomunicação'' foi introduzido por Gregory Bateson para
descrever um dos tipos de abstração disponíveis à comunicação verbal
humana para além da função meramente denotativa que empregamos numa
frase mundana como ``suas coisas estão aí''. Enquanto o nível
metalinguístico consiste de mensagens implícitas e explícitas cujo
objeto é o discurso (```coisas' se refere a objetos pertencentes a mais
de uma classe''), o metacomunicativo consiste de mensagens implícitas e
explícitas que se referem à relação entre os falantes (``meu propósito
em dizer que suas coisas estavam aí era amigável'').\footnote{Ver
  Gregory Bateson. ``A Theory of Play and Fantasy''. In: \emph{Steps to an
  Ecology of Mind}. Nova York: Ballantine, 1981.} Mensagens ou sinais
metacomunicativos definem, assim, o marco ou contexto (\emph{frame}, na
terminologia de Bateson) em que o conteúdo denotativo deve ser recebido.

Se compreendemos o termo desta maneira, está claro que uma dimensão
metacomunicativa já está presente tanto na épica"-didática brechtiana
quanto na estética da desestetização. No primeiro caso, a
metacomunicação ocorre no nível formal; trata"-se, obviamente, do famoso
\emph{Verfremdungseffekt} (``efeito de estranhamento'') brechtiano. A
ruptura com o naturalismo, com a narrativa psicologizante e com a
catarse comunica algo a respeito do próprio ato de comunicar: ``isto não
é a realidade, é uma representação; sua aparente `artificialidade' não é
um erro, mas uma escolha deliberada; ela está aqui para lembra"-lo de que
isto é uma representação''. Espera"-se do estranhamento assim produzido
que ele cave um espaço de reflexão não apenas sobre a realidade, mas
sobre o próprio ato de representá"-la e aquilo que naturalizamos a seu
respeito.

Já no segundo caso, a metacomunicação se realiza no nível das condições
materiais de produção, pensadas como devendo tornar"-se elas mesmas
expressivas. O subdesenvolvimento aparece aí como algo a ser revelado e
denunciado não apenas pela ação que se desenrola diante das câmeras, mas
também naquilo que ocorre por trás delas. Não é apenas que os temas da
obra de arte do Terceiro Mundo devam ser diferentes; as condições
materiais em que a arte é produzida no Terceiro Mundo são diferentes, e
a obra de arte não deve tentar apagar esta diferença material, mas antes
fazer dela um objeto, tornar sua marca visível, assumir a postura
anti"-ilusionística que a expõe no produto final.\footnote{Seria
  interessante comparar esta ideia com formulações semelhantes da parte
  de arquitetos ligados à chamada Escola Paulista, como Sérgio Ferro.
  Ver, neste sentido, Pedro Fiori Arantes (org.). \emph{Sérgio Ferro:
  arquitetura e trabalho livre}. São Paulo: Cosac Naify, 2006.} Esta
marca (visual, técnica, de \emph{production values}) deixa assim de
comunicar uma diferença \emph{relativa} entre o subdesenvolvido e o
desenvolvido --- a ``deficiência'' do primeiro em relação ao segundo,
que seria o modelo a ser alcançado --- e passa a comunicar uma
diferença \emph{em si} --- a singularidade do subdesenvolvido, sua
inserção numa divisão internacional do trabalho que o mantém em condição
de subdesenvolvimento, bem como a recusa de todo modelo externo. Ao
fazê"-lo, ela modifica o contexto em que a obra do Terceiro Mundo, com a
``cara'' que lhe é própria, deve ser interpretada: não como uma cópia
malsucedida, mas em seus próprios termos, que incluem a relação entre
centro e periferia capitalista como dado primário e princípio de
diferenciação. Não se trata unicamente de falar \emph{da} realidade, mas
de deixá"-la falar \emph{através} de sua representação --- daí,
justamente, que possamos falar em um ``meta"-realismo''.\footnote{Em uma
  de suas muitas polêmicas com o Cinema Novo, este é um elemento que a
  geração imediatamente posterior, do Cinema Marginal, acentuaria e
  levaria ao extremo, como pode"-se ver neste texto de Rogério Sganzerla:
  ``Enfim, meus filmes são antes de tudo óbvias autocríticas que os
  intelectuais jamais poderão entender: meus filmes são seus próprios
  defeitos: meus filmes são aquilo que a produção não conseguiu: meus
  filmes são exata e concretamente aquilo que nunca poderei filmar
  porque, como todo o mundo sabe, o cinema brasileiro é o máximo porque
  é impossível''. Rogério Sganzerla. ``A mulher de todos'', \emph{Jornal
  do Brasil}, 20/2/1970. Se ocasionalmente reconhecia o
  valor destes diretores, Glauber também parecia inclui"-los em sua
  crítica a um ``cinema esquerdista utopista {[}que{]} quer destruir o
  espetáculo cinematográfico (\ldots{}) mas não coloca um novo projeto.
  (\ldots{}) {[}O{]} destrutivismo das linguagens é um fenômeno das
  vanguardas pequeno"-burguesas radicais, como muito bem explica o Walter
  Benjamin, que querem fazer as revoluções nas gramáticas, mas não
  passam daí, do espaço do \emph{écran}, que assim fica vago para o
  reformismo''. Glauber Rocha. ``Entrevista Portuguesa''. In: \emph{Revolução do Cinema Novo}, op.~cit., p.~271.
  Cinco anos antes, em 1969, ele já afirmava: ``{[}Para Godard{]},
  existe hoje um cinema para quatro mil pessoas, de militante a
  militante. Eu entendo Godard. Um cineasta europeu, francês, é lógico
  que se ponha o problema de destruir o cinema. Mas nós não podemos
  destruir aquilo que não existe. (\ldots{}) Nós estamos em uma fase de
  libertação nacional que passa também pelo cinema, e o relacionamento
  com o público popular é fundamental. Nós não temos o que destruir, mas
  construir. Cinemas, Casas, Estradas, Escolas, etc.'' Glauber Rocha.
  ``Tropicalismo, antropologia, mito, ideograma''. In: \emph{Revolução do Cinema Novo}, op.~cit., p.~152.}

A originalidade da proposta de Glauber neste momento do final dos anos
1960 está, primeiro, em perceber estas duas dimensões metacomunicativas, a
formal e a material, como aspectos de um mesmo processo dialético (ou de
retroalimentação, como se queira). E, segundo, em pensar este movimento
como uma relação dialética com o próprio público, que o transforma
simultaneamente, e de maneira complementar e interligada, na direção da
aceitação de um outro tipo de arte e da participação na transformação
social. Juntam"-se aí num único movimento quatro vértices --- meios de
produção, pesquisa formal, conteúdo político e público --- que
conformam um processo em que o público, exposto a uma visão
desmistificada de sua realidade, em contraste com a falsificação
produzida pelo cinema imperialista, é tanto progressivamente conquistado
por uma estética nova, descolonizada, quanto descobre o imperialismo
como força determinante em sua vida. Ele pode, assim, relacionar"-se
praticamente não só com uma nova linguagem cinematográfica, mas também
com as questões políticas que esse novo cinema levanta; ao mesmo tempo
em que, assim fazendo, consolida o espaço, tanto comercial quanto
artístico e político, que essa nova cinematografia necessita para se
desenvolver, retroalimentando o processo como um todo. É como se o
procedimento brechtiano/benjaminiano de \emph{Umfunktionierung} --- a
``refuncionalização'' cuja divisa era ``não reproduzir o aparato de
produção {[}cultural{]} sem ao mesmo tempo transformá"-lo o máximo
possível na direção do socialismo''\footnote{Walter Benjamin. ``The
  Artist as Producer''. \emph{New Left Review} 62, 1970, p.~89.}
--- extravasasse os limites do campo artístico para contaminar a
sociedade como um todo.\footnote{Um claro limite da
  \emph{Umfunktionierung} glauberiana, quando pensada ao lado de outras
  experiências do período, tais como o Grupo Ukamau na Bolívia, está em
  sua aversão manifesta à ideia de processos participativos na
  construção do filme, o que denota uma concepção ainda bastante
  tradicional, algo romântica, da função do artista enquanto produtor.
  Assim, numa carta de 1971 ao diretor do Instituto Cubano de Arte e
  Indústria Cinematográfica (\versal{ICAIC}), falando do isolamento do Cinema
  Novo ``pela direita e pela esquerda'', Glauber comenta que o \versal{CPC} da
  \versal{UNE} quisera ``nos obrigar a discutir nossos roteiros com um grupo de
  críticos, escritores, atores e jornalistas que nem entediam de cinema
  e eram péssimos profissionais em suas atividades''; o único que
  aceitara submeter"-se fora Ruy Guerra, que participou de uma discussão
  sobre o roteiro de \emph{Os fuzis} ``que não serviu para nada''.
  Glauber Rocha. ``Carta a Alfredo Guevara, maio de 1971''. In:
  \emph{Glauber Rocha. Cartas ao mundo}. São Paulo:
  Companhia das Letras, 1997, p.~401. Pode"-se contrastar esta passagem com
  declarações de Jorge Sanjinés como: ``Um filme sobre o povo feito por
  um roteirista não é o mesmo que um filme feito pelo povo
  \emph{através} do roteirista, na medida em que o tradutor e intérprete
  daquele povo se torna seu veículo expressivo. Com uma mudança nas
  relações de criação vem uma mudança em conteúdo e, em paralelo, uma
  mudança de forma''. Jorge Sanjinés. \emph{Theory and Practice of a
  Cinema with the People}. Nova York: Curbstone, 1979, p.~40 (grifo no
  original). Ver ainda, neste sentido, a observação de Carlos Estevam
  Martins, primeiro diretor do \versal{CPC} da \versal{UNE}: ``quanto mais o sujeito
  tivesse ambições individuais, como artista, menos condições ele tinha
  de entrar para o \versal{CPC}. Veja um exemplo: o Glauber Rocha (\ldots{}), o que
  tinha mais esse `fogo sagrado' das coisas que \emph{ele} ia fazer. Ele
  o criador, ele o autor''. Carlos Estevam Martins, entrevistado em
  Jalusa Barcellos. \emph{\versal{CPC} da \versal{UNE}. Uma história de paixão e consciência}, op.~cit., p.~90 (grifo no original).}

\begin{quote}
Um produto cultural que se opõe à ideologia estética dos fascismos
dominantes e à \emph{estética do entorpecimento} do cinema
norte"-americano deve criar seu próprio mercado. Deve revolucionar o
mercado. À medida que o mercado se dilata para um novo tipo de filme, o
novo tipo de filme se desenvolve.\footnote{Glauber Rocha. ``Teoria e
  prática do cinema latino"-americano''. In: \emph{Revolução do Cinema Novo}, op.~cit, p. 86.}
\end{quote}

Esse é o programa que Glauber apresenta e generaliza para todo o
Terceiro Mundo em textos de 1967 como ``Revolução cinematográfica'',
``Tricontinental'' e ``Teoria e prática do cinema
latino"-americano''.\footnote{Tratam"-se, na verdade, de segmentos de um
  texto maior, conforme observado em: José Carlos Avellar. \emph{A ponte
  clandestina: teorias de cinema na américa latina}. São Paulo: 34/Ed\versal{USP}, 1995, p.~111, n. 44. Este texto, escrito em 1967 --- um excerto
  seria publicado nos \emph{Cahiers du cinéma} em novembro daquele ano
  --- responde à ``Mensagem à Conferência Tricontinental'', isto é, ao
  discurso à conferência de fundação da Organização de Solidariedade dos
  Povos da Ásia, África e América Latina em Havana, janeiro de 1966, em
  que Che Guevara, às vésperas de sua partida para a Bolívia, falava em
  ver florescer ``dois, três, muitos Vietnams''. Daí talvez venha parte
  do tom programático assumido por Glauber neste trabalho.} O papel do
cinema como ferramenta de transformação efetiva da realidade não se
exaure no nível dos conteúdos que ele retrata. Ele se realiza plenamente
no círculo virtuoso (movimento dialético, feedback positivo) que
estabelece com seu público, ao instigar neste uma consciência
revolucionária e, fazendo"-o, criar as condições para a expansão e
radicalização do cinema que instiga esta consciência. Para isto, era
preciso produzir e distribuir este novo tipo de filme, construir uma
``tradição estética e econômica para o futuro''\footnote{Glauber Rocha.
  ``O transe da América Latina''. In: \emph{Revolução do Cinema Novo}, op.~cit., p.~186.} a partir do esforço
coletivo organizado de cineastas independentes para tomar os meios de
produção e distribuição em suas próprias mãos.

\begin{quote}
A única maneira de lutar é produzir: aquele que não quer ver esta
realidade é cego ou idiota. Os cineastas do Terceiro Mundo devem
organizar a produção nacional e expulsar o cinema imperialista do
mercado nacional.

Se cada país do Terceiro Mundo tiver uma produção sustentada por seu
próprio mercado nascerá um cinema revolucionário
tricontinental.\footnote{Glauber Rocha. ``Das sequóias às palmeiras''. In: \emph{Revolução do Cinema Novo}, op.~cit., p.~236.}
\end{quote}

Mas é também nos seus filmes deste período que Glauber parece elaborar
este meta"-realismo de maneiras que vão além de suas referências de
origem, e que são relevantes para as reflexões que ele viria a fazer em
``Estética do sonho'', de 1971. Após \emph{Terra em transe}, sob uma
influência que ele atribui às longas tomadas de Jean"-Marie Straub e
Danièlle Huillet\footnote{Glauber Rocha. ``O transe da América Latina'', op.~cit.,
  p.~181.} e à sugestão de Jean Renoir de que ele passasse a usar som
direto,\footnote{Glauber Rocha. \emph{Cahiers du cinéma}. In: \emph{Revolução do Cinema Novo}, op.~cit., p.~213.} o método de trabalho e o estilo de Glauber mudam
significativamente. A partir de \emph{Câncer}, lançado em 1969, seus
filmes passam a ser menos feitos na mesa de montagem e ele começa a
experimentar com longas tomadas semi"-improvisadas que embaralham as
fronteiras entre representação e documentário, num procedimento cujo
análogo mais próximo talvez seja o trabalho de Jacques Rivette do mesmo
período.

Esta espécie paradoxal de ``\emph{cinema verité} de ficção'', em que se
confundem um uso narrativo do não"-diegético e um uso não"-realista do
documental, obviamente também funcionava como mais um efeito de
estranhamento a expor o artifício por trás do processo cinematográfico.
Mas ela interessava a Glauber sobretudo como via de acesso àquele
inconsciente (entendido sempre como coletivo) no qual ele vinha se
tornando cada vez mais interessado. Tratava"-se de um modo de dá"-lo a ver
não tal como elaborado --- analisado e sintetizado --- racionalmente
pelo diretor, mas em seu estado, por assim dizer, ``natural'': ativo
naquilo que, na representação que se desenrolava diante da câmera, não
era da ordem da simples representação.

Assim, por exemplo, \emph{Câncer} é um filme sobre a violência inerente
à sociedade brasileira em que, no interior das situações construídas por
Glauber, a interação entre os atores é toda improvisada --- a ideia
sendo que, à medida em que eles interagissem, esta violência
inevitavelmente afloraria à superfície. \emph{O dragão da maldade contra
o santo guerreiro} também traz exemplos neste sentido, especialmente nas
sequências de multidão em que a população da cidade em que o filme foi
rodado aparecia como ela mesma, representando a população de uma cidade
como aquela. Assim, Glauber relata que, quando tudo estava pronto para
filmar a cena da luta entre Antônio das Mortes e o cangaceiro Coirana,
ele explicou a cena para os figurantes e algumas mulheres começaram a
cantar cantigas populares adequadas à situação.

\begin{quote}
{[}N{]}ós nos acomodamos para ver o combate. Ao mesmo tempo, alguns
atores começaram a se movimentar sob o efeito da música e, então, eu vi
a representação que se desenhava. Coloquei em cena os atores, os
personagens do filme. Esta foi minha única intervenção. Eu estava
situado ao mesmo tempo como espectador e participante. Os participantes
encontraram o seu lugar naturalmente. (\ldots{}) \emph{Isto não é
espontaneismo, é um trabalho ligado às raízes da
representação}.\footnote{Ibidem, p.~210 (grifo do autor).}
\end{quote}

Na cena em que os seguidores do cangaceiro são massacrados, algo ainda
mais interessante tivera lugar:

\begin{quote}
Então eu lhes disse {[}aos figurantes locais{]}: ``Vocês vão morrer.
Eles vão mata"-los.'' Eles começaram a cantar e fizeram isso durante 45
minutos ou uma hora. (\ldots{}) Então eu disse aos outros personagens:
``Zombem deles antes de mata"-los''. Eles cantavam um xaxado e {[}o ator
que interpretava o jagunço{]} Mata"-Vaca, que é ator de teatro e veio de
família burguesa, entrou também neste clima. Eu fiz então dois longos
planos de quatro minutos cada um desta cena. Ao término de um minuto era
preciso parar. Os matadores estavam num tal clima que haviam tirado suas
facas e começavam a espetar os pés das pessoas. Se eu os tivesse deixado
continuar, eles começariam a ferir as pessoas. \emph{Eles atingiam uma
verdade completa porque ela vinha da tradição}. E reencontravam esse
espírito em 1969.\footnote{Ibidem, p.~218 (grifo do autor).}
\end{quote}

Na estética da desestetização, as condições materiais eram incorporadas
à representação precisamente como aquilo que resistia à incorporação
--- espécie de resto indeglutível que rompia a ilusão de continuidade e
autossuficiência da representação ao inserir"-se nela como marca de um
processo exterior que a enquadra, limita e condiciona. De modo análogo,
estes fragmentos de real captados no improvisado e no imprevisto aderem
à diegese como algo que recorda ao espectador: ``isto não é parte da
narrativa; é parte do mundo''. É como se a ideia de um acesso direto à
realidade, de faze"-la ``falar por si mesma'', fosse retomada aqui num
outro plano, após passar pelo crivo anti"-realista e desfazer"-se de um
naturalismo ingênuo. Trata"-se, aqui, de captar não a superfície da
realidade, mas o real que se deixa entrever em seus interstícios ---
nos automatismos que guiam a ação irrefletida dos indivíduos, naquilo
que irrompe em seu comportamento quando eles ``baixam a guarda'', nos
séculos de história e tradição sedimentadas que os habitam
inconscientemente todo o tempo, ainda que não estejam sempre visíveis
enquanto tal. Como nos documentários sobre a vida selvagem, é questão de
deixar a câmera rodar e esperar que as feras irrompam em cena; deixar a
razão adormecer para vê"-la produzir monstros. Ao que podemos ver
responder, de outro lado, a ideia da revolução como ``uma mágica'' que é
``o imprevisto dentro da razão dominadora'', do ``irracionalismo
liberador'' como ``a mais forte arte arma do revolucionário''.\footnote{Glauber
  Rocha. ``Eztetyka do sonho''. In: \emph{Revolução do Cinema Novo}, op.~cit., p.~250.}

Em condições muito diferentes daquelas que se instalaram a partir de
1964, poderia o programa imaginado por Glauber Rocha em 1967 ter se
realizado? A pergunta, infelizmente, pertence à abarrotada estante dos
contrafatuais históricos. Em todo caso, parece evidente que, a fim de
implementa"-lo, o Cinema Novo teria certamente precisado lançar mão de
circuitos alternativos, extra"-comerciais de distribuição tais como os
que os \versal{CPC}s tentaram desenvolver até que o golpe civil"-militar viesse
extingui"-los. Em condições políticas desfavoráveis e diante da
dificuldade de realizar a ambição de ``revolucionar'' o mercado --- em
larga medida por conta da falta de controle sobre os mecanismos de
distribuição ---, este projeto ficaria espremido entre a condição de
produto comercial no mercado da cultura de massa, por um lado, e a de
obra artística com uma pretensão de intervenção política, por outro. Com
o tempo, especialmente após a aproximação com a Embrafilme nos anos 1970,
isso acabaria significando, para diretores como Cacá Diegues e Arnaldo
Jabor, uma preocupação crescente com a viabilidade comercial ---
justificada, precisamente, como a busca daquela mesma ``comunicação''
que fora duramente criticada na polêmica do Cinema Novo contra o \versal{CPC}.

\section{Referências}

\begin{Parskip}

\textsc{arantes}, Pedro Fiori (org.). \emph{Sérgio Ferro: arquitetura e trabalho livre}. São Paulo: Cosac Naify, 2006.

\textsc{avellar}, José Carlos. \emph{A ponte clandestina: teorias de cinema na américa latina}. São Paulo: 34/Ed\versal{USP}, 1995.

\textsc{barcellos}, Jalusa. \emph{\versal{CPC} da \versal{UNE}. Uma história de paixão e consciência}. Rio de Janeiro: Nova Fronteira, 1994

\textsc{bateson}, Gregory. \emph{Steps to an Ecology of Mind}. Nova York: Ballantine, 1981.

\textsc{benjamin}, Walter. \emph{Illuminations}. London: Pimlico, 1999.

\_\_\_\_\_\_. ``The Artist as Producer''. \emph{New Left Review} 62, 1970.

\textsc{bentes}, Ivana (org.). \emph{Glauber Rocha. Cartas ao mundo}. São Paulo: Companhia das Letras, 1997.

\textsc{rocha}, Glauber. \emph{Revolução do Cinema Novo}. São Paulo: Cosac Naify, 2004.

\textsc{sanjinés}, Jorge. \emph{Theory and Practice of a Cinema with the People}. Nova York: Curbstone, 1979.

\textsc{sganzerla}, Rogério. ``A mulher de todos''. \emph{Jornal do Brasil}, 20/2/1970.


\end{Parskip}

\chapter*{``Popau Brasil'': pop, realismo e subdesenvolvimento em 1967}
\addcontentsline{toc}{chapter}{``Popau Brasil'': pop, realismo e subdesenvolvimento em 1967,\\ \emph{por Sérgio Bruno Martins}}

\begin{flushright}
\emph{Sérgio Bruno Martins}
\end{flushright}

\emph{Barge}, de Robert Rauschenberg, uma tela quase 10 metros de
largura, era o cartão de boas"-vindas da mostra \emph{Ambiente \versal{USA}:
1957-1967}, que compunha, juntamente a uma retrospectiva de Edward
Hopper, a representação norte"-americana na Bienal de São Paulo de 1967.
Pintada entre 1962 e 1963, a tela é parte da safra de 79 pinturas de
silkscreen que consagrou Rauschenberg com o Grande Prêmio Internacional
da Bienal de Veneza de 1964. É fato notório que, faturado o prêmio,
Rauschenberg telefonou para seu assistente em Nova York e ordenou a
destruição de todas as telas de silkscreen em seu ateliê, marcando assim
o abrupto encerramento da série. É difícil imaginar, dada a opção de
ostentar \emph{Barge} diretamente abaixo do título da mostra, que o
triunfo norte"-americano em Veneza não tenha pautado em alguma medida o
curador William C. Seitz. Talvez seja por isso que, numa das fotos
tiradas durante a abertura da mostra, o crítico Mário Pedrosa estampe
uma fisionomia deveras taciturna --- ou talvez não, afinal flagras
fotográficos podem dever"-se às razões mais banais.

Em todo caso, plausível o desgosto de Pedrosa é. Basta atentarmos para
seu juízo acerca da Pop naquele momento, como no trecho do artigo ``Do
Pop americano ao Sertanejo Dias'', publicado em 29 de outubro de 1967 no
\emph{Correio da Manhã}, em que ele passa em revista a tendência
norte"-americana:

\begin{quote}
Trata"-se, para {[}os artistas Pop{]}, tranquilamente, sem dramas,
verificar o que há, e produzir não para estetas, mas para consumidores
``normais''. Quando um {[}Tom{]} Wesselmann, poderoso artista na sua
naturalíssima sensibilidade, prega nos seus ``grandes nus'' americanos,
no lugar apropriado, uma boca"-aparelho feita e não pintada, semi"-aberta
e rosada de lábios grossos, com alvos dentes à mostra, o nu é um corpo
feliz oferecido na feira, e cujos dentes na sua alvura estão ali como
para fazer o reclame de um novíssimo dentifrício. O outro apresenta uma
vitrina de bolos fulgurantemente apetitosos como apetitosos são os
anúncios de belas saladas e guloseimas, no Life ou no Saturday Evening
Post. Todos esses artistas o que produzem são acessórios para o herói
positivo; no otimismo que os embala realçam acima de tudo as virtudes
positivas dos produtos, como o faz, todos os instantes, sem cessar, a
máquina da grande publicidade no frenético e insaciável afã de
intensificar o consumo de massa.\footnote{Mário Pedrosa. \emph{Dos
  murais de Portinari aos espaços de Brasília}. São Paulo: Ed. Perspectiva, 1981, p. 217-221.}
\end{quote}

A primeira conclusão a aparecer no trecho citado --- ``trata"-se de
verificar o que há'' --- já diz muito. É nessa clave que Pedrosa discorre
sob o que lhe parece uma adesão irrestrita dos trabalhos mencionados ao
\emph{ethos} da sociedade de consumo e de suas imagens publicitárias. Os bolos
são apetitosos, \emph{como} o são nos anúncios de revista; os dentes
estão \emph{como} que para fazer um anúncio de dentifrício. Trata"-se,
para o crítico, de uma adesão sem reservas: verificação, constatação.

Pedrosa não estava só em sua avaliação da Pop. Em resposta à mesma
Bienal, publicada em 8 de outubro também no \emph{Correio da Manhã},
Sérgio Ferro afirma que a Pop é ``a reafirmação contente da ilusória e
movimentada cultura de massa''.\footnote{Sérgio Ferro. ``\versal{IX} Bienal
  Mondrian, Op e nós''. \emph{Correio da Manhã}, 08/10/1967, 4º Caderno, p. 3.} Mesmo antes do %Op mesmo? Não pop?
desembarque da Pop na Bienal, esse já era o veredicto reinante. Numa
entrevista dada a Ferreira Gullar em 1966, em conjunto com Rubens
Gerchman, Antonio Dias também desdenha da Pop: ``eles constatam um
hambúrguer, e daí?''.\footnote{Antonio Dias. ``Entrevista de Antonio Dias e Rubens Gerchman
  a Ferreira Gullar''. \emph{Revista Civilização Brasileira}, n.~11-12,
  12/1966 -- 03/1967, p.~174.} Perguntado mais
recentemente sobre a má recepção da Pop, Dias recordou que as próprias
obras não chegavam ao Brasil naquela época --- o que chegava era sua
reprodução em jornais e revistas. Não causa espanto que uma reprodução
de um quadro de Roy Lichtenstein num jornal de fato aparente ser uma
mera reprodução de um quadrinho publicado no mesmo jornal, tornando
impossível a apreciação da complexa transação pictórica que se dava
entre as imagens apropriadas e o resultado final de seus quadros. Mais
espantoso, no entanto, é o fato de que, mesmo com a presença maciça de
obras desse tipo na Bienal de 1967, tal juízo tenha permanecido
praticamente o mesmo. São poucas as exceções: defronte o regime de
imagens deveras distinto de um quadro como \emph{Barge}, Pedrosa ainda
tem o cuidado de deixar Rauschenberg de fora do parágrafo que citei
acima, mas Sérgio Ferro não tem dúvida em dizer que o que reina nos
``puzzles'' tanto de Rauschenberg quanto de Wesselman é ``a admiração
pela orgia da mercadoria''.\footnote{Sérgio Ferro. ``\versal{IX} Bienal
  Mondrian, Op e nós'', op.~cit.}

A metáfora dos ``puzzles'' provavelmente diz respeito ao fato de que
tais quadros concatenem uma série de imagens diferentes. Sendo assim, os
quadros de Dias também poderiam ser caracterizados como puzzles. A orgia
que reina ali, no entanto, é outra, ou pelo menos é o que pensa Mário
Pedrosa no mesmo artigo previamente citado:

\begin{quote}
No mundo de Dias a vida pede seu espaço próprio. {[}\ldots{}{]} Aí queima"-se
a química vital, com seus cheiros e gorduras, seus fermentos e graxas,
seus gases e secreções. {[}\ldots{}{]} Na sua pintura, o volume, a
tridimensão não é fictícia, dada por truques e perspectivas pictóricas;
é real, em relevo por cujas bordas escorrem todos os expedientes e
secreções orgânicas --- sangue, excrementos, esperma, orgasmos, pus,
hormônios, com seus cheiros e suas cores.\footnote{Mário Pedrosa. \emph{Dos
  murais de Portinari aos espaços de Brasília}, op.~cit.}
\end{quote}

Uma descrição deveras hiperbólica para o que não passava de objetos de
tecido forrados com algodão e pintados com tinta vermelha ou preta. Mas
não há nada de fortuito no exagero de Pedrosa: ele crava uma oposição
fundamental entre aquela leitura da Pop como uma arte de imagens
simulacrais e uma combinação de imagem e objeto supostamente capaz de
referenciar diretamente uma realidade dura, sertaneja, subdesenvolvida.
Mais do que apenas uma resposta às obras expostas, o juízo de Pedrosa
sinaliza uma tomada de posição em meio a um intenso debate sobre
realismo e subdesenvolvimento que condicionava, naquele momento, as
principais interpretações da Pop no Brasil, e para o qual a figura de
Antonio Dias vinha ganhando importância central.

Antes de recapitular este debate, é preciso trazer à baila outra cena
fundamental para a Pop, ou melhor, para uma neovanguarda que ainda não
havia realizado seu desembarque triunfante em Veneza e nem sequer era
inequivocamente reconhecida ainda pelo nome Pop. A ``International
Exhibition of the New Realists'', ocorrida em 1962 na galeria Sidney
Janis, em Nova York, apresentava uma peculiaridade: a justaposição de
artistas franceses e americanos como Andy Warhol, Roy Lichenstein, Claes
Oldenburg, Arman e Yves Klein, entre vários outros. O nome em si já diz
muito: ``new realists'' alude inequivocamente ao \emph{Nouveau Réalisme},
isto é, ao movimento francês que tinha no crítico Pierre Restany seu
principal porta"-voz e promotor. Convidado a colaborar com a exposição,
Restany esperava aproveitá"-la para cimentar uma ponte intercontinental
batizada a partir do \emph{Nouveau Réalisme}, dando ao termo e às
poéticas por ele defendidas forte ascendência sobre as neovanguardas dos
dois lados do Atlântico. Era, portanto, um gesto ousado, uma tentativa
de assumir a ponta de lança da neovanguarda internacional; a mesma ponta
de lança que a Pop, capitaneada por Leo Castelli e Ileana Sonnabend,
terminaria por assumir.

A exposição na galeria Sidney Janis não apenas frustrou as expectativas
de Restany,~terminando por tornar"-se um dos marcos iniciais do
reconhecimento da Pop, como, em retrospecto, marcou o fracasso de sua
tentativa de ``conquista da América''.\footnote{Sobre a investida de
  Restany em Nova York, ver Kaira M. Cabañas. ```Maigres et
  poussiéreux': les Nouveau Réalistes à New York''. In: \emph{Le Nouveau
  Réalisme}. Paris: Galeries Nationales du Grand Palais, 2007, p.~124-135; ver também Ágnes Berecz. ``Close Encounters: On Pierre
  Restany and \emph{Nouveau Réalisme}''. In: \emph{New Realisms:
  1957-1962. Object Strategies and the Readymade}. Madri/Cambridge,
  \versal{MA}: Museo Nacional Reina Sofía/\versal{MIT} Press, 2010, p.~53-62.} Este fracasso
é fundamental para a compreensão dos motivos que guiavam Restany em suas
frequentes visitas ao Brasil ao longo dos anos 1960. Pode ser que
crítico considerasse o Brasil um front alternativo nessa mesma batalha
transatlântica, quiçá apenas um prêmio de consolação; em todo caso, o
fato é que seu interesse pelo país nos anos 1960 se inscreve no contexto
de um ativo esforço de promoção internacional do \emph{Nouveau Réalisme}.
Em diversas viagens ao país, algumas pagas do próprio bolso, Restany fez
questão de estabelecer amizades e contatos profissionais --- o colunista
Jayme Maurício, por exemplo, não só noticiava diariamente as passagens
de Restany, como se gabava de sua amizade com o crítico, a ponto de
reproduzir cartas que este lhe mandava agradecendo a hospitalidade. É
nesse contexto, enfim, que se dá o encontro entre o crítico e o então
novíssimo artista Antonio Dias.

Em 1964, Restany escreve o texto de apresentação da exposição individual
de Dias na Galeria Relevo. É, sem dúvida, o escrito mais importante do
artista até então --- com apenas 20 anos, Dias era apresentado por um
renomado crítico estrangeiro! Para compreender o texto, é preciso atentar
para os dois pontos que pautavam a atuação de Restany no Brasil naquele
momento. Em primeiro lugar, o crítico buscava superar a abstração
informal que dominava o pós"-guerra europeu, representada na França pela
pintura da chamada Escola de Paris. Restany então escreve que Dias fazia
parte de uma ``terceira geração'' de artistas do pós"-guerra, que
finalmente abandona a postura de ``recusa de um mundo absurdo'' das
gerações anteriores em prol do otimismo em relação às transformações
industriais e tecnológicas da sociedade daquele momento. Segundo o
crítico, uma geração que:

\begin{quote}
se apronta para viver a epopeia das grandes descobertas
interplanetárias; o olhar que ela lança sobre o mundo é ``novo'', no
sentido em que se pode falar de ideias ``novas'': uma ótica sem
parti"-pris, o olho da câmera fixado sobre a natureza moderna; esse
realismo objetivo da visão engendra uma arte de participação fortemente
influenciada por todas as técnicas contemporâneas da comunicação visual.
{[}\ldots{}{]} O artista de hoje põe os pés sobre a terra e procura os
elementos de uma participação orgânica no corpo social e em suas
estruturas técnicas, industriais, urbanas.\footnote{O ensaio de Restany
  foi republicado na coluna de Jayme Maurício: Pierre Restany. ``Da
  torre de marfim à Torre de Babel''. \emph{Correio da Manhã}, 4/12/1964.}
\end{quote}

Vale notar que a exposição na Relevo marca uma virada no próprio
trabalho de Dias, que deixa de lado uma abstração fortemente matérica,
inspirada na obra de pintores como os primos catalães Antonio Tapiès e
Modest Cuixart, para lançar"-se à linguagem pictórica que o consagraria
nos anos 1960, de viés mais gráfico e reminiscente das histórias em
quadrinhos, repleta de signos que remetem a partes de corpos em
situações violentas e/ou agressivamente sexuais. Restany certamente via
nessa virada mais um indício da superação das gerações anteriores. De
resto, a nova linguagem de Dias foi tão prontamente acolhida que alguns
colunistas da época chegaram batizá"-la de ``estilo Antonio Dias''. Em
seu texto, Restany minimiza cuidadosamente qualquer inclinação
expressionista: ele reconhece que a pintura de Dias é feita de ``pedaços
escolhidos de um diário íntimo'', mas argumenta que esse mesmo diário
diz tanto à ``introspecção interior'' quanto à ``reportagem exterior''.
Era, enfim, uma arte do ``Brasil urbano de 1964'': ``Lá existe sexo,
sangue, fatos diversos e muito fetichismo objetivo: enfim, toda a
herança de nossa natureza urbana e de nossa civilização industrial na
aurora de sua segunda mutação.''\footnote{Ibidem.}

Tal misticismo tecnofílico atrela"-se ao segundo ponto da pauta de
Restany: a negação não apenas do expressionismo, mas também da arte Pop.
O fracasso de Restany em Nova York torna a Pop o grande rival a ser
combatido, razão pela qual ele crescentemente denuncia o ``esteticismo''
gratuito dos norte"-americanos e propõe, contra este, um velho e
conhecido antídoto: aquela ``apropriação direta do real'' que ele
imputara ao \emph{Nouveau Réalisme} francês (daí a importância de sua
insistência no suposto caráter de ``reportagem'' da obra de Dias).

Esta dupla pauta de Restany --- oposição à abstração informal e oposição
à Pop --- será simultaneamente aproveitada e subvertida no Brasil dos
anos 1960. Analisemos primeiramente a questão da abstração. No
seminário \emph{Propostas 66}, ao apresentar uma primeira elaboração de
seu conceito de Nova Objetividade, Hélio Oiticica afirma que uma de suas
fontes é o Novo Realismo; não o Novo Realismo de Restany, mas sim o do
físico e crítico de arte Mário Schenberg. É precisamente no âmbito de
suas respectivas relações com as gerações anteriores que estes dois
``novos realismos'' se distinguem. Enquanto Restany rejeitava a
abstração, Schenberg a inseria num movimento dialético, reconhecendo ---
especialmente no caso da abstração de matriz construtiva --- seu papel
histórico fundamental de ter purgado a arte brasileira ``de formas
anacrônicas de naturalismo e realismo'' e aberto espaço, por
conseguinte, para o realismo da geração de Antonio Dias e Rubens
Gerchman. Tomando o movimento dialético de Schenberg como ponto de
partida, Oiticica termina por recusar o próprio nome ``realismo'',
talvez por pensar arraigada demais sua oposição ao termo ``abstração''
--- é o que explica sua opção pelo termo ``Nova Objetividade''.

Oiticica argumenta que a principal contribuição de \emph{Nota sobre a
morte imprevista} é a fusão entre o ``sentido estrutural'' da trajetória
que ele próprio e Lygia Clark vinham descrevendo desde o neoconcretismo
e um sentido ``ético"-político'' oriundo das referências figurativas
feitas pela pintura de Dias a questões urgentes como o ``massacre de
Hiroshima''.\footnote{Hélio Oiticica. ``Vivência do Morro do Quieto''.
  In: \emph{Tropicália: uma revolução na cultura brasileira}. Carlos
  Basualdo (org.). São Paulo: Cosac Naify, 2007, p.~218-219.} Em outras palavras,
ao progressivo questionamento do espaço da representação inerente à
dialética do objeto (que, desde o neoconcretismo, traduzia"-se no
questionamento do plano pictórico e do suporte do quadro), acopla"-se o
aporte ético"-político neofigurativo, resultando numa versão mais
fortemente engajada do que Oiticica chama de ``arte ambiental''. A
instauração da dimensão ambiental é guiada pela estrutura do objeto,
que, em \emph{Nota sobre a morte imprevista}, excede o plano pictórico
através da língua negra que se espraia pelo chão e invade o espaço do
espectador, espraiamento esse que é imantado desde a raiz pelas questões
políticas que o quadro sinaliza e organiza (e cuja legibilidade, na
forma da aproximação física de quem busca decifrar as referências
pintadas, é o mesmo movimento que traz o corpo do espectador para perto
da presença abjeta da língua negra). Se a obra é um ``anti"-quadro'',
como quer Oiticica, é por uma razão também dialética: é como se seu
recurso à figuração se amparasse em convenções pictóricas (isto é, ao
fato de que quadros tradicionalmente comunicam universos figurados), mas
não se contivesse em sua lógica representativa; como se seu ``sentido
ético"-político'' ampliasse o campo semântico da abertura ambiental que a
obra propõe e contribuísse para evitar uma versão meramente
``esteticista'' da arte ambiental.

Isso talvez fique mais claro numa das obras mais fundamentais de
Oiticica no período, o \emph{B33 Bólide Caixa 18, Homenagem a
Cara de Cavalo} (1965-1966). Ao extrair uma imagem feita com o propósito
de circular como troféu, celebrando o abatimento do inimigo público
número um (ou melhor, coroando a produção midiática deste inimigo
público), e reinscrevê"-la num objeto memorial, que confere aura de
mártir a Cara de Cavalo, Oiticica articula a consciência da opressão
social ao condicionamento subjetivo de um espectador conformado às
estruturas convencionais de experiência da obra de arte. O contraste
entre a experiência intimista que a estrutura formal do bólide sugere e
o impacto da imagem nele inscrita tensiona tal articulação --- novamente,
a dimensão ambiental depende precisamente desta tensão para efetuar sua
mediação entre os registros individual (na forma do projeto de
descondicionamento do espectador \emph{vis"-à"-vis} a experiência convencional da
arte) e social (na forma do projeto de descondicionamento do sujeito
\emph{vis"-à"-vis} as estruturas ideológicas nas quais ele inevitavelmente se
insere).

Passemos então ao problema da Pop. Como vimos, Restany propunha o
entusiasmo pela ``apropriação direta do real'' como antídoto contra o
esteticismo Pop, mas o que ele não pôde prever foi o registro crítico
--- isto é, contrário ao seu otimismo tecnofílico ---, e não mais
integrado, que este mesmo entusiasmo terminou adquirindo no Brasil. Daí
que um crítico como Mário Pedrosa fosse simpático a Restany, mas
narrasse o mesmo ``estilo Antonio Dias'' sem qualquer traço otimista,
como vimos anteriormente. Enquanto Restany recusava o suposto
esteticismo Pop em nome de visões de um futuro integrado e
tecnologicamente avançado, Pedrosa o fazia para celebrar a baixeza
material, atribuindo a esta o sentido realista daqueles trabalhos. Seu
realismo era, portanto, oposto à assepsia de uma arte Pop entendida como
puro simulacro.

Tal ênfase na baixeza material não é mera idiossincrasia; na verdade,
ela se relaciona com uma questão central na teoria econômica e social
daquele momento: o subdesenvolvimento. Em seu livro \emph{Cultura posta
em questão}, o poeta Ferreira Gullar afirma que tentar fazer no Brasil
esculturas como as de Max Bill é ``perder a noção de realidade: aquela
arte é produto de alto desenvolvimento técnico e industrial. Prova disso
é o que o mais evidente defeito das esculturas concretas realizadas no
Brasil era a sua péssima execução.''\footnote{Ferreira Gullar.
  \emph{Cultura posta em questão}. Rio de Janeiro: Editora Civilização
  Brasileira, 1965, p. 17.} Nada mais distante do imaginário de fusão
tecnológica e ``epopeias interplanetárias'' anunciado por Restany do que
o critério de adequação entre linguagem artística e desenvolvimento
econômico defendido por Gullar. Em 1967, o próprio Restany sente na pele
sua inadequação a este debate. Após reclamar, numa resenha sobre a
Bienal daquele ano, que as obras da representação brasileiras eram
``meio trôpegas'', Restany é repreendido pelo crítico Frederico Morais:

\begin{quote}
O que Gerchman e outros jovens estão fazendo no Brasil é uma Pop
antropofágica, uma ``Pop pau"-brasil''. Não Pop arte, mas popau, com já
se anunciou. O que {[}Restany{]} chama de `vanguarda trôpega' é a
consciência de que não há meio termo: ou devoramos \emph{rápida e
subdesenvolvidamente} o que nos impõem ou seremos devorados. {[}\ldots{}{]} O
nosso problema no Brasil não é fazer bem o mesmo (quantidade), mas,
ainda que mal, fazer o novo (qualidade).\footnote{Frederico Morais.
  ``Colonialismo cultural''. \emph{Diário de Notícias}, 11/10/1967.}
\end{quote}

Em última análise, o internacionalismo abrangente e expansivo de Restany
esbarrava na ressurgência de um vetor nacionalista na vanguarda
brasileira; não no sentido de uma vanguarda calcada no elogio, elevação
ou consolidação do nacional, mas de uma vanguarda que se pensava no
mesmo patamar de seus pares internacionais precisamente por assumir
dialeticamente a sua raiz nas contradições próprias de seu contexto
nacional. É verdade que a ideia de vanguarda é ela mesma condenada em
\emph{Cultura posta em questão}, livro intimamente ligado à experiência
do Centro Popular de Cultura (\versal{CPC}) no período imediatamente anterior ao
golpe militar; porém, ela deixa de sê"-lo no livro seguinte de Gullar,
\emph{Vanguarda e subdesenvolvimento}, publicado 1969, mas escrito ao
longo dos quatro ou cinco anos anteriores, incluindo o período em que
ele defende a vanguarda de Dias e Gerchman. Aqui, já com maior distância
crítica em relação aos seus anos de militância no \versal{CPC}, Gullar segue
problematizando a questão da vanguarda, mas termina por reabilitá"-la em
seus próprios termos --- ou seja, em termos \emph{realistas}.

O poeta lança mão do uso que Umberto Eco faz da teoria da informação em
seu famoso livro \emph{A obra aberta}, no qual a questão gira em torno
de um problema comunicativo: por um lado, uma mensagem com quantidade
reduzida de informações novas garante eficácia comunicativa, mas tende à
redundância; por outro lado, o novo é definido como acréscimo de
informação que, em excesso, aproxima"-se da incomunicabilidade.
Invertendo a doxa modernista, Gullar argumenta que a própria vanguarda
formalista moderna é, paradoxalmente, uma vanguarda da redundância: dado
que seus lances vanguardistas se dão no interior de uma linhagem e de
uma linguagem formal reconhecidas e autorreferenciais, ela então não
traria informações novas. Assim, juntando Eco e Lukács, Gullar argumenta
que quem tem informações novas a trazer é justamente o realismo:

\begin{quote}
A prioridade do conteúdo sobre a forma, na arte como na sociedade, é que
determina a transformação das estruturas, a renovação, a superação do
velho pelo novo. Assim, ao contrário do que pretendem afirmar corifeus
do vanguardismo formalista, a verdadeira renovação --- aquela que é
realmente revolucionária e consequente, na sociedade como na arte ---
resulta da emersão do conteúdo novo, isto é, da particularidade, do fato
histórico, social e culturalmente determinado, que exige a melhor forma
possível para se expressar.\footnote{Ferreira Gullar. \emph{Vanguarda e
  subdesenvolvimento: ensaios sobre arte}. Rio de Janeiro: Civilização
  Brasileira, 1969, p. 61.}
\end{quote}

Oiticica baseia"-se tanto em Gullar quanto em Schenberg; é daí que vem
sua repetida aversão ao ``esteticismo'', que, como Gullar, ele culpa por
produzir uma ``segunda natureza'' ilusória que cega os artistas de
integrar a questão ``estrutural'' à ``ética e política''.\footnote{Hélio
  Oiticica. ``Esquema geral da Nova Objetividade''. In: \emph{Tropicália:
  uma revolução na cultura brasileira}, op.~cit., p.~229.} Mas a admiração do artista pelo trabalho do
poeta neste momento não era uma via de mão dupla. Para Gullar, os
\emph{Parangolés} de Oiticica eram, por assim dizer, sinais de que o
artista havia finalmente reconhecido que seu próprio trabalho
construtivo estava atrelado àquela falsa natureza. Em outras palavras,
Gullar via o \emph{Parangolé} como uma tentativa desesperada, quiçá
afobada, de reinserir no bojo da obra o contato perdido com a realidade,
coisa que ele próprio, Gullar, já teria realizado anos antes, ao
abandonar o neoconcretismo. Em suma, é como se Oiticica apontasse na
direção correta, mas falhasse em engrenar a marcha dialética adequada
para avançar de forma consequente nessa direção.

Por outro lado, ainda para Gullar, é como se a obra de Dias e Gerchman não
tivesse passado por esse momento de ``mistificação especulativa'' que o
poeta atribuía ao formalismo modernista.\footnote{Ferreira Gullar.
\emph{Vanguarda e subdesenvolvimento: ensaios sobre arte}, op.~cit, p.~49.} Por reter a mediação da representação pictórica, esses artistas estariam caminhando desde sempre rumo à
produção de informações novas, mas comunicáveis, isto é, de produzir um
acréscimo de informações alheio aos lances formalistas da arte moderna.
O que também significa dizer, claro, que as dialéticas de Schenberg e
Gullar eram incompatíveis: para Schenberg, a passagem pela abstração
teria constituído um lance dialético capaz de contribuir
substantivamente para o novo realismo; para Gullar, no entanto, a marcha
dialética do realismo levaria à constatação de que a arte construtiva
fizera parte do erro formalista.

Em 1966, no auge do debate brasileiro sobre realismo, Dias muda"-se para
Paris. Lá, numa entrevista, o artista afirma que sua obra vinha se
beneficiando dos materiais mais avançados que ele tinha agora ao seu
dispor, muito superiores aos disponíveis num ambiente precário como o
brasileiro. \emph{Coletivo} (1967) e \emph{Solitário} (1967), por
exemplo, são objetos acabados em plástico laminado, e não mais em
madeira pintada, como era o caso das caixas em \emph{Nota sobre a morte
imprevista} e \emph{O meu retrato} (1967), entre outras. Por um lado,
Dias parece ter assimilado as lições de Gullar sobre adequação entre
forma e desenvolvimento técnico e econômico. Só que a ``adequação''
talvez não seja o termo correto aqui: é o que fica claro na descrição
que a crítica Sonia Salzstein faz de \emph{Coletivo} e \emph{Solitário}:
``elementos biomórficos parecem aí pressionar para vir à superfície, mas
só podem fazê"-lo como resíduos de uma alegoria remota, desmanchada sob
as formas irrelevantes do design e da geometria anódina que marcavam a
experiência do espaço na vida contemporânea.''\footnote{Sônia Salzstein.
  ``As muitas Mascarades de Antonio Dias''. In: \emph{Antonio
  Dias: Anywhere is my Land}. Zurich: Daros Latinamerica, 2009, p. 38.}
Nem mesmo a alegoria consegue mais extrair qualquer força das
contradições ou reorganizá"-las numa constelação fulminante; ela se
apresenta na descrição de Salzstein como um resíduo remoto, desprovido
de maior legibilidade. E não é só a alegoria que se torna ilegível: a
adequação proposta por Gullar entre desenvolvimento técnico e artístico
aqui já também não produz sentido; o desenvolvimento técnico ele próprio
afoga o signo artístico, desmanchando"-o sob as formas anódinas do mundo
material contemporâneo, ao passo em que estas mesmas formas ganham força
justamente por evidenciar a indigência desse mundo. Em \emph{Coletivo e
Solitário}, o que Dias parece buscar é esse sentido do resíduo como um
mínimo de resistência a qualquer assimilação, seja pela cultura Pop,
seja pelo formalismo vanguardista, seja por leituras como a do próprio
Pedrosa --- são trabalhos impermeáveis também àquela orgia de secreções
que Pedrosa, nesse mesmo ano de 1967, projeta sobre as pinturas que o
artista realizara ainda no Rio. Um sinal de que só assim --- tendo saído
do país ---, Dias parece ter sido capaz de recolocar noutro nível,
imantado pelo problema do desenvolvimento, mas não tanto pelos \emph{parti
pris} estéticos que o acompanhavam no Brasil, a discussão sobre o tipo de
sociedade e de cultura que possibilitava a arte Pop.

\pagebreak

\section{Referências}

\begin{Parskip}

\textsc{basualdo}, Carlos (org.). \emph{Tropicália: uma revolução na cultura brasileira}. São Paulo: Cosac Naify, 2007.

\textsc{berecz}, Ágnes \& \textsc{buchloh}, Benjamin \textsc{H. D.} \emph{New Realisms: 1957-1962. Object Strategies and the Readymade}. Madri/Cambridge, \versal{MA}: Museo Nacional Reina Sofía/\versal{MIT} Press, 2010.

\textsc{cabañas}, Kaira M. ```Maigres et poussiéreux': les Nouveau Réalistes à New York''. In: \emph{Le Nouveau Réalisme}. Paris: Galeries Nationales du Grand Palais, 2007.

\textsc{dias}, Antonio. ``Entrevista de Antonio Dias e Rubens Gerchman a Ferreira Gullar''. In: \textsc{gullar}, Ferreira; \textsc{dias}, Antonio Dias; \textsc{gerchman}, Rubens. \emph{Revista Civilização Brasileira}, n.~11-12, 12/1966 -- 03/1967.

\textsc{ferro}, Sérgio. ``\versal{IX} Bienal Mondrian, Op e nós''. \emph{Correio da Manhã}, 08/10/1967.

\textsc{gullar}, Ferreira. \emph{Cultura posta em questão}. Rio de Janeiro: Editora Civilização Brasileira, 1965.

\_\_\_\_\_\_. \emph{Vanguarda e subdesenvolvimento: ensaios sobre arte}. Rio de Janeiro: Civilização Brasileira, 1969.

\textsc{morais}, Frederico. ``Colonialismo cultural''. \emph{Diário de Notícias}, 11/10/1967.

\textsc{pedrosa}, Mário. \emph{Dos murais de Portinari aos espaços de Brasília}. São Paulo: Ed. Perspectiva, 1981.

\textsc{restany}, Pierre. ``Da torre de marfim à Torre de Babel''. \emph{Correio da Manhã}, 4/12/1964.

\textsc{salzstein}, Sônia. ``As muitas Mascarades de Antonio Dias''. In: \emph{Antonio Dias: Anywhere is my Land}. Zurich: Daros Latinamerica, 2009.


\end{Parskip}